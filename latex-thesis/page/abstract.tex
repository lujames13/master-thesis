% 中文摘要頁
\begin{ZhAbstract}
    \begin{ZhAbstractItems}
        % 關鍵詞,請自己填,請自己填,多個關鍵字以逗號(、)隔開
        \noindent \text 關鍵詞:區塊鏈、聯邦式學習、委員會佔領、驗證者共謀

    \end{ZhAbstractItems}

    \begin{ZhAbstractDescription}
       基於區塊鏈的聯邦式學習 (BCFL) 透過去中心化共識機制解決了信任與隱私問題。現有的 BCFL 系統依賴基於委員會的驗證機制,並假設委員會成員是誠實的或擁有誠實多數。此假設容易受到驗證者共謀的威脅,攻擊者可透過累積權益 (Stake) 來主導委員會。我們識別出一種新型威脅——漸進式委員會佔領攻擊 (PCCA),理性攻擊者利用激勵機制逐步累積權益,並佔領足夠的委員會席次以發動協同攻擊。一旦攻擊者取得委員會多數席次,現有的委員會架構便無法偵測或防範此類攻擊。為防禦 PCCA,我們提出一種審計驅動型委員會 BlockDFL (AC-BlockDFL),將安全性與委員會組成解耦:由小型委員會負責例行驗證以提供活性 (Liveness),而由全域共識支持的挑戰機制提供安全性保證。任何惡意聚合行為都將觸發密碼學驗證與罰沒懲罰,立即沒收參與共謀的委員會成員之全額質押。此機制將安全門檻從委員會多數轉移至全網共識,打破攻擊者依賴的權益累積正反饋循環。在 2000 輪的長期模擬實驗中,本機制將攻擊發生次數從 107 次壓制至 5 次,相較於現有方法實現超過 20 倍的攻擊抑制效果。我們的解耦設計亦允許更小的委員會規模,在不犧牲安全性的前提下提升運算效率。
    \end{ZhAbstractDescription}
    
\end{ZhAbstract}
