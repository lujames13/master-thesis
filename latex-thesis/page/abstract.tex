% 中文摘要頁
\begin{ZhAbstract}
    \begin{ZhAbstractItems}
        % 關鍵詞,請自己填,請自己填,多個關鍵字以逗號(、)隔開
        \noindent \text 關鍵詞:區塊鏈、聯邦式學習、委員會佔領、驗證者共謀

    \end{ZhAbstractItems}

    \begin{ZhAbstractDescription}
       基於區塊鏈的聯邦式學習 (BCFL) 透過去中心化共識機制解決了信任與隱私問題。現有的 BCFL 系統依賴基於委員會的驗證機制,並假設委員會成員是誠實的或擁有誠實多數。此假設容易受到驗證者共謀的威脅,攻擊者可透過累積權益 (Stake) 來主導委員會。我們識別出一種新型威脅——漸進式委員會佔領攻擊 (PCCA),理性攻擊者利用激勵機制逐步累積權益,並佔領足夠的委員會席次以發動協同攻擊。一旦攻擊者取得委員會多數席次,現有的委員會架構便無法偵測或防範此類攻擊。為防禦 PCCA,我們提出一種審計驅動型委員會 BlockDFL (AC-BlockDFL),將安全性與委員會組成解耦:由小型委員會負責例行驗證以提供活性 (Liveness),而由全域共識支持的挑戰機制提供安全性保證。任何惡意聚合行為都將觸發密碼學驗證、罰沒懲罰,並立即移除惡意驗證者——無論其在委員會中的席次多寡。此機制將安全門檻從委員會多數轉移至全網共識,從而瓦解委員會佔領攻擊。實驗結果顯示,當攻擊發生時,本機制能完全清除惡意委員會成員,而現有最先進的方法則允許攻擊者取得委員會完全控制權並執行不受制衡的攻擊。我們的解耦設計亦允許更小的委員會規模,在不犧牲安全性的前提下提升運算效率。
    \end{ZhAbstractDescription}
    
\end{ZhAbstract}

