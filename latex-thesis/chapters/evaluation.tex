\begin{ZhChapter}

\chapter{實驗評估 (Experimental Evaluation)}
\label{chap:evaluation}

本章透過系統性的實驗設計,驗證挑戰增強型委員會架構在應對漸進式委員會佔領攻擊時的防禦效能。本研究的核心關注點在於經濟安全性機制能否有效遏止理性攻擊者的惡意行為,進而維護系統的長期治理穩定性,而非傳統聯邦學習安全性研究所側重的模型準確率指標。為提供最嚴格的效能驗證,本章實驗採用「最壞情況分析」的設計哲學,假設攻擊者完全不顧經濟理性而執意發動所有可能的攻擊,藉此檢驗防禦機制在極端條件下的偵測與懲罰能力。這種實驗設計的深層意涵在於:若攻擊者在最壞情況下仍無法逃脫制裁,則理性攻擊者在預見此結果後將選擇不發動攻擊,系統便能在實務運作中自然趨向穩定均衡。

\section{實驗設置}
\label{sec:eval_setup}

\subsection{實驗配置概述}

本研究選用 MNIST 手寫數字資料集作為聯邦學習任務的測試平台,採用包含兩個卷積層與兩個全連接層的標準卷積神經網路作為訓練模型。訓練資料以獨立同分布的方式均勻分配給所有參與的客戶端節點,此設計選擇源於本研究防禦機制的本質特性:如第 \ref{chap:framework} 章所闘述,挑戰增強型委員會架構的核心防禦運作於共識層,其功能在於透過經濟懲罰威懾惡意的委員會行為,而攻擊能否成功取決於委員會組成與共識決策過程,這些因素與底層訓練資料的統計分佈特性相互獨立。

為確保評估結果的公平性,本實驗將挑戰增強型委員會架構與 BlockDFL 進行系統性對照比較,兩種架構均採用相同的委員會規模 $C=7$。攻擊場景嚴格遵循第 \ref{chap:threat-model} 章所定義的漸進式委員會佔領攻擊模型,攻擊者採取理性的兩階段策略:在潛伏階段完全模仿誠實行為以累積權益,當成功獲得委員會超過三分之二席位時進入佔領階段,根據控制範圍選擇戰略性餓死或全棧投毒策略。表 \ref{tab:exp_params} 彙整了本研究實驗所採用的完整系統參數配置。

\begin{table}[htbp]
    \centering
    \caption{實驗參數配置}
    \label{tab:exp_params}
    \renewcommand{\arraystretch}{1.3}
    \begin{tabular}{|l|l|}
        \hline
        \textbf{參數名稱} & \textbf{設定值} \\
        \hline
        訓練輪數 & $R = 300$(基礎實驗)/ $R = 2000$(長期實驗) \\
        \hline
        驗證者池規模 & $N = 100$ \\
        \hline
        委員會大小 & $C = 7$ \\
        \hline
        惡意節點數量 & $M = 30$(初始權益佔比 30\%) \\
        \hline
        初始權益分配 & 所有節點均分配 100 單位 \\
        \hline
        每輪獎勵 & 驗證者 1.0,聚合者 1.0,更新提供者 0.05 \\
        \hline
        罰沒規則 & 挑戰成功時全額沒收惡意節點質押 \\
        \hline
    \end{tabular}
\end{table}

惡意節點初始佔比設定為 30\% 代表了相當嚴峻的威脅情境,此比例已接近大多數拜占庭容錯系統所能容忍的理論上限。根據第 \ref{sec:committee-size-security} 節的超幾何分佈分析,在此條件下惡意節點於單輪中獲得委員會超過三分之二席位的機率約為 2.4\%,考量到聯邦學習訓練通常需要經歷數百甚至數千輪迭代,累積下來足以產生多次攻擊機會,為驗證防禦機制的長期效能提供了充分的測試場景。

\section{實驗結果與分析}
\label{sec:experimental_results}

\subsection{權益動態演化分析}
\label{sec:stake_dynamics}

本研究的核心主張在於挑戰增強型委員會架構能夠透過經濟懲罰機制有效打破漸進式委員會佔領攻擊所依賴的正反饋循環。權益比值作為衡量系統治理結構健康程度的核心指標,定義為惡意節點平均權益除以誠實節點平均權益,其變化軌跡直接反映了經濟安全性機制能否成功重塑長期的權力分配格局。

在缺乏挑戰機制的 BlockDFL 架構中,實驗數據呈現了權益分佈逐漸失衡的系統性趨勢。追蹤 2000 輪訓練過程中的權益演化軌跡可以發現,惡意節點的權益比值從初始的 1.0 穩定攀升,經過約 300 輪的初期波動後收斂至 1.3 左右並長期維持。這種持續性優勢根植於系統獎勵機制的內在結構:由於角色分配採用權益加權的隨機選舉,一旦某些節點累積了相對較高的權益,其在後續輪次被選為高獎勵角色的機率便隨之提升,形成「權益優勢帶來更多獎勵,更多獎勵鞏固權益優勢」的正反饋循環,這正是漸進式委員會佔領攻擊得以成立的經濟基礎。

\begin{figure}[htbp]
    \centering
    \includegraphics[width=0.85\textwidth]{figures/experiments/mnist_results_stack_comparison.png}
    \caption{BlockDFL 與 CACA 權益演化對比(300 輪基礎實驗)}
    \label{fig:stake_evolution_300}
\end{figure}

挑戰增強型委員會架構的權益演化軌跡則呈現截然不同的動態特徵。如圖 \ref{fig:stake_evolution_300} 所示,惡意節點的權益比值展現出明顯的「階梯式下降」模式,這種獨特軌跡精確對應了異步審計機制觸發罰沒制裁的時點:每當惡意節點嘗試利用委員會控制權發動攻擊時,挑戰者都能透過重新執行 Krum 演算法偵測異常,發起挑戰並觸發全網仲裁,一旦惡意行為經確認,所有參與共謀的惡意節點將被沒收全額質押,造成權益的瞬間大幅縮減。

\begin{figure}[htbp]
    \centering
    \includegraphics[width=0.85\textwidth]{figures/experiments/mnist_results_stack_comparison_2000_round.png}
    \caption{2000 輪長期模擬下的權益動態演化}
    \label{fig:stake_evolution_2000}
\end{figure}

將實驗觀察期延伸至 2000 輪的長期模擬,更能完整展現經濟懲罰機制作為「漸進式淨化」工具的運作特性。如圖 \ref{fig:stake_evolution_2000} 所示,本研究方法中的惡意節點權益軌跡呈現五次明確的階梯式下降,分別發生在第 15、136、695、815 與 1332 輪,將權益比值從初始的 1.0 逐步壓制至最終的 0.37,意味著實驗結束時惡意節點的平均權益僅為誠實節點的三分之一強。相較之下,BlockDFL 中惡意節點的權益比值穩定維持在 1.3 的優勢水平,這種對比清楚印證了罰沒機制成功打破正反饋循環的核心主張。

值得特別關注的是罰沒事件的時間間隔變化趨勢。隨著惡意節點權益基數的縮減,其成功組織委員會佔領攻擊的難度持續提高,導致罰沒事件的發生頻率逐漸降低:從第 15 輪到第 136 輪間隔 121 輪,從第 815 輪到第 1332 輪間隔已擴大至 517 輪,而第 1332 輪之後的 668 輪觀察期內則未再發生任何攻擊。這種動態變化印證了系統確實被引導至誠實節點主導的穩定均衡:當惡意節點權益下降後,其被選入委員會的機率降低,獲得獎勵的機會減少,恢復優勢的難度隨之提高,正反饋循環被有效切斷。

\subsection{攻擊偵測與懲罰效果分析}
\label{sec:attack_frequency}

權益動態的變化最終反映在攻擊事件的發生頻率與處置結果上。需要特別強調的是,本實驗採用最壞情況分析原則,假設攻擊者在每次獲得委員會控制權時都會不顧後果地發動攻擊,藉此驗證防禦機制在極端條件下的偵測與懲罰能力。在實際運作中,理性攻擊者若預見攻擊必然被偵測並遭受罰沒,將選擇不發動攻擊以保全質押資產,因此本節所記錄的攻擊次數代表防禦機制所成功處置的最壞情況,而非系統正常運作下預期會遭遇的攻擊頻率。

表 \ref{tab:attack_comparison} 彙整了兩種架構在 2000 輪長期實驗中的攻擊事件統計。BlockDFL 共記錄了 107 次委員會佔領攻擊事件,平均約每 19 輪發生一次,其中 18 次為戰略性餓死、89 次為全棧投毒,且全部未受任何經濟制裁。挑戰增強型委員會架構則僅記錄 5 次攻擊事件,包含 2 次戰略性餓死與 3 次全棧投毒,且這 5 次攻擊全部被成功偵測並執行罰沒制裁,沒有任何惡意行為能夠逃脫經濟懲罰。

\begin{table}[htbp]
    \centering
    \caption{攻擊事件統計對比(2000 輪長期實驗)}
    \label{tab:attack_events_stats}
    \renewcommand{\arraystretch}{1.3}
    \begin{tabular}{|l|c|c|}
        \hline
        \textbf{評估指標} & \textbf{BlockDFL} & \textbf{CACA} \\
        \hline
        攻擊總次數 & 107 & 5 \\
        \hline
        戰略性餓死 & 18 & 2 \\
        \hline
        全棧投毒 & 89 & 3 \\
        \hline
        成功偵測並罰沒 & 0 & 5 (100\%) \\
        \hline
        最終權益比值 & 1.30 & 0.37 \\
        \hline
    \end{tabular}
\end{table}

攻擊發生次數從 107 次大幅降至 5 次,這種數量級差異源於兩個相互強化的因素。其一是罰沒事件所造成的權益削減效應:惡意節點遭受罰沒後權益大幅縮水,其在後續輪次被選入委員會的機率隨之降低,從根本上減少了攻擊機會的出現頻率。其二是權益分佈改變對委員會組成機率的影響:隨著惡意節點相對權益佔比下降,即使被選入委員會,要同時獲得超過三分之二席位的難度也顯著提高。更關鍵的觀察在於,本研究方法中每一次攻擊嘗試都被完整偵測並受到懲罰,攻擊者無法從任何一次惡意行為中獲得淨收益,這與 BlockDFL 中攻擊者能夠持續累積不當利益的情況形成鮮明對比。

\subsection{系統穩定性分析}
\label{sec:system_stability}

攻擊頻率的降低與攻擊行為的有效懲罰自然轉化為整體系統穩定性的提升。為量化攻擊事件對系統運作造成的實質衝擊,本研究定義「最低不可用率」作為評估指標,衡量系統因遭受攻擊而處於效能顯著下降狀態的時間比例。根據實驗觀測,每次全棧投毒攻擊會導致模型準確率急劇下降,聯邦學習的自我修復機制通常需要約 5 至 25 輪才能使效能恢復正常。採用最保守的 5 輪恢復期估計,BlockDFL 因 89 次全棧投毒攻擊累積產生至少 445 輪的效能下降期間,對應約 22.3\% 的最低不可用率;本研究方法憑藉僅 3 次全棧投毒的記錄,將最低不可用率有效控制在 0.75\% 以下。

\begin{figure}[htbp]
    \centering
    \includegraphics[width=0.85\textwidth]{figures/experiments/mnist_results_convergence.png}
    \caption{模型準確率收斂曲線比較}
    \label{fig:convergence_curve}
\end{figure}

在模型收斂品質方面,兩種架構最終均能達成相近的準確率水平,這一結果符合預期並印證了本研究的設計哲學。如圖 \ref{fig:convergence_curve} 所示,BlockDFL 的準確率曲線在多個時間點出現明顯的突發性下降與後續緩慢回升,對應全棧投毒攻擊的瞬時衝擊與自我修復過程;本研究方法的準確率曲線則呈現更為平滑穩定的上升趨勢,反映較低攻擊頻率所帶來的訓練過程連續性優勢。兩種架構最終準確率相近這一現象,恰好印證了第 \ref{chap:framework} 章所闘述的核心觀點:聯邦學習的自癒特性意味著偶發性的模型偏差能夠被後續正常訓練自然修復,因此單純關注模型品質指標並不足以全面評估系統安全性。真正需要防禦的威脅是攻擊者透過權益累積機制逐步擴大影響力、最終奪取系統治理權的長期戰略性威脅,而本研究方法透過經濟懲罰機制有效阻斷了這種權益累積的正反饋循環。

\section{本章小結}
\label{sec:eval_summary}

本章透過系統性的最壞情況實驗分析,驗證了挑戰增強型委員會架構的防禦效能。2000 輪長期實驗的結果顯示,所有 5 次攻擊嘗試均被成功偵測並執行罰沒制裁,偵測率達到 100\%,沒有任何惡意行為能夠逃脫經濟懲罰。這項發現的深層意涵在於:理性攻擊者若能預見攻擊必然被偵測並遭受損失,將選擇不發動攻擊以保全質押資產,系統因此能在實務運作中自然趨向穩定均衡,維持第 \ref{sec:efficiency_analysis} 節所分析的 $O(c^2)$ 效率水平。

在經濟安全性機制的核心驗證方面,權益動態分析揭示了兩種架構的本質差異。BlockDFL 中惡意節點的權益比值穩定收斂至 1.3,這種優勢在整個實驗期間持續存在並轉化為 107 次未受懲罰的攻擊事件。本研究方法則透過五次階梯式的權益下降將惡意節點的權益比值從 1.0 逐步壓制至 0.37,每一次罰沒都進一步削弱了攻擊者組織後續攻擊的能力,第 1332 輪的最後一次罰沒之後再無任何成功攻擊的記錄。這些實驗結果從實證角度印證了第 \ref{sec:incentive_mechanism} 節的理論分析,證明經濟懲罰機制確實能夠重塑攻擊者的決策空間,使誠實行為成為長期均衡下的最優策略選擇。

\end{ZhChapter}