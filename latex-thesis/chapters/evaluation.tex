\begin{ZhChapter}

\chapter{實驗評估 (Experimental Evaluation)}
\label{chap:evaluation}

本章旨在透過系統性的實驗設計,驗證挑戰增強型委員會架構在應對漸進式委員會佔領攻擊時的防禦效能。有別於傳統聯邦學習研究以模型準確率作為主要評估指標,本研究的核心關注點在於經濟安全性機制能否有效重塑理性攻擊者的決策空間,使得誠實行為成為長期均衡下的最優策略。具體而言,實驗設計圍繞三個層次遞進的研究問題展開:罰沒機制能否打破惡意節點的權益累積循環、經濟威懾能否將攻擊頻率抑制至可忽略的水平、以及上述防禦效果能否在不犧牲模型品質的前提下達成。透過對權益動態演化、攻擊成功頻率與系統可用性等多維度指標的追蹤分析,本章將從實證角度檢驗第 \ref{chap:framework} 章所提出的理論主張。

\section{實驗設置}
\label{sec:eval_setup}

\subsection{資料集與模型配置}

本研究採用 MNIST 手寫數字資料集作為聯邦學習任務的基準測試平台,該資料集包含 60,000 張訓練影像與 10,000 張測試影像,每張影像為 28×28 像素的灰階圖像,對應 0 至 9 共十個數字類別。選擇此資料集的考量在於其廣泛的可比較性,MNIST 已成為聯邦學習安全性研究的標準基準,採用相同的測試平台有助於與現有文獻進行公平的效能對照。模型架構方面,本研究採用一個標準的卷積神經網路,該網路包含兩個卷積層與兩個全連接層,其參數規模足以展現聯邦學習的典型特徵,同時保持實驗的運算可行性。

在資料分佈的設置上,本研究採用獨立同分佈的配置方式,將訓練資料均勻地隨機分配給所有客戶端。這項設計選擇並非出於簡化實驗的考量,而是基於本研究防禦機制的架構特性所做的理論性決定。如第 \ref{chap:framework} 章所闘述,挑戰增強型委員會架構的防禦機制運作於共識層而非資料層,其核心功能在於透過經濟懲罰來威懾惡意的委員會行為,而攻擊能否成功取決於委員會的組成結構,與底層的資料分佈特性相互獨立。換言之,無論資料呈現獨立同分佈或高度異質的 Non-IID 分佈,攻擊者控制委員會的機率計算以及罰沒機制的觸發條件都維持不變。資料分佈影響的是攻擊成功後對模型造成的傷害程度,而非攻擊本身能否被阻止。這種共識層防禦與資料層特性的解耦,正是本架構相對於 Krum、Trimmed Mean 等資料層防禦方法的結構性優勢,後者的防禦效果會隨著資料異質性的增加而顯著衰減。基於此理論分析,IID 環境已足以驗證經濟安全性機制的核心效能,而無需額外引入 Non-IID 變因。

\subsection{基準方法與攻擊場景}

為了公平評估本研究所提出方法的效能,實驗將挑戰增強型委員會架構與 BlockDFL 進行對照比較。BlockDFL 代表了當前基於委員會機制的區塊鏈聯邦學習系統的主流設計,其採用固定規模的驗證委員會並依賴誠實多數假設來確保系統安全性。在本實驗中,兩種方法均採用相同的委員會規模 $C=7$,這是 BlockDFL 原始論文中建議的配置,旨在效率與基本安全性之間取得平衡。共識機制方面,兩者皆採用 $2/3$ 的 BFT 共識門檻,意即提案必須獲得超過三分之二委員會成員的同意方能通過。本研究方法與 BlockDFL 的關鍵差異在於引入了事後挑戰機制:在正常運作情況下,系統採用即時執行模式以維持高效率;當任何節點偵測到異常時,可透過質押押金發起挑戰,觸發全網仲裁流程並對確認的惡意行為執行罰沒。

攻擊場景的設計嚴格遵循第 \ref{chap:threat-model} 章所定義的漸進式委員會佔領攻擊模型。攻擊者採取理性的兩階段策略:在潛伏階段,攻擊者控制的節點完全模仿誠實行為,正常參與訓練並提交高品質的模型更新,目的在於累積權益並建立信譽,為後續攻擊奠定資源基礎;一旦攻擊者在某輪委員會選舉中獲得超過 $2/3$ 的席位,便立即進入佔領階段並根據當時的控制情況啟動攻擊。佔領階段的攻擊行為分為兩種場景:當攻擊者僅控制委員會多數但未同時控制聚合者時,採取戰略性餓死策略,系統性地拒絕包含誠實節點更新的提案,藉此獨佔該輪獎勵並阻止誠實節點的權益增長;當攻擊者同時控制委員會與聚合者時,則採取更激進的全棧投毒策略,直接將包含標籤翻轉等惡意內容的模型更新強制寫入區塊鏈。這兩種攻擊場景分別對應了經濟層面的治理操縱與技術層面的模型破壞,全面涵蓋了理性攻擊者可能採取的策略空間。

\subsection{實驗參數配置}

表 \ref{tab:exp_params} 彙整了本研究實驗所採用的系統參數配置。這些參數的設定參考了 BlockDFL \cite{qin2024blockdfl} 等主流區塊鏈聯邦學習研究的標準配置,以確保實驗結果的可比較性與可重現性。

\begin{table}[htbp]
    \centering
    \caption{實驗參數配置}
    \label{tab:exp_params}
    \renewcommand{\arraystretch}{1.3}
    \begin{tabular}{|l|l|}
        \hline
        \textbf{參數名稱} & \textbf{設定值} \\
        \hline
        訓練輪數 & $R = 300$(基礎實驗)/ $R = 2000$(長期實驗) \\
        \hline
        驗證者池規模 & $N = 100$ \\
        \hline
        委員會大小 & $C = 7$ \\
        \hline
        惡意節點數量 & $M = 30$(初始權益佔比 30\%) \\
        \hline
        初始權益分配 & 所有節點均分配 100 單位 \\
        \hline
        角色配置 & 聚合者 4 位,其餘為更新提供者 \\
        \hline
        每輪獎勵 & 驗證者 1.0,聚合者 1.0,更新提供者 0.05 \\
        \hline
        罰沒規則 & 挑戰成功時全額沒收惡意節點質押 \\
        \hline
        學習率 & $\eta = 0.01$(衰減率 0.99) \\
        \hline
        本地訓練 & 每輪 1 個 epoch,批次大小 32 \\
        \hline
    \end{tabular}
\end{table}

在參數設定的考量上,惡意節點初始佔比設定為 30\% 代表了一個相當嚴峻的威脅情境,這個比例接近大多數拜占庭容錯系統所能容忍的理論上限。委員會規模 $C=7$ 的選擇則反映了效率與安全性的權衡,根據第 \ref{sec:committee-size-security} 節的超幾何分佈分析,在此配置下惡意節點獲得委員會超過 $2/3$ 席位的單輪機率約為 2.4\%,這個機率雖然不高但在數百輪的訓練過程中足以產生多次成功攻擊的機會,為驗證防禦機制提供了充分的測試場景。獎勵機制的設定遵循 BlockDFL 的原始設計,驗證者與聚合者因承擔較高的運算與協調責任而獲得較高的單輪獎勵,更新提供者則獲得與其貢獻相稱的基礎獎勵,這種差異化的獎勵結構是權益累積動態的重要驅動因素。

\section{實驗結果與分析}
\label{sec:experimental_results}

\subsection{經濟安全性的核心驗證:權益動態分析}
\label{sec:stake_dynamics}

本研究的核心論點在於,挑戰增強型委員會架構能夠透過經濟懲罰機制打破漸進式委員會佔領攻擊所依賴的正反饋循環。為驗證此論點,本節首先分析兩種架構下惡意節點與誠實節點之間的權益演化動態,這是評估經濟安全性機制有效性的最直接指標。圖 \ref{fig:stake_evolution} 呈現了 300 輪訓練過程中雙方權益比例的變化軌跡,而圖 \ref{fig:stake_2000_sub} 則將觀察期延伸至 2000 輪以檢驗長期效應。

\begin{figure*}[htbp]
    \centering
    \begin{subfigure}[b]{0.48\textwidth}
        \centering
        \includegraphics[width=\textwidth]{figures/experiments/mnist_results_stack_comparison.png}
        \caption{BlockDFL 與 CACA 權益演化對比}
        \label{fig:stake_evolution_compare}
    \end{subfigure}
    \hfill
    \begin{subfigure}[b]{0.48\textwidth}
        \centering
        \includegraphics[width=\textwidth]{figures/experiments/mnist_results_stack_comparison_2000_round.png}
        \caption{2000 輪長期模擬}
        \label{fig:stake_2000_sub}
    \end{subfigure}
    \caption{權益演化動態分析。(a) 為 300 輪基礎實驗的對比,(b) 為 2000 輪長期模擬結果。}
    \label{fig:stake_evolution}
\end{figure*}

在缺乏挑戰機制的 BlockDFL 架構中,實驗數據清楚揭示了權益失衡的系統性趨勢。追蹤原始權益數據可以發現,惡意節點的平均權益在訓練過程中穩定維持在誠實節點的 1.1 至 1.2 倍,這種優勢並非偶然的統計波動,而是源於系統獎勵機制的結構性特徵。由於聚合者與驗證者角色所獲得的單輪獎勵顯著高於一般更新提供者,而角色分配機制又與節點權益直接掛鉤,一旦某些節點透過早期的隨機優勢累積了較高權益,其在後續輪次被選為高獎勵角色的機率便隨之提升,進而獲得更多獎勵並進一步鞏固優勢地位。這種「權益優勢帶來更多獎勵,更多獎勵鞏固權益優勢」的正反饋循環,正是第 \ref{chap:threat-model} 章所描述的漸進式委員會佔領攻擊得以成立的經濟基礎。值得注意的是,惡意節點的權益優勢並未呈現指數級的無限膨脹,而是穩定在 1.1 至 1.2 倍的區間,這是因為攻擊者無法完全操控角色分配的隨機性,且系統中仍有部分誠實節點會在各輪次中獲得獎勵。然而,即便是這種「有上界」的優勢,在長期運作中仍構成顯著的治理威脅。

挑戰增強型委員會架構的權益演化則呈現出截然不同的動態特徵。觀察圖 \ref{fig:stake_evolution_compare} 可以發現,惡意節點的平均權益並非呈現平滑的線性變化,而是展現出明顯的「台階式下降」模式。這種階梯狀的軌跡對應了異步審計機制觸發罰沒的具體時點:每當惡意節點嘗試利用委員會控制權發動攻擊時,挑戰者便會偵測到聚合結果與正確 Krum 運算之間的不一致,隨即發起挑戰並觸發全網仲裁,一旦惡意行為被確認,涉事節點的全額質押將被沒收。在 300 輪的觀察期內,這種罰沒事件在第 90 輪與第 229 輪各發生一次,導致惡意節點的平均權益在實驗結束時降至誠實節點的 0.56 倍。這個數值的意涵相當深遠:惡意節點不僅未能如在 BlockDFL 中那樣建立起權益優勢,反而陷入了相對於誠實節點的劣勢地位,其在未來委員會選舉中獲得多數席位的機率將顯著降低。

將觀察期延伸至 2000 輪的長期實驗進一步印證了經濟懲罰機制的威懾效果。如圖 \ref{fig:stake_2000_sub} 所示,在 BlockDFL 架構中,惡意節點的權益優勢在約 250 輪後便穩定維持在誠實節點的 1.2 倍左右,這種「強者恆強」的馬太效應使得攻擊者能夠長期維持其治理影響力。相較之下,本研究方法中的惡意節點權益軌跡在第 909 輪出現了決定性的轉折:該輪次發生的罰沒事件導致惡意節點的平均權益瞬間暴跌至誠實節點的 22.6\%,這次經濟重創產生了持久的邊緣化效果。在隨後長達 1091 輪的觀察期內,惡意節點因權益基礎過於薄弱而徹底喪失了競爭委員會多數席位的能力,再也未能成功發動任何一次佔領攻擊。這項實驗結果有力地證實了經濟安全性機制的核心效能:罰沒機制能夠將一次性的攻擊失敗轉化為永久性的治理排除,從根本上打破了漸進式委員會佔領攻擊所依賴的正反饋循環。

\subsection{攻擊抑制效果:頻率與成功率分析}
\label{sec:attack_frequency}

權益動態的變化最終會反映在攻擊成功頻率這一更直觀的安全性指標上。本節分析兩種架構在不同實驗規模下的攻擊事件記錄,以量化經濟威懾對攻擊者行為的實質影響。

在 300 輪的基礎實驗中,BlockDFL 共遭受了 10 次成功的委員會佔領攻擊,平均約每 30 輪便會發生一次安全事件。這個頻率與第 \ref{sec:committee-size-security} 節的理論分析大致吻合:在惡意節點佔全網 30\% 且委員會規模為 7 的配置下,單輪攻擊成功的機率約為 2.4\%,300 輪訓練中的期望攻擊次數約為 7.2 次,實際觀測的 10 次略高於期望值但仍在合理的統計波動範圍內。這些攻擊事件中,部分為戰略性餓死類型,攻擊者透過拒絕誠實提案來獨佔獎勵;部分為全棧投毒類型,攻擊者在同時控制委員會與聚合者的情況下直接注入標籤翻轉的惡意更新。後者對模型造成的傷害尤為顯著,例如第 68 輪的全棧投毒攻擊曾導致模型準確率從正常水平瞬間崩潰至 9.55\%,展現了單次成功攻擊可能造成的災難性後果。

相較之下,挑戰增強型委員會架構在相同的 300 輪實驗中僅遭受 2 次成功攻擊,攻擊頻率降低了 80\%。這項改善並非源於委員會組成機制的改變(兩種架構採用相同的權益加權選舉機制),而是來自兩個相互關聯的因素。其一是經濟威懾的直接效應:理性的攻擊者在評估攻擊決策時會將罰沒風險納入成本計算,當預期的懲罰損失超過潛在收益時,攻擊便不再是理性選擇,即使攻擊者恰好獲得了委員會的控制權,也可能選擇放棄攻擊以保全質押資產。其二是權益邊緣化的間接效應:如前節所述,罰沒機制導致惡意節點的權益持續萎縮,這直接降低了其在後續輪次被選入委員會的機率,從根本上減少了攻擊機會的出現頻率。

2000 輪的長期實驗更戲劇性地展現了上述效應的累積作用。在 BlockDFL 架構中,惡意節點在整個實驗期間共成功發動了 84 次委員會佔領攻擊,平均每 24 輪便會發生一次安全事件。這個高頻率的攻擊成功記錄意味著系統長期處於不穩定狀態,惡意節點能夠反覆利用權益優勢來操控委員會決策。本研究方法的表現則形成鮮明對比:2000 輪中僅發生 5 次成功攻擊,且這些攻擊全部集中在實驗的前 909 輪。如前節所述,第 909 輪的罰沒事件導致惡意節點權益暴跌至誠實節點的 22.6\%,在此之後的 1091 輪中,攻擊者再也未能成功佔領委員會,形成了事實上的永久性治理排除。這項結果的意涵超越了單純的頻率降低:它證明了經濟安全性機制能夠在長期演化中將系統導向「誠實者主導」的穩定均衡,使得攻擊不僅在單次決策中不划算,在長期賽局中更是不可行。

\subsection{系統穩定性與模型品質的連帶效益}
\label{sec:system_stability}

攻擊頻率的降低會自然轉化為系統穩定性的提升,這種連帶效益雖然並非本研究的核心貢獻,但對於評估防禦機制的實務價值仍具重要參考意義。本節從系統可用性與模型收斂兩個面向分析這些連帶效益。

\begin{figure*}[htbp]
    \centering
    \begin{subfigure}[b]{0.48\textwidth}
        \centering
        \includegraphics[width=\textwidth]{figures/experiments/mnist_results_convergence.png}
        \caption{模型準確率收斂曲線}
        \label{fig:convergence_curve}
    \end{subfigure}
    \hfill
    \begin{subfigure}[b]{0.48\textwidth}
        \centering
        \includegraphics[width=\textwidth]{figures/experiments/mnist_noniid_results_convergence.png}
        \caption{攻擊事件對準確率的瞬時衝擊}
        \label{fig:attack_impact}
    \end{subfigure}
    \caption{模型效能分析。(a) 為整體收斂趨勢比較,(b) 展示攻擊造成的準確率崩潰與恢復過程。}
    \label{fig:convergence_combined}
\end{figure*}

為量化攻擊對系統運作的實質衝擊,本研究定義「最低不可用率」作為評估指標,該指標衡量系統因攻擊而處於效能崩潰狀態的時間比例。根據實驗觀測,每次成功的全棧投毒攻擊會導致模型準確率急劇下降,而聯邦學習的自我修復機制需要約 5 至 25 輪才能使模型恢復至正常水平。採用最為保守的 5 輪恢復期估計,BlockDFL 在 300 輪實驗中因 10 次攻擊而產生至少 50 輪的不可用時間,對應約 16.7\% 的最低不可用率;若採用實際觀測中較常見的 15 輪恢復期,不可用率將攀升至 50\% 左右。本研究方法憑藉僅 2 次的攻擊記錄,將最低不可用率控制在 3.3\% 以下。這種差異在需要持續運作的關鍵任務場景中具有重要意義:醫療診斷、自動駕駛決策或金融風險評估等應用不僅要求最終模型的準確性,更要求訓練過程中模型狀態的穩定性,頻繁的效能崩潰即使最終能夠恢復,仍會嚴重影響系統的可靠性與可用性。

在模型收斂品質方面,兩種架構在 300 輪實驗結束時達成了相近的最終準確率,本研究方法為 98.63\%,BlockDFL 為 98.26\%,差異僅約 0.37 個百分點。這個結果並不令人意外,因為聯邦學習本身具備顯著的自我修復能力:只要攻擊者無法持續佔領委員會,後續輪次中來自誠實節點的正確更新便能逐步抵銷惡意梯度所造成的偏差,使模型重新收斂至正確方向。圖 \ref{fig:attack_impact} 清楚展示了這種自我修復的動態過程:BlockDFL 在第 68 輪遭受全棧投毒攻擊後準確率瞬間崩潰至 9.55\%,但在隨後數輪中便開始快速回升,約 15 輪後已恢復至接近正常的水平。這種自癒特性解釋了為何兩種架構的最終準確率差異不大,同時也凸顯了本研究關注經濟安全性而非單純模型品質的理論動機:在聯邦學習場景中,偶發的模型偏差可以被自然修復,真正需要擔憂的是攻擊者透過權益累積逐步掌控系統治理權的長期威脅。

值得強調的是,最終準確率的相近並不意味著兩種架構的效能等價。如前所述,BlockDFL 在訓練過程中經歷了遠為頻繁的效能崩潰與恢復週期,這種「過程不穩定性」在實務應用中可能造成嚴重後果。此外,聯邦學習的自癒能力存在其固有的界限:當攻擊頻率過高導致恢復期與下一次攻擊重疊時,模型可能陷入長期的低位震盪而無法有效累積學習成果。本研究方法透過將攻擊頻率抑制在極低水平,從根本上避免了這種連鎖惡化的風險,確保了模型能夠在絕大多數訓練時間內維持穩定的學習軌跡。

\section{本章小結}
\label{sec:eval_summary}

本章透過系統性的實驗設計,從權益動態、攻擊頻率與系統穩定性三個層面驗證了挑戰增強型委員會架構的防禦效能。實驗結果有力地支持了本研究的核心論點:經濟安全性機制能夠有效替代傳統的門檻安全性假設,在不擴大委員會規模的前提下達成等效甚至更優的安全保證。

在經濟安全性的核心驗證方面,權益動態分析揭示了兩種架構的本質差異。BlockDFL 中惡意節點能夠穩定維持 1.1 至 1.2 倍的權益優勢,這種優勢在 2000 輪的長期實驗中持續存在並轉化為對系統治理權的顯著影響力。本研究方法則透過罰沒機制成功打破了這種正反饋循環:惡意節點的權益軌跡呈現「台階式下降」的特徵,在第 909 輪的關鍵轉折點後權益暴跌至誠實節點的 22.6\%,達成了事實上的永久性治理排除。這一結果從實證角度印證了第 \ref{sec:incentive_mechanism} 節的理論分析,證明經濟懲罰機制確實能夠重塑理性攻擊者的決策空間,使得誠實行為成為長期均衡下的最優策略。

在攻擊抑制效果方面,實驗數據展現了經濟威懾的實質影響力。300 輪實驗中攻擊成功次數從 BlockDFL 的 10 次降至本研究方法的 2 次,降幅達 80\%;2000 輪長期實驗中的對比更為懸殊,BlockDFL 的 84 次對比本研究方法的僅 5 次,且後者的攻擊全部集中在前 909 輪,此後再無成功攻擊的記錄。這種攻擊頻率的大幅降低不僅源於經濟威懾的直接效應,更源於權益邊緣化所產生的間接效應:惡意節點因質押萎縮而逐漸喪失競爭委員會席位的能力,攻擊機會本身便大幅減少。

在系統穩定性的連帶效益方面,攻擊頻率的降低自然轉化為可用性的提升,最低不可用率從 BlockDFL 的約 16.7\% 降至本研究方法的 3.3\% 以下。同時,兩種架構的最終模型準確率相近(98.63\% 對 98.26\%),這證實了經濟安全性機制在強化防禦的同時並未犧牲聯邦學習的核心功能。這一發現呼應了第 \ref{chap:framework} 章的設計哲學:聯邦學習本身具備自我修復能力,真正需要防禦的不是偶發的模型偏差,而是攻擊者透過權益累積逐步奪取治理權的長期威脅。

綜上所述,實驗結果有力地驗證了挑戰增強型委員會架構的核心價值主張。透過將安全性保障從機率運算轉移至激勵相容,本架構成功解耦了安全性與委員會規模之間的傳統強耦合關係,開闢了一條兼顧效率與安全的技術路徑。這種基於經濟理性的自我維持安全機制,為大規模去中心化機器學習平台的實際部署奠定了可行的技術基礎。

\end{ZhChapter}