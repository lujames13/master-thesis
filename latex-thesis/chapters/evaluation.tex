\begin{ZhChapter}

\chapter{實驗評估}
\label{chap:evaluation}

本章透過系統性的實驗設計,驗證審計驅動型委員會 BlockDFL 在應對漸進式委員會佔領攻擊時的防禦效能。有別於傳統聯邦學習安全性研究將模型準確率視為首要評估指標的慣例,本研究的核心關注點在於經濟安全性機制能否有效遏止理性攻擊者的惡意行為,進而維護系統的長期治理穩定性。這種評估視角的轉移根植於第 \ref{chap:framework} 章所確立的設計哲學:當防禦機制的目標從「防止攻擊發生」轉向「確保攻擊無利可圖」時,衡量防禦效能的指標也應當相應地從模型品質轉向攻擊者的經濟決策空間。

為提供最嚴格的效能驗證,本章實驗採用「最壞情況分析」的設計哲學,假設攻擊者完全不顧經濟理性而執意發動所有可能的攻擊,藉此檢驗防禦機制在極端條件下的偵測與懲罰能力。這種實驗設計蘊含著一條重要的推論鏈:若機制在最壞情況下仍能確保每一次攻擊都被偵測並遭受懲罰,則理性攻擊者在事前評估預期收益時必然得出負值結論,從而自發地選擇不發動攻擊,系統便能在實務運作中自然趨向穩定均衡。換言之,本章所呈現的攻擊事件統計並非系統正常運作下預期會遭遇的攻擊頻率,而是防禦機制所能承受的最嚴苛壓力測試。

\section{實驗設置}
\label{sec:eval_setup}

\subsection{實驗配置概述}

本研究選用 MNIST 手寫數字資料集作為聯邦學習任務的測試平台,採用包含兩個卷積層與兩個全連接層的標準卷積神經網路作為訓練模型。訓練資料以獨立同分布(IID)的方式均勻分配給所有參與的客戶端節點,此項設計選擇並非出於簡化實驗的考量,而是源於本研究防禦機制的本質特性。如第 \ref{chap:framework} 章所闡述,審計驅動型委員會 BlockDFL 的核心防禦運作於共識層而非資料層,其功能在於透過經濟懲罰威懾惡意的委員會行為。攻擊能否成功取決於委員會組成與共識決策過程中的投票結果,而這些因素與底層訓練資料的統計分布特性相互獨立。無論資料呈現獨立同分布或高度異質的 Non-IID 特性,委員會成員是否選擇誠實投票這一決策,在邏輯上並不受資料分布的影響,因此 IID 設定足以驗證共識層防禦機制的有效性。

為確保評估結果的公平性與可比較性,本實驗將審計驅動型委員會 BlockDFL 與原始 BlockDFL 進行系統性對照比較,兩種架構均採用相同的委員會規模 $C=7$。攻擊場景嚴格遵循第 \ref{chap:threat-model} 章所定義的漸進式委員會佔領攻擊模型,攻擊者採取理性的兩階段策略:在潛伏階段完全模仿誠實行為以累積權益,當成功獲得委員會超過三分之二席位時進入佔領階段,根據其對系統各組件的控制範圍選擇戰略性餓死或全棧投毒策略。表 \ref{tab:exp_params} 彙整了本研究實驗所採用的完整系統參數配置。

\begin{table}[htbp]
    \centering
    \caption{實驗參數配置}
    \label{tab:exp_params}
    \renewcommand{\arraystretch}{1.3}
    \begin{tabular}{|l|l|}
        \hline
        \textbf{參數名稱} & \textbf{設定值} \\
        \hline
        訓練輪數 & $R = 300$(基礎實驗)/ $R = 2000$(長期實驗) \\
        \hline
        驗證者池規模 & $N = 100$ \\
        \hline
        委員會大小 & $C = 7$ \\
        \hline
        惡意節點數量 & $M = 30$(初始權益佔比 30\%) \\
        \hline
        每輪獎勵 & 驗證者 1.0,聚合者 1.0,更新提供者 0.05 \\
        \hline
        罰沒規則 & 挑戰成功時全額沒收惡意節點質押 \\
        \hline
    \end{tabular}
\end{table}

惡意節點初始佔比設定為 $30\%$ 代表了相當嚴峻的威脅情境,此比例已接近大多數拜占庭容錯系統所能容忍的理論上限。根據第 \ref{sec:committee-size-security} 節的超幾何分布分析,在此條件下惡意節點於單輪中獲得委員會超過三分之二席位的機率約為 $2.4\%$。這一機率乍看之下並不顯著,但考量到聯邦學習訓練通常需要經歷數百甚至數千輪迭代,累積下來足以產生數十次攻擊機會窗口,為驗證防禦機制在長期時間跨度中的持續效能提供了充分且嚴格的測試場景。

\section{實驗結果與分析}
\label{sec:experimental_results}

本節依循「機制驗證、長期生存、服務品質」的層層遞進邏輯,從三個逐步擴大的觀察尺度呈現實驗結果。分析首先聚焦於 300 輪基礎實驗中防禦機制的即時響應特性,確認經濟懲罰機制在微觀層面確實能夠靈敏地偵測並懲罰異常行為;繼而將觀察視野延伸至 2000 輪的長期模擬,驗證短期有效的機制是否足以引導系統趨向長期穩定的治理均衡;最終從系統服務品質的角度檢驗上述安全性保障是否以犧牲聯邦學習的核心效能為代價。這種從微觀機制到宏觀治理再到全局效能的遞進結構,旨在完整呈現經濟安全性機制在不同時間尺度與觀察維度上的防禦效能。

\subsection{微觀機制的即時驗證}
\label{sec:micro_mechanism}

驗證經濟懲罰機制有效性的第一步,在於確認其能否在攻擊發生的當下產生即時且明確的響應。300 輪基礎實驗為此提供了理想的觀察窗口,因為在這一相對短暫的時間跨度內,每一次罰沒事件的觸發條件、執行過程與權益衝擊都能被精確地追蹤與解讀,而不至於被長期累積的統計噪音所模糊。

\subsubsection{雙軌分歧:兩種架構的早期權益軌跡}

權益比值作為衡量系統治理結構健康程度的核心指標,定義為惡意節點平均權益除以誠實節點平均權益,其數值變化直接反映了第 \ref{chap:threat-model} 章所定義的「領先者優勢」是否正在被建立或瓦解。當此比值持續高於 1.0 時,意味著惡意節點正在累積治理層面的結構性優勢,其在委員會選舉中的入選機率將隨之提升,系統面臨逐步被滲透的風險;反之,當此比值被壓制至 1.0 以下時,則表明經濟激勵結構已成功將惡意節點邊緣化,使其在後續輪次中愈加難以組織有效的委員會佔領攻擊。基於此指標,本節透過 300 輪基礎實驗追蹤兩種架構下權益分布的演化軌跡,並結合具體的攻擊事件與罰沒資料,揭示經濟安全性機制在實驗早期即展現出的防禦效能。

\begin{figure}[htbp]
    \centering
    \includegraphics[width=0.85\textwidth]{figures/experiments/mnist_results_stack_comparison.png}
    \caption{BlockDFL 與 AC-BlockDFL 權益演化對比(300 輪基礎實驗)}
    \label{fig:stake_evolution_300}
\end{figure}

如圖 \ref{fig:stake_evolution_300} 所示,兩種架構的權益軌跡在實驗初期便呈現出截然相反的演化方向。在缺乏挑戰機制的 BlockDFL 架構中,300 輪實驗期間共發生 10 次委員會佔領攻擊事件,其中 4 次屬於攻擊者僅控制驗證委員會但未掌握聚合者角色的戰略性餓死攻擊,另外 6 次則是攻擊者同時控制委員會與聚合者的全棧投毒攻擊。這些攻擊事件平均約每 30 輪發生一次,而由於 BlockDFL 缺乏任何事後追責機制,全部 10 次攻擊均未受到經濟制裁,攻擊者得以在每次成功佔領中不受阻礙地獲取不當獎勵。這一資料為第 \ref{sec:pcca} 節的理論分析提供了直接的實證支持:獎勵機制所內建的正反饋特性確實使惡意節點的權益比值從初始的 1.0 開始穩定攀升,呈現出「權益優勢帶來更多獎勵,更多獎勵鞏固權益優勢」的累積效應。

\subsubsection{罰沒機制的運作實證:兩次懲罰事件的詳細剖析}

審計驅動型委員會 BlockDFL 在相同的 300 輪觀察期間內僅記錄到 2 次攻擊事件,分別發生在第 90 輪與第 229 輪,且兩次均為全棧投毒攻擊。相較於 BlockDFL 的 10 次未受懲罰的攻擊,這一資料差異的背後蘊含著罰沒機制對攻擊者決策空間的深層重塑。以下將逐一剖析這兩次懲罰事件的具體過程與經濟效果,藉此展示「偵測、舉證、懲罰」閉環在實務場景中的完整運作邏輯。

第一次攻擊發生在第 90 輪,此時惡意節點經過前 89 輪的誠實參與已累積了一定程度的權益優勢,其平均權益比值攀升至 1.25,意味著惡意節點的平均權益已高出誠實節點 $25\%$。在權益加權的隨機選舉機制下,這種優勢轉化為更高的委員會入選機率,最終在第 90 輪使 5 名惡意節點同時被選入規模為 7 的驗證委員會,超過了三分之二的控制閾值。由於攻擊者在該輪同時掌控了聚合者角色,遂發動全棧投毒攻擊,將惡意的模型更新寫入區塊鏈。然而,這一攻擊行為隨即被挑戰者偵測:挑戰者從 IPFS 下載該輪所有聚合提案的完整內容,在本地重新執行 Krum 演算法後發現委員會選定的結果與正確的 Krum 輸出存在明確偏離,遂提交挑戰交易並觸發全網仲裁。仲裁確認惡意行為後,智能合約自動執行罰沒操作,將參與共謀的 5 名惡意委員會成員的全額質押予以沒收。這次罰沒的經濟衝擊極為顯著:惡意節點的平均權益比值從罰沒前的 1.25 驟降至 0.62,在單一事件中便從「領先誠實節點 $25\%$」逆轉為「僅及誠實節點 $62\%$」,權益結構發生了根本性的翻轉。

這一數值變化的幅度值得從機制設計的角度進行深入解讀。5 名參與共謀的惡意委員會成員被全額沒收其自訓練開始以來所累積的全部權益,這意味著他們在前 89 輪誠實參與中透過正常獎勵機制所建立的經濟基礎被一次性清零。相較於該輪攻擊可能獲得的經濟收益上界,即當輪區塊獎勵的總和約 7.0 單位,罰沒所造成的損失遠超過其預期獲利,精確印證了第 \ref{sec:game_theory_analysis} 節博弈論分析所推導的理論預測。更為關鍵的是,這 5 名遭受罰沒的惡意節點並非僅僅損失了當期的經濟利益,而是喪失了在後續所有輪次中參與委員會選舉的權益基礎。由於 BlockDFL 的角色分配機制以權益為權重進行隨機選舉,權益歸零的節點實質上被永久排除在高獎勵角色之外,這種「永久性治理排除」效應遠超單次經濟懲罰的意義,它從系統的參與者結構層面削弱了攻擊者的長期作戰能力。

第二次攻擊發生在第 229 輪,距離第一次罰沒已相隔 139 輪。這一時間間隔本身即反映了第一次罰沒對攻擊能力的有效壓制:在權益比值降至 0.62 的條件下,惡意節點需要經歷顯著更長的等待期,才能在隨機選舉中再度湊齊超過三分之二的委員會席位。值得注意的是,在第 90 輪至第 229 輪的 139 輪間隔中,惡意節點的平均權益比值從罰沒後的 0.62 緩慢回升至 0.70,這種回升源於存活的惡意節點(即未參與第 90 輪共謀而未被罰沒的節點)持續透過誠實參與獲取的正常獎勵。然而,0.70 的回升水平仍顯著低於第一次攻擊前的 1.25,這表明單次罰沒事件對惡意群體的整體權益基數造成了難以完全恢復的結構性損傷。第 229 輪的攻擊同樣由 5 名惡意節點在委員會中形成多數後發動,挑戰者再度成功偵測並觸發罰沒,5 名共謀成員的全額質押被沒收,惡意節點的平均權益比值從 0.70 進一步下降至 0.52。

將兩次罰沒事件的資料進行對比,可以清晰地觀察到一個遞減的攻擊效力模式。第一次罰沒造成的權益比值降幅為 0.63(從 1.25 降至 0.62),第二次罰沒的降幅則縮小為 0.18(從 0.70 降至 0.52),這種降幅的收窄並非意味著罰沒機制的效力在衰減,而是反映了一個更為深層的動態:隨著惡意群體的權益基數在反覆罰沒中持續萎縮,每次被選入委員會的惡意節點所持有的質押金額也相應減少,因此單次罰沒所能沒收的絕對金額自然降低。然而,從攻擊者的視角來看,這種「可罰沒資產的縮減」並不構成任何安慰,因為與之同步縮減的還有其組織後續攻擊的能力:第二次罰沒後 0.52 的權益比值意味著惡意節點的平均權益僅略高於誠實節點的一半,在此條件下要透過隨機選舉同時將 5 名惡意節點送入規模為 7 的委員會,其機率已被壓縮至極低水平。事實上,從第 229 輪罰沒直至 300 輪實驗結束的 71 輪觀察期內,再未發生任何攻擊事件,這一觀測結果為上述分析提供了直接的實證支持。

\subsubsection{模型收斂品質與訓練穩定性}

從模型收斂品質的角度觀察,300 輪實驗結束時 BlockDFL 的模型準確率為 $98.26\%$,而審計驅動型委員會 BlockDFL 達到 $98.63\%$,兩者之間 0.37 個百分點的差距看似微小,但其背後隱含著截然不同的訓練穩定性特徵。BlockDFL 的 6 次全棧投毒攻擊意味著全域模型在 300 輪訓練過程中遭受了 6 次直接的惡意梯度注入,每一次注入都會導致模型準確率的急劇下降與後續數輪的修復過程,這種反覆的「下降與回升」不僅消耗了寶貴的訓練輪次,更使得收斂曲線呈現出顯著的波動特徵。相比之下,審計驅動型委員會 BlockDFL 僅經歷 2 次全棧投毒,模型所遭受的干擾次數減少了三分之二,訓練過程的連續性得到更好的保障,這正是其最終準確率略高的直接原因。

然而更為關鍵的觀察在於,兩種架構最終都能達到 $98\%$ 以上的準確率水平,這一事實恰好印證了第 \ref{chap:framework} 章所闡述的核心設計理念:聯邦學習的迭代訓練特性賦予了系統內在的自癒能力,偶發性的模型偏差能夠被後續正常訓練自然修復,因此單純的模型品質指標並不足以全面評估系統安全性。BlockDFL 雖然在 300 輪後仍達到 $98.26\%$ 的準確率,但其間經歷的 10 次未受懲罰的攻擊已使惡意節點建立起穩固的權益優勢,這種治理層面的失衡在短期內尚未對模型品質造成不可逆的損害,但隨著訓練輪次的延伸,攻擊頻率的持續攀升終將使自癒機制難以承受。這一觀察再次突顯了本研究聚焦於經濟安全性而非模型準確率的研究取向:真正需要防禦的並非偶發的模型偏差,而是攻擊者透過權益累積逐步奪取系統治理權的長期戰略性威脅。

\subsubsection{從短期響應到長期均衡:延伸實驗的必要性}

綜合 300 輪基礎實驗的各項觀測資料,兩種架構在權益分布演化上的對比結果清晰地揭示了經濟安全性機制的效能差異。在審計驅動型委員會 BlockDFL 中,罰沒機制展現出完整且有效的防禦閉環:2 次攻擊全部被成功偵測並執行罰沒,偵測率達  100\%,惡意節點的平均權益比值從首次攻擊前的 1.25 經由兩次階梯式下降被壓制至 0.55,實驗結束時惡意節點的平均權益僅為誠實節點的約一半,沒有任何惡意行為能夠逃脫經濟制裁。與此形成鮮明對比的是,缺乏事後追責機制的 BlockDFL 中,惡意節點在 10 次未受懲罰的攻擊中持續鞏固權益優勢,其平均權益比值在 300 輪後攀升至 1.15,意味著攻擊者的經濟實力已超越誠實節點約 15\%,在委員會選舉中享有更高的入選機率,為後續更頻繁的攻擊創造了結構性條件。兩條權益演化軌跡在方向上的根本分歧,清楚呈現了經濟懲罰機制對治理結構的重塑效應:一方將惡意節點向邊緣化方向驅趕,另一方則任由攻擊者逐步鞏固優勢地位。然而,300 輪的觀察期對於驗證經濟安全性機制的長期效能而言仍顯不足,這是因為 PCCA 攻擊的本質在於其漸進性與累積性,攻擊者能否透過長期的耐心等待逐步恢復被削減的權益,最終突破罰沒機制的持續壓制,是一個只有在更長時間尺度上才能獲得回答的問題。基於此考量,下一節將實驗觀察期延伸至 2000 輪,以完整呈現經濟懲罰機制作為「漸進式淨化」工具的長期運作特性,並驗證系統最終是否能夠收斂至一個誠實節點主導的穩定均衡狀態。

\subsection{宏觀治理的長期均衡}
\label{sec:macro_equilibrium}

短期實驗確認了經濟懲罰機制在微觀層面的即時有效性,但一個更為關鍵的問題隨之浮現:這種有效性能否在長期時間跨度中持續發揮作用,並最終將系統引導至一個攻擊自然消亡的穩定均衡?2000 輪長期模擬實驗正是為回答這一問題而設計,其結果揭示了經濟懲罰機制作為「漸進式淨化」工具的完整動態過程,構成本章實驗分析的核心發現。

\subsubsection{漸進式淨化:從短期懲罰到長期治理逆轉}

\begin{figure}[htbp]
    \centering
    \includegraphics[width=0.85\textwidth]{figures/experiments/mnist_results_stack_comparison_2000_round.png}
    \caption{2000 輪長期模擬下的權益動態演化}
    \label{fig:stake_evolution_2000}
\end{figure}

如圖 \ref{fig:stake_evolution_2000} 所示,將觀察期延伸至 2000 輪後,兩種架構在治理結構層面的本質差異得到了充分的展現。BlockDFL 中惡意節點的權益比值在經歷約 300 輪的初期波動後穩定收斂至 1.3 左右,並在整個實驗期間持續維持在此水平。1.3 這一看似溫和的數值背後隱藏著深刻的治理危機:從超幾何分布的角度分析,權益比值為 1.3 意味著惡意節點在委員會選舉中享有顯著高於其初始比例的入選機率,這種機率優勢在 2000 輪的時間跨度中累積轉化為持續不斷的攻擊能力。更值得關注的是,由於 BlockDFL 缺乏任何事後追責機制,攻擊者從每一次成功佔領中獲取的不當利益反而進一步鞏固了其權益優勢,實證確認了第 \ref{sec:pcca} 節所預測的正反饋循環與常態化治理失衡。

AC-BlockDFL 的長期軌跡則呈現一幅截然不同的圖景。惡意節點的權益比值經歷五次明確的階梯式下降,分別發生在第 15、136、695、815 與 1332 輪,從初始的 1.0 逐步被壓制至最終的 0.37。這一最終數值意味著實驗結束時惡意節點的平均權益僅為誠實節點的三分之一強,相較於 BlockDFL 中惡意節點維持 1.3 倍優勢的情況,兩種架構之間的權益比值差距達到 $1.3 / 0.37 \approx 3.5$ 倍。這種量級的差異並非漸進式的微幅改善,而是代表了系統治理結構的根本性逆轉:從攻擊者主導的失衡狀態轉變為誠實節點佔據明確優勢的健康格局。

\subsubsection{攻擊事件:權益下降的因果證據}

圖 \ref{fig:stake_evolution_2000} 中五次階梯式下降的背後,是五次被完整偵測並懲罰的攻擊事件。表 \ref{tab:attack_events_stats} 彙整了兩種架構在 2000 輪實驗中的攻擊事件統計,為理解權益為何下降提供了直接的因果證據。

\begin{table}[htbp]
    \centering
    \caption{攻擊事件統計對比(2000 輪長期實驗)}
    \label{tab:attack_events_stats}
    \renewcommand{\arraystretch}{1.3}
    \begin{tabular}{|l|c|c|}
        \hline
        \textbf{評估指標} & \textbf{BlockDFL} & \textbf{AC-BlockDFL} \\
        \hline
        攻擊總次數 & $107$ & $5$ \\
        \hline
        戰略性餓死 & $18$ & $2$ \\
        \hline
        全棧投毒 & $89$ & $3$ \\
        \hline
        成功偵測並罰沒 & $0$ & $5$ ($100\%$) \\
        \hline
        最終權益比值 & $1.30$ & $0.37$ \\
        \hline
    \end{tabular}
\end{table}

BlockDFL 共記錄了 $107$ 次委員會佔領攻擊事件,平均約每 $19$ 輪發生一次,其中 $18$ 次為戰略性餓死、$89$ 次為全棧投毒,且全部 $107$ 次攻擊未受任何形式的經濟制裁。AC-BlockDFL 則僅記錄 $5$ 次攻擊事件,包含 $2$ 次戰略性餓死與 $3$ 次全棧投毒,而這 $5$ 次攻擊全部被成功偵測並執行罰沒制裁,偵測率達到 $100\%$。攻擊次數從 $107$ 次降至 $5$ 次,這種超過二十倍的數量級差異源於兩個相互強化的因果機制。第一個機制是罰沒事件對權益基數的直接削減:惡意節點在遭受全額質押罰沒後權益大幅縮水,其在後續輪次中被選入委員會的機率隨之降低,從根本上減少了攻擊機會窗口的出現頻率。第二個機制涉及權益分布變化對委員會控制門檻的影響:即使惡意節點在權益削減後仍被選入委員會,要在 $7$ 名成員中同時佔據 $5$ 名以上席位的難度也因權益比例的下降而顯著提高。兩者共同作用的結果遠超任何單一機制的效果,這種雙重因果結構正是第 \ref{chap:framework} 章博弈論分析中所預測的「永久性治理排除效應」的實證體現。

\subsubsection{罰沒事件間隔的遞增趨勢:走向靜默}

五次罰沒事件的時間間隔呈現出一種值得特別關注的遞增趨勢,此趨勢為判斷系統是否正在趨向長期均衡提供了關鍵線索。具體而言,第一次與第二次罰沒之間的間隔為 $121$ 輪(第 $15$ 輪至第 $136$ 輪),第二次與第三次之間的間隔擴大至 $559$ 輪(第 $136$ 輪至第 $695$ 輪),第三次與第四次的間隔為 $120$ 輪(第 $695$ 輪至第 $815$ 輪),第四次與第五次的間隔再度擴大至 $517$ 輪(第 $815$ 輪至第 $1332$ 輪),而第五次罰沒之後的 $668$ 輪觀察期內則未再發生任何攻擊事件。

這種間隔遞增的整體趨勢並非偶然的統計波動,而是罰沒機制作用於權益分布後的必然數學結果。每一次罰沒事件都會削減惡意節點的權益基數,而根據超幾何分布的性質,惡意節點權益佔比的下降會直接降低其在後續委員會選舉中同時獲得五個以上席位的機率。以初始的 $30\%$ 權益佔比為參照,惡意節點獲得委員會控制權的單輪機率約為 $2.4\%$;當權益比值被壓制至 $0.37$(對應約 $20\%$ 的權益佔比)時,此機率將降至不足 $0.5\%$。機率的降低直接體現為攻擊機會窗口的稀疏化,進而表現為罰沒事件間隔的拉長。

更深層地看,這種動態揭示了一個自我強化的良性循環:罰沒削減權益,權益削減降低攻擊機率,攻擊機率降低減少罰沒的觸發頻率,而較低的觸發頻率又意味著系統在更長的時間段內以正常效率運作,印證了第 \ref{sec:efficiency_analysis} 節關於「挑戰觸發機率 $p$ 在長期均衡中趨近於零」的理論預測。第 $1332$ 輪之後長達 $668$ 輪的「靜默期」尤其具有說明力:在此期間惡意節點的權益已被削減至極低水平,其同時在委員會中獲得足夠席位以發動攻擊的可能性已變得微乎其微。從博弈論的視角來看,即使這些惡意節點在此後的某一輪中僥倖獲得委員會控制權,前四次罰沒的經驗已經清楚展示了攻擊的必然後果,攻擊事件的消失因此是一個結構性的、而非偶然性的結果。

\subsubsection{$100\%$ 偵測率的理論印證與博弈論意涵}

AC-BlockDFL 中 $5$ 次攻擊全部被成功偵測並罰沒的結果,為第 \ref{sec:security_guarantee} 節的形式化安全性定理提供了直接的實證驗證。定理 \ref{thm:detection_completeness} 從理論上證明了在 $1$-of-$N$ 誠實假設下,任何偏離正確 Krum 結果的委員會決策必然被偵測;定理 \ref{thm:punishment_certainty} 進一步證明了被偵測的惡意行為在全網三分之二誠實假設下必然遭受罰沒。實驗中 $100\%$ 的偵測率精確對應了這兩個定理的聯合預測,確認了理論分析在實際模擬環境中的有效性。

這一結果的深層意涵在於,它直接支撐了第 \ref{sec:game_theory_analysis} 節中激勵相容性分析的核心前提。在預期收益公式 $E[\text{Payoff}] = P_{\text{success}} \cdot (G_{\text{attack}} - L_{\text{slash}})$ 中,$100\%$ 偵測率意味著 $P_{\text{caught}} = 1$,而 $L_{\text{slash}} = 500$ 單位($5$ 個惡意節點各損失 $100$ 單位質押)遠大於 $G_{\text{attack}} \leq 7.0$ 單位(壟斷全部驗證獎勵的上界),因此無論攻擊成功的機率 $P_{\text{success}}$ 取何值,預期收益都嚴格為負。與 BlockDFL 中 $107$ 次攻擊全部「免費」的情況相對照,AC-BlockDFL 中每一次攻擊嘗試都伴隨著災難性的經濟後果,這種對比凸顯了兩種架構在安全性典範上的根本分歧:BlockDFL 的門檻安全性依賴於降低攻擊成功的機率,但對於成功的攻擊不施加任何事後代價;AC-BlockDFL 的經濟安全性則不執著於消除攻擊成功的可能性,而是確保每一次成功的攻擊都面臨遠超其收益的經濟懲罰。前者的防禦效果受限於機率運算的固有不確定性,後者的威懾效果則建立在確定性的經濟損失之上。


\subsection{安全性保障下的服務品質驗證}
\label{sec:service_quality}

前兩節的分析已從微觀機制的即時響應與宏觀治理的長期均衡兩個維度,確認了 AC-BlockDFL 經濟安全性機制的有效性。一個隨之而來且同樣重要的問題是:這種安全性保障是否以犧牲聯邦學習的核心效能為代價?本節從系統可用性與模型收斂品質兩個互補的面向展開分析,其目的並非將模型準確率視為獨立的評估維度,而是作為一項合理性驗證,確認防禦機制在有效維護治理安全的同時,並未破壞系統作為機器學習基礎設施的實用價值。

\subsubsection{系統可用性:從攻擊頻率到服務連續性}

為量化攻擊事件對系統連續運作能力的實質衝擊,本研究定義「最低不可用率」作為評估指標,衡量系統因遭受全棧投毒攻擊而處於效能顯著下降狀態的時間比例。之所以聚焦於全棧投毒而非戰略性餓死,是因為前者直接注入惡意模型更新並導致全域模型準確率的急劇下降,其對系統功能的衝擊更為顯著且易於量化。根據實驗觀測,每次全棧投毒攻擊會導致模型準確率急劇下降,而聯邦學習的自我修復機制通常需要約 $5$ 至 $25$ 輪的正常訓練才能使效能恢復至攻擊前的水平。

採用最保守的 $5$ 輪恢復期估計,BlockDFL 因 $89$ 次全棧投毒攻擊累積產生至少 $89 \times 5 = 445$ 輪的效能下降期間,對應約 $445 / 2000 = 22.3\%$ 的最低不可用率。這意味著在整個 $2000$ 輪的訓練過程中,系統有超過五分之一的時間處於模型品質受損的狀態,對於依賴模型輸出進行實時決策的應用場景(如自動駕駛或醫療診斷)而言,如此高的不可用率顯然難以接受。反觀 AC-BlockDFL,憑藉僅 $3$ 次全棧投毒的記錄,最低不可用率被有效控制在 $3 \times 5 / 2000 = 0.75\%$ 以下,相較於 BlockDFL 實現了超過 $96\%$ 的改善幅度。值得注意的是,這一改善並非源自對攻擊行為本身的更好抵禦(兩種架構在面對全棧投毒時的單次受損程度相當),而是完全歸因於經濟懲罰機制對攻擊頻率的有效壓制。

\subsubsection{模型收斂品質:訓練穩定性而非終點準確率}

在確認系統可用性得到了實質性保障後,本節進一步檢驗模型訓練過程本身的品質特徵,探討防禦機制如何從根本上影響機器學習任務的執行效率。由於 AC-BlockDFL 在前 $300$ 輪訓練內便已透過連續的罰沒事件大幅削弱了惡意節點的攻擊能力,這段集中交戰期間的模型表現格外值得深入剖析。它揭示了防禦機制如何在抵禦攻擊的同時,維護訓練過程的連續性與參數最佳化的穩定性,而非僅僅追求最終的一個數值指標。圖 \ref{fig:convergence_curve} 呈現了兩種架構在 $300$ 輪基礎實驗中的準確率演化軌跡,從中可以清晰辨識出截然不同的訓練動態特徵,這反映出底層治理結構對上層模型品質的深層形塑作用。

\begin{figure}[htbp]
    \centering
    \includegraphics[width=0.85\textwidth]{figures/experiments/mnist_results_convergence.png}
    \caption{模型準確率收斂曲線比較}
    \label{fig:convergence_curve}
\end{figure}

BlockDFL 的準確率曲線呈現出顯著的鋸齒狀波動,其中每一次突發性的效能下降都精確對應一次全棧投毒攻擊的瞬時衝擊。在 $300$ 輪的實驗期間內,BlockDFL 共遭受 $6$ 次破壞性的全棧投毒攻擊,而值得特別注意的是,系統最早遭受的威脅並非表現為準確率的下降。第 $69$ 輪發生的首次攻擊採取了一種更為隱蔽的「戰略性餓死策略」,惡意委員會透過共謀排斥誠實委員的提案選擇,轉而批准能夠將獎勵分配給最多惡意更新提供者的聚合結果。這種攻擊的目標並非直接摧毀模型品質,而是操縱經濟激勵的流向,藉此在不被察覺的情況下加速權益累積。從圖 \ref{fig:convergence_curve} 的準確率監測角度觀察,這類戰略性餓死攻擊在模型效能指標上幾乎不留痕跡。傳統的準確率監控機制完全無法識別此類治理層面的攻擊,這也印證了第 \ref{chap:threat-model} 章所強調的核心論點:僅依賴模型品質指標進行安全性評估存在根本性的盲區。

隨著 BlockDFL 中惡意節點透過戰略性餓死策略逐步建立起權益優勢,攻擊發生的頻率呈現出明顯的加速趨勢。實驗觀測顯示,惡意勢力平均約每 $19$ 輪便能發動一次成功的委員會佔領攻擊,這種頻率遞增現象揭示了一種漸進式委員會佔領攻擊的正反饋邏輯。早期的戰略性餓死攻擊為惡意節點積累了足夠的權益基數,而提升的權益又轉化為後續輪次中獲得委員會控制權的高機率,使得攻擊窗口出現得愈發頻繁。當後續攻擊從隱蔽的戰略性餓死策略升級為破壞性的全棧投毒時,每一次成功的佔領都使模型在準確率曲線上留下深刻的凹陷,導致模型不得不被迫在偏離狀態與正常軌跡之間反覆擺盪。儘管聯邦學習的自癒特性使系統能在攻擊後逐步恢復,但這種持續的「偏離與修正」循環實質上消耗了大量本應用於模型最佳化的有效週期,大幅降低了運算資源的利用率。

相較之下,AC-BlockDFL 的準確率曲線展現出截然不同的動態穩定性特徵。在 $300$ 輪實驗期間內,AC-BlockDFL 僅遭受 $2$ 次全棧投毒攻擊,其中第 $90$ 輪的極端案例提供了觀察防禦機制韌性的重要指標。在該輪次中,惡意節點執行了顯著的模型翻轉攻擊,將模型準確率從正常水準驟降至 $9.5\%$,這一數值已接近 MNIST 十分類任務中的隨機猜測基準線。然而,系統展現了極強的自我修復能力,在經歷約 $20$ 輪的正常訓練後成功恢復至攻擊前的水準。更有意義的發現是,隨著訓練的深入推進,模型對同類攻擊的抵抗力與恢復速度呈現持續改善的趨勢。至訓練後期,即使面臨同等強度的擾動,恢復所需的輪次已從初期的 $20$ 輪大幅縮短至僅約 $5$ 輪,這歸因於模型在長期訓練後建立的穩健參數結構提高了系統的容錯上限。

從最終收斂品質的角度審視,BlockDFL 在 $300$ 輪訓練結束時達到 $98.26\%$ 的準確率,而 AC-BlockDFL 則達到 $98.63\%$,兩者之間僅有 $0.37$ 個百分點的微小差距。這一結果進一步印證了第 \ref{chap:framework} 章第 \ref{sec:no_rollback} 節所制定的「不回滾策略」在實務上的合理性:聯邦學習的迭代本質賦予了系統天然的自癒力,使偶發性的模型偏差能在後續正常訓練中被逐步消化,因此系統並不需要承擔回滾操作所帶來的巨大協調成本。然而,終點準確率的相近並不意味著兩種架構在訓練品質上等價。AC-BlockDFL 憑藉罰沒機制將全棧攻擊壓縮至僅 $2$ 次,為模型訓練營造了近乎無干擾的連續最佳化環境,使運算資源得以更高效地轉化為模型效能的增益,而非被浪費在修復攻擊損害上。這對於運算資源受限的邊緣部署場景而言,具有尤為突出的實務意義。

\section{本章小結}
\label{sec:eval_summary}

本章透過「機制驗證、長期生存、服務品質」的三階段遞進分析,從逐步擴大的觀察尺度驗證了審計驅動型委員會 BlockDFL 的防禦效能。$300$ 輪基礎實驗首先確認了經濟懲罰機制在微觀層面的即時響應能力:每一次惡意委員會決策都被挑戰者靈敏偵測,並透過全網仲裁觸發罰沒制裁,權益軌跡的階梯式下降為機制的靈敏度與精準度提供了直觀的視覺證據。$2000$ 輪長期模擬則進一步揭示了這種短期有效性如何轉化為長期的治理均衡:五次罰沒事件將惡意節點的權益比值從 $1.0$ 逐步壓制至 $0.37$,罰沒間隔的遞增趨勢與最終長達 $668$ 輪的靜默期,共同印證了系統確實被引導至攻擊自然消亡的穩定狀態。$100\%$ 的攻擊偵測率為第 \ref{sec:game_theory_analysis} 節的博弈論分析提供了實證基礎:理性攻擊者若能預見攻擊必然被偵測並遭受損失,將選擇不發動攻擊以保全質押資產,系統因此能在實務運作中自然趨向穩定均衡,維持第 \ref{sec:efficiency_analysis} 節所分析的 $O(C^2)$ 效率水平。

上述安全性保障的達成並非以犧牲系統效能為代價。服務品質分析顯示,AC-BlockDFL 將最低不可用率從 BlockDFL 的 $22.3\%$ 壓制至 $0.75\%$ 以下,同時為模型訓練提供了更為平穩且不受干擾的最佳化環境。這些實驗結果整體構成了一條完整的推論鏈:最壞情況測試證明所有攻擊皆被偵測,理性攻擊者因此預見懲罰而選擇不攻擊,系統在無攻擊的均衡狀態下同時維持了高效率與高品質的服務水準。

\end{ZhChapter}