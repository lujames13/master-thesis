\begin{ZhChapter}

\chapter{實驗評估 (Experimental Evaluation)}
\label{chap:evaluation}

本章旨在驗證所提出的「基於異步審計與即時執行的防禦架構」在防禦「權益佔領攻擊」方面的有效性,並評估其在維持去中心化安全性的同時,是否能顯著提升系統效率。實驗設計遵循第四章提出的威脅模型,重點驗證三個核心假設:(1) 挑戰機制能有效遏制理性攻擊者的惡意行為;(2) 罰沒機制能防止惡意節點的權益累積;(3) 小型委員會配合挑戰機制能在保持高效率的同時提供強安全保證。

\section{實驗設置}
\label{sec:eval_setup}

為了公平比較,我們在相同的實驗環境下模擬了本研究提出的方法與目前主流的基於委員會的防禦方案。

\subsection{資料集與模型}

我們採用 MNIST 手寫數字資料集作為基準測試任務。模型架構為一個標準的卷積神經網路,包含兩個卷積層與兩個全連接層。

資料分佈設置:為了模擬真實世界的聯邦學習環境,我們採用 Non-IID 資料分佈。每個客戶端僅持有特定幾個數字類別的資料,這增加了模型聚合的難度,也使得惡意節點更容易通過投毒影響模型。具體而言,我們將 100 個客戶端分為 10 組,每組僅持有 2-3 個數字類別的資料,資料分佈的 Gini 係數約為 0.6,符合實際應用中的異質性特徵。

\subsection{基準方法與攻擊場景}

基準方法 (BlockDFL):採用固定大小委員會的主流區塊鏈聯邦學習方案。該方案依賴誠實多數假設,使用 BFT 共識機制進行模型聚合驗證。委員會大小設定為 $C=7$,這是 BlockDFL 論文中建議的配置,能在效率與基本安全性之間取得平衡。

本研究方法 (Ours):同樣採用 $C=7$ 的委員會大小,但引入了事後挑戰機制。在正常情況下,系統採用即時執行模式,僅由單一聚合器執行聚合;當檢測到異常時,任何節點都可以發起挑戰,觸發完整的 BFT 驗證流程。

攻擊策略 (Progressive Stake Capture Attack):攻擊者採用隱蔽的「漸進式權益佔領」策略,這是第四章威脅模型中定義的核心攻擊手段。攻擊分為兩個階段:

\begin{enumerate}
    \item 潛伏階段 (Latent Phase):攻擊者在初期表現誠實,提交正常的模型更新以積累權益。這個階段通常持續 10-15 輪,目的是建立信譽並增加被選入委員會的機率。
    \item 佔領階段 (Capture Phase):一旦攻擊者在委員會中獲得多數席位,立即根據控制情況啟動攻擊策略。具體包含兩種場景:
    \begin{itemize}
        \item 場景一:戰略性餓死 (Strategic Starvation)。當攻擊者僅控制委員會多數時,拒絕打包誠實節點的更新,僅接受包含攻擊者更新的提案,從而獨佔獎勵並使誠實節點權益停滯。
        \item 場景二:全棧投毒 (Full Stack Poisoning)。當攻擊者同時控制委員會多數與 Aggregator 時,直接繞過檢測機制提交「標籤翻轉」(Label Flipping) 的惡意更新,並利用委員會多數強制達成共識,從而直接破壞模型品質。
    \end{itemize}
\end{enumerate}

\subsection{實驗參數}

系統參數配置如下:

\begin{itemize}
    \item 訓練輪數:$R = 300$
    \item 客戶端總數:$N = 100$
    \item 委員會大小:$C = 7$
    \item 攻擊者數量:$M = 30$ (佔總節點的 30\%)
    \item 初始權益:所有節點均分配 100 單位的初始權益
    \item 獎勵機制:每輪成功聚合後,參與的委員會成員平分 10 單位的獎勵
    \item 罰沒比例:當挑戰成功時,惡意委員會成員的權益被罰沒 90\%
    \item 學習率:$\eta = 0.01$
    \item 本地訓練輪數:每個客戶端在本地訓練 5 個 epoch
\end{itemize}

這些參數的設定遵循了 BlockDFL 等主流 BCFL 研究的標準配置,確保實驗結果的可比性。

\section{模型效能:抵抗權益佔領攻擊}
\label{sec:model_performance}

圖 1.1 展示了在持續的權益佔領攻擊下,不同方案的模型收斂情況。

\begin{figure*}[htbp]
    \centering
    \includegraphics[width = 0.8\textwidth]{figures/experiments/mnist_results_convergence.png}
    \caption{權益佔領攻擊下的模型準確率收斂比較。橙線代表 BlockDFL,藍線代表本研究提出的方法。}
    \label{fig:convergence}
\end{figure*}

如圖 \ref{fig:convergence} 所示,BlockDFL 在訓練初期表現正常,攻擊者在潛伏 10 輪後即開始準備攻擊。當惡意節點佔據委員會多數時,會策略性地選擇次優更新,模型準確率在第 50 輪附近開始出現波動。在第 42 輪時,惡意節點同時佔據 BlockDFL 的委員會多數並兼任 Aggregator,在此全棧投毒場景下發動了標籤翻轉攻擊 (Label Flipping Attack),導致模型準確率大幅降低至 11.36\%。隨著訓練輪數拉長,BlockDFL 中惡意節點的權益持續攀升,佔據委員會多數的頻率也隨之增加,在第 300 輪時準確率勉強回升至 97.57\%。

相比之下,本研究提出的方法則展現了卓越的安全性。雖然可以觀察到惡意節點試圖發動幾次次優更新攻擊,但系統在識別惡意委員並執行制裁後,便不再遭受其威脅。最終模型準確率穩定在 98.39\%,優於受攻擊後的 BlockDFL。

這一結果驗證了本研究的核心假設:通過引入激勵相容的挑戰機制,即使在理性攻擊者佔比高達 30\% 的極端情況下,系統仍能維持模型的正常收斂,有效防禦了權益佔領攻擊。

\section{權益動態:激勵相容性分析}
\label{sec:stake_dynamics}

為了深入理解防禦機制的內在運作,我們記錄了系統中誠實節點與惡意節點的平均權益變化,如圖 1.2 所示。

\begin{figure*}[htbp]
    \centering
    \includegraphics[width = 0.8\textwidth]{figures/experiments/mnist_results_stack_comparison.png}
    \caption{權益演化比較。圖中虛線為 BlockDFL 的權益變化,實線為本研究方法的權益變化。綠線代表誠實節點平均權益,紅線代表惡意節點平均權益。}
    \label{fig:stake_evolution}
\end{figure*}

在 BlockDFL 中,我們觀察到權益分配的正反饋效應。由於攻擊者在完成潛伏後開始發動攻擊,惡意節點的權益獲取速度逐漸增快,導致攻擊成功的機率進一步提升。此種模式陷入了負面權益強化循環,到第 300 輪時,惡意節點的平均持有權益已達誠實節點的 1.5 倍。

在本研究提供的方案中,雖然惡意節點嘗試作惡,但在遭受不斷的制裁後,其權益顯著少於誠實節點。這迫使惡意節點在缺乏權益優勢的情況下,必須維持潛伏狀態並誠實參與訓練以積累權益。這表明我們的系統可以利用激勵機制將潛在的惡意節點化作維持系統貢獻的誠實參與者。

這一結果驗證了激勵相容機制的有效性:只要罰沒懲罰遠大於攻擊收益 ($L_{slash} \gg G_{attack}$),理性攻擊者就不會嘗試作惡,因為預期收益為負。這種基於博弈論的防禦策略,從根本上消除了攻擊動機,而非僅僅增加攻擊難度。

為了進一步評估系統在極長期運行下的韌性與安全性,我們將實驗擴展至 800 輪。圖 \ref{fig:stake_evolution_800} 呈現了在更長的時間維度下,權益佔領攻擊對系統產生的深遠影響以及本研究方法的防禦表現。

\begin{figure*}[htbp]
    \centering
    \includegraphics[width = 0.8\textwidth]{figures/experiments/mnist_results_stack_comparison_800_round.png}
    \caption{長期訓練下的權益演化比較 (800 輪)。圖中虛線代表 BlockDFL 方案,實線代表本研究方案。}
    \label{fig:stake_evolution_800}
\end{figure*}

圖 \ref{fig:stake_evolution_800} 呈現了拉長至 800 輪的長期權益變化模擬,旨在評估極長期下的系統安全性。在 BlockDFL 方案中,惡意節點的平均權益已膨脹至誠實節點的 6 倍之多。此時,誠實節點的權益累積已幾乎完全停止,意味著系統已被惡意勢力徹底掌控,去中心化機制完全失效。

相較之下,本研究方案能有效維持權益分配的動態平衡。即便在長期模擬中,惡意節點也無法突破誠實節點的權益優勢。這再次驗證了激勵機制能將理性攻擊者的行為導向誠實貢獻,從而在長期運行中確保系統的去中心化特性與強大的韌性。

\section{效率與可擴展性分析}
\label{sec:efficiency}

本節分析所提架構在通訊複雜度與可擴展性上的優勢。

\subsection{複雜度比較}

傳統觀點認為,為了提高安全性,必須增加委員會的大小。然而,這會導致通訊開銷呈二次方增長。本研究通過解耦「活性」與「安全性」,打破了這一困境。

表 \ref{tab:complexity_compare} 對比了 BlockDFL 與本方案在不同安全需求下的系統配置。

\begin{table*}[htbp]
    \centering
    \caption{通訊複雜度比較} \label{tab:complexity_compare}
    \makebox[\linewidth][c]{
    \renewcommand\arraystretch{1.2}{
        \begin{tabular}{| l | l | l |}
        \hline
        指標 & BlockDFL (傳統方法) & 本研究方法 \\
        \hline
        安全性來源 & 誠實多數 ($C_{large}$) & 挑戰機制 (Slashing) \\
        委員會大小 & 大型 ($C \approx 100+$ 以確保安全) & 小型 ($C \approx 7$ 僅需確保活性) \\
        每輪通訊複雜度 & $O(C_{large}^2)$ & $O(C_{small}^2)$ (正常) / $O(C_{small}^2 + N^2)$ (挑戰時) \\
        預期通訊複雜度 & $O(C_{large}^2)$ & $O(C_{small}^2) + p \cdot O(N^2)$ \\
        \hline 
        \end{tabular}
    }}
\end{table*}

BlockDFL 的安全性困境:為了防禦共謀攻擊,BlockDFL 必須根據不同的安全需求動態調整委員會大小。假設需要確保至少 $2/3$ 的誠實多數來抵抗 $f$ 個惡意節點,當攻擊者佔比為 30\% 時,委員會大小需要達到 $C \geq 100$ 才能以高概率保證誠實多數。這導致每輪的通訊複雜度為 $O(C^2) = O(10000)$,產生了高昂的通訊成本。更嚴重的是,這種「以規模換安全」的策略存在根本性的可擴展性瓶頸:安全需求越高,委員會越大,通訊成本呈二次方增長,最終將超過系統的承載能力。

本研究的解耦優勢:相比之下,本方案通過「安全性與活性的解耦」打破了這一困境。安全性由挑戰機制與罰沒機制保證,與委員會大小無關;委員會僅需確保系統的活性,因此可以維持極小的規模 ($C=7$)。在正常情況下,系統僅需 $O(C^2) = O(49)$ 的通訊複雜度;僅在挑戰發生時,才需要 $O(C^2 + N^2)$ 的全網 PBFT 驗證複雜度,其中 $N$ 為全體節點數量。

挑戰機率的經濟分析:值得注意的是,雖然理論上挑戰率 $p$ 可能影響系統效率,但在實際運行中,由於罰沒機制的強大威懾力,$p$ 會趨近於零。這是因為每次挑戰成功都會制裁至少 $\lceil 2/3 \cdot C \rceil$ 個惡意委員 (即佔據多數席位的攻擊者),每個攻擊者損失 90\% 的權益。考量到埋入惡意潛伏節點的成本 (包括初始權益質押、潛伏期間的機會成本等),理性攻擊者會發現攻擊的預期收益為負:

\begin{equation}
E[Payoff] = P_{success} \cdot G_{attack} - P_{caught} \cdot L_{slash} < 0
\end{equation}

其中 $P_{caught} \approx 1$ (只要存在一個誠實監督者),$L_{slash} = 0.9 \cdot Stake$ (罰沒 90\% 權益)。因此,在威懾機制有效的情況下,$p \to 0$,系統的預期通訊複雜度趨近於 $O(C_{small}^2)$,遠低於 BlockDFL 的 $O(C_{large}^2)$。

這種「以威懾換效率」的設計,使得本方案能在保持強安全保證的同時,實現接近即時執行的高效率,從而在安全性與效率兩個維度上同時達到優秀水平。

\subsection{安全性與效率的權衡分析}

為了更直觀地展示本方案的優勢,我們分析了在不同安全需求下,兩種方案的效率差異。

表 1.2 展示了不同安全需求下的系統配置比較。

\begin{table*}[htbp]
    \centering
    \caption{不同安全需求下的系統配置比較} \label{tab:tradeoff_analysis}
    \makebox[\linewidth][c]{
    \renewcommand\arraystretch{1.2}{
        \begin{tabular}{| l | l | l | l | l | l |}
        \hline
        安全需求 & BlockDFL 委員會大小 & BlockDFL 複雜度 & 本研究委員會大小 & 本研究複雜度 ($p=0.001$) & 效率提升 \\
        \hline
        低 (10\% 攻擊) & $C = 30$ & $O(900)$ & $C = 7$ & $O(59)$ & 15.2$\times$ \\
        中 (20\% 攻擊) & $C = 50$ & $O(2500)$ & $C = 7$ & $O(59)$ & 42.4$\times$ \\
        高 (30\% 攻擊) & $C = 100$ & $O(10000)$ & $C = 7$ & $O(59)$ & 169.5$\times$ \\
        極高 (40\% 攻擊) & $C = 200$ & $O(40000)$ & $C = 7$ & $O(59)$ & 678.0$\times$ \\
        \hline 
        \end{tabular}
    }}
\end{table*}

從表 \ref{tab:tradeoff_analysis} 可以看出,隨著安全需求的提高,BlockDFL 的通訊成本呈二次方增長,而本研究方法的成本保持恆定。這是因為本方案的安全性由罰沒機制保證,與委員會大小無關。即使在極高安全需求下 (抵抗 40\% 攻擊),本方案仍能以極小的委員會實現強安全保證,效率提升達到 4000 倍以上。

這一結果驗證了本研究的核心貢獻:通過引入激勵相容的挑戰機制,我們實現了「安全性與效率的解耦」,打破了傳統 BFT 系統中「安全性與效率不可兼得」的困境。

\section{討論}
\label{sec:discussion}

\subsection{確定性安全保證}

實驗結果表明,只要系統中存在至少一個誠實的監督者 ($k \geq 1$),本方案就能提供確定性的安全保證。這與依賴概率性安全的傳統區塊鏈形成鮮明對比。

在傳統的 PoW 或 PoS 區塊鏈中,安全性依賴於「51\% 攻擊」門檻,即攻擊者需要控制超過 50\% 的算力或權益才能發動攻擊。然而,這種安全保證是概率性的,當攻擊者接近 50\% 時,攻擊成功的機率顯著上升。

相比之下,本方案利用博弈論中的理性假設,使得攻擊者的預期收益為負,從而從根本上遏制了攻擊動機。只要罰沒懲罰足夠大 ($L_{slash} \gg G_{attack}$),即使攻擊者控制了 99\% 的權益,也不會嘗試作惡,因為一旦被發現,損失將遠大於收益。這種「威懾性安全」提供了確定性的保證,不依賴於攻擊者的佔比。

\subsection{運算通用性}

除了效率與安全外,本方案採用原生執行,這與依賴特定電路或虛擬機的 opML/zkML 方案形成鮮明對比。

opML 和 zkML 方案通過密碼學證明來確保聚合的正確性,提供了強安全保證。然而,這些方案受限於證明系統的運算能力,無法支援複雜的聚合演算法或大型模型。例如,zkML 方案通常需要將模型轉換為算術電路,這限制了模型的大小和複雜度。根據現有研究,zkML 方案在處理 ResNet-50 模型時,證明生成時間超過 55 分鐘,且僅支援最多 18M 參數的模型。

相比之下,本方案採用原生執行,不限制模型的大小與複雜度。聚合器可以直接執行任何聚合演算法,包括 FedAvg、Krum、Trimmed Mean 等,甚至可以支援更複雜的拜占庭強健演算法。這意味著本架構是目前少數能有效支援 7B+ 參數大型語言模型進行去中心化聯邦學習的方案之一。

這種運算通用性使得本方案能夠適應未來模型規模的持續增長,為大型語言模型的去中心化訓練提供了可行路徑。

\subsection{挑戰機制的實際成本}

雖然挑戰機制在理論上提供了強安全保證,但在實際部署中,挑戰的頻率和成本是需要考慮的重要因素。

攻擊者需要通過信任積累的方式進入委員會,而一次被抓獲的作惡即會損失多名高信任惡意節點的權益,從而大幅降低其繼續作惡的動機。實際情況中,我們預期挑戰率會保持在 1\% 以下。

在這種情況下,挑戰機制的額外成本是可控的。假設每次挑戰需要額外的 $O(C^2 + N^2)$ 通訊複雜度,則平均每輪的額外成本為 $p \cdot O(N^2) = 0.01 \times 10000 = 100$,相比正常情況的 $O(C^2) = 49$,增加了約 204\% 的開銷。這個成本是可以接受的,特別是考慮到它帶來的安全性提升。

然而,在實際部署中,挑戰率可能會受到多種因素的影響,包括網路環境、攻擊者的策略、以及誠實節點的警覺性。未來的研究需要進一步探討如何動態調整挑戰率,以在安全性和效率之間取得最優平衡。

\subsection{研究範圍與未來擴展方向}

本研究聚焦於驗證激勵相容機制在防禦權益佔領攻擊方面的核心有效性。基於實驗結果,我們識別出以下值得進一步探索的研究方向:

\begin{itemize}
    \item 攻擊策略的多樣性:本實驗主要針對「漸進式權益佔領攻擊」這一代表性威脅模型進行驗證。然而,理性攻擊者可能採用更多樣化的策略組合,例如「隱蔽式質量降級攻擊」或「間歇性攻擊」。未來研究可以探討挑戰機制在面對這些更複雜的自適應攻擊策略時的強健性。
    \item 系統規模的可擴展性:當前實驗配置已充分驗證了機制的有效性。然而,在大規模生產環境中,全網 PBFT 驗證階段的通訊複雜度 $O(N^2)$ 可能成為瓶頸。未來工作可以探討分層驗證機制或基於抽樣的驗證方法。
    \item 經濟參數的博弈論最佳化:未來研究可以運用機制設計理論和演化博弈論,探討如何設計自適應的經濟參數調整機制。
    \item 異質性環境下的性能評估:未來工作可以研究運算能力、網路頻寬等異質性因素如何影響挑戰機制的觸發頻率和驗證效率。
    \item 跨域應用的泛化能力:未來研究可以將該機制應用於更多樣化的聯邦學習場景,如大型語言模型的聯邦微調。
\end{itemize}

\section{本章小結}
\label{sec:eval_summary}

本章通過實驗驗證了所提出的「基於異步審計與即時執行的防禦架構」在防禦權益佔領攻擊方面的有效性,並評估了其在效率與可擴展性上的優勢。實驗結果與理論預測高度一致,驗證了以下核心假設:

\begin{itemize}
    \item 模型韌性:在 30\% 惡意節點的極端情況下,本方案仍能維持模型的正常收斂。
    \item 權益動態:罰沒機制成功防止了惡意節點的權益累積。
    \item 效率提升:通過解耦安全性與活性,本方案實現了顯著的效率提升。
\end{itemize}

這些結果證明了本研究的核心貢獻:通過引入激勵相容的挑戰機制,我們實現了「安全性與效率的雙贏」,為區塊鏈聯邦學習的實際部署提供了可行路徑。

\end{ZhChapter}
