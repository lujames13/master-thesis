\begin{ZhChapter}

\chapter{實驗評估 (Experimental Evaluation)}
\label{chap:evaluation}

本章旨在驗證所提出的「基於異步審計與即時執行的防禦架構」在防禦「權益佔領攻擊」方面的有效性,並評估其在維持去中心化安全性的同時,是否能顯著提升系統效率。實驗設計遵循第四章提出的威脅模型,重點驗證三個核心假設:(1) 挑戰機制能有效遏制理性攻擊者的惡意行為;(2) 罰沒機制能防止惡意節點的權益累積;(3) 小型委員會配合挑戰機制能在保持高效率的同時提供強安全保證。

\section{實驗設置}
\label{sec:eval_setup}

為了公平比較,我們在相同的實驗環境下模擬了本研究提出的方法與目前主流的基於委員會的防禦方案。

\subsection{資料集與模型}

我們採用 MNIST 手寫數字資料集作為基準測試任務。模型架構為一個標準的捲積神經網路,包含兩個捲積層與兩個全連接層。

資料分佈設置:為了全面評估系統性能,本研究考量了獨立同分佈 (IID) 與非獨立同分佈 (Non-IID) 兩類環境。在 IID 設置中,資料被均勻地隨機分配給所有客戶端。而在 Non-IID 設置中,我們採用基於 Dirichlet 分佈 ($\text{Dir}(\alpha)$) 的資料劃分,並將濃度參數設定為 $\alpha=0.5$。這種設定會導致每個客戶端持有的類別分佈呈現高度異質性,模擬了真實場景中資料分佈極度不均的情況,從而增加模型聚合與抗攻擊的挑戰。

\subsection{基準方法與攻擊場景}

基準方法 (BlockDFL):採用固定大小委員會的主流區塊鏈聯邦學習方案。該方案依賴誠實多數假設,使用 BFT 共識機制進行模型聚合驗證。委員會大小設定為 $C=7$,這是 BlockDFL 論文中建議的配置,能在效率與基本安全性之間取得平衡。我們設定 BFT 的共識門檻為 $2/3$,即必須有超過 $2/3$ 的成員同意才能通過提案。

本研究方法 (Ours):同樣採用 $C=7$ 的委員會大小,但引入了事後挑戰機制。在正常情況下,系統採用即時執行模式,僅由單一聚合器執行聚合;當檢測到異常時,任何節點都可以發起挑戰,觸發完整的 BFT 驗證流程。

攻擊策略 (Progressive Stake Capture Attack):攻擊者採用隱蔽的「漸進式權益佔領」策略,這是第四章威脅模型中定義的核心攻擊手段。攻擊分為兩個階段:

\begin{enumerate}
    \item 潛伏階段 (Latent Phase):只要攻擊者尚未獲得委員會的控制權 (即未達 2/3 席位),皆會維持潛伏狀態並表現誠實,透過提交正常的模型更新來穩定積累權益。此階段的目的是建立信譽並增加權益佔比,從而提升未來被選入委員會的機率,為發動攻擊奠定基礎。
    \item 佔領階段 (Capture Phase):一旦攻擊者在委員會中獲得超過 2/3 席位,立即根據控制情況啟動攻擊策略。具體包含兩種場景:
    \begin{itemize}
        \item 場景一:戰略性餓死 (Strategic Starvation)。當攻擊者僅控制委員會超過 2/3 席位時,拒絕打包誠實節點的更新,僅接受包含攻擊者更新的提案,從而獨佔獎勵並使誠實節點權益停滯。
        \item 場景二:全棧投毒 (Full Stack Poisoning)。當攻擊者同時控制委員會超過 2/3 席位與 Aggregator 時,直接繞過檢測機制提交「標籤翻轉」(Label Flipping) 的惡意更新,並利用委員會多數強制達成共識,從而直接破壞模型品質。
    \end{itemize}
\end{enumerate}

\subsection{實驗參數}

本研究實驗採用的系統參數配置如表 \ref{tab:exp_params} 所示。這些參數的設定遵循了 BlockDFL \cite{qin2024blockdfl} 等主流 BCFL 研究的標準配置,確保實驗結果的可比性。

\begin{table}[htbp]
    \centering
    \caption{實驗參數配置 (Experimental Parameter Configurations)}
    \label{tab:exp_params}
    \renewcommand{\arraystretch}{1.3}
    \begin{tabular}{|l|l|}
        \hline
        \textbf{參數名稱} & \textbf{設定值} \\
        \hline
        訓練輪數 & $R = 300$ \\
        \hline
        客戶端總數 & $N = 100$ (Verifier Pool Size) \\
        \hline
        委員會大小 & $C = 7$ \\
        \hline
        攻擊者數量 & $M = 30$ (初始權益佔比 30\%) \\
        \hline
        初始權益分配 & 所有參與節點初始均分配 100 單位 \\
        \hline
        設備池分配 & Aggregator: 4 位,Provider: 其餘節點 \\
        \hline
        獎勵機制 (每輪) & Verifier: 1.0, Aggregator: 1.0, Provider: 0.05 \\
        \hline
        罰沒機制 & 挑戰成功時,惡意委員全額罰沒 (Full Slashing) \\
        \hline
        學習率 & $\eta = 0.01$ (衰減率 0.99) \\
        \hline
        本地訓練參數 & Epochs = 1, Batch Size = 32 \\
        \hline
        資料分佈環境 & IID 及 Dirichlet-based Non-IID ($\alpha=0.5$) \\
        \hline
    \end{tabular}
\end{table}

\section{實驗結果與分析}
\label{sec:experimental_results}

\subsection{模型效能與攻擊表現分析}
\label{sec:model_performance}

本節針對系統在不同資料分佈下的收斂性與遭受攻擊的頻率進行量化分析。圖 \ref{fig:convergence_iid} 至圖 \ref{fig:convergence_noniid} 分別展示了 IID 與 Non-IID 環境下,BlockDFL 與本研究方法(Ours)的表現。

\begin{figure*}[htbp]
    \centering
    \begin{subfigure}[b]{0.48\textwidth}
        \centering
        \includegraphics[width=\textwidth]{figures/experiments/mnist_results_convergence.png}
        \caption{IID 環境 (均勻分佈)}
        \label{fig:convergence_iid}
    \end{subfigure}
    \hfill
    \begin{subfigure}[b]{0.48\textwidth}
        \centering
        \includegraphics[width=\textwidth]{figures/experiments/mnist_noniid_results_convergence.png}
        \caption{Non-IID 環境 ($\alpha=0.5$)}
        \label{fig:convergence_noniid}
    \end{subfigure}
    \caption{模型準確率收斂比較。(a) 為 IID 環境,(b) 為 Non-IID 環境。}
    \label{fig:convergence_combined}
\end{figure*}

\subsubsection*{1) 顯性攻擊影響與收斂穩定性}
實驗結果顯示,BlockDFL 在兩類環境下均展現出明顯的安全性漏洞。

\textbf{攻擊頻率:} 在 300 輪訓練中,BlockDFL 分別遭受了 10 次(IID)與 12 次(Non-IID)成功的委員會佔領。相較之下,本研究方法透過異步審計機制,在 IID 中僅遭受 2 次佔領,在更具挑戰性的 Non-IID 環境中也僅遭受 3 次佔領,顯示出極強的韌性。

\textbf{瞬時破壞力:} 以圖 \ref{fig:convergence_noniid}(Non-IID)為例,BlockDFL 於第 68 輪遭受標籤翻轉(Label Flipping)攻擊時,準確度由正常水平瞬間崩潰至 9.55\%。這證明了在傳統 BCFL 框架下,單次成功的委員會佔領即可對全球模型造成致命打擊。

\textbf{顯性攻擊與聯邦學習的自癒性:} 觀察圖 \ref{fig:convergence_noniid}(Non-IID)可以發現,BlockDFL 在第 68 輪遭受標籤翻轉攻擊後,準確度雖瞬間崩潰至 9.55\%,但隨後幾輪呈現快速回升。這印證了聯邦學習具備顯著的自我修復能力(Self-healing capacity):只要攻擊者無法持續佔領委員會,後續輪次的誠實更新即可逐步抵銷惡意梯度產生的噪聲。因此,單次的標籤翻轉攻擊雖會造成系統震盪,但通常不會導致模型不可逆的毀滅。

\textbf{Non-IID 強健性解釋:} 值得注意的是,即便在 $\alpha=0.5$ 的高度異質資料分佈下,本系統仍能維持與 IID 相似的收斂速度。此現象源於系統採用的「基於驗證的選優機制」(Selection-based mechanism),透過全局驗證集有效過濾了 Non-IID 引起的權重發散(Client Drift)。

\subsubsection*{2) 系統穩定性與最低不可用率分析}

為了進一步量化攻擊對系統運行的實質衝擊,本研究定義「最低不可用率」(Minimum Unavailability Rate)為指標。我們保守地假設每次受擊後的恢復期僅需 5 輪(此為實驗觀測結果 5--25 輪之最小值),並據此運算系統處於效能崩潰狀態的比例。

\textbf{下限估計與效能鴻溝:}
根據實驗資料的量化分析,在 Non-IID 環境下,BlockDFL 由於遭受了 12 次成功的委員會佔領攻擊,即便採用最為樂觀的 5 輪恢復期進行運算,系統在 300 輪的訓練過程中仍有至少 20\%(即 60 輪)的時間處於不可用狀態。若進一步考量到實驗中實際觀察到的最大恢復期(25 輪),其實際癱瘓時間將遠超此比例。

相比之下,本研究提出的方法憑藉「異步審計機制」,將成功受擊次數大幅壓制在 3 次以內。在同樣的保守估計準則下,本系統的最低不可用率僅為 5\%(15/300 輪)。這項數據對比清晰地證明:儘管聯邦學習具有「自癒性」,但頻繁的受擊仍會使傳統框架在訓練過程中陷入極大的不穩定;而本方法則能確保系統在 95\% 以上的訓練時間內,始終維持高品質的服務能力。

\textbf{連續受擊的連鎖反應:} 
此外,BlockDFL 的高受擊頻率(平均每 25 輪一次)與恢復期(5--25 輪)在時間軸上高度重疊。這意味著在 Non-IID 較複雜的收斂過程中,BlockDFL 極易在尚未從前次攻擊完全恢復時再次受擊,導致模型準確度長期在低位震盪,無法累積有效的全局知識。

\subsubsection*{3) 最終準確率對比}
在經歷 300 輪的攻防博弈後,兩者的最終訓練結果如下:

\begin{itemize}
    \item \textbf{IID 環境:} 本研究方法最終準確率達到 98.63\%,BlockDFL 為 98.26\%。
    \item \textbf{Non-IID 環境:} 本研究方法達到 98.67\%,BlockDFL 為 98.57\%。
\end{itemize}

誠然,BlockDFL 展現了聯邦學習的自癒特性,但在 Non-IID 環境下,每次受擊後的恢復期至少需要 5 輪。保守估計,BlockDFL 在訓練過程中有超過 20\% 的時間處於不可用狀態。本研究方法透過異步審計將攻擊頻率降低了 80\%,確保了模型在整個週期內維持高水準的服務能力。這種「過程穩定性」在需要實時部署的關鍵任務中,其價值遠超最終 0.1\% 的準確率增益。

\subsection{安全動態與治理風險深層分析}
\label{sec:governance_risk}

本節進一步探討權益演化與隱蔽攻擊的內在邏輯,揭示基於權益選拔(Stake-based Selection)系統中的固有治理風險。

\begin{figure*}[htbp]
    \centering
    \begin{subfigure}[b]{0.48\textwidth}
        \centering
        \includegraphics[width=\textwidth]{figures/experiments/mnist_results_stack_comparison.png}
        \caption{IID 環境}
        \label{fig:stake_evolution_iid}
    \end{subfigure}
    \hfill
    \begin{subfigure}[b]{0.48\textwidth}
        \centering
        \includegraphics[width=\textwidth]{figures/experiments/mnist_noniid_results_stack_comparison.png}
        \caption{Non-IID 環境}
        \label{fig:stake_evolution_noniid}
    \end{subfigure}
    \caption{權益演化比較。(a) 為 IID 環境,(b) 為 Non-IID 環境。}
    \label{fig:stake_evolution}
\end{figure*}

\subsubsection*{1) 權益優勢的建立與自我強化機制}
透過對原始權益資料的追蹤發現,在 BlockDFL 中,攻擊者平均持有的權益穩定維持在誠實節點的 1.1 至 1.2 倍。這種優勢地位的建立具有其系統必然性:

\textbf{任務價值差異:} 系統中執行運算量較大或關鍵性較高的任務(如 Aggregator 或 Committee 成員)所獲取的獎勵遠高於普通 Provider。

\textbf{正向回饋循環:} 由於角色分配機制與權益掛鉤,一旦節點獲得初步權益優勢,其未來被選中擔任重要角色的機率隨之增加,進而獲得更多獎勵。

\textbf{增長上限分析:} 攻擊者權益比未能呈現指數級成長,是因為其無法完全操控隨機的角色分配邏輯。即便惡意委員會策略性地選擇有利於惡意節點的更新,系統中仍有部分誠實節點(UP 或 AG)會獲得獎勵,從而形成了 1.1–1.2 倍的動態平衡區間。然而,只要「高貢獻任務獲得高獎勵」的分配邏輯不變,這種\textbf{領先者優勢(Leader Advantage)}便會轉化為長期的治理威脅。

\subsubsection*{2) NEPO 攻擊的普遍性與隱蔽性}
進一步分析揭示,NEPO 攻擊的隱蔽性並非僅限於 Non-IID 環境,而是系統層面的普遍風險。

\textbf{模型指標的局限性:} 如實驗資料所示(例如 Non-IID 第 239 輪),即便委員會已被惡意佔領且正在執行 NEPO 攻擊,全球模型的準確度仍可能維持上升。這是因為攻擊者可透過保留部分高質量更新來偽裝其行為。

\textbf{解耦威脅:} 這種現象顯示了\textbf{「模型效能」與「系統誠信」的解耦}。若缺乏本研究提出的罰沒機制(Slashing),攻擊者可以長期隱藏在系統中累積權益,直到達成「全棧共謀」(Full-stack Collusion)的條件。

\subsubsection*{3) 罰沒機制與權益抑制的動態演化}
圖 \ref{fig:stake_evolution} 記錄了 300 輪內節點權益的動態變化,這不僅反映了系統的獎懲邏輯,更揭示了惡意節點在攻擊過程中的資源損耗特徵。

\textbf{1. 台階式下降的制裁特徵:} 觀察圖 \ref{fig:stake_evolution} 可以發現,惡意節點的平均權益並非線性遞減,而是呈現顯著的「台階式下降」。這種現象對應了本研究異步審計機制觸發 Slashing 的具體時點: \begin{itemize} \item \textbf{IID 環境:} 在第 90 輪與第 229 輪發生兩次大幅度的權益減損,最終降至誠實節點的 0.56 倍。 \item \textbf{Non-IID 環境:} 在第 90、216 與 261 輪分別觸發制裁,導致其權益在第 300 輪時僅剩誠實節點的 0.43 倍。 \end{itemize} 每一次「台階」的出現,都代表一次成功的惡意行為攔截與經濟懲罰。

\textbf{2. 經濟資本的不可逆損耗:} 雖然在 300 輪的觀測期內,攻擊發生的頻率未呈現明顯的早晚期差異,但惡意節點的經濟資源(Stake)已處於持續枯竭狀態。由於本系統採用基於權益的角色選拔機制,攻擊者每次發動攻擊都面臨著喪失「治理資本」的風險。

\textbf{3. 長期治理安全性的推論:} 儘管短期內攻擊者仍能憑藉剩餘權益參與競爭,但 0.43--0.56 倍的權益差距已構成實質性的進入門檻。 \begin{itemize} \item \textbf{先行者優勢轉移:} 誠實節點透過穩定訓練持續累積權益,擴大了與惡意節點的貧富差距。 \item \textbf{攻擊難度遞增:} 隨著訓練輪數繼續增加,惡意節點若要再次達成「委員會佔領」所需的席位,其權益權重將顯得捉襟見肘。 \end{itemize} 這種「台階式」的權益縮減證明了本機制能有效剝奪攻擊者的治理資源,從經濟層面限制了惡意行為的擴張潛力。

\subsection{長期賽局中的經濟嚇阻力分析}
\label{sec:long_term_deterrence}

為了驗證本研究提出的防禦機制在長期運作下的穩定性與嚇阻效果,我們將實驗模擬輪數擴展至 2000 輪。圖 \ref{fig:stake_2000} 展示了長期賽局下的權益動態變化,這些數據揭示了兩種機制在經濟誘因設計上的根本差異。

\begin{figure}[htbp]
    \centering
    \includegraphics[width=1.0\linewidth]{figures/experiments/mnist_results_stack_comparison_2000_round.png}
    \caption{2000 輪長期模擬下的權益動態比較}
    \label{fig:stake_2000}
\end{figure}

\subsubsection*{1) BlockDFL 的財富固化與持續威脅}

\textbf{強者恆強的馬太效應:}
在 BlockDFL 的長期模擬中,我們觀察到顯著的財富固化現象。數據顯示,攻擊者的平均權益在約 250 輪後,便穩定維持在誠實節點的 1.2 倍左右。這種 20\% 的權益優勢源於該機制缺乏有效的負向反饋迴路(Negative Feedback Loop)。一旦攻擊者透過初期優勢累積了較高的權益,其被選入委員會並獲得獎勵的機率便隨之提升,進而鞏固其經濟地位。

\textbf{高頻率的治理失效:}
這種權益優勢直接轉化為對系統治理權的掌控。在總計 2000 輪的模擬中,惡意節點成功攻佔委員會多數高達 84 次。這意味著在 BlockDFL 架構下,攻擊者不僅能長期存活,更能平均每 24 輪就發動一次成功的委員會劫持,形成持續性的安全漏洞。

\subsubsection*{2) 本研究方法的經濟嚇阻與邊緣化效應}

\textbf{不對稱的攻擊風險:}
相較之下,本研究方法展現了極強的經濟嚇阻力。在相同的 2000 輪測試中,惡意節點僅成功佔領委員會 5 次。這巨大的差異(84 次對 5 次)證明了引入罰沒機制後,攻擊者的期望收益被大幅壓縮,迫使其在大部分時間必須保持誠實以避免資產歸零。

\textbf{攻擊者的經濟致死螺旋:}
觀察圖 \ref{fig:stake_2000} 的第 909 輪可發現一個具決定性的轉折點:攻擊者在發動第五次攻擊後隨即受到異步審計機制的制裁 (Slashing),導致其平均權益瞬間暴跌至誠實節點的 22.6\%。

\textbf{永久性的治理排除:}
這一經濟重創產生了長期的邊緣化效果。在隨後的 1091 輪(超過總時長的一半)中,攻擊者因權益基礎過低,徹底失去了競爭委員會席次的能力,再也無法成功發動任何一次佔領攻擊。這項結果有力地證實了本系統能有效將一次性的攻擊失敗轉化為永久性的治理排除,從而確保系統在長期演化中趨向於「誠實者主導」的穩定態。

\section{效率與可擴展性分析}
\label{sec:efficiency}
\subsection{系統開銷與安全性需求對比}

為了評估系統在極端壓力下的效能表現,我們設定基準安全性要求為「受攻擊頻率(成功被劫持機率 $p$)低於 $1\%$」。在總節點數 $N=100$、惡意節點佔比 $f=30\%$ 的環境下,對比 BlockDFL 與本研究所提出的方案。

\begin{table}[htbp]
    \centering
    \caption{不同防禦機制在相同安全性水平 ($p < 0.01$) 下的複雜度對比 ($N=100, f=30\%$)}
    \label{tab:complexity_comparison_new}
    \makebox[\linewidth][c]{
    \renewcommand\arraystretch{1.3}
    \begin{tabular}{|l|l|l|l|}
        \hline
        \textbf{評估維度} & \textbf{傳統方案 (BlockDFL)} & \textbf{本研究方法 (Ours)} & \textbf{差異性質分析} \\
        \hline
        核心安全性模型 & 門檻安全性 (Threshold-based) & 經濟安全性 (Economic-based) & 信任多數 vs. 激勵相容 \\
        \hline
        設計哲學 & 悲觀併發控制 (Pessimistic) & 樂觀執行與異步審計 (Optimistic) & 預防 vs. 治理 \\
        \hline
        委員會大小 ($c$) & $c = 9$ & $c = 5$ & 資源佔用降低 $44.4\%$ \\
        \hline
        常態通訊複雜度 & $O(c^2) = 81$ & $O(c^2) = 25$ & 顯著降低日常頻寬負載 \\
        \hline
        安全維護成本 & 固定開銷 (Fixed Cost) & 條件式開銷 (Conditional Cost) & 靜態冗餘 vs. 動態防禦 \\
        \hline
        挑戰觸發代價 & 無 (N/A) & $O(p \cdot N^2)$ & 僅在檢測異常時觸發 \\
        \hline
        長期穩定狀態 & 固定於 $O(81)$ & 趨近於 $O(25)$ & 基於經濟嚇阻的博弈均衡 \\
        \hline
    \end{tabular}
    }
\end{table}

\subsection{複雜度差異與經濟安全性分析}

本小節針對表 \ref{tab:complexity_comparison_new} 中的複雜度模型進行深度分析,揭示兩者在處理安全性威脅時的本質差異:

\subsubsection*{1. 預防溢價與資源冗餘 (Pessimistic Overhead)}

BlockDFL 採用的是一種「預防性策略」。為了將攻擊成功率壓制在 $0.01$ 以下,系統必須在每一輪都維持高達 $c=9$ 的大型委員會進行 BFT 共識。即便在系統完全誠實運行的狀態下,這份 $O(81)$ 的高昂通訊代價也是不可減免的「預防溢價」。這種設計雖然安全,但缺乏對實際威脅程度的自適應性,導致資源長期處於冗餘狀態。

\subsubsection*{2. 基於罰沒機制的博弈均衡 (Economic Deterrence)}

相較之下,本研究方法將安全性保證由「事前攔截」轉為「事後追責」。透過引入罰沒機制 (Slashing Mechanism),我們成功改變了攻擊者的收益預期:

\begin{itemize}
    \item \textbf{激勵不相容 (Incentive Incompatibility):}雖然本研究採用的 $c=5$ 委員會在理論上被佔領的風險較高,但由於存在 $O(N^2)$ 的全量審計與全額罰沒風險,對於「理性攻擊者」而言,發動攻擊的期望收益將遠低於潛在的經濟損失。
    
    \item \textbf{$p$ 值的動態演化:}雖然在模擬環境中我們考慮了 $p < 0.01$ 的頻率,但在真實部署環境中,一旦首位攻擊者遭到處罰並被剔除,後續節點將因「經濟致死螺旋」的威懾而選擇誠實策略。因此,實際的挑戰觸發頻率 $p$ 將隨時間迅速遞減,使得系統的攤銷成本 (Amortized Cost) 極度趨近於 $O(c_{low}^2)$,從而在極低開銷下實現了與大型委員會等效的安全等級。
\end{itemize}

\section{本章小結}
\label{sec:eval_summary}

本章透過多維度的實驗設計與複雜度建模,全面驗證了所提出的「挑戰增強型委員會架構」在動態攻防環境下的優越性。實驗結果不僅支持了本文的核心假設,更揭示了該架構在去中心化治理中的深層潛力,具體總結如下:

\begin{itemize}
    \item \textbf{動態防禦的韌性:} 在 MNIST 任務的 IID 與 Non-IID 環境下,本研究方法均展現了極強的抗攻擊能力。數據顯示,相較於傳統固定委員會方案(BlockDFL),本架構將受擊頻率降低了約 80\%,並將系統的「最低不可用率」從 20\% 大幅壓制至 5\% 以下。這證明了挑戰機制能有效彌補小型委員會在即時防禦上的不足,確保模型訓練過程的連續性與穩定性。

    \item \textbf{經濟治理的有效性:} 長期賽局實驗(2000 輪)證實,引入罰沒機制(Slashing)能對惡意行為產生實質性的經濟嚇阻。透過追蹤權益演化發現,攻擊者的治理資本會因挑戰觸發而陷入「致死螺旋」,最終其權益佔比降至誠實節點的 22.6\%,達成永久性的治理排除。此結果說明了「挑戰增強」不只是技術層面的補救,更是一種從經濟誘因上根除惡意行為的治理手段。

    \item \textbf{效率與安全性的雙贏:} 複雜度對比分析顯示,在相同的安全性邊界($p < 0.01$)要求下,本架構成功打破了安全性與通訊開銷的強耦合關係。透過解耦共識流程,系統在常態下僅需維持 $c=5$ 的輕量級運作($O(c^2) = 25$),相較於必須維持 $c=9$ 的傳統方案($O(c^2) = 81$),顯著降低了系統整體的通訊冗餘。
\end{itemize}

綜上所述,實驗數據有力地支撐了本文論點:「挑戰增強型委員會架構」能以極低的常態通訊成本,換取等同甚至優於大型委員會的安全保證,為大規模區塊鏈聯邦學習的部署提供了一條具備高擴展性的技術路徑。

\end{ZhChapter}
