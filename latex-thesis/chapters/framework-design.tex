\begin{ZhChapter}

\chapter{挑戰增強型委員會架構 (Challenge-Augmented Committee Architecture)}
\label{chap:framework}

區塊鏈聯邦學習系統在追求去中心化安全性的同時,往往面臨著執行效率的嚴峻挑戰,而傳統拜占庭容錯共識機制雖能提供強大的安全保證,其高昂的通訊成本卻難以適應需要頻繁迭代更新的機器學習場景。第 \ref{chap:threat-model} 章的威脅分析揭示了現有委員會機制的根本缺陷:小規模委員會雖然能夠顯著降低通訊複雜度,但其固有的集中化特性使得理性攻擊者能夠透過漸進式的權益累積來逐步控制驗證權力,而現有的防禦機制過度依賴「誠實多數假設」,缺乏對策略性攻擊者的有效威懾。為了突破這一困境,本章提出「挑戰增強型委員會架構」(Challenge-Augmented Committee Architecture, CACA),該架構建立在第 \ref{sec:blockdfl-baseline} 節所定義的 BlockDFL 委員會模型之上,透過引入異步審計機制與內部罰沒協議,實現了從傳統「門檻安全性」向「經濟安全性」的典範轉移。

本架構的核心設計哲學在於認識到聯邦學習與金融交易系統在本質上的根本差異,這一認識為效率與安全的重新平衡提供了理論基礎。金融交易系統要求每一筆交易都必須具備即時的、不可逆的正確性保證,因為任何錯誤都可能導致資產的永久損失,這種特性迫使傳統區塊鏈系統必須在每次狀態變更前達成全網共識。然而,機器學習過程本身具備天然的抗噪性與自我修復能力,模型參數在訓練過程中的微小偏差通常不會導致災難性的後果,而是能夠透過後續的訓練迭代逐步修正。CACA 正是基於這一洞察,將安全性驗證從同步的阻塞式流程轉變為異步的非阻塞式審計機制,成功地將效率優化與安全保障解耦,使得系統能夠在正常情況下維持極高的執行效率,同時保留在異常情況下動員全網資源進行仲裁的能力。更重要的是,透過引入經濟懲罰機制,本架構從根本上重塑了攻擊者的理性決策空間,使得任何試圖操縱委員會共識的行為都將面臨遠超其潛在收益的經濟損失,從而消除了發動攻擊的經濟誘因。

本章的結構安排如下:首先在 \ref{sec:arch_overview} 節中概述 CACA 相對於 BlockDFL 的架構創新,闡明挑戰者角色與異步審計機制如何嵌入現有的委員會流程;接著在 \ref{sec:async_audit} 節深入探討異步審計與究責機制的運作原理,包括挑戰流程的觸發條件、仲裁機制的執行邏輯,以及「僅懲罰不回滾」策略的設計考量;隨後在 \ref{sec:security_guarantee} 節論證雙層信任模型如何提供等同於全網共識的安全保障;在 \ref{sec:efficiency_analysis} 節透過通訊複雜度分析與概率模型推導,量化評估本架構的效率優勢;最後在 \ref{sec:incentive_mechanism} 節探討激勵機制的經濟學基礎,說明如何透過罰沒與獎勵的精心設計實現激勵相容性。


\section{系統架構概覽}
\label{sec:arch_overview}

挑戰增強型委員會架構的設計目標在於建立一個既具備經濟安全性又能保持高執行效率的去中心化學習平台,而這一目標的實現建立在對現有 BlockDFL 架構的繼承與創新之上。如第 \ref{sec:blockdfl-baseline} 節所詳述,BlockDFL 透過角色分離的設計理念,將參與者劃分為更新提供者、聚合者與驗證者三種角色,並透過權益加權的隨機選舉機制決定每輪的角色分配,這種設計在效率與基本安全性之間取得了當時文獻中的最佳平衡。CACA 完整保留了 BlockDFL 的訓練流程與角色定義,包括更新提供者的本地訓練職責、聚合者的提案生成流程,以及驗證委員會的 Krum 評分與 PBFT 共識機制,這些經過驗證的設計元素構成了本架構運作的基礎框架。然而,CACA 的核心創新在於引入了第四種角色——挑戰者,以及與之配套的異步審計機制,這一創新從根本上改變了系統的安全性保障方式,將防禦策略從「事前預防」轉向「事後追責」。

圖 \ref{fig:caca_arch} 展示了 CACA 的完整運作流程,清晰呈現了挑戰機制如何嵌入現有的委員會共識流程。在每一輪次的正常運作中,系統首先根據前一區塊的雜湊值進行動態角色分配,隨後更新提供者執行本地訓練並將結果提交給聚合者,聚合者生成提案後交由驗證委員會進行 Krum 評分與 PBFT 投票,這一流程與 BlockDFL 完全一致。關鍵的差異出現在共識達成之後:在 BlockDFL 中,委員會的決策即為最終決策,系統缺乏對委員會潛在惡意行為的事後追責能力;而在 CACA 中,委員會達成共識後系統立即執行模型更新(即時執行策略),但同時開啟了一個異步的審計窗口,允許任何持有足夠質押的節點作為挑戰者對委員會的決策進行事後驗證。這種「先執行後審計」的設計哲學使得系統能夠在絕大多數正常情況下以最小的通訊開銷快速完成模型更新,同時保留了在檢測到異常行為時啟動全網仲裁的能力。

\begin{figure}[htbp]
    \centering
    \includegraphics[width=0.9\textwidth]{figures/utils/Challenge-Augmented-Committee-Architecture.drawio.png}
    \caption{Challenge-Augmented Committee Architecture (CACA) 系統架構與工作流程圖}
    \label{fig:caca_arch}
\end{figure}

挑戰者角色的設計體現了 CACA 對去中心化監督的核心承諾,這一角色向所有持有足夠質押的節點開放,而非僅限於特定的特權群體。挑戰者的職責在於持續監聽鏈上資料,獨立重新執行 Krum 演算法的運算,並將計算結果與委員會選定的全域更新進行比對。由於 Krum 演算法是一個完全確定性的數學運算,給定相同的輸入必然產生相同的輸出,因此委員會無法透過資訊不對稱來掩蓋其惡意行為,任何偏離正確結果的決策都將被挑戰者精確識別。一旦挑戰者發現委員會選定的結果與正確的 Krum 運算答案存在不一致,便可以質押規定金額的押金發起挑戰交易,觸發全網仲裁機制。這種開放式的監督設計本質上將監督權力從少數委員會成員民主化到了整個網路,創造了一個「人人都是潛在監督者」的環境,確保了即使委員會被惡意控制,攻擊行為也能夠被及時發現並受到懲罰。

演算法 \ref{alg:caca_execution} 與演算法 \ref{alg:caca_challenge} 分別以形式化的方式呈現了 CACA 的即時執行協議與異步挑戰機制。即時執行協議描述了從角色分配到模型更新的完整流程,其核心特徵在於委員會達成共識後立即提交全域模型更新,無需等待任何額外的確認期。這種設計選擇體現了對系統「活性」的優先保障,只要委員會能夠達成共識,系統就能夠持續前進。異步挑戰機制則作為系統的安全後盾在背景中持續運作,其核心邏輯在於:挑戰者持續驗證委員會決策的正確性,一旦發現異常便發起挑戰,觸發全網重新驗證;若惡意行為被確認,系統執行罰沒操作沒收惡意節點的質押金,但值得注意的是,已經提交的模型更新不會被回滾。這種「僅懲罰不回滾」的策略是 CACA 設計中的重要考量,其理論基礎將在下一節詳細闘述。

\begin{algorithm}[t]
\caption{CACA Execution Protocol (Instant Update)}
\label{alg:caca_execution}
\begin{algorithmic}[1]
\Require Current Round $r$, Total Stake Weighted Nodes $\mathcal{N}$
\Ensure Updated Global Model $w_{r+1}$
\vspace{0.1cm}
\State \textbf{Role Assignment:} 
\State Blockchain selects $\mathcal{V}$ (Committee), $\mathcal{A}$ (Aggregators), $\mathcal{U}$ (Update Providers) from $\mathcal{N}$ based on stake and randomness.
\vspace{0.1cm}
\State \textbf{Training \& Aggregation:}
\State Each $u \in \mathcal{U}$ trains using $w_r$, broadcasts updates to $\mathcal{A}$.
\State Each $a \in \mathcal{A}$ aggregates updates into proposal $p_a$, sends to $\mathcal{V}$.
\vspace{0.1cm}
\State \textbf{Consensus \& Update:}
\State $\mathcal{V}$ runs Krum on all proposals $\{p_a\}$.
\State $\mathcal{V}$ votes on the best proposal via PBFT.
\State Commit $w_{r+1}$ to blockchain \textbf{immediately}.
\State Distribute rewards to $\mathcal{U}, \mathcal{A}, \mathcal{V}$.
\end{algorithmic}
\end{algorithm}

\begin{algorithm}[t]
\caption{Asynchronous Challenge Mechanism (Slash-Only)}
\label{alg:caca_challenge}
\begin{algorithmic}[1]
\Require Challengers $\mathcal{C}$
\Ensure Punishment for Malicious Acts
\vspace{0.1cm}
\For{each Challenger $c \in \mathcal{C}$}
    \State $c$ retrieves committee inputs and re-executes Krum.
    \If{$c$ detects outcome mismatch with $w_{r+1}$}
        \State $c$ posts \textbf{Challenge Transaction} with deposit.
        \State \textbf{Arbitration Triggered:} All nodes re-verify.
        \If{Malicious Consensus Confirmed}
            \State \textbf{Burn/Slash} stake of malicious $\mathcal{V}$.
            \State Reward Challenger $c$ and all nodes.
            \State \textit{// Note: Model $w_{r+1}$ is NOT reverted.}
        \EndIf
        \State \textbf{Exit Loop}.
    \EndIf
\EndFor
\end{algorithmic}
\end{algorithm}


\section{異步審計與究責機制}
\label{sec:async_audit}

異步審計機制是 CACA 架構中最具創新性的設計要素,其核心理念在於將傳統區塊鏈系統中同步驗證與即時執行之間的緊密耦合關係予以解構,從而在不犧牲長期安全性的前提下最大化系統的執行效率。傳統的拜占庭容錯系統要求在每次狀態變更之前必須達成全網共識,這種「悲觀併發控制」的設計哲學雖然能夠提供強大的即時正確性保證,卻也導致了系統吞吐量與延遲性能的嚴重退化。CACA 則採用了「樂觀執行」的設計哲學,允許系統在委員會達成共識後立即更新模型,而將嚴格的正確性驗證推遲到異步的背景審計流程中進行。這種設計選擇的理論基礎在於認識到聯邦學習與金融交易在容錯需求上的本質差異:金融交易的錯誤是不可逆的資產損失,而機器學習的偶發偏差則能夠透過後續訓練迭代逐步修正,這種固有的自我修復能力為樂觀執行策略提供了安全邊際。

挑戰流程的設計確保了即使委員會的決策存在問題,這些問題也能夠被及時發現並受到適當的懲罰,而這種事後究責能力正是 CACA 實現經濟安全性的關鍵。挑戰流程的觸發條件相當明確且易於驗證:挑戰者透過持續監控鏈上的公開資料,獲取每一輪次中所有聚合者提交的提案以及委員會最終選定的全域更新,隨後在本地重新執行 Krum 演算法,計算出理論上應該被選中的最優提案,並將其與委員會實際選定的結果進行比對。若兩者不一致,則意味著委員會的決策過程存在問題,無論是由於計算錯誤還是惡意操縱,都構成了發起挑戰的充分理由。挑戰者提交挑戰交易時必須附帶規定金額的質押金,這筆質押金的設計具有雙重目的:一方面防止惡意節點透過大量無效挑戰來發動拒絕服務攻擊,另一方面為成功的挑戰者提供經濟激勵,使得監督委員會行為成為一項有利可圖的活動。

當挑戰交易被提交到區塊鏈後,系統進入仲裁階段,這是整個挑戰機制中最為關鍵的環節。智能合約首先鎖定相關的質押金,包括挑戰者的押金以及被挑戰的委員會成員的質押,隨後調取該輪次中鏈上緩存的所有聚合提案資料,這些資料在委員會共識階段就已經被完整地記錄在區塊鏈上,確保了仲裁過程的資料完整性與不可篡改性。系統隨即觸發全網仲裁機制,所有驗證節點都被要求重新執行 Krum 演算法的運算,這個過程本質上是將原本由小委員會執行的驗證任務擴展到了全網範圍,從而將安全性等級提升到了與全網 PBFT 共識相當的高度。全網驗證者透過 PBFT 協議對仲裁結果進行投票,若超過三分之二的節點確認委員會的決策確實存在錯誤,則挑戰成立,系統將執行罰沒操作沒收惡意委員會成員的全額質押金,並將部分罰沒資金分配給挑戰者作為獎勵。

當仲裁確認委員會存在惡意行為時,CACA 採用「僅懲罰不回滾」的處置策略,這一設計選擇基於對聯邦學習系統特性的深刻理解。機器學習模型具備顯著的自我修復能力,即使某一輪次的更新受到惡意操縱而包含了有偏差的梯度資訊,後續輪次中來自誠實節點的正確更新也能夠逐步抵銷這種負面影響,使模型重新收斂到正確的方向,這種特性在第 \ref{chap:evaluation} 章的實驗中將得到驗證。相對地,若選擇回滾模型狀態,則從被攻擊的輪次開始之後所有輪次的訓練成果都將被作廢,考慮到聯邦學習通常需要經歷數百甚至數千個訓練輪次,這種回滾將造成極為嚴重的運算資源浪費。更重要的是,全網仲裁的時間延遲意味著當仲裁最終判定某個早期輪次存在問題時,該輪次的影響很可能已經透過後續的正常訓練被大幅稀釋,此時強行回滾不僅缺乏實質意義,反而會破壞系統訓練過程的連續性。因此,CACA 將處置重點放在對惡意行為的經濟懲罰而非對歷史狀態的修正上,透過高額的質押金罰沒來建立強大的經濟威懾力,使得攻擊在經濟上變得完全不理性。


\section{安全性保證}
\label{sec:security_guarantee}

CACA 架構的安全性建立在一個精心設計的雙層信任模型之上,該模型透過巧妙地分配不同層級的安全職責,成功地在維持高效率的同時提供了等同於全網共識的安全保障。傳統的區塊鏈系統通常採用單一層級的信任假設,要求每一次狀態變更都必須經過全網共識的嚴格驗證,這種設計雖然能夠提供強大的安全保證,但其高昂的通訊成本使其難以應用於需要頻繁更新的應用場景。CACA 透過引入分層信任的概念,將效率優化與安全保障解耦,使得系統能夠根據實際威脅的性質動態調整其安全等級,在正常情況下以小委員會的效率運行,而在異常情況下能夠迅速升級至全網共識的安全等級。

雙層信任模型的第一層是檢測層,其採用了極為寬鬆但極其有效的「1-of-N 誠實假設」,這個假設的含義是只要全網 $N$ 個參與節點中存在至少一個誠實節點願意擔任挑戰者的角色,任何委員會層級的惡意行為就能夠被成功揭露。這種假設的寬鬆程度遠超過傳統拜占庭容錯系統所要求的「三分之二誠實節點」假設,因為它僅需要單一誠實節點的存在而非多數誠實節點的協調行動。從概率角度來看,在一個擁有數百或數千個參與者的大型網路中,所有節點同時選擇沉默或串謀的可能性極其微小,幾乎可以視為不可能事件。檢測層的設計巧妙地利用了區塊鏈系統的資料透明性特質,由於所有聚合提案都被完整地記錄在鏈上,任何節點都能夠獨立地重新執行驗證計算,這使得委員會的惡意行為無法被隱藏或掩蓋,攻擊者即使成功控制了當前輪次的整個委員會,也無法阻止其他節點訪問相同的資料並發現異常。

雙層信任模型的第二層是仲裁層,其採用了更為嚴格但同樣標準的「全網三分之二誠實假設」,當挑戰被發起並進入仲裁階段後,最終的判決權力從小委員會回歸到全網範圍。這個階段的安全假設要求網路中誠實節點的數量必須超過總節點數的三分之二,即 $N_{total} > 3f$,其中 $f$ 為惡意節點的上限數量,這是幾乎所有拜占庭容錯共識協議的標準假設,也是區塊鏈系統普遍依賴的安全基礎。在仲裁階段,所有參與驗證的節點透過 PBFT 協議對挑戰的正當性進行投票,只有當超過三分之二的節點確認委員會確實存在錯誤時,挑戰才會被判定為成立,這種高門檻的設計確保了仲裁結果的可靠性,防止了錯誤挑戰或惡意挑戰對系統造成的干擾。

這兩層信任機制的結合創造了一個強大而靈活的安全框架,其核心優勢在於顯著提高了成功攻擊所需的資源投入。若攻擊者希望發動一次完整的攻擊並確保不被懲罰,其必須同時滿足兩個極為苛刻的條件:第一個條件是收買當前輪次委員會中超過三分之二的成員以確保其惡意提案能夠透過委員會的 PBFT 共識;第二個條件是收買或壓制全網足夠數量的節點以確保沒有任何誠實節點會發起挑戰,或者即使有挑戰發起也能在仲裁階段控制超過三分之一的投票權以阻擋共識達成。第二個條件的達成難度遠超第一個,因為在檢測層面攻擊者面臨的是「1-of-N 誠實假設」的挑戰,要確保沒有任何節點發起挑戰,攻擊者理論上需要控制或買通全部 $N$ 個可能的挑戰者,這在大型網路中幾乎是不可能完成的任務。

將這兩個條件的成本累加,我們可以得出總攻擊成本的數學表達,設單個委員會成員的平均質押額為 $s_c$,委員會規模為 $C$,則控制委員會所需的成本約為 $\frac{2}{3}C \cdot s_c$;設全網單個節點的平均質押額為 $s_n$,全網節點總數為 $N_{total}$,則在仲裁階段阻擋共識所需的成本約為 $\frac{1}{3}N_{total} \cdot s_n$,總攻擊成本為這兩者之和,即 $Cost_{total} = \frac{2}{3}C \cdot s_c + \frac{1}{3}N_{total} \cdot s_n$。關鍵的觀察在於,雖然 CACA 使用了小委員會來提升效率,但其安全性並未隨之降低到僅依賴小委員會的水平,透過異步挑戰機制的引入,系統的安全性實質上由全網規模 $N_{total}$ 決定而非委員會規模 $C$,這意味著攻擊成本從原本單純控制小委員會的 $O(C)$ 量級大幅提升到了需要控制全網的 $O(N_{total})$ 量級,實現了安全性的顯著擴展,從而優雅地解決了去中心化系統中效率與安全之間的經典兩難困境。


\section{效率分析}
\label{sec:efficiency_analysis}

為了全面評估 CACA 架構的實際運作效率,本節透過嚴格的通訊複雜度分析與概率模型推導,論證該架構如何在理論層面實現效率與安全的最佳平衡。通訊複雜度是分散式系統性能的核心指標之一,它直接決定了系統的吞吐量、延遲以及可擴展性,在區塊鏈聯邦學習的場景中尤為重要,因為模型參數的傳輸往往涉及大量的資料交換,而共識協議又要求多輪的訊息往返。傳統的全網 PBFT 共識機制雖然能夠提供強大的拜占庭容錯能力,但其通訊複雜度呈現 $O(N^2)$ 的二次方增長特性,當網路規模從 10 個節點擴展到 100 個節點時,通訊成本將增長約 100 倍而非線性的 10 倍,這種指數級的增長在大規模網路中成為了嚴重的性能瓶頸。

BlockDFL 等先前研究提出的固定小委員會方案透過將驗證職責限制在一個規模為 $C$ 的小型委員會內,成功將通訊複雜度降低到 $O(C^2)$,然而這種方法面臨著效率與安全的兩難困境:若要維持足夠的安全性就必須使用較大的委員會,但這又會削弱效率優勢。CACA 架構透過引入異步挑戰機制成功突破了這一困境,其通訊複雜度特性需要分兩種情況討論。在正常運作情況下,當委員會誠實執行其職責且沒有挑戰發起時,系統的通訊複雜度與固定小委員會方案完全相同,均為 $O(C^2)$,關鍵的區別在於 CACA 能夠安全地使用比 BlockDFL 更小的委員會規模,因為異步挑戰機制提供了額外的安全保障。在異常情況下,當挑戰被發起並觸發全網仲裁時,系統的通訊複雜度會暫時上升到 $O(C^2) + O(N^2)$,但這種高成本狀態僅在極少數情況下出現而非每一輪次都必須承擔。

為了量化分析系統的期望通訊複雜度,我們引入挑戰發生的概率 $p$,這個概率代表了在任意給定輪次中系統需要進行全網仲裁的可能性。在理性行為假設下,由於挑戰機制所帶來的高額經濟懲罰,潛在的攻擊者會意識到發動攻擊的期望收益為負值,因此傾向於選擇誠實行為,這意味著在均衡狀態下挑戰發生的概率 $p$ 應該趨近於零。基於這個概率,系統的期望通訊複雜度可表示為:
\begin{equation}
E[Comm] = (1-p) \cdot O(C^2) + p \cdot (O(C^2) + O(N^2)) = O(C^2) + p \cdot O(N^2)
\end{equation}
當 $p \to 0$ 時,上式中的第二項趨近於零,整體的期望複雜度近似於 $O(C^2)$,這個結果表明在絕大多數時間裡 CACA 的通訊效率與最優化的小委員會方案相當,但同時享有由異步挑戰機制所提供的全網級別安全保障。

除了通訊複雜度分析,我們還需要透過概率模型來論證小委員會在配備異步挑戰機制後的安全性。這個分析的核心問題是:在給定網路規模 $N$ 和惡意節點比例 $f$ 的情況下,委員會被惡意控制的機率為何。由於委員會成員的選擇是一個無放回抽樣過程,委員會中惡意節點數量 $X$ 服從超幾何分佈,$X = k$ 的機率可以表示為:
\begin{equation}
P(X = k) = \frac{\binom{fN}{k} \binom{(1-f)N}{C-k}}{\binom{N}{C}}
\end{equation}
我們關心的安全性指標是惡意節點在委員會中佔據超過三分之二席位的機率,因為這是 PBFT 共識機制的臨界點,若惡意節點數量達到或超過 $\lfloor 2C/3 \rfloor + 1$,則攻擊者能夠控制委員會的共識結果。因此,委員會被惡意控制的風險概率 $P_{mal}$ 可以表示為:
\begin{equation}
P_{mal} = P(X \ge \lfloor 2C/3 \rfloor + 1) = \sum_{k=\lfloor 2C/3 \rfloor + 1}^{C} \frac{\binom{fN}{k} \binom{(1-f)N}{C-k}}{\binom{N}{C}}
\end{equation}

為了具體理解這個概率模型的含義,考察一個實際的數值案例:假設驗證者總池規模 $N = 100$,網路中惡意節點的比例 $f = 0.3$,即存在 30 個惡意節點和 70 個誠實節點,這是一個相對極端的假設,因為 30\% 的惡意比例已經接近大多數拜占庭容錯系統所能容忍的上限。當委員會規模 $C=5$ 時,惡意節點需要至少佔據 4 個席位才能達到控制閾值,透過超幾何分佈的計算這種情況發生的機率約為 2.74\%;當委員會規模增加到 $C=9$ 時,惡意節點需要佔據至少 7 個席位才能達到三分之二的控制閾值,此時風險機率驟降至約 0.28\%,這個數值已經低於許多實際系統所設定的風險容忍度。這些數據揭示了一個重要的洞察:即使在相當高的惡意節點比例下,只需要一個規模適中的委員會(如 9 到 13 個成員)就能將被惡意控制的風險壓制到極低的水平,而這個風險水平是在沒有考慮異步挑戰機制的情況下計算的,當我們將挑戰機制納入考量後,即使這低於 1\% 的概率事件真的發生,攻擊者也將在事後面臨全額質押金的罰沒,從而使得攻擊在經濟上變得不可行。

這個概率分析的結論具有深遠的實踐意義,它證明了 CACA 能夠安全地使用極小的委員會規模而不會顯著增加安全風險。相比之下,若要達到相同的安全保障水平,傳統的全網 PBFT 需要所有 100 個節點參與共識,其通訊複雜度為 $O(100^2) = O(10000)$,而 CACA 在使用 9 個成員的委員會時,通訊複雜度僅為 $O(9^2) = O(81)$,效率提升超過 100 倍。更進一步地,當網路規模擴大時這種效率優勢會變得更加顯著,若驗證者池增長到 $N=1000$,全網 PBFT 的複雜度將膨脹到 $O(1000000)$,而 CACA 依然可以使用相同規模的小委員會(因為概率分析顯示在更大的池中抽取相同規模的委員會風險反而會進一步降低),其複雜度保持在 $O(81)$ 的量級,這種可擴展性特質使得 CACA 特別適合應用於大規模的去中心化聯邦學習平台。


\section{激勵機制}
\label{sec:incentive_mechanism}

激勵機制是維持去中心化系統長期穩定運行的根本動力,其設計的優劣直接決定了系統能否在沒有中心化權威的情況下自發形成良性的治理秩序。在 CACA 架構中,激勵機制的設計遵循博弈論與機制設計理論的核心原則,旨在創造一個激勵相容的環境,使得誠實行為成為所有理性參與者的最優策略。與傳統的區塊鏈系統依賴持續增發代幣來支付安全成本不同,CACA 採用了一種更為可持續且經濟高效的方法,即透過對違規者的資產罰沒來支付審計與仲裁的相關費用,這種「懲罰驅動」的激勵模式既避免了通貨膨脹對代幣價值的長期侵蝕,又確保了安全成本由真正造成風險的行為者承擔而非由全體參與者分攤。

罰沒機制的核心設計理念在於建立一個極度不對稱的風險收益結構,使得攻擊行為在經濟上變得完全不理性。每個願意擔任驗證者或聚合者角色的節點都必須預先質押一定數量的代幣作為其誠實行為的保證金,這個質押金的規模被精心設定在一個足夠高的水平,確保其價值遠超過任何單次攻擊所能獲得的潛在收益。當節點被證實存在惡意行為時,系統將立即沒收其全額質押金,這種懲罰的嚴厲程度傳遞了一個明確的訊號:在 CACA 系統中任何作惡嘗試都將導致災難性的經濟損失,而這種損失是即刻的、確定的且不可逆轉的。相對地,誠實參與者雖然需要承擔質押金被鎖定的機會成本,但能夠獲得穩定且可預期的區塊獎勵,這種穩定收益的累積在長期內將遠超過任何一次性攻擊所能帶來的非法所得。

被罰沒的資金並非簡單地從系統中移除或銷毀,而是透過精心設計的分配機制來強化激勵相容性。資金分配的首要受益者是成功發起挑戰的挑戰者,他們將獲得罰沒金額中相當可觀的一部分作為獎勵,這種設計確保了監督委員會行為成為一項有利可圖的經濟活動,從而吸引足夠數量的節點願意投入資源進行持續的審計工作。挑戰者的獎勵必須足夠高以覆蓋其進行驗證計算的運算成本、質押挑戰押金的機會成本,以及承擔錯誤挑戰被反向懲罰的風險溢價,在實際設計中挑戰成功後的獎勵通常被設定為罰沒金額的 30\% 到 50\%。剩餘的罰沒資金則分配給全體誠實參與者,特別是那些在被攻擊輪次中提供了正確更新的訓練者以及積極參與了仲裁過程的驗證節點,這種廣泛的獎勵分配機制不僅補償了誠實節點因系統遭受攻擊而承受的潛在損失,更重要的是創造了一種集體監督的文化,使得每個參與者都有動力關注系統的整體健康狀況。

從長期均衡的角度來分析,CACA 的激勵機制創造了一個穩定且可持續的經濟生態。對於誠實節點而言,參與系統的期望收益來自於兩個管道:一是擔任驗證者、聚合者或訓練者時獲得的常規區塊獎勵,這是一種穩定且可預測的收入流;二是在極少數系統遭受攻擊時透過參與挑戰或仲裁而獲得的額外獎勵。關鍵在於這種誠實參與是低風險的,節點只需按照協議規則執行其職責就能確保獲得獎勵而不會面臨質押金損失的風險。相對地,對於潛在的攻擊者其決策邏輯則完全不同,發動一次成功的攻擊能夠帶來的收益是有限的,主要體現在該輪次中對模型更新方向的控制權,而這種控制權的價值在聯邦學習場景中往往並不高,因為單次的模型偏移很快就會被後續的誠實更新所修正。然而攻擊的成本卻是極為高昂的,不僅包括控制委員會所需的大量質押金投入,更包括一旦攻擊被發現後全額質押金的損失以及永久失去驗證者資格所帶來的未來收益流損失。

這種極度不對稱的風險收益結構是 CACA 激勵機制設計的核心成就,它不依賴於對參與者道德水平的假設,而是透過純粹的經濟邏輯來引導行為。更重要的是,這種懲罰機制成功打破了第 \ref{chap:threat-model} 章所描述的漸進式委員會佔領攻擊所依賴的正反饋循環,在沒有罰沒機制的系統中攻擊者可以透過操縱委員會來獲取不當獎勵進而增加其質押權重最終逐步掌控整個系統,而在 CACA 中任何作惡嘗試都會導致質押的減少而非增加,從而從根本上切斷了這種惡性循環的可能性,確保了系統長期治理的穩定性與公正性。這種激勵相容性確保了系統能夠在沒有中心化監管的情況下實現自我治理,為去中心化聯邦學習平台的長期穩定運作奠定了堅實的經濟基礎。


\section{本章小結}

本章提出的挑戰增強型委員會架構代表了區塊鏈聯邦學習系統設計理念的一次重要轉變,其核心創新在於透過異步審計與經濟懲罰機制的引入,成功地將傳統上互相衝突的效率與安全性目標統一到一個連貫的框架之中。CACA 建立在第 \ref{sec:blockdfl-baseline} 節所定義的 BlockDFL 委員會模型之上,完整保留了其經過驗證的訓練流程與角色定義,同時透過引入挑戰者角色與異步審計機制,實現了從傳統「門檻安全性」向「經濟安全性」的典範轉移。這種轉移並非透過在效率與安全之間尋求妥協而達成,而是透過重新思考安全性的實現方式,將同步驗證的即時成本轉化為異步審計的條件成本,從而在不犧牲長期安全保障的前提下最大化了系統的執行效率。

CACA 的安全性保障建立在雙層信任模型的堅實基礎之上,該模型巧妙地利用了聯邦學習系統固有的容錯特性以及區塊鏈網路的資料透明性。透過將檢測職責開放給所有願意參與的節點,系統將監督門檻降低到了極致,只需存在單一誠實節點願意執行挑戰,任何委員會層級的惡意行為都將無所遁形。通訊複雜度分析揭示了 CACA 架構的效率優勢,在絕大多數正常運作的情況下系統的通訊成本維持在小委員會共識的 $O(C^2)$ 量級,相較於全網 PBFT 的 $O(N^2)$ 複雜度實現了數量級的降低。激勵機制的創新設計則確保了這種架構不僅在理論上可行,在實踐中也能夠長期穩定運作,透過建立極度不對稱的風險收益結構,CACA 使得誠實行為成為所有理性參與者的最優策略,而任何試圖操縱系統的行為都將面臨災難性的經濟後果,從根本上打破了漸進式委員會佔領攻擊所依賴的正反饋循環。

總體而言,本章所提出的 CACA 架構為區塊鏈聯邦學習系統提供了一條突破效率與安全兩難困境的可行路徑,它不僅解決了現有系統面臨的技術挑戰,更重要的是提供了一套完整的理論框架可以指導未來更多去中心化機器學習應用的設計。然而,理論分析終究需要實證驗證來支撐其有效性,下一章將透過多維度的模擬實驗,在各種攻擊場景下驗證 CACA 架構的實際性能表現,特別是其在面對第 \ref{chap:threat-model} 章所描述的漸進式委員會佔領攻擊時的防禦能力與系統穩健性,從而為本研究的理論主張提供實證基礎。


\end{ZhChapter}