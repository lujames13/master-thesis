\begin{ZhChapter}

\chapter{系統架構設計 (System Architecture Design)}
\label{chap:framework}

區塊鏈聯邦學習系統在邁向大規模部署的過程中,始終面臨效率與安全性之間難以調和的張力。傳統拜占庭容錯共識機制固然能夠提供堅實的安全保證,但其伴隨的 $O(n^2)$ 通訊成本卻與機器學習場景中頻繁迭代更新的需求產生根本性衝突。第 \ref{chap:threat-model} 章的威脅分析已經揭示了現有委員會機制在面對理性攻擊者時的結構性缺陷:小規模委員會雖然顯著降低了通訊複雜度,但其固有的集中化特性為攻擊者提供了透過漸進式權益累積逐步掌控驗證權力的可乘之機,而現有防禦機制對「誠實多數假設」的過度依賴更使得系統缺乏對策略性攻擊者的有效威懾手段。為突破這一困境,本章提出「審計驅動型委員會 BlockDFL」(Audit-driven Committee BlockDFL, AC-BlockDFL),該架構建立在第 \ref{sec:blockdfl-baseline} 節所定義的 BlockDFL 委員會模型之上,透過引入異步審計機制與內部罰沒協議,實現從傳統「門檻安全性」向「經濟安全性」的典範轉移。

本架構的核心設計哲學源於對區塊鏈系統狀態最終性需求與聯邦學習迭代訓練特性之間關係的重新審視。金融交易系統對每一筆交易都要求即時且不可逆的正確性保證,因為任何錯誤都可能導致資產的永久損失,這種特性迫使傳統區塊鏈系統必須在每次狀態變更前達成全網共識。然而,AC-BlockDFL 認識到聯邦學習的訓練過程具備多輪迭代的特性,單一輪次中的偏差可透過後續訓練逐步修正,這為將安全性驗證從同步的阻塞式流程轉變為異步的非阻塞式審計機制提供了設計空間。在此基礎上,AC-BlockDFL 透過引入經濟懲罰機制從根本上重塑了攻擊者的理性決策空間,使得任何試圖操縱委員會共識的行為都將面臨遠超其潛在收益的經濟損失,從而在博弈論層面消除了發動攻擊的經濟誘因。

本章的結構安排如下:第 \ref{sec:arch_overview} 節概述 AC-BlockDFL 的系統架構與元件互動關係,闡明挑戰者角色與鏈下儲存整合如何嵌入現有的委員會流程;第 \ref{sec:protocol_flow} 節描述運作協議的逐步執行流程;第 \ref{sec:async_audit} 節深入探討異步審計與挑戰機制的運作原理,涵蓋挑戰觸發邏輯、內生動態質押模型,以及狀態最終性與不回滾策略的設計考量;第 \ref{sec:security_guarantee} 節以形式化的定理與證明論證雙層信任模型所提供的安全保障;第 \ref{sec:efficiency_analysis} 節透過通訊、運算與儲存三個維度的開銷分析,量化評估本架構的效率優勢。


\section{系統架構概覽}
\label{sec:arch_overview}

審計驅動型委員會 BlockDFL的設計目標在於建立一個兼具經濟安全性與高執行效率的去中心化學習平台,而這一目標的實現建立在對現有 BlockDFL 架構的繼承與創新之上。如第 \ref{sec:blockdfl-baseline} 節所詳述,BlockDFL 透過角色分離的設計理念,將參與者劃分為更新提供者、聚合者與驗證者三種角色,並透過權益加權的隨機選舉機制決定每輪的角色分配。AC-BlockDFL 完整保留了 BlockDFL 的訓練流程與角色定義,包括更新提供者的本地訓練職責、聚合者的提案生成流程,以及驗證委員會的 Krum \cite{blanchard2017machine} 評分與 PBFT 共識機制,這些經過驗證的設計元素構成了本架構運作的基礎框架。

AC-BlockDFL 的核心創新體現在三個層面的架構擴展。第一個層面是引入第四種角色,即「挑戰者」(Challenger),以及與之配套的異步審計機制。這一角色的設計遵循開放准入原則,任何網路參與者只要願意質押規定數額的代幣,即可在該輪次中承擔挑戰者的監督職責,這種開放式的准入設計確保了監督權力不會集中於少數節點之手,從而避免了在解決委員會信任問題的同時引入新的中心化風險。挑戰者的核心職責在於持續監聽鏈上資料,獨立重新執行 Krum 演算法的運算,並將運算結果與委員會選定的全域更新進行比對,由於 Krum 演算法是一個完全確定性的數學運算,給定相同的輸入必然產生相同的輸出,因此委員會無法透過資訊不對稱來掩蓋其惡意行為。

第二個層面是鏈下儲存架構的整合。考量到模型更新的資料量通常遠大於一般區塊鏈交易,AC-BlockDFL 將沉重的模型梯度與權重資料儲存於星際檔案系統 (InterPlanetary File System, IPFS) 之上,僅將資料的內容識別符 (Content Identifier, CID) 與相關元資料記錄於鏈上。這種設計將鏈上儲存複雜度從 $O(\text{ModelSize})$ 降至 $O(\text{HashSize})$,有效緩解了區塊鏈帳本膨脹的問題。為確保審計期間的資料可用性,參與節點在異步審計窗口 (Challenge Window) 的存續期間內持續釘選 (pin) 相關的 IPFS 資料,待審計窗口關閉且未發生挑戰後,節點即可解除釘選以釋放儲存空間。這種基於生命週期管理的儲存策略,在審計所需的資料可用性與長期儲存成本之間取得了合理的平衡。

第三個層面是「先執行後審計」的安全性保障模式。在 BlockDFL 中,委員會的決策即為最終決策,系統缺乏對委員會潛在惡意行為的事後追責能力。而在 AC-BlockDFL 中,委員會達成共識後系統立即執行模型更新,但同時開啟了一個異步的審計窗口,允許挑戰者對委員會的決策進行事後驗證。圖 \ref{fig:ac_blockdfl_arch} 展示了 AC-BlockDFL 的完整運作流程,清晰呈現了挑戰機制如何嵌入現有的委員會共識流程。這種設計使得系統能夠在絕大多數正常情況下以最小的通訊開銷快速完成模型更新,同時保留了在檢測到異常行為時啟動全網仲裁的能力,本質上將監督權力從少數委員會成員民主化到了整個網路。

\begin{figure}[htbp]
    \centering
    \includegraphics[width=0.9\textwidth]{figures/utils/Challenge-Augmented-Committee-Architecture.drawio.png}
    \caption{Audit-driven Committee BlockDFL (AC-BlockDFL) 系統架構與工作流程圖}
    \label{fig:ac_blockdfl_arch}
\end{figure}


\section{運作協議流程}
\label{sec:protocol_flow}

本節描述 AC-BlockDFL 在正常運作情境下的逐步執行流程,即系統未遭受攻擊時的「快樂路徑」(Happy Path)。演算法 \ref{alg:ac_blockdfl_execution} 以形式化的方式呈現了從角色分配到模型更新的完整協議,其核心特徵在於委員會達成共識後立即提交全域模型更新,無需等待任何額外的確認期,這種設計選擇體現了對系統「活性」(Liveness) 的優先保障。

\begin{algorithm}[!htbp]
\caption{AC-BlockDFL Execution Protocol (Instant Update)}
\label{alg:ac_blockdfl_execution}
\begin{algorithmic}[1]
\Require Current Round $r$, Total Stake Weighted Nodes $\mathcal{N}$
\Ensure Updated Global Model $w_{r+1}$
\vspace{0.1cm}
\State \textbf{Phase 1 --- Role Assignment:}
\State Blockchain selects $\mathcal{V}$ (Committee), $\mathcal{A}$ (Aggregators), $\mathcal{U}$ (Update Providers) from $\mathcal{N}$ based on stake-weighted randomness derived from previous block hash.
\vspace{0.1cm}
\State \textbf{Phase 2 --- Training \& Off-chain Storage:}
\State Each $u \in \mathcal{U}$ trains locally using $w_r$, broadcasts updates to $\mathcal{A}$.
\State Each $a \in \mathcal{A}$ aggregates updates into proposal $p_a$.
\State Each $a$ uploads $p_a$ to IPFS $\rightarrow$ obtains Content Identifier $\text{CID}_a$.
\State Each $a$ submits $\text{CID}_a$ and metadata to $\mathcal{V}$ via on-chain transaction.
\vspace{0.1cm}
\State \textbf{Phase 3 --- On-chain Consensus \& Instant Update:}
\State $\mathcal{V}$ retrieves proposals from IPFS using $\{\text{CID}_a\}$, verifies data availability.
\State $\mathcal{V}$ runs Krum scoring on all proposals $\{p_a\}$.
\State $\mathcal{V}$ votes on the best proposal via PBFT.
\State Commit $w_{r+1}$ to blockchain \textbf{immediately}.
\State Record winning $\text{CID}^*$ and voter identities on-chain.
\State Distribute rewards to contributing $\mathcal{U}, \mathcal{A}, \mathcal{V}$.
\vspace{0.1cm}
\State \textbf{Phase 4 --- Audit Window Opens:}
\State Asynchronous challenge period begins (see Section \ref{sec:async_audit}).
\State Participating nodes pin relevant IPFS data for the duration of the challenge window.
\end{algorithmic}
\end{algorithm}

協議的第一階段為角色分配。當新一輪訓練開始時,所有參與者根據最新區塊的雜湊值與當前權益分布,確定性地運算出本輪的角色分配結果,此運算過程僅依賴公開可驗證的鏈上資訊,任何參與者皆可獨立驗證角色分配的正確性而無需依賴中央協調者。第二階段涵蓋本地訓練與鏈下儲存,更新提供者在各自的私有資料集上執行模型訓練並將本地更新發送給聚合者,聚合者完成篩選與聚合運算後,將提案上傳至 IPFS 以取得內容識別符,隨後透過鏈上交易將此識別符與相關元資料提交給驗證委員會。這種將沉重資料負載置於鏈下的設計,確保了區塊鏈帳本僅記錄輕量級的雜湊參照,從而大幅降低鏈上儲存壓力。

第三階段是鏈上共識與即時更新,構成整個協議的核心環節。驗證委員會的成員透過鏈上記錄的 CID 從 IPFS 取得各聚合提案的完整內容,在確認資料可用性後執行 Krum 演算法進行提案評分,並依據評分結果透過 PBFT 協議進行投票表決。當某提案獲得超過三分之二驗證者的贊成票時,該提案被正式接受,對應的全域模型更新立即寫入區塊鏈,獲勝提案的 CID 與投票者身份同步記錄於鏈上,為後續可能的審計提供完整的可追溯資訊。獎勵隨即分配給對本輪全域模型更新有實質貢獻的更新提供者、聚合者與驗證者。第四階段標誌著異步審計窗口的開啟,系統進入背景監督狀態,參與節點在此期間持續釘選相關的 IPFS 資料以確保審計所需的資料可用性,挑戰機制的具體運作邏輯將在下一節詳述。


\section{異步審計與挑戰機制}
\label{sec:async_audit}

異步審計機制是 AC-BlockDFL 架構中最具創新性的設計要素,其核心理念在於將傳統區塊鏈系統中同步驗證與即時執行之間的緊密耦合關係予以解構。傳統的拜占庭容錯系統要求在每次狀態變更之前必須達成全網共識,這種「悲觀併發控制」的設計哲學雖然能夠提供強大的即時正確性保證,卻也導致了系統吞吐量與延遲性能的嚴重退化。AC-BlockDFL 則採用了「樂觀執行」的設計哲學,允許系統在委員會達成共識後立即更新模型,而將嚴格的正確性驗證推遲到異步的背景審計流程中進行。這種設計使得系統能夠在不犧牲長期安全性的前提下最大化執行效率,其安全邊際來自於經濟懲罰機制對理性攻擊者的威懾效果。本節將依序闡述挑戰流程的觸發邏輯、質押金額的內生動態定價模型,以及不回滾策略的設計考量。

\subsection{挑戰觸發邏輯}
\label{sec:challenge_trigger}

挑戰流程的觸發條件建立在確定性運算與區塊鏈資料透明性的交互作用之上,其設計確保了即使委員會的決策存在問題,這些問題也能夠被任何持有足夠質押的網路參與者及時發現。具體而言,挑戰者透過持續監控鏈上記錄的 CID 參照,從 IPFS 下載每一輪次中所有聚合者提交的提案完整內容,隨後在本地重新執行 Krum 演算法,運算出理論上應該被選中的最優提案,並將其與委員會實際選定的結果進行比對。由於 Krum 演算法的確定性特質保證了給定相同輸入必然產生唯一且一致的輸出,任何偏離正確結果的委員會決策都將被挑戰者精確識別,無論該偏離源於運算錯誤還是蓄意操縱。

演算法 \ref{alg:ac_blockdfl_challenge} 以形式化的方式呈現了異步挑戰機制的完整邏輯。當挑戰者偵測到委員會決策與正確 Krum 結果之間的不一致時,需提交一筆挑戰交易並附帶規定數額的質押金。這筆質押金的設計具有雙重目的:一方面防止惡意節點透過大量無效挑戰來發動拒絕服務攻擊,另一方面為成功的挑戰者提供經濟激勵,使得監督委員會行為成為一項有利可圖的經濟活動。挑戰交易被提交到區塊鏈後,智能合約自動鎖定相關質押金並觸發全網仲裁機制,所有驗證節點透過 IPFS 取得該輪次的完整提案資料並獨立重新執行 Krum 運算。全網驗證者隨後透過 PBFT 協議對仲裁結果進行投票,若超過三分之二的節點確認委員會的決策確實存在偏離,則挑戰成立,系統執行罰沒操作沒收所有在該輪次中對偏離結果投贊成票的惡意委員會成員之全額質押金。值得注意的是,已經提交的模型更新不會被回滾,此策略的設計考量將在第 \ref{sec:no_rollback} 節詳述。

\begin{algorithm}[!htbp]
\caption{Asynchronous Challenge Mechanism (Slash-Only)}
\label{alg:ac_blockdfl_challenge}
\begin{algorithmic}[1]
\Require Challengers $\mathcal{C}$, On-chain CID references, IPFS data store
\Ensure Punishment for Malicious Committee Acts
\vspace{0.1cm}
\For{each Challenger $c \in \mathcal{C}$}
    \State $c$ retrieves all proposal CIDs from on-chain records.
    \State $c$ downloads full proposal data $\{p_a\}$ from IPFS using CIDs.
    \State $c$ re-executes Krum algorithm on $\{p_a\}$.
    \If{$c$ detects outcome mismatch with committed $w_{r+1}$}
        \State $c$ posts \textbf{Challenge Transaction} with deposit $D_{\text{challenge}}$.
        \State \textbf{Arbitration Triggered:} All nodes download IPFS data via CID and re-verify.
        \If{Malicious Consensus Confirmed by $> 2/3$ of network}
            \State \textbf{Slash} full stake of colluding committee members $\mathcal{V}_{\text{mal}}$.
            \State Reward Challenger $c$ from slashed funds.
            \State Distribute remaining slashed funds to honest participants.
            \State \textit{// Note: Model $w_{r+1}$ is NOT reverted (see Section \ref{sec:no_rollback}).}
        \Else
            \State Forfeit Challenger $c$'s deposit $D_{\text{challenge}}$.
        \EndIf
        \State \textbf{Exit Loop}.
    \EndIf
\EndFor
\end{algorithmic}
\end{algorithm}

罰沒資金的分配遵循強化激勵相容性的設計原則。成功發起挑戰的挑戰者將獲得罰沒金額中相當可觀的比例作為獎勵,這種設計確保了監督活動在經濟上具有吸引力,從而維持足夠數量的節點願意投入資源進行持續審計。剩餘的罰沒資金則分配給在被攻擊輪次中提供了正確更新的訓練者以及積極參與仲裁過程的驗證節點,這種廣泛的獎勵分配機制不僅補償了誠實節點因系統遭受攻擊而承受的潛在損失,更重要的是創造了一種集體監督的文化,使得每個參與者都有動力關注系統的整體治理健康狀況。

\subsection{內生動態質押模型}
\label{sec:endogenous_staking}

質押金額的定價機制直接關係到挑戰機制能否在不同經濟環境下持續有效運作,是維繫系統經濟安全性的核心基石。根據去中心化系統的設計原則,所有核心參數都應當源自系統內部的可驗證資訊,而非仰賴外部基礎設施的資料饋送,因為任何形式的外部依賴都可能成為潛在的攻擊面或單點故障來源。以鏈上預言機引入代幣對法幣匯率來動態調整質押門檻為例,此種做法雖然在邏輯上直觀,卻顯著增加了系統的外部依賴程度,更使得質押機制本身暴露於預言機操控與報價延遲的風險之下。基於這一考量,AC-BlockDFL 的質押定價完全錨定於系統內部的經濟活動指標,透過將懲罰力度定義為當輪區塊獎勵的倍數,確保無論代幣的法幣計價如何波動,攻擊行為的相對經濟損失始終顯著高於其潛在的相對經濟收益,從而在無需外部資料源的情況下實現了機制的自我穩定性。

這一定價策略的關鍵在於準確辨識出攻擊者的根本經濟動機:在理性博弈假設下,發動委員會佔領攻擊所能獲取的最大即時利益,歸根結柢就是當輪的區塊獎勵 $R_{\text{round}}$。既然攻擊收益以 $R_{\text{round}}$ 為嚴格上界,那麼只要將懲罰規模同樣以 $R_{\text{round}}$ 為基準並乘以足夠大的倍數,便能在系統內部建立起一套自洽且穩定的經濟威懾結構。據此,本研究將惡意委員會成員遭受罰沒時的懲罰金額 $D_{\text{slash}}$ 定義為:
\begin{equation}
D_{\text{slash}} = k \times R_{\text{round}}, \quad k \gg 1
\label{eq:slash_amount}
\end{equation}
其中 $k$ 為安全倍數參數,其取值需確保即使攻擊者成功壟斷整輪的全部獎勵,罰沒損失仍遠超其潛在收益。在實驗配置中,每位驗證者的初始質押設定為 $100$ 單位,而單輪獎勵 $R_{\text{round}}$ 面向驗證者的分配約為 $1.0$ 單位,故 $k$ 的實質取值約為 $100$,意味著一次罰沒所造成的損失相當於一百輪正常獎勵的總和。面對如此極端不對稱的風險收益結構,理性攻擊者在進行成本效益評估時,必然會得出攻擊的預期淨收益為負的結論,作惡的經濟誘因也因此從根本上被消除。

挑戰者提交挑戰交易時所需質押的門檻 $D_{\text{challenge}}$ 同樣遵循內生定價的設計邏輯,其數值的設定需要同時滿足經濟可持續性與准入可及性這兩項看似對立的約束條件。就經濟可持續性而言,當挑戰失敗而質押被沒收時,沒收的資金應足以補償全網參與仲裁的運算成本,以防止惡意節點透過發起大量無效挑戰來消耗全網資源。設全網參與仲裁的節點數量為 $N_{\text{arb}}$,每個節點執行一次完整 Krum 驗證運算的邊際成本為 $\epsilon$,則門檻需滿足 $D_{\text{challenge}} \geq N_{\text{arb}} \cdot \epsilon$ 的基本邊界條件。在實務上,$D_{\text{challenge}}$ 可進一步設定為 $\alpha \times R_{\text{round}}$,其中 $\alpha$ 為平衡監督積極性與抗攻擊能力的調整係數。由於 $R_{\text{round}}$ 隨網路經濟活動規模自然增減,當獎勵池增大時,攻擊成本與挑戰成本將同步實現線性擴展,確保了懲罰力度始終能夠覆蓋潛在的區塊獎勵竊取收益,實現了質押機制在不同市場環境下的自適應調節。

\subsection{狀態最終性與不回滾策略}
\label{sec:no_rollback}

當仲裁確認委員會存在惡意行為時,AC-BlockDFL 採用「僅懲罰不回滾」的處置策略,即對惡意節點執行經濟罰沒但不撤銷已經提交的模型更新。這一設計選擇並非基於對機器學習模型固有魯棒性的樂觀假設,而是主要出於分散式系統穩定性與區塊鏈狀態最終性的考量。

從區塊鏈系統設計的角度而言,狀態回滾與帳本不可篡改性這一核心原則之間存在根本性衝突。區塊鏈的價值主張建立在每一個經共識確認的區塊都具備最終性 (Finality) 的保證之上,一旦允許因事後審計結果而回滾歷史區塊的狀態,便為長程攻擊 (Long-Range Attack) 等利用歷史重寫的攻擊向量開啟了空間。在聯邦學習的場景中,全網仲裁的時間延遲意味著當仲裁最終判定某個早期輪次存在問題時,該輪次之後可能已經累積了數十甚至數百個後續區塊,撤銷這些區塊將摧毀所有中間交易的最終性,對系統的可信賴性造成災難性的打擊。

從工程實踐的角度來看,全域模型狀態的回滾涉及極高的協調複雜度。回滾操作要求所有網路節點同步恢復至歷史狀態,並在此基礎上重新執行從被攻擊輪次開始的所有後續訓練,這不僅會造成大量運算資源的浪費,更需要設計複雜的分散式協調協議來確保所有節點一致地完成狀態回溯。考慮到聯邦學習通常需要經歷數百甚至數千個訓練輪次,回滾操作的沉沒成本與協調難度都將隨著被攻擊輪次與當前輪次之間的距離增長而急劇攀升,使其在實務上幾乎不可行。

因此,AC-BlockDFL 採用「前向修正」(Forward Correction) 的策略取代歷史回滾:透過對惡意行為者施加嚴厲的經濟懲罰來消除未來的攻擊誘因,同時依賴後續輪次中誠實節點的正常訓練來漸進式地修正模型軌跡的偏差。這種處置方式在本質上是一種工程權衡,以接受單次偶發性的模型精度波動為代價,換取帳本不可篡改性的完整保全與系統運作連續性的穩定維持。第 \ref{chap:evaluation} 章的實驗結果將從實證角度展示,在罰沒機制有效威懾理性攻擊者的前提下,系統在長期訓練過程中能夠自然收斂至與無攻擊環境相當的模型品質水準。


\section{安全性分析}
\label{sec:security_guarantee}

AC-BlockDFL 架構的安全性建立在一個精心設計的雙層信任模型之上,該模型透過將不同層級的安全職責分配給不同的參與者群體,成功地在維持高效率的同時提供了等同於全網共識的安全保障。傳統的區塊鏈系統通常採用單一層級的信任假設,要求每一次狀態變更都必須經過全網共識的嚴格驗證,這種設計雖然能夠提供強大的安全保證,但其高昂的通訊成本使其難以應用於需要頻繁更新的場景。AC-BlockDFL 透過引入分層信任的概念,使得系統在正常情況下以小委員會的效率運行,而在異常情況下能夠迅速升級至全網共識的安全等級。本節將透過三個形式化的安全性定理及其證明,嚴謹地論證此雙層信任模型所提供的安全保障。

\subsection{檢測層:1-of-N 誠實假設}
\label{sec:detection_layer}

雙層信任模型的第一層是檢測層,其採用了極為寬鬆的「1-of-N 誠實假設」。這個假設的含義是:只要全網 $N$ 個參與節點中存在至少一個誠實節點願意擔任挑戰者的角色,任何委員會層級的惡意行為就能夠被成功揭露。這種假設的寬鬆程度遠超過傳統拜占庭容錯系統所要求的「三分之二誠實節點」條件,因為它僅需要單一誠實節點的存在而非多數誠實節點的協調行動。從概率角度而言,在一個擁有數百或數千個參與者的大型網路中,所有節點同時選擇沉默或串謀的可能性極其微小。檢測層的設計巧妙地利用了區塊鏈系統的資料透明性特質:由於所有聚合提案的 CID 都被記錄在鏈上且對應的完整資料可透過 IPFS 公開存取,任何節點都能夠獨立地重新執行驗證運算,攻擊者即使成功控制了當前輪次的整個委員會,也無法阻止其他節點發現異常。

\begin{theorem}[檢測完備性]
\label{thm:detection_completeness}
令 $\mathcal{V}_r$ 為第 $r$ 輪的驗證委員會,$\text{Krum}(\{p_a\})$ 為對所有聚合提案執行 Krum 演算法所得的確定性正確結果,$w_{r+1}$ 為委員會實際選定並寫入區塊鏈的全域更新。若 $w_{r+1} \neq \text{Krum}(\{p_a\})$,且全網 $N$ 個節點中存在至少一個誠實節點 $c^*$ 願意擔任挑戰者角色,則此偏離行為必然被偵測。
\end{theorem}

\begin{proof}
此定理的證明建立在 Krum 演算法的確定性特質與區塊鏈資料的公開可驗證性之上。Krum 演算法的運算過程完全由其輸入決定:給定同一組聚合提案 $\{p_a\}$,任何執行者無論身份與位置,都將得到唯一且一致的輸出結果 $\text{Krum}(\{p_a\})$。在 AC-BlockDFL 的協議設計中,所有聚合提案的 CID 在委員會共識階段即被記錄於區塊鏈,對應的完整提案資料可透過 IPFS 公開存取,任何持有區塊鏈帳本的節點都能透過 CID 取得這些資料。因此,誠實挑戰者 $c^*$ 可以從 IPFS 下載與委員會完全相同的輸入集合 $\{p_a\}$,在本地獨立執行 Krum 演算法,所得結果必然為 $\text{Krum}(\{p_a\})$。當 $c^*$ 將此結果與委員會實際選定的 $w_{r+1}$ 進行比對時,若兩者不一致,則 $c^*$ 即可確認委員會的決策存在偏離,並據此發起挑戰交易。由於 $c^*$ 的驗證過程僅依賴公開可存取的鏈上 CID 參照、IPFS 資料與確定性演算法,委員會無法透過隱藏資訊或製造歧義來規避偵測。因此,只要存在至少一個誠實且具備質押能力的挑戰者,任何偏離正確 Krum 結果的委員會決策都必然被偵測。
\end{proof}

此定理的實務意涵在於,攻擊者若希望其惡意決策不被偵測,唯一的途徑是確保全網沒有任何一個誠實節點願意擔任挑戰者,這在大規模網路中幾乎不可能實現。相較於傳統 BFT 系統要求三分之二誠實節點的嚴格條件,1-of-N 假設將偵測門檻降至理論最低限度,極大地擴展了安全性的適用範圍。

\subsection{仲裁層:全網三分之二誠實假設}
\label{sec:arbitration_layer}

雙層信任模型的第二層是仲裁層,其採用了「全網三分之二誠實假設」。當挑戰被發起並進入仲裁階段後,最終的判決權力從小委員會回歸到全網範圍,要求網路中誠實節點的數量必須超過總節點數的三分之二,即 $N_{\text{total}} > 3f$。這是幾乎所有拜占庭容錯共識協議的標準假設,也是區塊鏈系統普遍依賴的安全基礎。在仲裁階段,所有參與驗證的節點透過 IPFS 下載相關提案資料並重新執行 Krum 運算,隨後透過 PBFT 協議對挑戰的正當性進行投票,只有當超過三分之二的節點確認委員會確實存在錯誤時,挑戰才會被判定為成立。

\begin{theorem}[懲罰確定性]
\label{thm:punishment_certainty}
令全網節點總數為 $N_{\text{total}}$,其中惡意節點數量 $f$ 滿足 $N_{\text{total}} > 3f$。若挑戰者依據定理 \ref{thm:detection_completeness} 成功偵測到委員會的惡意決策並提交了有效的挑戰交易,則此惡意行為必然在仲裁階段被確認,且參與共謀的委員會成員必然遭受質押金的全額罰沒。
\end{theorem}

\begin{proof}
仲裁過程的核心是全網範圍的 PBFT 共識。當挑戰交易被提交後,智能合約自動從鏈上調取該輪次的所有提案 CID,並要求全網驗證節點從 IPFS 下載完整提案資料後獨立重新執行 Krum 演算法。由於 Krum 的確定性特質(如定理 \ref{thm:detection_completeness} 的證明所述),所有誠實驗證節點將得到一致的正確結果 $\text{Krum}(\{p_a\})$,並能據此判斷 $w_{r+1}$ 是否偏離正確值。在 $N_{\text{total}} > 3f$ 的假設下,至少有 $N_{\text{total}} - f > 2N_{\text{total}}/3$ 個誠實節點參與仲裁投票,這些誠實節點基於相同的確定性運算結果將一致地投票確認委員會決策存在偏離。由於 PBFT 協議要求超過三分之二的贊成票即可達成共識,而誠實節點的數量已超過此門檻,因此仲裁共識必然成立。共識達成後,智能合約自動執行預定義的罰沒邏輯,沒收所有在該輪次中對偏離結果投贊成票的委員會成員之全額質押金,此過程由智能合約的確定性執行保證,不受任何外部干預。
\end{proof}

定理 \ref{thm:detection_completeness} 與定理 \ref{thm:punishment_certainty} 的結合構成了 AC-BlockDFL 安全性保障的完整邏輯鏈:前者確保惡意行為「必然被發現」,後者確保被發現的惡意行為「必然受到懲罰」。這兩層保障的疊加效果是,攻擊者在發動攻擊之前即可預見其行為將面臨偵測與懲罰的雙重確定性後果,這種確定性正是經濟安全性得以成立的邏輯前提。

\subsection{攻擊成本的形式化分析}
\label{sec:attack_cost_analysis}

基於前述兩層信任機制所提供的安全保障,本節將進一步形式化地分析攻擊者若試圖在 AC-BlockDFL 架構中發動一次獲利且確保不受懲罰的攻擊,所需跨越的關鍵經濟門檻。在第 \ref{sec:endogenous_staking} 節所建立的內生動態質押模型中,攻擊成本不再被視為一個靜態且固定的數值,而是被賦予了與系統整體質押規模及內部經濟指標 $R_{\mathrm{round}}$ 緊密掛鉤的動態屬性。這種設計的核心哲學在於建立一套自適應的平衡機制,確保無論代幣的市場價格如何劇烈波動,攻擊者所面臨的潛在罰沒損失始終能與其可能的非法獲利保持量級上的顯著落差。透過將安全成本內生化於區塊鏈自身的經濟循環中,AC-BlockDFL 成功在無需依賴外部資料源的前提下,從經濟維度建立起一道難以逾越的准入障礙,使得攻擊行為在理性博弈的框架下因期望收益轉負而變得無效。

理解這一形式化分析的前提,在於精確區分攻擊過程中兩種性質截然不同的經濟負擔及其對應的實現難度,這兩者共同構成了系統分層防禦的理論邊界。首先,委員會的佔領在本質上屬於一個具有強烈不確定性的機率性事件,雖然攻擊者可以透過策略性地累積更高的權益佔比,來提升其在特定輪次選舉中獲得超過三分之二席位的機率,但其最終能否在目標輪次實際取得驗證主導權,始終受制於具備密碼學可驗證隨機性的選舉結果。相比之下,仲裁階段的規避則是一個純粹的確定性資本門檻問題,根據定理 \ref{thm:detection_completeness} 與定理 \ref{thm:punishment_certainty} 的結論,只要攻擊者無法在物理層面阻止全網 PBFT 仲裁共識的達成,其惡意行為便必然會在審計窗口期內被揭露並遭受罰沒。而要阻止共識,攻擊者必須控制網路總投票權的三分之一以上,這構成了一個堅實的物理界限。

\begin{theorem}[無懲罰攻擊的資本門檻]
\label{thm:attack_cost_bound}
在 AC-BlockDFL 架構中,攻擊者若要完成一次惡意委員會決策且完全規避隨後而來的經濟懲罰,其在全網中必須掌握的權益資本需滿足以下下界:
\begin{equation}
\mathrm{Cost}_{\mathrm{total}} \geq \frac{1}{3} N_{\mathrm{total}} \cdot s_n
\label{eq:attack_cost}
\end{equation}
其中 $N_{\mathrm{total}}$ 代表全網節點的總體規模,$s_n$ 則為所有參與節點的平均質押額數。此外,即便攻擊者滿足上述資本條件而具備了阻斷仲裁活性之能力,其仍須在委員會隨機選舉中以機率性的方式獲得超過三分之二的席位,方能實際對當前輪次的共識結果產生實質影響。
\end{theorem}

\begin{proof}
攻擊者若要達成「攻擊成功且不受懲罰」的最終目標,必須同時克服兩個不同層級的安全防線。在委員會層級,攻擊者必須在目標輪次的隨機選舉中恰好分配到超過三分之二的驗證者席位,此條件的實現機率由權益佔比決定,具有天然的隨機與不確定性,攻擊者僅能透過增加權益來提升勝算,卻無法將其轉化為必然。在全網層級,根據定理 \ref{thm:punishment_certainty},一旦挑戰被任何誠實節點發起並經全網驗證確認,所有惡意節點的質押將被全額罰沒。遵循拜占庭容錯理論,PBFT 協議要求獲得超過三分之二的權重同意方能達成共識,因此攻擊者若要逃避懲罰,唯一途徑是控制全網至少 $\lceil N_{\mathrm{total}}/3 \rceil$ 的投票權重,以破壞仲裁共識的活性,使其無法執行罰沒。此確定性的資本門檻與全網規模成線性正比,其量級為 $O(N_{\mathrm{total}})$。由於規避懲罰是發動獲利攻擊的邏輯前提,故總體攻擊成本的下界必然為 $\frac{1}{3} N_{\mathrm{total}} \cdot s_n$。
\end{proof}

此定理所揭示的深層意涵在於,AC-BlockDFL 透過異步審計與全網仲裁機制,成功將攻擊者面對的經濟門檻從局部的委員會規模提升至全域的網路規模。在缺乏事後追責機制的傳統 BlockDFL 架構中,攻擊者的作惡成本僅取決於其控制小型委員會所需的資源,其複雜度始終維持在 $O(C)$ 的水平,這使得具備一定資本實力的攻擊者能在較短時間內完成對系統的佔領。AC-BlockDFL 則迫使攻擊者在發動具體攻擊之前,必須先解決如何抗衡全網三分之二誠實多數的問題,其成本量級發生了跨越式的躍升。考慮到實際應用中全網節點數 $N_{\mathrm{total}}$ 通常遠大於委員會規模 $C$(例如在本研究的基準實驗配置中,$N_{\mathrm{total}} = 100$ 而 $C = 7$),這種分層的安全防線為系統提供了極強的魯棒性。

值得進一步討論的是,即便攻擊者在財力上足以支撐全網三分之一以上的權益佔比,其所面對的仍然不是一條通往非法收益的坦途。控制全網三分之一的節點雖然提供了阻止懲罰執行的確定性能力,卻無法保證在每一輪的委員會選舉中都能如願獲得主導地位,後者依然受到不可預測的隨機分佈約束。換言之,攻擊者需要投入大量的沉沒成本來確保即使作惡也不會損失本金,但這筆高昂的資本投入僅換取了一個不確定的攻擊機會,這種「高昂確定性投入換取微小不確定收益」的結構,在博弈論層面極大地削弱了理性參與者的作惡衝動。由此可見,AC-BlockDFL 成功地在維持執行效率的同時,利用全網的累計權益為小型委員會的運行提供了深度的安全保障。


\subsection{激勵相容性的博弈論分析}
\label{sec:game_theory_analysis}

基於前述安全性定理,本節運用博弈論的分析框架論證 AC-BlockDFL 的經濟懲罰機制如何使誠實行為成為所有理性參與者的最優策略。第 \ref{chap:threat-model} 章在安全目標中提出了激勵相容性的數學條件,要求攻擊者的預期收益必須為負值,本節將在 AC-BlockDFL 的具體架構參數下展開這一分析。

對於理性攻擊者而言,其決策問題可建模為單次博弈的期望收益計算。設攻擊者成功控制委員會後所能獲得的單輪最大經濟收益為 $G_{\text{attack}}$,被全額罰沒的質押金損失為 $L_{\text{slash}}$,則攻擊的預期收益可表示為:
\begin{equation}
E[\text{Payoff}] = P_{\text{success}} \cdot G_{\text{attack}} - P_{\text{caught}} \cdot L_{\text{slash}}
\label{eq:expected_payoff}
\end{equation}
其中 $P_{\text{success}}$ 為攻擊者在特定輪次成功控制委員會的機率,$P_{\text{caught}}$ 為惡意行為被偵測並受到懲罰的機率。定理 \ref{thm:detection_completeness} 與定理 \ref{thm:punishment_certainty} 的結合表明,在 1-of-N 誠實假設與全網三分之二誠實假設同時成立的條件下,$P_{\text{caught}}$ 趨近於 1。需要注意的是,$P_{\text{success}}$ 衡量的是攻擊者在委員會選舉中獲得多數席位的機率,$P_{\text{caught}}$ 衡量的是惡意決策被偵測的機率,兩者分屬不同層面的事件:攻擊者只有在 $P_{\text{success}}$ 對應的條件實現時才能發動攻擊,而一旦攻擊發動,$P_{\text{caught}}$ 趨近於 1 確保其必然面臨懲罰。因此式 (\ref{eq:expected_payoff}) 可簡化為:
\begin{equation}
E[\text{Payoff}] = P_{\text{success}} \cdot (G_{\text{attack}} - L_{\text{slash}})
\label{eq:simplified_payoff}
\end{equation}

激勵相容性的充分條件由此清晰浮現:只要 $L_{\text{slash}} > G_{\text{attack}}$,則無論攻擊成功機率 $P_{\text{success}}$ 取何值,預期收益都嚴格為負。在內生動態質押模型下,$L_{\text{slash}} = k \times R_{\text{round}}$ 而 $G_{\text{attack}}$ 的上界約為 $C \times R_{\text{round}}$(即攻擊者壟斷全部驗證獎勵),由於 $k \gg C$,此條件穩定成立且不受代幣市場價值波動的影響。以本研究的實驗參數進行具體的數值分析:委員會規模 $C = 7$,每位驗證者的單輪獎勵為 $1.0$ 單位,初始質押為 $100$ 單位。攻擊者即使成功壟斷全部驗證獎勵,單輪最大收益 $G_{\text{attack}}$ 上界約為 $7.0$ 單位。然而,一旦惡意行為被偵測,參與共謀的至少 $5$ 個惡意委員會成員各自損失全額質押 $100$ 單位,攻擊者陣營的總損失 $L_{\text{slash}} = 500$ 單位,懲罰力度約為潛在收益的 $71$ 倍。

如此極端的風險收益不對稱結構,使得理性攻擊者即使考慮到可能低估被偵測機率的僥倖心理,只要 $P_{\text{caught}}$ 超過 $G_{\text{attack}} / L_{\text{slash}} \approx 1.4\%$ 的極低門檻,攻擊的預期收益即轉為負值。而 AC-BlockDFL 的安全性定理保證了 $P_{\text{caught}}$ 趨近於 1,遠遠超過這一最低門檻。從長期均衡的角度而言,AC-BlockDFL 的罰沒機制成功打破了第 \ref{chap:threat-model} 章所描述的漸進式委員會佔領攻擊所依賴的正反饋循環。在沒有罰沒機制的系統中,攻擊者可透過操縱委員會來獲取不當獎勵,進而增加質押權重並逐步掌控系統。而在 AC-BlockDFL 中,任何作惡嘗試都會導致質押的大幅減少而非增加,遭受罰沒的惡意節點不僅損失了當下的質押資產,更喪失了透過未來輪次逐步恢復影響力的經濟基礎,這種永久性的治理排除效應從根本上切斷了惡性循環的可能性 \cite{chiu2018incentive}。


\section{效率與開銷分析}
\label{sec:efficiency_analysis}

第 \ref{sec:committee-size-security} 節的分析揭示了傳統委員會架構面臨的根本性困境:在 BlockDFL 等現有系統中,安全性的保障完全依賴於「委員會中誠實節點佔據多數」這一機率性條件,而要提高此條件成立的機率,唯一的途徑便是擴大委員會規模,這又直接推高了通訊成本。本節將從通訊、運算與儲存三個維度論證 AC-BlockDFL 如何透過將安全性保障從「門檻安全性」轉移至「經濟安全性」,在維持等效安全保證的前提下實現顯著的效率提升。

\subsection{通訊複雜度對比分析}

BlockDFL 的安全性論證建立在超幾何分佈的機率運算之上:若要將委員會被惡意控制的風險壓制在可接受水準之下,系統必須維持足夠大的委員會規模。以第 \ref{sec:committee-size-security} 節的數值分析為例,在全網節點數 $N=100$、惡意節點佔比 $f=30\%$ 的威脅環境下,若將風險閾值設定為 $p < 0.01$,委員會規模至少需要達到 $c=9$。這種設計的深層問題在於其「悲觀併發控制」的本質:BlockDFL 預設每一輪都可能遭受攻擊,因此必須在每一輪都部署足以抵禦攻擊的防禦資源。然而在實際運作中,攻擊者成功控制委員會的情況畢竟屬於少數輪次,絕大多數時候系統處於正常運作狀態,此時維持大型委員會所付出的通訊成本便成為一種恆常的「預防溢價」。

AC-BlockDFL 對效率問題的回應並非追求「更好的機率保證」,而是從根本上改變了安全性的實現方式。傳統的門檻安全性聚焦於「如何降低委員會被攻破的機率」,這種思路必然導向更大的委員會規模。AC-BlockDFL 則採取截然不同的策略:與其執著於將被攻破的機率壓制至趨近於零,不如確保即使委員會被攻破,攻擊者也無法從中獲取正向收益。在經濟安全性的框架下,委員會被攻破的機率不再是唯一的安全性指標,因為異步挑戰機制確保了任何惡意行為都將面臨全額質押金的罰沒。由此,委員會規模的選擇便不再完全受制於安全性的機率運算,系統得以在滿足基本安全閾值的前提下採用相對較小的委員會來獲取效率優勢。

表 \ref{tab:efficiency_comparison} 呈現了兩種架構在「委員會被惡意控制的風險低於 1\%」這一統一安全性基準下的通訊成本對比。BlockDFL 必須採用 $c=9$ 的委員會規模,每輪通訊複雜度固定為 $O(81)$;AC-BlockDFL 透過經濟安全性的補充保障,得以採用 $c=7$ 的委員會規模,常態通訊複雜度降至 $O(49)$,實現了約 39.5\% 的通訊成本削減。

\begin{table}[htbp]
    \centering
    \caption{BlockDFL 與 AC-BlockDFL 在相同安全性水平下的效率對比 ($N=100$, $f=30\%$, $p_{\text{risk}} < 0.01$)}
    \label{tab:efficiency_comparison}
    \renewcommand{\arraystretch}{1.3}
    \begin{tabular}{|l|l|l|l|}
        \hline
        \textbf{評估維度} & \textbf{BlockDFL} & \textbf{AC-BlockDFL} & \textbf{差異分析} \\
        \hline
        安全性實現方式 & 門檻安全性 & 經濟安全性 & 機率保證 vs. 激勵相容 \\
        \hline
        所需委員會規模 & $c = 9$ & $c = 7$ & 規模縮減 22.2\% \\
        \hline
        常態通訊複雜度 & $O(c^2) = O(81)$ & $O(c^2) = O(49)$ & 通訊成本降低 39.5\% \\
        \hline
        安全性維護模式 & 每輪固定開銷 & 條件式觸發開銷 & 預防性 vs. 響應性 \\
        \hline
    \end{tabular}
\end{table}

從系統運作的動態視角來看,AC-BlockDFL 的通訊成本呈現條件式的特徵。在正常運作下,系統僅需支付 $O(c^2) = O(49)$ 的委員會共識成本;唯有當挑戰被觸發並進入全網仲裁時,才會產生額外的 $O(N_{\text{total}}^2)$ 通訊開銷。由於經濟懲罰機制有效消除了理性攻擊者的作惡誘因,挑戰觸發的機率 $p$ 在長期均衡中將趨近於零,據此系統的期望通訊複雜度可表示為:
\begin{equation}
E[\text{Comm}] = (1-p) \cdot O(c^2) + p \cdot (O(c^2) + O(N_{\text{total}}^2)) = O(c^2) + p \cdot O(N_{\text{total}}^2)
\end{equation}
當 $p \to 0$ 時,期望複雜度近似於常態值 $O(c^2)$,這意味著全網仲裁的高昂成本僅作為威懾手段存在而實際上鮮少被觸發。這種「按需付費」的安全模式,相較於 BlockDFL 每輪都必須支付的固定「預防溢價」,在資源利用上更為經濟。

\subsection{儲存開銷的權衡分析}

傳統的鏈上儲存方案要求每輪訓練中所有聚合提案的完整模型參數都被永久記錄於區塊鏈帳本之中,隨著訓練輪次的累積,帳本的體積將持續膨脹,對節點的儲存資源構成沉重負擔。AC-BlockDFL 透過將模型參數的沉重負載遷移至 IPFS 並搭配嚴格的生命週期管理策略,顯著緩解了這一問題。在此設計下,區塊鏈帳本僅記錄固定長度的 CID 雜湊值與相關元資料,鏈上儲存複雜度從 $O(\text{ModelSize})$ 降至 $O(\text{HashSize})$,對於典型的卷積神經網路模型而言,這意味著數個數量級的儲存節省。

IPFS 上的臨時儲存成本是此設計為獲得審計能力所付出的必要代價。在異步審計窗口的存續期間,參與節點需要釘選相關的提案資料以確保挑戰者能夠存取驗證所需的完整輸入,這會佔用一定的本地儲存空間。然而,由於審計窗口具有明確的時間上限(即生存時間 TTL),一旦窗口關閉且未發生挑戰,節點即可解除釘選以釋放空間。因此,每個節點在任一時刻所需維護的 IPFS 儲存量僅與當前處於審計窗口內的少數輪次相關,而非與整個訓練歷史成正比。這種設計在審計所需的資料可用性與長期儲存成本之間建立了合理的平衡,使得 AC-BlockDFL 在引入異步審計能力的同時,不會對節點的儲存資源造成不可承受的負擔。

\subsection{效率提升的本質:架構層面的解耦創新}

綜合上述通訊與儲存兩個維度的分析,AC-BlockDFL 相對於 BlockDFL 的效率優勢並非源自共識協議本身的改進,而是源自架構層面的根本性創新,即將安全性與委員會規模之間的強耦合關係予以弱化。在 BlockDFL 的設計中,委員會規模是安全性的唯一保障手段,追求更高的安全性必然要求更大的委員會,而更大的委員會必然帶來更高的通訊成本。AC-BlockDFL 透過引入異步挑戰機制與經濟懲罰協議,為安全性開闢了獨立於委員會規模的第二條保障路徑,從而打破了這種強耦合。

這種解耦的實踐意義在於,系統設計者得以根據效率需求選擇較小的委員會規模,而無需過度顧慮安全性的機率運算。雖然 $c=7$ 的委員會在純機率意義上的安全性略低於 $c=9$,但經濟懲罰機制提供的額外威懾力足以彌補這一差距:攻擊者或許更容易獲得控制委員會的機會,但每一次攻擊嘗試都面臨著災難性的經濟後果,這種威懾足以使理性攻擊者放棄攻擊意圖。最終,系統在實際運作中達成了一種新的均衡,較小的委員會提供了效率優勢,而幾乎不會發生的攻擊確保了這種效率優勢不會被全網仲裁的開銷所侵蝕。


\section{本章小結}

本章提出的審計驅動型委員會 BlockDFL代表了區塊鏈聯邦學習系統設計理念的一次重要轉變。AC-BlockDFL 建立在 BlockDFL 委員會模型之上,完整保留了其經過驗證的訓練流程與角色定義,同時透過三個層面的架構創新實現了從「門檻安全性」向「經濟安全性」的典範轉移。在角色擴展方面,挑戰者角色的設計遵循開放准入原則,任何持有足夠質押的節點均可擔任,將監督權力從少數委員會成員民主化到整個網路。在儲存架構方面,IPFS 整合與基於生命週期管理的釘選策略,將鏈上儲存複雜度從 $O(\text{ModelSize})$ 降至 $O(\text{HashSize})$,在確保審計期間資料可用性的同時有效控制了長期儲存成本。在質押定價方面,內生動態質押模型消除了對外部預言機的依賴,將罰沒金額錨定於系統內部經濟指標 $R_{\text{round}}$ 的倍數,確保風險收益的不對稱結構在任何代幣市場環境下都穩定成立。

AC-BlockDFL 的安全性保障由三個形式化定理構成完整的邏輯鏈。定理 \ref{thm:detection_completeness} 證明了在 1-of-N 誠實假設下任何偏離正確 Krum 結果的委員會決策必然被偵測,定理 \ref{thm:punishment_certainty} 進一步證明了在全網三分之二誠實假設下被偵測的惡意行為必然受到罰沒制裁,而定理 \ref{thm:attack_cost_bound} 則量化了攻擊者完全規避懲罰所需的經濟成本下界。三者的結合揭示了 AC-BlockDFL 的核心安全特性:系統的安全性實質上由全網規模 $N_{\text{total}}$ 決定而非委員會規模 $C$。通訊複雜度分析確認了在相同安全性要求下,AC-BlockDFL 得以採用 $c=7$ 的委員會規模,相較於 BlockDFL 所需的 $c=9$ 實現了約 39.5\% 的通訊成本削減。博弈論分析則確保了這種架構在實踐中能夠長期穩定運作,在本研究的實驗配置下懲罰力度約為潛在收益的 71 倍,使得誠實行為成為所有理性參與者的最優策略,從根本上打破了漸進式委員會佔領攻擊所依賴的正反饋循環。下一章將透過多維度的模擬實驗驗證這些理論主張的有效性。

\end{ZhChapter}