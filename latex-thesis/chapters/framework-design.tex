\begin{ZhChapter}

\chapter{挑戰增強型委員會架構 (Challenge-Augmented Committee Architecture)}
\label{chap:framework}

區塊鏈聯邦學習系統在追求去中心化安全性的同時,往往面臨著執行效率的巨大挑戰。傳統的拜占庭容錯共識機制雖然能夠提供強大的安全保證,卻因其高昂的通訊成本而難以應用於需要頻繁更新的機器學習場景。為了突破這一困境,本章提出「挑戰增強型委員會架構」(Challenge-Augmented Committee Architecture, CACA),該架構建立在第二章 \ref{sec:system_model} 節所定義的基準委員會模型之上,透過引入異步審計機制與內部罰沒協議,實現了從傳統「門檻安全性」向「經濟安全性」的典範轉移。此設計哲學的核心在於認識到聯邦學習與金融交易系統在本質上的差異:機器學習過程具備天然的抗噪性與自我修復能力,這使得我們能夠在不犧牲長期安全性的前提下,優先保障系統的即時執行效率。

本架構的設計理念源自於對現有委員會機制根本缺陷的深刻洞察。小規模委員會雖然能夠顯著降低通訊複雜度,但其固有的集中化特性使得攻擊者能夠透過漸進式的權益累積來逐步控制驗證權力。CACA 透過將安全性驗證從同步的、阻塞式的流程轉變為異步的、非阻塞式的審計機制,成功地將效率優化與安全保障解耦。這種解耦使得系統能夠在正常情況下維持極高的執行效率,同時保留了在異常情況下動員全網資源進行仲裁的能力。更重要的是,透過引入經濟懲罰機制,本架構將攻擊者的理性決策空間重新塑造:任何試圖操縱委員會共識的行為都將面臨遠超其潛在收益的經濟損失,從而從根本上消除了發動攻擊的經濟誘因。

本章的結構安排如下:首先在 \ref{sec:arch_overview} 節中概述 CACA 的整體架構與設計哲學,詳細闡明各組件之間的協作關係;接著在後續各節中深入探討異步審計機制的運作原理、雙層信任模型的安全性論證、通訊複雜度的理論分析,以及激勵機制的經濟學基礎。透過將理論分析與概率模型相結合,本章將論證 CACA 如何在維持極高執行效率的同時,提供具備激勵相容性的強大安全保障,從而為第三章所提出的五項安全目標提供完整的技術實現路徑。


\section{系統架構概覽}
\label{sec:arch_overview}

挑戰增強型委員會架構的設計目標在於建立一個既具備經濟安全性又能保持高執行效率的去中心化學習平台。相較於傳統的區塊鏈共識機制需要在每次狀態更新時達成全網共識,本架構採用了更為靈活的分層驗證策略,將日常的效率需求與極端情況下的安全需求巧妙地分離開來。圖 \ref{fig:caca_arch} 展示了 CACA 的完整運作流程,該流程涵蓋了從初始的角色分配、本地模型訓練、聚合結果驗證,直到潛在的挑戰仲裁等各個階段。這種設計使得系統能夠在絕大多數正常情況下以最小的通訊開銷快速完成模型更新,同時保留了在檢測到異常行為時啟動全網仲裁的能力。

\begin{figure}[htbp]
    \centering
    \includegraphics[width=0.9\textwidth]{figures/utils/Challenge-Augmented-Committee-Architecture.drawio.png}
    \caption{Challenge-Augmented Committee Architecture (CACA) 系統架構與工作流程圖}
    \label{fig:caca_arch}
\end{figure}

本系統的運作依賴於四個核心角色之間的精密協作,每個角色都承擔著特定的職責並受到相應的激勵約束。訓練者 (Update Provider, UP) 是系統中持有本地私有資料的參與節點,他們的主要任務是在本地執行模型訓練並將運算出的本地更新提交給指定的聚合者。這些訓練者構成了聯邦學習系統的基礎,其資料隱私性透過本地訓練的方式得到保護,無需將原始資料暴露於網路之中。聚合者 (Aggregator, AG) 則負責收集來自多個訓練者的本地更新,執行初步的彙整運算並生成聚合更新,隨後將此聚合結果作為「提案」提交給驗證委員會。聚合者的角色設計旨在減少驗證委員會需要處理的資料量,同時透過多個聚合者之間的競爭機制來降低單點故障的風險。

驗證委員會 (Verifier Committee, VC) 是整個架構中最為關鍵的組件,由透過質押權重選出的小型委員會組成。委員會的核心職責在於針對多個聚合者提交的提案運行 Krum 評分演算法,並透過 PBFT 共識機制投票決定其中哪一份提案應被採納為該輪次的全域更新。這種設計使得驗證過程能夠在保持高效率的同時,具備一定程度的拜占庭容錯能力。值得注意的是,委員會的規模被刻意保持在較小的範圍內,這是基於對通訊複雜度與安全性之間權衡的深思熟慮。雖然小委員會在理論上更容易被攻擊者控制,但透過後續將介紹的異步挑戰機制,這種看似的安全性劣勢實際上轉化為了效率優勢。最後,挑戰者 (Challenger) 這一角色向所有持有足夠質押的節點開放,他們在背景中異步監聽鏈上資料並重新執行 Krum 演算法的運算。一旦發現委員會選定的全域更新與正確的 Krum 運算結果不符,任何挑戰者都可以發起挑戰程序,從而觸發全網仲裁機制。這種開放式的監督設計確保了即使委員會被惡意控制,攻擊行為也能夠被及時發現並受到懲罰。

系統的工作流程被精心設計為一個連貫且高效的循環過程,每個階段都與前後階段緊密銜接。在每一輪次開始時,區塊鏈系統根據前一區塊的哈希值進行動態角色抽選,這種基於隨機性的分配機制能夠有效防止攻擊者預先布局。抽選機率與節點的質押權益成正比,且嚴格遵循驗證者、聚合者、更新提供者的優先順序進行分配。這種優先序設計確保了最重要的驗證角色能夠由權益最大的節點擔任,從而提高了系統的整體安全性。角色分配完成後,被選定為更新提供者的節點使用其本地資料進行模型訓練,並將運算結果傳遞給當輪選定的聚合者。聚合者收集到足夠數量的本地更新後,執行初步的聚合運算並將結果作為提案提交給驗證委員會。

驗證委員會收到所有聚合提案後,針對每一份提案運行 Krum 演算法進行評分,該演算法能夠有效識別出偏離正常分布的異常更新。委員會成員隨後透過 PBFT 共識機制對評分最優的提案進行投票,一旦達成共識,該提案即被確定為最終的全域更新。這裡的關鍵設計在於,系統採用了即時更新策略而非傳統的等待確認機制。一旦委員會達成共識,區塊鏈立即更新全域模型並根據貢獻度向更新提供者、聚合者與驗證委員會成員發放獎勵。此過程完全非阻塞,下一輪訓練可以立即基於新模型開始,從而確保了系統的持續高效運作。這種設計哲學的核心在於認識到聯邦學習系統具備天然的容錯能力,短暫的不完美更新能夠透過後續輪次逐步修正,因此無需為了追求絕對的即時正確性而犧牲整體執行效率。

在大多數情況下,系統的運作將在此階段順利完成並進入下一輪次。然而,CACA 的創新之處在於引入了異步挑戰這一選用階段,該階段作為系統的安全後盾,在背景中持續運作而不影響正常流程。任何擔任挑戰者角色的節點都可以持續監控鏈上資料,重新執行 Krum 演算法並驗證委員會的決策是否正確。若挑戰者發現委員會選定的結果與正確的 Krum 運算答案存在不一致,便可以質押一定金額的押金發起挑戰。挑戰一旦成立,區塊鏈系統將啟動全參與者的 PBFT 仲裁程序,調動全網資源重新執行 Krum 運算以判定真實的正確答案。這種設計巧妙地將效率與安全性解耦:在正常情況下,系統以小委員會的效率運行;在異常情況下,系統能夠迅速升級至全網共識的安全等級。透過這種分層設計,CACA 成功地在維持高執行效率的同時,為系統提供了等同於全網共識的安全保障。

\begin{algorithm}[t]
\caption{CACA Execution Protocol (Instant Update)}
\label{alg:caca_execution}
\begin{algorithmic}[1]
\Require Current Round $r$, Total Stake Weighted Nodes $\mathcal{N}$
\Ensure Updated Global Model $w_{r+1}$
\vspace{0.1cm}
\State \textbf{Role Assignment:} 
\State Blockchain selects $\mathcal{V}$ (Committee), $\mathcal{A}$ (Aggregators), $\mathcal{U}$ (Update Providers) from $\mathcal{N}$ based on stake and randomness.
\vspace{0.1cm}
\State \textbf{Training \& Aggregation:}
\State Each $u \in \mathcal{U}$ trains using $w_r$, broadcasts updates to $\mathcal{A}$.
\State Each $a \in \mathcal{A}$ aggregates updates into proposal $p_a$, sends to $\mathcal{V}$.
\vspace{0.1cm}
\State \textbf{Consensus \& Update:}
\State $\mathcal{V}$ runs Krum on all proposals $\{p_a\}$.
\State $\mathcal{V}$ votes on the best proposal via PBFT.
\State Commit $w_{r+1}$ to blockchain \textbf{immediately}.
\State Distribute rewards to $\mathcal{U}, \mathcal{A}, \mathcal{V}$.
\end{algorithmic}
\end{algorithm}

\begin{algorithm}[t]
\caption{Asynchronous Challenge Mechanism (Slash-Only)}
\label{alg:caca_challenge}
\begin{algorithmic}[1]
\Require Challengers $\mathcal{C}$
\Ensure Punishment for Malicious Acts
\vspace{0.1cm}
\For{each Challenger $c \in \mathcal{C}$}
    \State $c$ retrieves committee inputs and re-executes Krum.
    \If{$c$ detects outcome mismatch with $w_{r+1}$}
        \State $c$ posts \textbf{Challenge Transaction} with deposit.
        \State \textbf{Arbitration Triggered:} All nodes re-verify.
        \If{Malicious Consensus Confirmed}
            \State \textbf{Burn/Slash} stake of malicious $\mathcal{V}$.
            \State Reward Challenger $c$ and all nodes.
            \State \textit{// Note: Model $w_{r+1}$ is NOT reverted.}
        \EndIf
        \State \textbf{Exit Loop}.
    \EndIf
\EndFor
\end{algorithmic}
\end{algorithm}

\section{異步審計與究責機制}
\label{sec:async_audit}

異步審計機制是 CACA 架構中最具創新性的設計要素,其核心理念在於將傳統區塊鏈系統中同步驗證與即時執行之間的緊密耦合關係予以解構。傳統的拜占庭容錯系統要求在每次狀態變更之前必須達成全網共識,這種設計雖然能夠提供強大的即時正確性保證,卻也導致了系統吞吐量與延遲性能的嚴重退化。然而,當我們深入審視聯邦學習系統的本質特性時,會發現這種對即時正確性的執著追求實際上並非必要。機器學習過程本身具備顯著的抗噪性,模型參數在訓練過程中的微小偏差通常不會導致災難性的後果,而是能夠透過後續的訓練迭代逐步修正。基於這一洞察,CACA 採用了「先執行後審計」的設計哲學,允許系統在委員會達成共識後立即更新模型,而將嚴格的正確性驗證推遲到異步的背景審計流程中進行。

這種即時執行策略的實施機制相當直接但極具威力。當驗證委員會對某一聚合提案達成共識後,該提案所對應的模型更新會立即被視為有效並寫入區塊鏈。全域模型參數隨即更新,所有訓練者節點都可以基於這個最新的模型狀態開始下一輪的本地訓練。這個過程不需要等待任何額外的確認期或審計結果,從而確保了系統的端到端延遲能夠降至最低。在理想情況下,CACA 的執行效率幾乎與完全無防禦機制的中心化系統相當,這是因為正常流程中唯一的額外開銷僅來自於小規模委員會內部的 PBFT 共識,而這個開銷相較於全網共識而言幾乎可以忽略不計。這種設計選擇體現了對系統「活性」(Liveness) 的優先保障:只要委員會能夠達成共識,系統就能夠持續前進,而不會因為等待完整的安全驗證而陷入停滯。

然而,即時執行策略的採用並不意味著系統放棄了對安全性的追求,而是將安全保障從同步的阻塞式驗證轉移到了異步的非阻塞式審計。挑戰機制的設計確保了即使委員會的決策存在問題,這些問題也能夠被及時發現並受到適當的懲罰。挑戰流程的核心在於其所依賴的「數學確定性」,這是一個至關重要的設計要素。由於 Krum 演算法是一個完全確定性的數學運算,給定相同的輸入必然產生相同的輸出,因此委員會無法透過資訊不對稱來掩蓋其惡意行為。所有參與挑戰的節點都能夠獨立地重新執行 Krum 運算,並驗證委員會的選擇是否符合演算法的正確結果。這種基於數學證明的驗證方式消除了傳統審計機制中常見的主觀判斷空間,使得挑戰過程具備了客觀性和不可辯駁性。

挑戰流程的觸發條件設計得相當明確且易於驗證。挑戰者透過持續監控鏈上的公開資料,獲取每一輪次中所有聚合者提交的提案以及委員會最終選定的全域更新。挑戰者在本地重新執行 Krum 演算法,計算出理論上應該被選中的最優提案,並將其與委員會實際選定的結果進行比對。若兩者不一致,則意味著委員會的決策過程存在問題,無論是由於計算錯誤還是惡意操縱,都構成了發起挑戰的充分理由。挑戰者此時可以提交一筆挑戰交易並附帶規定金額的質押金。這筆質押金的設計具有雙重目的:一方面,它能夠防止惡意節點透過大量無效挑戰來發動拒絕服務攻擊,因為錯誤的挑戰會導致質押金的損失;另一方面,它也為成功的挑戰者提供了經濟激勵,使得監督委員會行為成為一項有利可圖的活動。

當挑戰交易被提交到區塊鏈後,系統進入仲裁階段,這是整個挑戰機制中最為關鍵的環節。智能合約會立即鎖定相關的質押金,包括挑戰者的押金以及被挑戰的委員會成員的質押。隨後,合約調取該輪次中鏈上緩存的所有聚合提案資料,這些資料在委員會共識階段就已經被完整地記錄在區塊鏈上,確保了仲裁過程的資料完整性。接下來,系統觸發全網仲裁機制,所有驗證節點都被要求重新執行 Krum 演算法的運算。這個過程本質上是將原本由小委員會執行的驗證任務擴展到了全網範圍,從而將安全性等級提升到了與全網 PBFT 共識相當的高度。全網驗證者透過 PBFT 協議對仲裁結果進行投票,若超過三分之二的節點確認委員會的決策確實存在錯誤,則挑戰成立,系統將執行相應的懲罰措施。這種從小委員會到全網共識的動態升級機制,巧妙地平衡了日常運作的效率需求與極端情況下的安全需求。

當仲裁確認委員會存在惡意行為時,系統的處置策略展現出了與傳統分散式系統截然不同的設計哲學。CACA 採用「僅懲罰不回滾」(Slash-Only) 的政策,這一決策基於對聯邦學習系統特性的深刻理解以及對系統整體效益的全面考量。這種處置方式的第一個重要考量在於算力效率與機器學習系統的自癒特性。若選擇回滾模型狀態,則意味著從被攻擊的輪次開始,之後所有輪次的訓練成果都將被作廢,這將造成極為嚴重的運算資源浪費。考慮到聯邦學習通常需要經歷數百甚至數千個訓練輪次,即使僅回滾數十個輪次也將導致大量誠實節點的貢獻付之東流。更重要的是,機器學習模型具備顯著的自我修復能力,即使某一輪次的更新受到惡意操縱而包含了有偏差的梯度資訊,後續輪次中來自誠實節點的正確更新也能夠逐步抵銷這種負面影響,使模型重新收斂到正確的方向。

第二個關鍵考量涉及仲裁機制的時效性問題。全參與者的 PBFT 仲裁雖然能夠提供最高等級的安全保證,但其通訊複雜度和時間延遲都顯著高於小委員會共識。在實際運作中,從挑戰發起到仲裁完成,往往需要經歷相當長的時間窗口,在此期間系統可能已經完成了數十個新的訓練輪次。模型參數在這個過程中持續演進,當仲裁最終判定某個早期輪次存在問題時,該輪次的影響很可能已經透過後續的正常訓練被大幅稀釋。在這種情況下,強行回滾不僅缺乏實質意義,反而會破壞系統訓練過程的連續性,導致更大的效率損失。因此,CACA 選擇接受這種由時間延遲帶來的不完美性,將重點放在對惡意行為的懲罰而非對歷史狀態的修正上。這種選擇體現了對系統整體效益的優先考量,認識到在快速演進的機器學習過程中,持續前進往往比追求歷史的完美更為重要。

第三個支持「僅懲罰不回滾」策略的理論基礎來自於機器學習領域的正規化效應。從更廣闊的視角來看,偶爾出現的次優更新(Sub-optimal Updates)實際上可以被視為向訓練過程中引入的隨機噪音。機器學習理論告訴我們,適度的噪音注入在特定情境下能夠起到正規化的作用,幫助模型避免過度擬合訓練資料,從而提升在未見資料上的泛化能力。雖然這種「意外的正規化」效果不應被視為系統設計的主要目標,但它確實提供了一個有趣的理論視角,說明了為何少量的非最佳更新未必會對最終的模型品質造成災難性影響。這種認知進一步支持了不回滾策略的合理性,因為它暗示著聯邦學習系統具備足夠的韌性來容忍偶發的偏差。

基於上述三點深入的學術考量,CACA 的處置方式聚焦於經濟懲罰而非狀態回退。當仲裁確認委員會的惡意行為後,系統立即執行罰沒(Slashing)操作,沒收惡意委員會成員以及涉案聚合者的全額質押金。這種懲罰的力度是極為嚴厲的,因為質押金的規模通常被設定在足夠高的水平,以確保攻擊的潛在收益遠小於被發現後的損失。被罰沒的資金並非簡單地銷毀或歸入系統金庫,而是被精心分配以維持激勵相容性。挑戰者作為揭露惡意行為的功臣,將獲得其中相當一部分作為獎勵,這確保了監督機制具備持續的經濟驅動力。剩餘的資金則分配給全體誠實參與者,包括那些在被攻擊輪次中提供了正確更新的訓練者,以及參與了仲裁過程的驗證節點,作為對他們所承受風險和付出努力的補償。

與此同時,系統對於受影響的模型更新採取了保留而非回退的策略。被操縱的更新紀錄會完整地保存在區塊鏈上,連同相關的懲罰記錄一起成為系統歷史的一部分。這種透明化處理方式不僅有助於學術研究和系統審計,也為其他參與者提供了寶貴的警示資訊。更重要的是,系統明確依賴聯邦學習演算法自身的強健性,透過後續輪次中誠實更新的持續累積,逐步稀釋並覆蓋惡意更新所帶來的負面影響。這種做法雖然在短期內可能允許模型參數存在一定程度的偏差,但從長期來看,模型將在誠實節點的主導下重新收斂到正確的狀態。這種設計體現了對機器學習過程本質的深刻理解,認識到模型訓練是一個持續優化的過程而非一次性的精確計算,因此能夠容忍和吸收過程中的局部擾動。

這種「僅懲罰不回滾」策略的最終效果在於建立了極為強大的經濟威慑力。從攻擊者的理性決策視角來看,即使成功控制了某一輪次的委員會並注入了惡意更新,其所能獲得的收益是相當有限的。單次的模型操縱充其量只能影響一個訓練輪次的參數更新,而這種影響很快就會被後續的正常訓練所稀釋。相對地,一旦攻擊行為被挑戰者發現並經仲裁證實,攻擊者將面臨全額質押金的損失,並被永久性地從治理委員會中除名,失去未來獲得驗證獎勵的機會。這種極度不對稱的風險收益比使得發動攻擊在經濟上變得完全不理性。更重要的是,這種懲罰機制成功打破了漸進式委員會佔領攻擊(PCCA)所依賴的正反饋循環。在沒有罰沒機制的系統中,攻擊者可以透過操縱委員會來獲取不當獎勵,進而增加其質押權重,最終逐步掌控整個系統。而在 CACA 中,任何作惡嘗試都會導致質押的減少而非增加,從而從根本上切斷了這種惡性循環的可能性,確保了系統長期治理的穩定性與公正性。


\section{安全性保證}
\label{sec:security_guarantee}

CACA 架構的安全性建立在一個精心設計的雙層信任模型之上,該模型透過巧妙地分配不同層級的安全職責,成功地在維持高效率的同時提供了等同於全網共識的安全保障。這種設計的核心洞察在於認識到,在去中心化系統中,不同類型的安全威脅需要不同程度的防禦機制,而將所有安全職責都交給同一層級的共識機制既沒有必要也不符合效益。傳統的區塊鏈系統通常採用單一層級的信任假設,要求每一次狀態變更都必須經過全網共識的嚴格驗證。這種設計雖然能夠提供強大的安全保證,但其高昂的通訊成本使其難以應用於需要頻繁更新的應用場景。CACA 透過引入分層信任的概念,將效率優化與安全保障解耦,使得系統能夠根據實際威脅的性質動態調整其安全等級。

雙層信任模型的第一層是檢測層(Detection Layer),其採用了極為寬鬆但極其有效的「1-of-N 誠實假設」。這個假設的含義是,只要全網 $N$ 個參與節點中存在至少一個誠實節點願意擔任挑戰者的角色,任何委員會層級的惡意行為就能夠被成功揭露。這種假設的寬鬆程度遠超過傳統拜占庭容錯系統所要求的「三分之二誠實節點」假設,因為它僅需要單一誠實節點的存在而非多數誠實節點的協調行動。從概率角度來看,在一個擁有數百或數千個參與者的大型網路中,所有節點同時選擇沉默或串謀的可能性極其微小,幾乎可以視為不可能事件。更重要的是,這個誠實節點可以是任何類型的參與者,無論是被選入委員會的候補成員、未被選中的閒置節點,還是專門從事監督工作的獨立審計者,只要他們能夠訪問鏈上的公開資料並執行 Krum 演算法的驗證,就具備了發起挑戰的能力。

檢測層的設計巧妙地利用了區塊鏈系統的資料透明性特質。由於所有聚合提案都被完整地記錄在鏈上,任何節點都能夠獨立地重新執行驗證計算,這使得委員會的惡意行為無法被隱藏或掩蓋。攻擊者即使成功控制了當前輪次的整個委員會,也無法阻止其他節點訪問相同的資料並發現異常。這種設計本質上將監督權力從少數特權節點民主化到了整個網路,創造了一個「人人都是潛在監督者」的環境。值得注意的是,檢測層並不要求挑戰者必須在攻擊發生的當下立即發現問題,而是允許在一個合理的時間窗口內進行事後審計。這種靈活性進一步降低了監督的門檻,因為挑戰者可以在方便的時候批次處理多個輪次的驗證工作,而不需要持續保持實時監控的高強度狀態。

雙層信任模型的第二層是仲裁層(Arbitration Layer),其採用了更為嚴格但同樣標準的「全網三分之二誠實假設」。當挑戰被發起並進入仲裁階段後,最終的判決權力從小委員會回歸到全網範圍或至少是一個大規模的陪審團。這個階段的安全假設要求網路中誠實節點的數量 $N_{honest}$ 必須超過總節點數 $N_{total}$ 的三分之二,即 $N_{total} > 3f$,其中 $f$ 為惡意節點的上限數量。這是幾乎所有拜占庭容錯共識協議的標準假設,也是區塊鏈系統普遍依賴的安全基礎。在仲裁階段,所有參與驗證的節點透過 PBFT 協議對挑戰的正當性進行投票,只有當超過三分之二的節點確認委員會確實存在錯誤時,挑戰才會被判定為成立。這種高門檻的設計確保了仲裁結果的可靠性,防止了錯誤挑戰或惡意挑戰對系統造成的干擾。

這兩層信任機制的結合創造了一個強大而靈活的安全框架。在正常運作情況下,系統主要依賴檢測層的低門檻監督來威懾潛在的攻擊行為。攻擊者明確知道,即使只有一個誠實節點存在,其惡意行為也有被揭露的風險,而這種風險所對應的懲罰是全額質押金的損失。這種認知極大地提高了發動攻擊的心理門檻,使得多數理性的攻擊者在權衡利弊後選擇誠實行為。當異常情況真的發生並觸發挑戰時,系統能夠迅速升級到仲裁層,透過全網共識來確保判決的公正性。這種設計的優雅之處在於,它將小委員會的效率優勢與大網路的安全優勢完美結合。小委員會負責日常的快速決策,其可能的錯誤或惡意行為由檢測層持續監督;大網路則作為最終的裁判,在需要時提供不可辯駁的仲裁結果。

從攻擊成本的角度來分析,雙層信任模型顯著提高了成功攻擊所需的資源投入。若攻擊者希望發動一次完整的攻擊並確保不被懲罰,其必須同時滿足兩個極為苛刻的條件。第一個條件是收買當前輪次委員會中超過三分之二的成員,以確保其惡意提案能夠透過委員會的 PBFT 共識。假設委員會規模為 $C$,且成員的選擇基於質押權重,攻擊者需要控制的質押金額至少為全體委員會成員質押總額的三分之二以上。這已經是一筆相當可觀的投資,考慮到質押金的設定通常較高以增加攻擊門檻。然而,僅僅控制委員會還遠遠不夠,因為攻擊者還必須防止其惡意行為被檢測和懲罰。這就引出了第二個更為嚴苛的條件:攻擊者需要收買或壓制全網足夠數量的節點,確保沒有任何誠實節點會發起挑戰,或者即使有挑戰發起,也能在仲裁階段控制超過三分之一的投票權以阻擋共識達成。

這第二個條件的達成難度遠超第一個。在檢測層面,攻擊者面臨的是「1-of-N 誠實假設」的挑戰,這意味著只要有一個節點保持誠實並願意發起挑戰,攻擊就會被揭露。要確保沒有任何節點發起挑戰,攻擊者理論上需要控制或買通全部 $N$ 個可能的挑戰者,這在大型網路中幾乎是不可能完成的任務。即使退一步假設攻擊者無法阻止挑戰的發起,而是選擇在仲裁階段透過操縱投票來逃避懲罰,其所需控制的資源也極為龐大。根據 PBFT 的安全假設,攻擊者需要控制全網至少三分之一以上的節點才能阻止仲裁達成正確的共識。若設全網節點總數為 $N_{total}$,攻擊者需要控制的節點數量至少為 $\lfloor N_{total}/3 \rfloor + 1$。在一個擁有數百個驗證者的網路中,這意味著攻擊者需要同時操控數十個甚至上百個獨立的節點,所需的質押金總額將達到天文數字。

將這兩個條件的成本累加,我們可以得出總攻擊成本的數學表達。設單個委員會成員的平均質押額為 $s_c$,委員會規模為 $C$,則控制委員會所需的成本約為 $\frac{2}{3}C \cdot s_c$。設全網單個節點的平均質押額為 $s_n$,全網節點總數為 $N_{total}$,則在仲裁階段阻擋共識所需的成本約為 $\frac{1}{3}N_{total} \cdot s_n$。總攻擊成本為這兩者之和,即 $Cost_{total} = \frac{2}{3}C \cdot s_c + \frac{1}{3}N_{total} \cdot s_n$。關鍵的觀察在於,雖然 CACA 使用了小委員會來提升效率,但其安全性並未隨之降低到僅依賴小委員會的水平。相反,透過異步挑戰機制的引入,系統的安全性實質上由全網規模 $N_{total}$ 決定而非委員會規模 $C$。這意味著攻擊成本從原本單純控制小委員會的 $O(C)$ 量級,大幅提升到了需要控制全網的 $O(N_{total})$ 量級,實現了安全性的顯著擴展。這種設計使得 CACA 能夠在保持小委員會的效率優勢的同時,享有等同於全網共識的安全保障,從而優雅地解決了去中心化系統中效率與安全之間的經典兩難困境。


\section{效率分析}
\label{sec:efficiency_analysis}

為了全面評估 CACA 架構的實際運作效率,本節透過嚴格的通訊複雜度分析與概率模型推導,論證該架構如何在理論層面實現效率與安全的最佳平衡。通訊複雜度是分散式系統性能的核心指標之一,它直接決定了系統的吞吐量、延遲以及可擴展性。在區塊鏈聯邦學習的場景中,通訊成本的重要性尤為突出,因為模型參數的傳輸往往涉及大量的資料交換,而共識協議又要求多輪的訊息往返。因此,任何試圖在去中心化環境中實現高效機器學習的系統,都必須在通訊複雜度上做出創新性的優化。

傳統的全網 PBFT 共識機制雖然能夠提供強大的拜占庭容錯能力,但其通訊複雜度呈現二次方增長的特性,這在大規模網路中成為了嚴重的性能瓶頸。在標準的 PBFT 協議中,每個驗證節點都需要向其他所有節點廣播其提案或投票訊息,並接收來自其他所有節點的回應。若網路中有 $N$ 個驗證節點,則每一輪共識所需傳遞的訊息數量大致為 $N \times (N-1)$,其漸近複雜度為 $O(N^2)$。這種二次方的增長意味著,當網路規模從 10 個節點擴展到 100 個節點時,通訊成本將增長約 100 倍,而非線性的 10 倍。在聯邦學習場景中,這種指數級的通訊成本增長將嚴重限制系統的實用性,使得全網 PBFT 僅適用於小規模的封閉環境而難以擴展到真正的去中心化網路。

為了緩解這一問題,BlockDFL 等先前研究提出了固定小委員會的解決方案。透過將驗證職責限制在一個規模為 $C$ 的小型委員會內,共識過程僅需在委員會成員之間進行,從而將通訊複雜度降低到 $O(C^2)$。由於 $C$ 通常遠小於全網節點總數 $N$,這種優化能夠帶來顯著的效率提升。然而,這種方法的問題在於,委員會規模的縮小直接導致了安全性的降低。較小的委員會更容易被攻擊者透過累積質押權重來逐步控制,而一旦委員會被控制,整個系統的安全性就蕩然無存。因此,BlockDFL 面臨著一個兩難困境:若要維持足夠的安全性,就必須使用較大的委員會,但這又會削弱效率優勢;若要最大化效率,就必須使用極小的委員會,但這又會帶來不可接受的安全風險。這種固有的矛盾使得固定小委員會方案難以在實際應用中取得理想的效果。

CACA 架構透過引入異步挑戰機制,成功地突破了這一兩難困境。本架構的通訊複雜度特性需要分兩種情況來討論。在正常運作情況下,當委員會誠實執行其職責且沒有挑戰發起時,系統的通訊複雜度與固定小委員會方案完全相同,均為 $O(C^2)$。這是因為驗證過程僅在委員會內部進行,無需全網參與。關鍵的區別在於,CACA 能夠安全地使用比 BlockDFL 更小的委員會規模,因為異步挑戰機制提供了額外的安全保障,使得小委員會的脆弱性不再是致命缺陷。在異常情況下,當挑戰被發起並觸發全網仲裁時,系統的通訊複雜度會暫時上升到 $O(C^2) + O(N^2)$,其中 $O(C^2)$ 代表原始的委員會共識成本,而 $O(N^2)$ 代表全網 PBFT 仲裁的成本。表面上看,這似乎比全網 PBFT 的成本更高,但關鍵在於這種高成本狀態僅在極少數情況下出現,而非每一輪次都必須承擔。

為了量化分析系統的期望通訊複雜度,我們引入挑戰發生的概率 $p$。這個概率代表了在任意給定輪次中,系統需要進行全網仲裁的可能性。在理性行為假設下,由於挑戰機制所帶來的高額經濟懲罰,潛在的攻擊者會意識到發動攻擊的期望收益為負值,因此傾向於選擇誠實行為。這意味著在均衡狀態下,挑戰發生的概率 $p$ 應該趨近於零。即使考慮到偶發的系統錯誤或非理性攻擊者的存在,在一個運作良好的系統中,$p$ 的數值也應該保持在極低的水平,例如 0.1\% 到 1\% 之間。基於這個概率,我們可以計算系統的期望通訊複雜度。每一輪次的通訊成本要麼是正常情況下的 $O(C^2)$(發生概率為 $1-p$),要麼是挑戰情況下的 $O(C^2) + O(N^2)$(發生概率為 $p$)。因此,期望通訊複雜度可表示為:
\begin{equation}
E[Comm] = (1-p) \cdot O(C^2) + p \cdot (O(C^2) + O(N^2)) = O(C^2) + p \cdot O(N^2)
\end{equation}

當 $p \to 0$ 時,上式中的第二項趨近於零,整體的期望複雜度近似於 $O(C^2)$。這個結果表明,在絕大多數時間裡,CACA 的通訊效率與最優化的小委員會方案相當,但同時享有由異步挑戰機制所提供的全網級別安全保障。這種設計實現了一個重要的經濟學原理:將罕見但嚴重的風險事件(委員會作惡)的處理成本推遲到該事件實際發生時才支付,而不是預先在每一輪次中都為此付出代價。這種「按需付費」的安全模式使得系統能夠在正常運作中保持極高的效率,同時具備在需要時迅速升級安全等級的能力。

除了通訊複雜度分析,我們還需要透過概率模型來論證小委員會在配備異步挑戰機制後的安全性。這個分析的核心問題是:在給定網路規模 $N$ 和惡意節點比例 $f$ 的情況下,最小的委員會規模 $C$ 應該設定為多少,才能確保惡意節點控制委員會的機率低於可接受的風險閾值。這個問題的數學建模需要使用超幾何分佈(Hypergeometric Distribution),因為委員會成員的選擇是一個無放回抽樣過程。假設驗證者總池包含 $N$ 個節點,其中惡意節點的數量為 $f \cdot N$,誠實節點的數量為 $(1-f) \cdot N$。當我們從這個池中隨機抽取 $C$ 個節點組成委員會時,委員會中惡意節點數量 $X$ 服從超幾何分佈。

超幾何分佈的概率質量函數描述了在無放回抽樣中獲得特定數量成功樣本的機率。在我們的場景中,「成功」被定義為抽到一個惡意節點。因此,$X = k$ 的機率可以表示為:
\begin{equation}
P(X = k) = \frac{\binom{fN}{k} \binom{(1-f)N}{C-k}}{\binom{N}{C}}
\end{equation}
其中 $\binom{n}{m}$ 表示二項式係數,代表從 $n$ 個元素中選擇 $m$ 個元素的方式數量。這個公式的分子部分計算了選擇 $k$ 個惡意節點和 $C-k$ 個誠實節點的所有可能組合方式,而分母則是從 $N$ 個節點中選擇 $C$ 個節點的總組合數。我們關心的安全性指標是惡意節點在委員會中佔據超過三分之二席位的機率,因為這是 PBFT 共識機制的臨界點。若惡意節點數量 $X$ 達到或超過 $\lfloor 2C/3 \rfloor + 1$,則攻擊者能夠控制委員會的共識結果。因此,委員會被惡意控制的風險概率 $P_{mal}$ 可以表示為:
\begin{equation}
P_{mal} = P(X \ge \lfloor 2C/3 \rfloor + 1) = \sum_{k=\lfloor 2C/3 \rfloor + 1}^{C} \frac{\binom{fN}{k} \binom{(1-f)N}{C-k}}{\binom{N}{C}}
\end{equation}

為了具體理解這個概率模型的含義,讓我們考察一個實際的數值案例。假設驗證者總池規模 $N = 100$,網路中惡意節點的比例 $f = 0.3$,即存在 30 個惡意節點和 70 個誠實節點。這是一個相對極端的假設,因為 30\% 的惡意比例已經接近大多數拜占庭容錯系統所能容忍的上限。在這種情況下,我們可以計算不同委員會規模下被惡意控制的風險。當委員會規模 $C=5$ 時,惡意節點需要至少佔據 4 個席位才能達到控制閾值。透過超幾何分佈的計算,這種情況發生的機率約為 2.74\%。雖然這個風險不算高,但在某些對安全性要求極為嚴格的應用中可能仍不夠理想。若將委員會規模增加到 $C=7$,惡意節點需要至少 5 個席位才能控制,此時風險機率約為 2.42\%,略有下降但改善不明顯。

當委員會規模進一步增加到 $C=9$ 時,情況出現了顯著變化。此時惡意節點需要佔據至少 7 個席位才能達到三分之二的控制閾值,而這種情況發生的機率驟降至約 0.28\%。這個數值已經低於許多實際系統所設定的風險容忍度(通常為 1\%)。繼續增加委員會規模,當 $C=11$ 時風險進一步降至約 0.25\%,當 $C=13$ 時則降至約 0.21\%。這些數據揭示了一個重要的洞察:即使在相當高的惡意節點比例(30\%)下,只需要一個規模適中的委員會(如 9 到 13 個成員)就能將被惡意控制的風險壓制到極低的水平。更重要的是,這個風險水平是在沒有考慮異步挑戰機制的情況下計算的。當我們將挑戰機制納入考量後,即使這低於 1\% 的概率事件真的發生,攻擊者也將在事後面臨全額質押金的罰沒,從而使得攻擊在經濟上變得不可行。

這個概率分析的結論具有深遠的實踐意義。它證明了 CACA 能夠安全地使用極小的委員會規模,例如 9 到 15 個成員,而不會顯著增加安全風險。相比之下,若要達到相同的安全保障水平,傳統的全網 PBFT 需要所有 100 個節點參與共識,其通訊複雜度為 $O(100^2) = O(10000)$。而 CACA 在使用 9 個成員的委員會時,通訊複雜度僅為 $O(9^2) = O(81)$,效率提升超過 100 倍。這種巨大的效率差異使得 CACA 能夠在實際應用中達到接近中心化系統的性能,同時保持去中心化架構所帶來的安全性和抗審查性。更進一步地,當網路規模擴大時,這種效率優勢會變得更加顯著。若驗證者池增長到 $N=1000$,全網 PBFT 的複雜度將膨脹到 $O(1000^2) = O(1000000)$,而 CACA 依然可以使用相同規模的小委員會(因為概率分析顯示,在更大的池中抽取相同規模的委員會,風險反而會進一步降低),其複雜度保持在 $O(81)$ 的量級。這種可擴展性特質使得 CACA 特別適合應用於大規模的去中心化聯邦學習平台,為數千甚至數萬參與者的協作學習提供了理論基礎。


\section{激勵機制}
\label{sec:incentive_mechanism}

激勵機制是維持去中心化系統長期穩定運行的根本動力,其設計的優劣直接決定了系統能否在沒有中心化權威的情況下自發形成良性的治理秩序。在 CACA 架構中,激勵機制的設計遵循博弈論與機制設計理論的核心原則,旨在創造一個激勵相容(Incentive Compatible)的環境,使得誠實行為成為所有理性參與者的最優策略。與傳統的區塊鏈系統依賴持續增發代幣來支付安全成本不同,CACA 採用了一種更為可持續且經濟高效的方法,即透過對違規者的資產罰沒(Slashing)來支付審計與仲裁的相關費用。這種「懲罰驅動」的激勵模式具有多重優勢,既避免了通貨膨脹對代幣價值的長期侵蝕,又確保了安全成本由真正造成風險的行為者承擔,而非由全體參與者分攤。

罰沒機制的核心設計理念在於建立一個極度不對稱的風險收益結構,使得攻擊行為在經濟上變得完全不理性。每個願意擔任驗證者或聚合者角色的節點,都必須預先質押一定數量的代幣作為其誠實行為的保證金。這個質押金的規模被精心設定在一個足夠高的水平,確保其價值遠超過任何單次攻擊所能獲得的潛在收益。當節點被證實存在惡意行為,例如驗證委員會成員串謀選擇了錯誤的聚合結果,或者聚合者提交了惡意構造的虛假提案,系統將立即沒收其全額質押金。這種懲罰的嚴厲程度傳遞了一個明確的訊號:在 CACA 系統中,任何作惡嘗試都將導致災難性的經濟損失,而這種損失是即刻的、確定的且不可逆轉的。相對地,誠實參與者雖然需要承擔質押金被鎖定的機會成本,但能夠獲得穩定且可預期的區塊獎勵,這種穩定收益的累積在長期內將遠超過任何一次性攻擊所能帶來的非法所得。

被罰沒的資金並非簡單地從系統中移除或銷毀,而是透過精心設計的分配機制來強化激勵相容性。資金分配的首要受益者是成功發起挑戰的挑戰者,他們將獲得罰沒金額中相當可觀的一部分作為獎勵。這種設計確保了監督委員會行為成為一項有利可圖的經濟活動,從而吸引足夠數量的節點願意投入資源進行持續的審計工作。挑戰者的獎勵必須足夠高,以覆蓋其進行驗證計算的運算成本、質押挑戰押金的機會成本,以及承擔錯誤挑戰被反向懲罰的風險溢價。在實際設計中,挑戰成功後的獎勵通常被設定為罰沒金額的 30\% 到 50\%,這個比例確保了挑戰活動具備充分的經濟激勵。剩餘的罰沒資金則分配給全體誠實參與者,特別是那些在被攻擊輪次中提供了正確更新的訓練者,以及積極參與了仲裁過程的驗證節點。這種廣泛的獎勵分配機制不僅補償了誠實節點因系統遭受攻擊而承受的潛在損失,更重要的是創造了一種集體監督的文化,使得每個參與者都有動力關注系統的整體健康狀況。

激勵機制的另一個關鍵設計要素是動態調整能力,使得系統能夠根據實際運作情況自適應地優化參數設定。若系統在一段長時間內未發生任何挑戰事件,這可能暗示著兩種截然不同的情況:一種可能是威懾機制運作良好,所有參與者都選擇了誠實行為;另一種可能是挑戰門檻設置過高,抑制了潛在挑戰者的監督意願。為了區分這兩種情況並確保監督機制的活躍性,系統可以在長期無挑戰的情況下適當降低挑戰者的質押門檻,或者增加挑戰成功後的獎勵比例,從而鼓勵更多節點參與到監聽工作中。這種調整應該是漸進且謹慎的,避免因過度降低門檻而引發惡意挑戰的濫用問題。相反地,若系統在某一時期內挑戰頻發,這可能意味著當前的懲罰力度不足以威懾攻擊者,或者質押金要求過低使得攻擊成本可以被接受。在這種情況下,系統可以提高委員會成員和聚合者的最低質押要求,同時增加罰沒比例,以強化經濟威懾效果。這種動態調整機制使得 CACA 能夠在面對不斷演變的威脅環境時保持適應性,確保激勵結構始終處於最優配置狀態。

從長期均衡的角度來分析,CACA 的激勵機制創造了一個穩定且可持續的經濟生態。對於誠實節點而言,參與系統的期望收益來自於兩個管道:一是擔任驗證者、聚合者或訓練者時獲得的常規區塊獎勵,這是一種穩定且可預測的收入流;二是在極少數情況下,當系統遭受攻擊時,透過參與挑戰或仲裁而獲得的額外獎勵。由於攻擊事件在均衡狀態下極為罕見,後者僅能視為偶發的獎金而非穩定收入。然而,即使只依賴前者,誠實參與的長期回報率也足以吸引理性的節點持續參與系統。關鍵在於,這種誠實參與是低風險的,節點只需按照協議規則執行其職責,就能確保獲得獎勵而不會面臨質押金損失的風險。相對地,對於潛在的攻擊者,其決策邏輯則完全不同。發動一次成功的攻擊能夠帶來的收益是有限的,主要體現在該輪次中對模型更新方向的控制權,而這種控制權的價值在聯邦學習場景中往往並不高,因為單次的模型偏移很快就會被後續的誠實更新所修正。然而,攻擊的成本卻是極為高昂的,不僅包括控制委員會所需的大量質押金投入,更包括一旦攻擊被發現後全額質押金的損失。

更進一步地,攻擊者還必須考慮長期的機會成本。在 CACA 系統中,被罰沒的節點將永久失去其驗證者資格,這意味著他們不僅失去了當前的質押金,也失去了未來所有輪次中獲得驗證獎勵的機會。若我們假設系統將長期穩定運行,這種未來收益流的淨現值可能遠超一次性攻擊所能獲得的短期利益。因此,任何理性的攻擊者在進行成本效益分析時,都會得出攻擊在經濟上完全不划算的結論。這種極度不對稱的風險收益結構,是 CACA 激勵機制設計的核心成就。它不依賴於對參與者道德水平的假設,而是透過純粹的經濟邏輯來引導行為,使得即使是完全自利且缺乏道德約束的理性行為者,也會選擇誠實參與而非發動攻擊。這種激勵相容性確保了系統能夠在沒有中心化監管的情況下實現自我治理,為去中心化聯邦學習平台的長期穩定運作奠定了堅實的經濟基礎。


\section{本章小結}

本章提出的挑戰增強型委員會架構代表了區塊鏈聯邦學習系統設計理念的一次重要轉變,其核心創新在於透過異步審計與經濟懲罰機制的引入,成功地將傳統上互相衝突的效率與安全性目標統一到一個連貫的框架之中。這種統一並非透過在兩者之間尋求妥協而達成,而是透過重新思考安全性的實現方式,將同步驗證的即時成本轉化為異步審計的條件成本,從而在不犧牲長期安全保障的前提下,最大化了系統的執行效率。透過移除傳統區塊鏈系統中無所不在的確認等待期,CACA 使得聯邦學習訓練過程能夠以接近中心化系統的速度持續推進,每一輪模型更新都能夠在委員會達成共識後立即生效,而不需要等待冗長的全網確認。

CACA 的安全性保障建立在雙層信任模型的堅實基礎之上,該模型巧妙地利用了聯邦學習系統固有的容錯特性以及區塊鏈網路的資料透明性。透過將檢測職責開放給所有願意參與的節點,系統將監督門檻降低到了極致,只需存在單一誠實節點願意執行挑戰,任何委員會層級的惡意行為都將無所遁形。與此同時,透過保留全網仲裁作為最終判決機制,系統確保了在真正需要時能夠動員全網資源來確保判決的公正性與不可辯駁性。本章透過嚴格的超幾何分佈分析證明,即使在相當高的惡意節點比例下,只需要一個規模極小的委員會配合異步挑戰機制,就能夠將系統被攻擊的風險控制在可接受的範圍內,而一旦這種小概率風險真的發生,經濟懲罰機制將確保攻擊者付出遠超其收益的代價。

通訊複雜度分析進一步揭示了 CACA 架構的效率優勢。在絕大多數正常運作的情況下,系統的通訊成本維持在小委員會共識的 $O(C^2)$ 量級,相較於全網 PBFT 的 $O(N^2)$ 複雜度實現了數量級的降低。即使在極少數需要觸發全網仲裁的異常情況下,由於這種情況的發生概率趨近於零,其對系統整體期望性能的影響也微乎其微。這種設計使得 CACA 能夠在保持極高執行效率的同時,享有等同於全網共識的安全保障,從而優雅地解決了去中心化系統設計中的經典難題。激勵機制的創新設計則確保了這種架構不僅在理論上可行,在實踐中也能夠長期穩定運作。透過建立極度不對稱的風險收益結構,CACA 使得誠實行為成為所有理性參與者的最優策略,而任何試圖操縱系統的行為都將面臨災難性的經濟後果。

總體而言,本章所提出的 CACA 架構為區塊鏈聯邦學習系統提供了一條突破效率與安全兩難困境的可行路徑。它不僅解決了現有系統面臨的技術挑戰,更重要的是提供了一套完整的理論框架,可以指導未來更多去中心化機器學習應用的設計。然而,理論分析終究需要實證驗證來支撐其有效性。下一章將透過多維度的模擬實驗,在各種攻擊場景下驗證 CACA 架構的實際性能表現,特別是其在面對第三章所描述的漸進式委員會佔領攻擊時的防禦能力與系統穩健性,從而為本研究的理論主張提供實證基礎。


\end{ZhChapter}