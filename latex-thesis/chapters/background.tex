\chapter{相關技術背景}
\label{chapter:background}

本章節旨在建立理解本研究所需之技術基礎,涵蓋聯邦學習、拜占庭容錯機制、區塊鏈架構以及現代共識系統中的經濟安全設計。本章內容採說明性敘述,為後續章節之問題分析與系統設計提供理論框架。

\section{聯邦學習與拜占庭容錯 (Federated Learning and Byzantine Resilience)}
\label{sec:fl_byzantine}

聯邦學習(Federated Learning, FL)是由McMahan等人於2017年正式提出之分散式機器學習框架 \cite{mcmahan2017communication}。其核心目標在於多個參與方(Clients)協同訓練模型,而無需將原始資料集中於中央伺服器,從而保護資料隱私。

\subsection{聯邦學習數學定義}
在標準聯邦學習架構中,目標是最小化全域損失函數 $F(w)$:
\begin{equation}
    \min_{w} F(w) = \sum_{k=1}^{K} \frac{n_k}{n} F_k(w)
\end{equation}
其中 $K$ 為參與客戶端總數,$n_k$ 為第 $k$ 個客戶端之本地樣本數,$F_k(w)$ 為其本地損失函數。經典的 \textit{FederatedAveraging} (FedAvg) 演算法透過週期性地收集客戶端模型更新 $w_{t+1}^k$,並在伺服器端進行加權聚合:
\begin{equation}
    w_{t+1} \leftarrow \sum_{k=1}^{K} \frac{n_k}{n} w_{t+1}^k
\end{equation}
此方法雖然顯著降低了通訊開銷,但其安全性建立在中央聚合器完全誠實且客戶端皆非惡意的假設之上。

\subsection{拜占庭故障與強健聚合}
在開放或非受信任環境中,部分參與者可能發生拜占庭故障(Byzantine fault),即發送任意、甚至是蓄意偽造的梯度向量。Blanchard等人證明了線性聚合規則(如算術平均)無法抵禦即便只有一個拜占庭節點的攻擊 \cite{blanchard2017machine}。為了應對此威脅,研究界提出了多種拜占庭容錯(Byzantine-robust)聚合機制,例如:
\begin{itemize}
    \item \textbf{Krum/Multi-Krum}:透過運算各向量間的歐幾里得距離,選擇與周圍節點距離最近的向量,以排除偏離較大的惡意更新。
    \item \textbf{座標式裁剪均值 (Trimmed Mean)}:在各維度上移除最大與最小的部分觀測值,對剩餘值取平均。Yin等人證明了該方法在特定機率分佈下可達到階數最優統計誤差率 \cite{yin2018byzantine}。
    \item \textbf{座標式中位數 (Coordinate-wise Median)}:取各維度上的中位數作為聚合結果,具有較高的崩潰點(Breakdown point)。
\end{itemize}
同樣地,Bulyan 演算法 \cite{el2018hidden} 被提出用於縮減高維度下投毒攻擊的空間。
儘管這些機制增強了對惡意客戶端的防禦力,但它們均隱含了一個關鍵的前提:執行這些規則的「中央聚合器」必須是絕對誠實的。

\section{基於區塊鏈的聯邦學習 (Blockchain-based Federated Learning)}
\label{sec:bcfl}

為了消除對單一中央伺服器的依賴,研究者引入區塊鏈技術,提出區塊鏈式聯邦學習(BCFL)架構。在此架構中,去中心化帳本取代了傳統聚合器,提供不可篡改性與透明性。

\subsection{BCFL 架構演進}
BCFL的發展經歷了從全節點共識到委員會機制的演進:
\begin{enumerate}
    \item \textbf{早期架構 (PoW-based)}:如 BlockFL \cite{kim2020blockchained} 使用工作量證明(PoW)達成共識,雖具備高度去中心化特性,但面臨高能耗與高延遲問題。Lu等人則探討了將訓練品質(PoQ)與共識結合的工業物聯網架構 \cite{lu2020blockchain}。
    \item \textbf{委員會共識 (Committee-based)}:如 BFLC \cite{li2021blockchain} 與 BlockDFL \cite{qin2024blockdfl}。為了提升效能,系統從全體參與者中選出一個子集(委員會)負責驗證與聚合。此種機制將通訊複雜度從 $O(n^2)$ 降低至 $O(C^2)$,其中 $C$ 為委員會大小。隨後的研究如 VBFL \cite{chen2021robust} 與 VFChain \cite{peng2022vfchain} 分別引入了基於權益的共識與可審計的聚合證明。
\end{enumerate}

\subsection{基於質押的參與機制}
現代 BCFL 系統常借鑑權益質押(Staking)概念,要求節點質押代幣以獲得成為委員會成員的權利。此機制建立了經濟進入門檻,並將參與者的利益與系統的整體安全綁定,為進階的激勵與懲罰機制奠定了基礎。

\section{激勵與罰沒機制基礎 (Incentive and Slashing Foundations)}
\label{sec:staking_slashing}

加密經濟安全性(Crypto-economic Security)的核心在於確保「攻擊成本高於潛在收益」。這主要透過權益質押與罰沒(Slashing)機制來達成。

\subsection{權益質押與經濟終局性}
在主流共識協議(如 Casper FFG \cite{buterin2017casper}、Tendermint \cite{tendermint}、Cosmos \cite{kwon2016cosmos} 或 Polkadot \cite{wood2016polkadot})中,驗證者必須存入押金。若驗證者違反協定規則(例如雙重投票),其部分或全部押金將被系統自動沒收。這種機制確保了系統具有「可問責性」(Accountability),即任何破壞安全性的行為都能被識別並追究經濟責任 \cite{chiu2018incentive}。

\subsection{罰沒機制的設計原則}
一個完善的罰沒機制具備以下特性:
\begin{itemize}
    \item \textbf{即時性}:違規證據一經提交,處罰應在區塊鏈上立即生效。
    \item \textbf{相關性懲罰 (Correlation Penalty)}:如 Gasper \cite{gasper} 所採用,當多個節點在同一時間段內發生違規行為時,罰沒比例會非線性增加,以有效遏止大規模協同共謀。
    \item \textbf{激勵相容性}:設計旨在使誠實行為成為理性參與者的最優策略。
\end{itemize}

總結而言,聯邦學習提供了協同訓練的框架,拜占庭容錯提供了演算法層面的防禦,而區塊鏈及其經濟機制則提供了去中心化的信任根基。然而,當這些技術結合時,如何確保「監督者(委員會)」本身不被攻陷,仍是現有技術未能完全解決的問題。
