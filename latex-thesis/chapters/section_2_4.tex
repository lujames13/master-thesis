\subsection{基準系統模型:BlockDFL 委員會架構}
\label{sec:blockdfl-baseline}

為深入分析區塊鏈聯邦學習中委員會機制的安全性,本研究採用 BlockDFL \cite{qin2024blockdfl} 作為基準系統模型。BlockDFL 於 2024 年發表於 WWW 會議,代表當前完全去中心化點對點聯邦學習架構的最新進展。該系統透過角色分離、權益加權選舉與拜占庭容錯共識的結合,在效率與安全性之間取得了當前文獻中的最佳平衡。本節將詳細介紹其系統架構、運作流程與獎勵機制,作為後續威脅分析的基礎框架。

\subsubsection{系統角色與職責定義}

BlockDFL 採用角色分離的設計理念,將參與者依據其在每輪訓練中承擔的職責劃分為三種角色:更新提供者(Update Provider)、聚合者(Aggregator)與驗證者(Verifier)。這種分工模式源於一個核心洞察:在去中心化環境中,若由單一節點同時負責訓練、聚合與驗證,將難以建立有效的制衡機制。透過將這三項職責分派給不同的參與者群體,系統得以在各環節引入相互監督,降低單點惡意行為對全域模型的影響。

更新提供者構成系統中的多數參與者,其職責是利用本地私有資料執行模型訓練,並將訓練所得的模型更新提交給聚合者。由於訓練資料始終保留在本地裝置,更新提供者的隱私得以保護,這體現了聯邦學習「資料不動、模型動」的核心價值。聚合者的職責則是收集來自多個更新提供者的本地更新,執行篩選與聚合運算,將結果打包為全域更新提案並提交給驗證者。每輪訓練中可能有多個聚合者同時運作,各自獨立收集更新並提交競爭性的提案,這種設計避免了單一聚合者壟斷聚合權的風險。驗證者組成委員會,負責評估各聚合者提交的提案品質,並透過拜占庭容錯共識機制選出最佳提案寫入區塊鏈。驗證者的數量通常遠小於更新提供者,以維持共識效率。

角色的分配並非固定不變,而是在每輪訓練開始時根據上一區塊的雜湊值與各參與者的權益(stake)重新決定。具體而言,區塊雜湊被映射至一個雜湊環,每個參與者依據其權益大小佔據環上相應比例的空間。系統依序從雜湊環上選出聚合者與驗證者,未被選中者則成為更新提供者。這種機制確保了角色分配的確定性與不可預測性:一方面,給定相同的區塊雜湊與權益分布,角色分配結果完全確定,便於驗證;另一方面,由於區塊雜湊在區塊產生前無法預知,參與者難以提前操控自身角色。更重要的是,權益加權的設計使得持有較多權益的參與者更有可能被選為聚合者或驗證者,這反映了系統對「高權益者傾向誠實」的信任假設。

\subsubsection{訓練輪次的運作流程}

BlockDFL 的每輪訓練遵循一個結構化的流程,從角色分配開始,經由本地訓練、聚合、驗證與共識,最終完成全域模型更新與獎勵分配。圖 \ref{fig:blockdfl-workflow} 呈現了此流程的概念架構,以下將逐一說明各階段的運作邏輯。

% 此處預留流程圖位置
% \begin{figure}[htbp]
% \centering
% \includegraphics[width=\textwidth]{figures/blockdfl-workflow.pdf}
% \caption{BlockDFL 單輪訓練流程示意圖}
% \label{fig:blockdfl-workflow}
% \end{figure}

訓練輪次始於角色分配階段。當新一輪開始時,所有參與者根據最新區塊的雜湊值與當前權益分布,確定性地計算出本輪的角色分配結果。由於計算過程僅依賴公開可驗證的資訊,任何參與者皆可獨立驗證角色分配的正確性,無需依賴中央協調者。角色確定後,更新提供者隨即進入本地訓練階段,在各自的私有資料集上執行若干輪隨機梯度下降,產生本地模型更新並將其發送給聚合者。

聚合階段是 BlockDFL 架構的關鍵環節。每個聚合者獨立收集來自更新提供者的本地更新,當收集數量達到預設閾值後,開始執行篩選與聚合程序。篩選的目的在於過濾潛在的惡意更新:聚合者首先依據更新提供者的權益進行抽樣,優先納入來自高權益節點的更新;接著透過本地推論測試評估各更新的品質,排除表現異常者。通過篩選的更新被聚合為一個全域更新提案,由聚合者簽署後提交給驗證者委員會。值得注意的是,由於多個聚合者同時運作且各自獨立收集更新,同一輪中將產生多個競爭性的提案,這為後續的驗證階段提供了選擇空間。

驗證與共識階段決定了哪個提案將被接受並寫入區塊鏈。當驗證者委員會收到足夠數量的提案後,驗證程序啟動。每位驗證者獨立使用 Krum 演算法 \cite{blanchard2017machine} 對所有提案進行評分,Krum 分數較低者代表與其他提案的整體距離較小,被視為品質較高。基於評分結果,驗證者對各提案進行投票:僅當某提案的 Krum 分數優於三分之二以上的其他提案時,驗證者才會投下贊成票。投票過程遵循簡化的 PBFT 協議,當某提案獲得超過三分之二驗證者的贊成票時,該提案被正式接受。委員會的領導者隨即將接受的提案連同相關資訊打包成新區塊,廣播至全網。所有參與者收到新區塊後,依據其中的全域更新同步更新本地模型,完成本輪訓練。

\subsubsection{獎勵機制與激勵設計}

BlockDFL 的獎勵機制遵循「有貢獻才有回報」的設計原則,旨在解決分散式系統中普遍存在的搭便車問題。在傳統聯邦學習中,無論參與者是否真正貢獻高品質的訓練成果,皆可獲得最終全域模型的使用權,這削弱了誠實參與的激勵。BlockDFL 透過將權益獎勵與「被選中」直接綁定,確保只有對本輪全域模型更新有實質貢獻的參與者才能獲得回報,從而建立起正向的激勵結構。

具體而言,當某一提案通過委員會共識並被寫入區塊鏈時,系統將權益增量分配給三類參與者。第一類是提交該提案的聚合者,其承擔了收集更新、執行篩選與聚合運算的工作,並承受提案可能未被選中的風險。第二類是本地更新被納入該提案的更新提供者,他們貢獻了訓練計算資源與本地資料的價值。第三類是對該提案投下贊成票的驗證者,他們執行了驗證運算並參與了共識決策。這三類參與者均分本輪的權益獎勵,其身份被明確記錄於區塊之中,確保獎勵分配的透明與可驗證。

相對地,未對本輪全域模型更新做出貢獻的參與者則不獲得任何獎勵。這包括:提案未被選中的其他聚合者、本地更新未被納入獲選提案的更新提供者、以及對獲選提案投下反對票或未參與投票的驗證者。這種設計創造了明確的激勵導向:聚合者有動機提交高品質的提案以提高被選中的機率;更新提供者有動機提交優質的本地更新以增加被納入提案的可能性;驗證者則有動機投票支持真正優質的提案,因為只有投票與最終結果一致時才能獲得獎勵。

然而,此獎勵機制在激勵誠實行為的同時,也創造了一個具有正反饋特性的動態系統。當參與者獲得權益獎勵時,其總權益增加,這直接提升了該參與者在未來輪次被選為聚合者或驗證者的機率,進而增加其獲得更多獎勵的機會。BlockDFL 的設計者預期此正反饋將使「持續誠實貢獻的參與者逐漸累積優勢,而惡意參與者的影響力則相對削弱」\cite{qin2024blockdfl}。這項預期建立在一個關鍵假設之上:獲得高權益的參與者必然是長期誠實貢獻者。然而,若攻擊者能夠在潛伏期間偽裝成誠實參與者並成功累積權益,此正反饋機制反而可能成為攻擊者鞏固優勢的工具。這一安全隱患將在本章第 \ref{sec:research-gap} 節進一步探討。

\subsubsection{本研究採用此模型的理由}

本研究選擇 BlockDFL 作為基準系統模型,基於以下四項考量。首先,BlockDFL 代表了區塊鏈聯邦學習委員會架構的最新技術水準,其於 2024 年發表於頂級網路研討會,融合了角色分離、權益加權選舉、雙層評分機制與拜占庭容錯共識等多項先進設計,具有高度的代表性。其次,BlockDFL 的系統定義清晰完整,論文詳細說明了角色職責、運作流程與獎勵規則,這為形式化的威脅分析提供了堅實基礎。相較於部分僅提供概念性描述的研究,BlockDFL 的明確定義使得安全性分析能夠建立在具體的系統行為之上,而非抽象的假設。

第三,BlockDFL 的委員會機制與獎勵設計具有廣泛的通用性。儘管不同 BCFL 系統在具體實現上存在差異,但「小型委員會執行共識」與「權益驅動的角色選舉」已成為此領域的主流設計範式。因此,針對 BlockDFL 所發現的安全問題與提出的防禦機制,在原理上可推廣至採用類似架構的其他系統。最後,BlockDFL 已有公開的實驗數據與效能基準,這為本研究後續的防禦機制評估提供了可比較的參照點。綜合以上考量,BlockDFL 是本研究進行威脅建模與防禦設計的理想分析對象。