\begin{ZhChapter}

\chapter{威脅模型 (Threat Model)}
\label{chap:threat-model}

本章定義本研究所針對的威脅模型,特別聚焦於區塊鏈聯邦學習系統中的「委員會佔領攻擊」(Committee Capture Attack)。如第三章文獻分析所示,現有研究主要關注數據層的惡意客戶端攻擊,而系統性地忽略了共識層的驗證者共謀問題。本章將詳細描述系統模型、攻擊者能力、攻擊向量,並重點定義「漸進式權益佔領攻擊」(Progressive Committee Capture Attack, PCCA),為後續章節的防禦機制設計提供明確的安全目標。

\section{系統模型與假設}
\label{sec:system_model}

\subsection{網絡模型}

本研究考慮一個去中心化的區塊鏈聯邦學習系統,由以下組件構成:

客戶端集合 $\mathcal{C} = \{c_1, c_2, ..., c_N\}$:共 $N$ 個參與訓練的客戶端,每個客戶端 $c_i$ 持有本地私有數據集 $\mathcal{D}_i$。客戶端負責在本地數據上訓練模型,並提交模型更新 (梯度或權重) 至區塊鏈網絡。

驗證者集合 $\mathcal{V} = \{v_1, v_2, ..., v_M\}$:共 $M$ 個驗證者節點,每個驗證者 $v_j$ 持有一定數量的權益 $s_j$。驗證者負責驗證客戶端提交的更新、執行聚合算法,並通過共識機制將聚合結果記錄上鏈。

委員會選擇機制:在每個訓練輪次 $r$,系統從驗證者集合中選擇一個大小為 $C$ 的委員會 $\mathcal{V}_r \subset \mathcal{V}$。選擇機制基於權益加權隨機抽樣,驗證者 $v_j$ 被選中的機率為:

\begin{equation}
P(v_j \in \mathcal{V}_r) = \frac{s_j}{\sum_{k=1}^{M} s_k}
\end{equation}

這種機制確保權益較高的驗證者有更高機率參與委員會,從而獲得更多獎勵機會。

\subsection{聚合與共識流程}

在每個訓練輪次,系統執行以下流程:

\begin{enumerate}
    \item 本地訓練:客戶端在本地數據上訓練模型,計算模型更新 $\Delta w_i$。
    \item 更新提交:客戶端將更新提交至區塊鏈網絡,等待委員會處理。
    \item 委員會聚合:委員會 $\mathcal{V}_r$ 收集客戶端更新,執行拜占庭魯棒聚合算法 (如 Krum、Trimmed Mean) 計算全局更新 $\Delta w_g$。
    \item 共識驗證:委員會成員通過 BFT 共識協議 (如 PBFT) 對聚合結果達成一致,並將結果記錄上鏈。
    \item 獎勵分配:成功參與聚合的委員會成員獲得獎勵 $R$,平分給所有參與成員。每個成員的權益增加 $\Delta s = R / C$。
\end{enumerate}

\subsection{權益動態機制}

權益在系統中扮演雙重角色:

\begin{itemize}
    \item 選擇權重:權益決定驗證者與其被選入委員會的機率,權益越高,參與機會越多。
    \item 經濟激勵:參與委員會的驗證者獲得獎勵,進一步增加其權益,形成正反饋循環。
\end{itemize}

這種機制設計的初衷是激勵誠實行為:誠實驗證者通過持續參與獲得獎勵,權益不斷增長,從而鞏固其在系統中的影響力。然而,如本章後續所示,這種機制也可能被惡意驗證者利用,通過排他性的獎勵分配實現權益壟斷。

\subsection{系統假設}

本研究基於以下假設:

\begin{itemize}
    \item 網絡假設:網絡為部分同步模型,消息最終會被傳遞,但傳遞延遲有上界。
    \item 密碼學假設:密碼學原語 (如數字簽名、哈希函數) 是安全的,攻擊者無法偽造簽名或碰撞哈希。
    \item 誠實客戶端存在:系統中至少存在一定比例的誠實客戶端,其提交的更新是基於真實數據的正常訓練結果。
    \item 可驗證性假設:聚合結果的正確性可以被驗證。任何節點都可以重新執行聚合算法,驗證委員會提交的結果是否正確。
\end{itemize}

\section{攻擊者模型}
\label{sec:adversary_model}

\subsection{攻擊者類型:理性攻擊者}

本研究考慮的攻擊者為理性攻擊者 (Rational Adversary),而非傳統的拜占庭攻擊者。兩者的關鍵區別在於動機:

\begin{itemize}
    \item 拜占庭攻擊者:以破壞系統為目標,可能採取任意惡意行為,即使損害自身利益也在所不惜。
    \item 理性攻擊者:以利益最大化為目標,僅在預期收益大於成本時才發動攻擊。如果攻擊的預期收益為負,理性攻擊者不會嘗試作惡。
\end{itemize}

這種區分至關重要,因為它為基於博弈論的防禦機制提供了理論基礎。如果能夠設計激勵機制,使得攻擊的預期收益為負,理性攻擊者將自發地選擇誠實行為,無需依賴傳統的多數誠實假設。

\subsection{攻擊者目標}

理性攻擊者的主要目標包括:

\begin{itemize}
    \item 經濟利益:通過操縱委員會,獨佔訓練獎勵,排擠誠實節點的收益。
    \item 權益壟斷:通過阻止誠實節點的權益增長,逐步提高自身在系統中的權益佔比,最終控制委員會選擇過程。
    \item 網絡控制:當攻擊者的權益佔比足夠高時,可以持續控制委員會,進而控制整個聯邦學習過程,包括模型更新的接受與拒絕。
\end{itemize}

值得注意的是,理性攻擊者的目標不僅僅是短期的經濟收益,更重要的是長期的網絡控制權。這種攻擊不同於傳統的模型投毒攻擊,後者僅影響模型品質,而前者則從根本上顛覆了去中心化系統的安全假設。

\subsection{攻擊者能力}

本研究假設攻擊者具有以下能力:

\begin{itemize}
    \item 節點控制:攻擊者可以控制系統中一定比例的驗證者節點,記為 $f$。在典型場景下,假設 $f \leq 0.3$,即攻擊者控制不超過 30\% 的節點。
    \item 協同作惡:被攻擊者控制的節點可以相互協調,共同執行攻擊策略。例如,當多個惡意節點同時被選入委員會時,它們可以串通一致地投票。
    \item 策略性行為:攻擊者可以根據系統狀態動態調整策略。例如,在權益較低時表現誠實以積累信譽,在獲得委員會多數席位時發動攻擊。
    \item 觀察能力:攻擊者可以觀察區塊鏈上的所有公開信息,包括其他節點的權益、歷史行為、委員會組成等,並據此制定攻擊策略。
\end{itemize}

\subsection{攻擊者限制}

同時,攻擊者受到以下限制:

\begin{itemize}
    \item 密碼學限制:攻擊者無法破解密碼學原語,無法偽造其他節點的簽名或篡改已上鏈的數據。
    \item 算力限制:攻擊者無法控制全網多數節點,無法單獨發動 51\% 攻擊。
    \item 經濟約束:攻擊者受經濟激勵約束,如果攻擊的預期成本大於收益,理性攻擊者不會嘗試攻擊。
    \item 可驗證性:攻擊者無法阻止其他節點驗證聚合結果的正確性。任何節點都可以重新執行聚合算法,檢測委員會是否正確執行協議。
\end{itemize}

\section{攻擊向量分析}
\label{sec:attack_vectors}

區塊鏈聯邦學習系統面臨多層次的安全威脅。本節分析不同層次的攻擊向量,並說明本研究的關注目點。

\subsection{數據層攻擊:已有防禦}

數據層攻擊主要指惡意客戶端通過投毒攻擊破壞模型品質:

\begin{itemize}
    \item 數據投毒 (Data Poisoning):惡意客戶端在本地訓練時使用被污染的數據集,導致訓練出的模型更新偏離正常分佈。
    \item 模型投毒 (Model Poisoning):惡意客戶端直接構造惡意的模型更新,而非基於真實訓練過程。
\end{itemize}

針對這類攻擊,現有研究已提出多種拜占庭魯棒聚合算法,如 Krum、Trimmed Mean、Median 等。這些算法通過統計方法識別並過濾異常更新,在一定比例的惡意客戶端存在時仍能保證模型收斂。

然而,這些防禦方法存在一個關鍵假設:執行聚合算法的驗證者是誠實的。如果驗證者本身是惡意的,它們可以選擇不執行這些防禦算法,或者篡改算法的執行結果,從而使數據層的防禦完全失效。

\subsection{共識層攻擊:本研究重點}

共識層攻擊針對的是執行聚合和驗證的委員會本身:

\begin{itemize}
    \item 驗證者共謀 (Verifier Collusion):多個惡意驗證者協同作惡,共同通過惡意的聚合結果。
    \item 委員會佔領 (Committee Capture):攻擊者通過操縱委員會選擇機制,逐步增加惡意節點在委員會中的佔比,最終控制委員會。
\end{itemize}

如第三章文獻分析所示,現有區塊鏈聯邦學習研究存在系統性的「驗證層盲點」:約 93\% 的研究假設驗證者是誠實的或滿足誠實多數,僅有極少數研究 (如 KFC) 明確考慮惡意驗證者的場景。

本研究聚焦於共識層攻擊,特別是委員會佔領攻擊。這種攻擊的危險性在於:

\begin{itemize}
    \item 繞過數據層防禦:惡意委員會可以直接接受惡意更新,無需執行 Krum 等防禦算法。
    \item 隱蔽性強:攻擊者在初期表現誠實,不易被檢測,等到權益足夠高時才發動攻擊。
    \item 自我強化:一旦攻擊成功,攻擊者的權益會進一步增加,形成正反饋,使得攻擊越來越容易。
\end{itemize}

\subsection{攻擊層次對比}

表 \ref{tab:attack_comparison} 對比了不同層次攻擊的特徵與現有防禦情況。

\begin{table*}[htbp]
\centering
\caption{攻擊層次對比}
\label{tab:attack_comparison}
\begin{tabular}{|l|l|l|l|l|l|}
\hline
攻擊層次 & 攻擊者 & 攻擊目標 & 現有防禦 & 防禦假設 & 本研究關注 \\ \hline
數據層 & 惡意客戶端 & 模型品質 & Krum, Trimmed Mean & 驗證者誠實 & 否 \\ \hline
共識層 & 惡意驗證者 & 網絡控制 & 誠實多數假設 & 多數驗證者誠實 & 是 \\ \hline
\end{tabular}
\end{table*}

從表中可以看出,數據層攻擊已有成熟的防禦方法,但這些方法依賴於驗證者誠實執行的假設。相比之下,共識層攻擊的防禦仍依賴於誠實多數假設,缺乏針對理性攻擊者的激勵相容機制。

\section{漸進式權益佔領攻擊 (Progressive Committee Capture Attack)}
\label{sec:pcca}

本節詳細定義本研究針對的核心威脅:漸進式權益佔領攻擊 (Progressive Committee Capture Attack, PCCA)。這是一種針對基於權益的委員會選擇機制的隱蔽性攻擊,通過兩階段策略逐步實現網絡控制。

\subsection{攻擊定義}

漸進式權益佔領攻擊是指攻擊者通過以下策略,逐步增加其在系統中的權益佔比,最終控制委員會選擇過程:

\begin{enumerate}
    \item 潛伏階段:攻擊者在初期表現誠實,提交正常的模型更新,積累權益與信譽。
    \item 佔領階段:當攻擊者在委員會中獲得多數席位時,啟動「戰略性餓死」策略,拒絕打包誠實節點的更新,獨佔獎勵。
    \item 權益壟斷:由於誠實節點無法獲得獎勵,其權益停滯;而惡意節點持續獲得獎勵,權益呈指數增長,進一步提高其在未來委員會中的佔比。
\end{enumerate}

這種攻擊的關鍵在於利用了權益機制的正反饋特性:權益高的節點更容易被選入委員會,獲得更多獎勵,進而權益更高。攻擊者通過操縱這一循環,實現權益的指數增長與網絡控制權的轉移。

\subsection{攻擊階段詳述}

\subsubsection{階段一:潛伏階段 (Latent Phase)}

在潛伏階段,攻擊者的目標是積累初始權益並建立信譽,具體策略包括:

\begin{itemize}
    \item 誠實行為:攻擊者控制的節點在被選入委員會時,誠實執行聚合算法,提交正確的聚合結果。
    \item 權益積累:通過誠實參與,攻擊者節點獲得獎勵,權益逐漸增加。
    \item 等待時機:攻擊者持續觀察委員會組成,等待多個惡意節點同時被選入委員會,形成多數席位的時機。
\end{itemize}

潛伏階段的持續時間取決於攻擊者的初始權益佔比與委員會大小。假設攻擊者控制 $f = 0.3$ 的節點,委員會大小 $C = 7$,則攻擊者需要至少 4 個節點被選入委員會才能形成多數。根據超幾何分佈,這種情況發生的機率為:

\begin{equation}
P(\text{多數}) = \sum_{k=\lceil C/2 \rceil}^{\min(fM, C)} \frac{\binom{fM}{k} \binom{(1-f)M}{C-k}}{\binom{M}{C}}
\end{equation}

當 $f = 0.3, C = 7$ 時,這一機率約為 10-15\%,意味著攻擊者平均需要等待 7-10 輪才能獲得一次攻擊機會。

\subsubsection{階段二:佔領階段 (Capture Phase)}

一旦攻擊者在委員會中形成多數,立即啟動佔領階段:

\begin{itemize}
    \item 戰略性餓死 (Strategic Starvation):惡意委員會拒絕打包誠實節點提交的更新,僅接受攻擊者自身控制的客戶端更新。
    \item 排他性獎勵:由於只有惡意節點的更新被接受,獎勵僅分配給惡意委員會成員,誠實節點無法獲得任何獎勵。
    \item 權益分化:經過一輪攻擊後,惡意節點的平均權益增加 $\Delta s = R / C$,而誠實節點的權益保持不變。隨著攻擊輪次增加,權益差距呈線性擴大。
    \item 循環強化:權益差距擴大後,攻擊者在未來委員會中的佔比進一步提高,獲得攻擊機會的頻率增加,形成自我強化的惡性循環。
\end{itemize}

\subsection{核心機制分析}

PCCA 的成功依賴於三個核心機制:

\subsubsection{機制一:次優更新 (Sub-optimal Update)}

惡意委員會不一定需要提交完全錯誤的聚合結果,而是可以提交「次優更新」:

\begin{itemize}
    \item 選擇性接受:僅接受部分客戶端的更新,排除誠實節點的貢獻。
    \item 偏向性聚合:在執行聚合算法時,給予惡意客戶端更新更高的權重。
\end{itemize}

這種策略的隱蔽性在於,次優更新仍然可能使模型收斂,只是收斂速度較慢或最終準確率較低。在訓練初期,這種差異不易被察覺,使得攻擊者有足夠時間積累權益。

\subsubsection{機制二:戰略性餓死 (Strategic Starvation)}

戰略性餓死是 PCCA 的關鍵策略,其目標不是直接破壞模型,而是通過排他性的獎勵分配阻止誠實節點的權益增長:

\begin{itemize}
    \item 獎勵壟斷:惡意委員會獨佔訓練獎勵,誠實節點即使提交了正常更新,也無法獲得任何收益。
    \item 相對權益下降:雖然誠實節點的絕對權益不變,但其相對權益佔比持續下降,因為惡意節點的權益在不斷增長。
    \item 選擇機率下降:由於委員會選擇基於權益加權,誠實節點被選入委員會的機率逐輪下降,進一步減少其獲得獎勵的機會。
\end{itemize}

\subsubsection{機制三:權益指數增長}

在沒有外部干預的情況下,PCCA 會導致惡意節點的權益呈指數增長:

\begin{itemize}
    \item 初始階段:假設攻擊者初始權益佔比為 $f_0 = 0.3$。
    \item 首次攻擊:當攻擊者首次獲得委員會多數時,獨佔獎勵 $R$,權益增加至 $S_{mal}(1) = S_{mal}(0) + R$。
    \item 循環攻擊:隨著權益增加,攻擊者獲得委員會多數的機率提高,攻擊頻率增加。假設每 $k$ 輪成功攻擊一次,則經過 $t$ 輪後,惡意節點的平均權益為:
\end{itemize}

\begin{equation}
S_{mal}(t) = S_{mal}(0) + \frac{t}{k} \cdot R
\end{equation}

而誠實節點的權益保持 $S_{hon}(t) = S_{hon}(0)$,導致權益比例為:

\begin{equation}
\frac{S_{mal}(t)}{S_{hon}(t)} = \frac{S_{mal}(0) + \frac{t}{k} \cdot R}{S_{hon}(0)}
\end{equation}

隨著 $t$ 增加,這一比例趨向無窮,意味著攻擊者最終將完全控制系統。

\subsection{攻擊效果與影響}

PCCA 對系統造成多層次的破壞:

\begin{itemize}
    \item 模型品質下降:由於惡意委員會可能接受次優更新或排除部分誠實更新,模型收斂速度變慢,最終準確率下降。在極端情況下,如果惡意委員會完全拒絕誠實更新,模型將無法收斂。
    \item 網絡控制權轉移:隨著惡意節點權益佔比的提高,它們在委員會中的佔比也持續上升。最終,攻擊者可以持續控制委員會,完全掌握聯邦學習過程。
    \item 去中心化假設崩潰:區塊鏈聯邦學習的核心價值在於去中心化,避免單點故障與中心化信任。然而,PCCA 通過權益壟斷,實質上將系統重新中心化至攻擊者手中,違背了去中心化的初衷。
    \item 經濟激勵扭曲:誠實節點發現無論如何努力,都無法獲得獎勵,可能選擇退出系統。這進一步降低了誠實節點的佔比,加速了系統的崩潰。
\end{itemize}

\subsection{與傳統攻擊的區別}

PCCA 與傳統的拜占庭攻擊或數據投毒攻擊有本質區別,如表 \ref{tab:comparison_traditional} 所示。

\begin{table*}[htbp]
\centering
\caption{與傳統攻擊的區別}
\label{tab:comparison_traditional}
\begin{tabular}{|l|l|l|}
\hline
特徵 & 傳統攻擊 & PCCA \\ \hline
攻擊目標 & 模型品質 & 網絡控制權 \\ \hline
攻擊者動機 & 破壞 & 利益最大化 \\ \hline
攻擊策略 & 直接投毒 & 漸進式滲透 \\ \hline
隱蔽性 & 低 (立即可檢測) & 高 (初期表現誠實) \\ \hline
自我強化 & 無 & 有 (權益正反饋) \\ \hline
防禦方法 & 數據層防禦 & 需要激勵相容機制 \\ \hline
\end{tabular}
\end{table*}

傳統攻擊可以通過 Krum 等數據層防禦方法應對,但 PCCA 繞過了這些防禦,直接攻擊共識層。這種攻擊的隱蔽性與自我強化特性,使得傳統的誠實多數假設不再可靠。

\section{安全目標}
\label{sec:security_goals}

基於上述威脅模型,本研究的防禦機制需要達成以下安全目標:

\subsection{防止委員會被惡意節點控制}

核心目標:即使攻擊者在某一輪獲得委員會多數席位,也無法持續控制委員會。

具體要求:
\begin{itemize}
    \item 攻擊者無法通過單次成功攻擊獲得長期優勢。
    \item 系統能夠檢測並懲罰惡意委員會的行為。
    \item 懲罰機制足以剝奪攻擊者的作惡能力,防止其再次獲得委員會多數。
\end{itemize}

\subsection{確保誠實節點的權益公平增長}

核心目標:誠實節點通過正常參與系統,能夠持續獲得獎勵,權益穩定增長。

具體要求:
\begin{itemize}
    \item 惡意委員會無法阻止誠實節點獲得應得的獎勵。
    \item 即使在攻擊發生時,誠實節點仍有機制保障其權益不受損害。
    \item 長期來看,誠實節點的權益佔比應保持穩定或增長,而非下降。
\end{itemize}

\subsection{維持模型收斂性與準確性}

核心目標:在存在 PCCA 攻擊的情況下,系統仍能保證模型正常收斂,達到預期準確率。

具體要求:
\begin{itemize}
    \item 防禦機制能夠識別並拒絕次優更新。
    \item 即使部分輪次受到攻擊影響,整體訓練過程仍能收斂。
    \item 最終模型準確率與無攻擊場景相當。
\end{itemize}

\subsection{保持系統的去中心化特性}

核心目標:防禦機制本身不應引入新的中心化風險或信任假設。

具體要求:
\begin{itemize}
    \item 不依賴可信第三方或中心化仲裁者。
    \item 不依賴誠實多數假設,而是基於激勵相容的博弈論機制。
    \item 任何節點都能參與驗證與挑戰,無需特殊權限。
\end{itemize}

\subsection{激勵相容性}

核心目標:使得理性攻擊者的最優策略是誠實行為,而非發動攻擊。

具體要求:
\begin{itemize}
    \item 攻擊的預期收益必須為負,即 $E[\text{Payoff}] = P_{success} \cdot G_{attack} - P_{caught} \cdot L_{slash} < 0$。
    \item 懲罰機制 $L_{slash}$ 必須遠大於潛在收益 $G_{attack}$,使得即使攻擊成功機率較高,預期收益仍為負。
    \item 獎勵機制應激勵誠實行為,使得誠實節點的長期收益高於攻擊者。
\end{itemize}

\section{本章小結}
\label{sec:threat_summary}

本章定義了本研究針對的威脅模型,重點聚焦於區塊鏈聯邦學習系統中的「漸進式權益佔領攻擊」(PCCA)。與傳統的數據層攻擊不同,PCCA 針對的是共識層的驗證者,通過兩階段策略 (潛伏$\rightarrow$佔領) 逐步實現網絡控制權的轉移。

PCCA 的核心機制包括:
\begin{itemize}
    \item 次優更新:惡意委員會提交次優聚合結果,隱蔽性強。
    \item 戰略性餓死:通過排他性獎勵分配,阻止誠實節點權益增長。
    \item 權益指數增長:利用權益機制的正反饋特性,實現權益壟斷。
\end{itemize}

這種攻擊的危險性在於其隱蔽性、自我強化性,以及對去中心化假設的根本性顛覆。現有的數據層防禦方法 (如 Krum) 無法應對這種攻擊,因為它們依賴於驗證者誠實執行的假設。

基於這一威脅模型,本研究提出了五個安全目標:防止委員會控制、確保權益公平增長、維持模型收斂、保持去中心化特性,以及實現激勵相容性。下一章將介紹本研究提出的防禦機制,展示如何通過激勵相容的挑戰與罰沒機制,在不依賴誠實多數假設的前提下,有效防禦 PCCA 攻擊。

\end{ZhChapter}
