\begin{ZhChapter}

\chapter{威脅模型 (Threat Model)}
\label{chap:threat-model}

基於第二章所建立的委員會架構系統模型,本章將深入剖析該架構在面對理性攻擊者時所呈現的安全脆弱性。本章的核心任務在於定義並分析「漸進式委員會佔領攻擊」(Progressive Committee Capture Attack, PCCA),這是一種專門針對權益機制設計缺陷的隱蔽性攻擊手法。透過揭示攻擊者如何利用權益機制內建的正反饋特性逐步實現網路控制權的轉移,本章為後續章節的防禦機制設計提供明確的安全目標與理論基礎。值得特別強調的是,這種攻擊與傳統的模型投毒攻擊存在本質性差異,其危險性並非體現在對單一模型品質的破壞,而是從根本上顛覆了去中心化系統的安全假設,能夠將表面上維持去中心化形態的聯邦學習系統,實質上重新集權化至攻擊者手中。

\section{攻擊者模型}
\label{sec:adversary_model}

\subsection{攻擊者類型:理性攻擊者}

本研究所考慮的攻擊者屬於理性攻擊者 (Rational Adversary) 範疇,這與傳統區塊鏈安全研究中常見的拜占庭攻擊者存在本質性差異。拜占庭攻擊者的行為動機往往是純粹的破壞性,他們可能採取任意惡意行為來癱瘓系統,即使這些行為會導致自身利益受損也在所不惜,這種攻擊模型源自於分散式系統理論中對最壞情況的假設。然而,在實際的區塊鏈應用場景中,攻擊者往往具有明確的經濟動機而非單純追求破壞,他們的行為模式遵循經濟理性原則,首要目標是利益最大化。這意味著理性攻擊者會仔細評估每次攻擊行為的預期收益與成本,只有當預期收益明顯大於成本時才會採取行動,而如果能夠透過機制設計使得攻擊的預期收益為負,理性攻擊者將自發地選擇誠實行為,無需依賴傳統的誠實多數假設。這種區分為基於博弈論的防禦機制提供了理論基礎,也是本研究設計激勵相容機制的關鍵前提。

理性攻擊者的目標體系呈現出多層次性與長期性的特徵,這種複雜的目標結構使得攻擊行為更加隱蔽且難以偵測。在最直接的層面,攻擊者追求經濟利益的最大化,具體表現為透過操縱委員會來獨佔訓練獎勵,將誠實節點排除在獎勵分配機制之外。然而,這種短期經濟收益只是攻擊者目標體系的表層,更深層的目標在於權益壟斷與網路控制,透過系統性地阻止誠實節點的權益增長,攻擊者能夠逐步提高自身在整個系統中的權益佔比,這種權益佔比的提升會直接轉化為在委員會選擇過程中的優勢地位。當攻擊者的權益佔比達到某個臨界點後,他們將能夠更頻繁地控制委員會的組成,進而掌握聯邦學習過程中的關鍵決策權,包括決定哪些模型更新會被接受、哪些會被拒絕。這種從經濟收益到網路控制的轉變體現了攻擊者策略的長期性與系統性,也是 PCCA 攻擊之所以危險的根本原因,因為它並非僅僅影響模型品質,而是從根本上顛覆了去中心化系統的權力結構。

\subsection{攻擊者能力與限制}

在能力方面,本研究假設攻擊者能夠控制系統中一定比例的驗證者節點,這個比例記為 $f$,在典型的威脅場景下我們假設 $f \leq 0.3$,即攻擊者最多控制全網 30\% 的節點。這個假設並非任意設定,而是基於實際區塊鏈系統中攻擊者資源有限的現實考量,因為控制更高比例的節點需要投入大量的經濟資源與協調成本。被攻擊者控制的這些節點並非孤立運作,而是能夠相互協調並共同執行精心設計的攻擊策略,例如當多個惡意節點同時被選入同一個委員會時,它們可以串通一致地投票,形成協同作惡的局面。更值得注意的是,攻擊者具備策略性調整能力,能夠根據系統的動態狀態靈活改變行為模式,在權益積累的早期階段可能完全表現誠實以建立信譽並累積資源,而一旦獲得委員會的多數席位便會立即切換至攻擊模式。此外,攻擊者擁有完整的觀察能力,可以追蹤區塊鏈上的所有公開資訊,包括其他節點的權益分布、歷史行為記錄、委員會組成變化等,並基於這些資訊進行精確的策略規劃。

然而,攻擊者的能力並非無限,其行為同時受到多個維度的約束,這些約束為防禦機制的設計提供了重要的切入點。從密碼學角度來看,攻擊者無法突破系統所採用的密碼學原語,這意味著他們既無法偽造其他節點的數位簽章,也無法篡改已經寫入區塊鏈的歷史資料,區塊鏈的不可篡改性為系統提供了可靠的審計基礎。在網路控制層面,攻擊者的節點數量受到經濟成本的限制,無法達到發動傳統 51\% 攻擊所需的絕對多數,這使得攻擊者必須採用更為精細的策略來實現其目標。更關鍵的是,理性攻擊者的行為受到經濟激勵的根本性約束,如果精心設計的防禦機制能夠確保攻擊的預期成本大於潛在收益,那麼理性攻擊者將不會嘗試發動攻擊。此外,系統的可驗證性特徵為防禦提供了重要基礎:攻擊者無法阻止其他節點獨立驗證聚合結果的正確性,任何參與者都可以重新執行聚合演算法並檢測委員會是否正確遵守協議規則,這種透明性與可驗證性為後續設計挑戰機制奠定了技術可行性基礎。

\section{攻擊向量分析}
\label{sec:attack_vectors}

區塊鏈聯邦學習系統作為一個多層次的複雜架構,其安全威脅同樣呈現出層次化的特徵,不同層次的攻擊具有截然不同的目標、手法與防禦需求。本節的目標是系統性地分析不同層次的攻擊向量,釐清各層防禦的現狀與局限,進而明確本研究的關注焦點。這種層次化的分析框架不僅有助於理解 PCCA 攻擊的獨特性,也能揭示現有研究在安全分析上存在的系統性盲點,為後續的防禦機制設計提供清晰的問題定位。

\subsection{資料層攻擊:已有防禦}

資料層攻擊主要針對聯邦學習的訓練階段,透過污染訓練資料或模型更新來破壞最終模型的品質,這類攻擊在聯邦學習安全研究中已經得到廣泛的關注與深入的探討。具體而言,惡意客戶端可能採用資料投毒 (Data Poisoning) 手段,在本地訓練時刻意使用被污染的資料集,導致產生的模型更新偏離正常分布,從而影響全域模型的收斂方向。另一種更直接的攻擊方式是模型投毒 (Model Poisoning),惡意客戶端不經過真實的訓練過程,而是直接構造精心設計的惡意模型更新向量,這些更新可能包含後門觸發器或導向特定的錯誤分類行為。針對這類資料層威脅,現有的聯邦學習研究已經發展出相對成熟的防禦框架,其中最具代表性的是拜占庭強健聚合演算法,如 Krum \cite{blanchard2017machine}、Trimmed Mean \cite{yin2018byzantine}、Median 等方法,這些演算法的核心思想是利用統計學方法識別並過濾異常的模型更新,即使在存在一定比例惡意客戶端的情況下,仍能保證全域模型朝著正確的方向收斂。

然而,這些看似完備的防禦方法實際上建立在一個關鍵但往往被忽視的假設之上:執行這些防禦演算法的驗證者本身是誠實的。這個假設在傳統的中心化聯邦學習場景中或許是合理的,因為中心化伺服器的可信度通常由組織層面的信任保證,但在去中心化的區塊鏈聯邦學習系統中,驗證者同樣是由網路中的普通節點擔任,並沒有任何外部的信任背書。如果驗證者本身受到攻擊者控制,他們完全可以選擇不執行這些拜占庭強健演算法,或者更隱蔽地篡改演算法的執行結果,宣稱執行了防禦措施但實際上接受了惡意更新。在這種情況下,無論資料層的防禦演算法設計得多麼精妙,都將完全失去效力。這揭示了一個根本性的問題:資料層防禦的有效性依賴於共識層的安全性,如果共識層本身被攻陷,資料層的所有防線都將不攻自破,這種層次間的依賴關係構成了現有防禦體系的結構性弱點。

\subsection{共識層攻擊:本研究重點}

相較於已經得到充分研究的資料層攻擊,針對共識層的攻擊則構成了本研究的核心關注對象,這類攻擊的目標不是訓練資料或模型更新本身,而是負責執行聚合和驗證工作的委員會機制。驗證者共謀 (Verifier Collusion) 是這類攻擊的典型形式,多個惡意驗證者可以透過事先協調,在投票環節協同作惡,共同通過明顯包含錯誤或惡意特徵的聚合結果。更具威脅性的是委員會佔領 (Committee Capture) 攻擊,攻擊者不滿足於偶然的共謀機會,而是試圖系統性地操縱委員會選擇機制,逐步增加惡意節點在委員會中的席位佔比,最終實現對委員會的持續性控制。如第 \ref{chap:background-related} 章的文獻回顧所揭示的,現有區塊鏈聯邦學習研究在這個層面存在系統性的「驗證層盲點」,統計資料顯示約 93\% 的相關研究在設計系統時隱含地假設驗證者是誠實的或者至少滿足誠實多數的條件,僅有極少數研究明確考慮了惡意驗證者可能存在的場景並嘗試設計相應的防禦機制。

更值得關注的是,即使在引入了 Verifier 機制的 BlockDFL 類系統中,大多數研究仍然假設 Aggregator 和 Verifier 之間在利益上是相互獨立的,或者至少 Verifier 群體內部維持著誠實多數。本研究指出了一個被普遍忽視的風險:Verifier 和 Aggregator 完全可能形成利益集團 (Cartel),攻擊者可以同時滲透委員會與聚合節點,形成從上游到下游的完整控制鏈。這種「全棧控制」的風險是對現有 BlockDFL 架構安全分析的重要補充,也是 PCCA 攻擊得以成功的關鍵條件之一。共識層攻擊之所以比資料層攻擊更加危險,在於其具有三個顯著特徵:首先是防禦繞過能力,一旦委員會被惡意節點控制,所有的資料層防禦機制都可以被直接忽略或篡改;其次是隱蔽性,攻擊者在權益積累的早期階段可以完全表現誠實,不會觸發任何異常檢測機制;第三是自我強化特性,一旦攻擊成功,攻擊者將獲得更多獎勵,導致其權益進一步增加,形成正反饋循環。

\subsection{攻擊層次對比}

為了更清晰地呈現不同層次攻擊的特徵差異與防禦現狀,表 \ref{tab:attack_comparison} 提供了系統性的對比分析,這種對比有助於理解本研究選擇聚焦於共識層攻擊的理論依據。

\begin{table}[htbp]
\centering
\caption{攻擊層次對比}
\label{tab:attack_comparison}
\begin{tabular}{|l|l|l|l|l|l|}
\hline
攻擊層次 & 攻擊者 & 攻擊目標 & 現有防禦 & 防禦假設 & 本研究關注 \\ \hline
資料層 & 惡意客戶端 & 模型品質 & Krum, Trimmed Mean & 驗證者誠實 & 否 \\ \hline
共識層 & 惡意驗證者 & 網路控制 & 誠實多數假設 & 多數驗證者誠實 & 是 \\ \hline
\end{tabular}
\end{table}

從表中可以清楚地看到,資料層攻擊已經發展出相對完善的防禦方法體系,但這些方法的有效性建立在驗證者誠實執行協議的假設之上,相比之下,共識層攻擊的防禦仍然停留在依賴誠實多數假設的階段,缺乏針對理性攻擊者的激勵相容機制。這種防禦上的不對稱性正是本研究需要填補的關鍵空白。更深層次地看,資料層防禦與共識層防禦之間存在著依賴關係:前者的有效性完全取決於後者的可靠性,因此即使投入再多的研究資源去最佳化資料層的拜占庭強健演算法,如果不能從根本上解決共識層的安全問題,整個防禦體系仍然建立在不穩固的基礎之上。這種認識促使本研究將焦點放在共識層的安全性分析與防禦機制設計上,而非繼續在資料層防禦的技術細節上進行增量式的改進。

\section{漸進式委員會佔領攻擊 (Progressive Committee Capture Attack)}
\label{sec:pcca}

本節將詳細定義本研究針對的核心威脅:漸進式委員會佔領攻擊 (Progressive Committee Capture Attack, PCCA)。這是一種專門針對基於權益的委員會選擇機制的隱蔽性攻擊手法,其獨特之處在於透過精心設計的兩階段策略,利用權益機制內在的正反饋特性,實現從小規模滲透到顯著優勢地位的漸進式轉變,最終建立起對委員會決策的持續性影響力。

\subsection{攻擊定義與核心機制}

PCCA 的本質是一種針對權益衍生系統的經濟攻擊,其核心在於利用「權益-選舉-獎勵-權益」這一閉環機制中存在的正反饋特性。在正常運作的權益證明系統中,節點的權益決定了其被選入委員會的機率,而成功參與委員會工作又會獲得獎勵從而增加權益,這種設計的初衷是激勵節點誠實參與,但攻擊者可以將這一機制轉化為累積優勢的工具。PCCA 的攻擊策略分為兩個明確的階段:在潛伏階段,攻擊者控制的節點完全遵守協議規則,表現得與誠實節點無異,目的是積累初始權益並建立良好的信譽記錄,這個階段的持續時間取決於攻擊者的初始資源與委員會的隨機選擇結果。攻擊者會持續觀察系統狀態,等待一個關鍵的時機窗口:當多個惡意節點恰好同時被選入同一個委員會,且其席位數超過委員會總席位的三分之二時,攻擊便進入第二階段。

在佔領階段,攻擊者利用在委員會中的多數優勢,啟動「戰略性餓死」(Strategic Starvation) 策略,這種策略的核心不是直接破壞模型品質,而是透過操縱投票結果來控制獎勵分配。具體而言,惡意委員會會系統性地拒絕由誠實節點主導的聚合提案,即使這些提案包含高品質的模型更新,由於區塊鏈聯邦學習系統通常採用「提案-投票-獎勵」的連動機制,被拒絕的提案意味著相關的 Aggregator 和 Update Providers 都無法獲得本輪獎勵。透過持續執行這種排他性策略,惡意節點能夠獲得相對於誠實節點更高比例的系統獎勵,逐步擴大其權益優勢。隨著攻擊者權益佔比的提升,其在未來委員會選舉中獲得多數席位的機率也會相應提高,形成自我強化的正反饋循環。演算法 \ref{alg:pcca_strategy} 以形式化的方式呈現了 PCCA 的決策邏輯,清晰展示了攻擊者如何根據當前控制比例動態調整其行為模式。

\begin{algorithm}[!htbp]
\caption{High-Level Strategy of Progressive Committee Capture Attack (PCCA)}
\label{alg:pcca_strategy}
\begin{algorithmic}[1]
\Require Current Committee $\mathcal{V}$, Adversary Controlled Nodes $\mathcal{C}_{adv}$
\Ensure Action for the current round
\State \textbf{Check Phase:} Calculate control ratio $r = \frac{|\mathcal{V} \cap \mathcal{C}_{adv}|}{|\mathcal{V}|}$
\If{$r \leq 2/3$} 
    \Statex \textit{State 1: Shadow Mode (Lurking)}
    \State Follow the protocol honestly to accumulate stake and await majority.
\Else 
    \Statex \textit{State 2: Capture Mode (Occupying)}
    \If{\textbf{Aggregator is Adversarial}}
        \State \textbf{Full Stack Poisoning}: Force approve malicious proposal.
    \Else
        \State \textbf{Strategic Starvation}: Force reject honest proposal.
    \EndIf
\EndIf
\end{algorithmic}
\end{algorithm}

攻擊者在每一輪開始時都會計算其在當前委員會中的控制比例 $r$,這個比例決定了攻擊者採取的行為模式。當控制比例未超過三分之二時,攻擊者進入「影子模式」,嚴格遵守協議規則以避免暴露身份並持續積累權益,一旦控制比例超越臨界值,攻擊者立即切換至「佔領模式」。此時的具體策略取決於當輪 Aggregator 的身份:如果 Aggregator 本身也受攻擊者控制,那麼整個提案-驗證鏈條都在攻擊者掌握之中,此時可以執行更激進的「全棧投毒」策略,直接將包含惡意內容的模型更新寫入區塊鏈;如果 Aggregator 為誠實節點,攻擊者則採用相對保守的「戰略性餓死」策略,透過拒絕誠實提案來實現經濟層面的打擊,同時避免在技術層面留下明顯的攻擊痕跡。

\subsection{攻擊階段詳述}

\subsubsection{階段一:潛伏階段 (Latent Phase)}

潛伏階段是 PCCA 攻擊成功的關鍵前提,其核心目標是在不引起任何懷疑的情況下,為後續的佔領階段創造必要條件。在這個階段,攻擊者面臨的主要挑戰是如何在誠實行為與權益積累之間取得平衡,由於委員會的選擇基於權益加權的隨機抽樣,攻擊者的初始權益佔比直接決定了其節點被選入委員會的機率,進而影響多個惡意節點同時入選的可能性。假設攻擊者控制全網 $f = 0.3$ 的節點,而委員會大小為 $C = 7$,那麼要形成超過三分之二的多數優勢,至少需要 5 個惡意節點同時被選中。根據第 \ref{sec:committee-size-security} 節的超幾何分布分析,這種情況發生的機率約為 2.4\%,這意味著攻擊者平均需要等待約 42 輪才能獲得一次發動攻擊的機會,這種低頻率的攻擊窗口使得潛伏階段可能持續相當長的時間。

在這漫長的等待期間,攻擊者必須維持完美的誠實表現以避免被識別為可疑節點。當攻擊者控制的節點被選為 Update Provider 時,它們會基於本地資料集進行真實的模型訓練,提交符合協議規範的高品質更新;當被選為 Aggregator 時,它們會正確執行聚合演算法,包括運行 Krum 等拜占庭強健機制來過濾異常更新;當被選為 Verifier 時,它們會認真驗證聚合結果的正確性,對誠實的提案投贊成票,對存在問題的提案投反對票。這種全方位的誠實表現不僅能夠幫助攻擊者積累權益,更重要的是建立起良好的歷史記錄,使得其他節點和監督機制都將其視為可信的誠實參與者。潛伏階段的持續時間是彈性的,攻擊者會根據權益積累的速度與委員會組成的隨機結果動態調整策略,在確保安全的前提下耐心等待最佳的攻擊時機。

\subsubsection{階段二:佔領階段 (Capture Phase)}

當攻擊者在系統中累積了足夠的權益並成功控制了某一輪委員會的超過三分之二席位時,PCCA 進入最關鍵的佔領階段。與傳統攻擊採取單一破壞模式不同,PCCA 在佔領階段展現出高度的策略彈性,根據攻擊者對系統不同組件的控制程度採取不同層次的攻擊手法,這種分層策略設計使得攻擊既能最大化經濟收益,又能根據實際情況控制暴露風險。

\paragraph{場景一:戰略性餓死 (Strategic Starvation via Committee Capture)}

在第一種場景中,攻擊者成功控制了 Verifier 委員會的絕對多數席位,但當輪的 Aggregator 角色仍由誠實節點擔任或未完全受攻擊者控制,這種非對稱的控制狀態為攻擊者提供了一種獨特的攻擊機會。其核心策略是透過操縱投票結果來重新分配系統的經濟激勵,基於 BlockDFL 架構中普遍採用的獎勵連鎖機制,只有當聚合提案獲得委員會的批准並成功寫入區塊鏈時,相關的 Aggregator 和 Update Providers 才能獲得本輪的獎勵分配。攻擊者正是利用這一機制設計的關鍵環節,透過控制委員會的投票權來決定誰能獲得獎勵、誰將被排除在外。惡意委員會會採取系統性的差別對待策略:對於由誠實 Aggregator 提交的聚合提案,即使這些提案基於高品質的模型更新並且聚合過程完全正確,惡意委員會仍然會協同投出反對票,使其無法達到所需的三分之二多數支持。

戰略性餓死攻擊的破壞力主要體現在經濟層面而非技術層面,這種攻擊的精妙之處在於其高度的隱蔽性。從模型品質的角度看,由於系統仍然接受了某種形式的模型更新,訓練過程並未完全停滯,只是收斂速度相對放緩,這使得攻擊行為不易被外部觀察者識別為明顯的惡意行為。然而,從經濟激勵的角度看,這種攻擊造成了顯著的後果:誠實節點發現無論自己多麼努力地訓練模型、提交高品質更新,最終都會在委員會投票環節被系統性地排除,無法獲得應得的經濟回報。這種「付出努力但得不到回報」的狀態會導致兩種效應:一方面,誠實節點因為無法獲得獎勵而使其權益增長停滯,在未來的委員會選舉中其被選中的機率相對下降;另一方面,惡意節點透過獲取更高比例的獎勵實現權益的相對增長,其在下一輪委員會中的佔比優勢進一步擴大。這種馬太效應形成了正反饋循環,使得攻擊者的優勢隨時間推移而不斷鞏固。

\paragraph{場景二:全棧投毒 (Full Stack Poisoning)}

第二種場景代表了 PCCA 攻擊的最極端形態,攻擊者不僅控制了委員會的絕對多數,同時也成功滲透了當輪的 Aggregator 角色,這種「全棧控制」狀態意味著從模型聚合到結果驗證的整個流程都處於攻擊者的掌控之下,系統原本設計的多層防禦機制完全失效。在這種情況下,攻擊者的目標從經濟打擊轉向直接的技術破壞,透過向區塊鏈中注入惡意的模型更新來破壞全域模型的性能。全棧投毒攻擊的執行過程展現了多層防禦失效的連鎖反應:在聚合層面,惡意 Aggregator 可以選擇性地接收來自惡意 Update Providers 的投毒更新,這些更新可能採用標籤翻轉 (Label Flipping)、梯度反轉或後門注入等多種投毒技術;在驗證層面,由惡意委員會對這個明顯包含問題的聚合結果進行投票表決,即使任何具備計算能力的節點都可以重新執行聚合演算法並發現結果的異常,但由於委員會成員超過三分之二都是惡意的,他們會協同投出贊成票,強制使該提案達到共識所需的支持門檻。

全棧投毒攻擊的後果是全方位的,從模型品質角度看,被污染的更新一旦寫入區塊鏈並被全網採用,將直接導致全域模型的準確率大幅下降,在某些精心設計的後門攻擊場景下,模型甚至可能在特定輸入下表現出完全違背預期的行為。從經濟層面看,由於惡意 Aggregator 和惡意 Update Providers 瓜分了本輪的全部獎勵,攻擊者不僅成功破壞了模型,還進一步鞏固了其經濟優勢,使得系統越來越難以透過正常的選舉機制實現自我恢復。值得強調的是,全棧投毒場景的出現揭示了一個被廣泛忽視的系統性風險:在現有的 BlockDFL 架構中,Aggregator 和 Verifier 雖然在協議設計上被視為相互制約的獨立角色,但在實際攻擊場景下,它們完全可能被同一利益集團所控制形成合謀關係,這是對現有安全分析框架的重要挑戰。

\subsection{權益增長動態分析 (Stake Growth Dynamics Analysis)}

為了更精確地理解 PCCA 攻擊的長期影響,我們需要建立權益演化的數學模型,量化分析在沒有外部干預的情況下攻擊者的權益佔比如何隨時間推移而變化。假設系統初始狀態下,攻擊者控制的節點總權益為 $S_{mal}(0)$,誠實節點的總權益為 $S_{hon}(0)$,攻擊者的初始權益佔比為 $f_0 = \frac{S_{mal}(0)}{S_{mal}(0) + S_{hon}(0)} = 0.3$。在潛伏階段,雙方的權益都保持正常增長,攻擊者透過誠實參與獲得獎勵,權益佔比維持在初始水平附近。關鍵的轉折點出現在攻擊者首次獲得委員會超過三分之二席位的時刻,此時戰略性餓死策略開始生效。

然而,值得特別注意的是,即使在佔領階段,惡意節點的權益增長也並非呈現指數式的無限擴張。這是因為在 BlockDFL 的獎勵機制中,誠實節點仍然能夠透過擔任 Update Provider 角色獲得部分獎勵,即使他們的提案被惡意委員會拒絕,他們作為被選中提案的 Update Provider 時仍可分得相應的獎勵份額。這種機制設計意味著惡意節點無法完全壟斷系統的全部獎勵,而是會與誠實節點形成一種動態的權益分配平衡。具體而言,假設系統每輪分配的總獎勵為 $R$,惡意委員會雖然能夠透過操縱投票使獎勵更傾向於流向惡意節點,但誠實節點作為 Update Provider 的貢獻仍會獲得部分補償,這使得雙方的權益比例會趨向於某個穩定的比值而非無限分化。

基於上述分析,權益演化的動態可以表示為一個有界的增長模型。設 $\alpha$ 為惡意節點在成功控制委員會時能夠獲得的獎勵比例優勢係數,由於誠實節點仍能透過 Update Provider 角色獲得部分獎勵,這個係數 $\alpha$ 具有上界,通常在 1.1 至 1.2 之間,這意味著惡意節點每輪獲得的獎勵大約是誠實節點的 1.1 到 1.2 倍,而非無限倍數。這種有界的優勢係數導致雙方的權益比例會收斂到一個穩定值:

\begin{equation}
\lim_{t \to \infty} \frac{S_{mal}(t)}{S_{hon}(t)} = \alpha \cdot \frac{S_{mal}(0)}{S_{hon}(0)}
\end{equation}

從系統動力學的角度看,這是一個具有穩定平衡點的正反饋系統,而非傳統理解中的發散系統。這種平衡的存在並不意味著 PCCA 攻擊不具威脅性,相反地,它揭示了攻擊的另一種危險形態:攻擊者能夠建立並維持一種持久的「領先者優勢」,即使無法實現完全壟斷,也能長期保持對系統治理的顯著影響力。這種持續性的權益優勢使得攻擊者能夠更頻繁地控制委員會,形成一種「常態化」的治理失衡狀態,而現有系統缺乏打破這種平衡的內在機制。

\subsection{攻擊效果與影響}

PCCA 攻擊對區塊鏈聯邦學習系統造成的破壞是多維度且層層遞進的,其影響範圍涵蓋了技術性能、經濟激勵、系統治理等多個關鍵層面。在模型品質層面,即使攻擊者採取相對溫和的戰略性餓死策略,系統的訓練效能也會受到明顯影響。由於惡意委員會傾向於批准次優更新而拒絕最優更新,每一輪訓練對全域模型的改進幅度都會小於正常情況,導致收斂速度顯著放緩。在某些情況下,如果被批准的次優更新與全域模型的最佳改進方向存在較大偏差,甚至可能出現訓練震盪或陷入局部最優的情況。在全棧投毒場景下,模型品質的損害更加直接和嚴重,被注入的惡意更新可能包含精心設計的後門觸發器或針對特定類別的偏差,使得模型在大部分正常輸入上表現正常,但在特定條件下產生攻擊者預期的錯誤行為。

從網路治理權的角度看,PCCA 實現了權力結構的顯著傾斜。在攻擊的初期階段,系統表面上仍然維持著去中心化的形態,委員會的組成看起來是透過隨機選舉產生的,各個節點都有機會參與。但隨著攻擊者權益佔比的持續上升並穩定在優勢水平,這種表面上的去中心化逐漸演變為實質上的寡頭主導。當攻擊者的權益佔比穩定在較高水平時,他們獲得委員會多數席位的機率將顯著高於隨機分布的預期值,意味著從統計意義上他們能夠在更多輪次中控制委員會,形成一種「軟性壟斷」的治理狀態。這種從分散到集中的權力轉移過程,雖然不會達到完全壟斷的程度,但已經足以嚴重損害區塊鏈系統的核心價值主張。

在經濟激勵層面,PCCA 造成了激勵機制的扭曲與部分失靈。對於誠實節點而言,他們會發現一個令人沮喪的現實:即使投入大量計算資源進行本地訓練、提交高品質的模型更新,在委員會投票環節仍然面臨被系統性排斥的風險,獲得的經濟回報明顯低於預期。這種「付出與回報不成比例」的狀態會逐漸瓦解誠實節點的參與動機,理性的節點會進行成本效益分析,當持續的低回報無法覆蓋參與系統所需的計算成本、網路成本和時間成本時,部分節點可能選擇降低參與程度或退出系統。這種節點流失會形成另一層負面效應:誠實節點的退出會進一步提高惡意節點的相對權益佔比,使得系統更容易被控制,這又會加速更多誠實節點的離開,形成一種緩慢但持續的惡性循環。

\subsection{與傳統攻擊的區別}

為了更清晰地凸顯 PCCA 攻擊的獨特性與威脅性,表 \ref{tab:comparison_traditional} 提供了與傳統拜占庭攻擊和資料投毒攻擊的系統性對比,這種對比有助於理解為何現有的防禦機制難以有效應對 PCCA。

\begin{table*}[htbp]
\centering
\caption{與傳統攻擊的區別}
\label{tab:comparison_traditional}
\begin{tabular}{|l|l|l|}
\hline
特徵 & 傳統攻擊 & PCCA \\ \hline
攻擊目標 & 模型品質 & 網路控制權 \\ \hline
攻擊者動機 & 破壞 & 利益最大化 \\ \hline
攻擊策略 & 直接投毒 & 漸進式滲透 \\ \hline
隱蔽性 & 低(立即可檢測) & 高(初期表現誠實) \\ \hline
自我強化 & 無 & 有(權益正反饋) \\ \hline
防禦方法 & 資料層防禦 & 需要激勵相容機制 \\ \hline
\end{tabular}
\end{table*}

從攻擊目標來看,傳統的資料投毒或模型投毒攻擊主要關注破壞機器學習模型的性能指標,例如降低分類準確率、植入後門、造成特定類別的誤判等,這類攻擊的影響主要局限在機器學習的技術層面,即使攻擊成功,系統的治理結構和參與者組成並不會發生根本改變。相比之下,PCCA 的目標是奪取系統的治理權,控制決定模型演化方向的委員會機制,一旦攻擊成功,攻擊者不僅能夠影響模型品質,更能決定哪些節點可以參與、哪些提案會被接受,實質上控制了系統的未來走向。從攻擊者動機角度來看,傳統拜占庭攻擊者的行為模式往往基於最壞情況假設,他們可能出於意識形態、惡意競爭或純粹的破壞慾望而發動攻擊,即使這些行為會導致自身經濟利益受損也在所不惜;而 PCCA 則建立在理性經濟人的假設之上,攻擊者的每一步行動都經過精心計算,目標是最大化長期的經濟收益,這種基於理性的攻擊模型更貼近現實世界中的威脅場景。

從攻擊策略的時間維度來看,傳統攻擊通常採取直接而迅速的方式,惡意節點從一開始就提交明顯異常的更新或投票,試圖在短時間內對系統造成最大破壞,這種「一次性」的攻擊模式雖然可能在短期內造成嚴重影響,但也使得攻擊行為容易被檢測系統識別。PCCA 則採用漸進式的長期策略,攻擊者願意在潛伏階段投入大量時間和資源來建立信譽,只在時機成熟時才發動攻擊,這種耐心的策略使得攻擊具有極強的隱蔽性,因為在攻擊的大部分時間裡,惡意節點的行為與誠實節點完全無法區分。更關鍵的是,PCCA 具有傳統攻擊所不具備的自我強化特性:傳統攻擊即使成功也不會改變攻擊者與誠實節點之間的力量對比,下一輪攻擊仍然面臨同樣的難度;但 PCCA 每成功一次,攻擊者的相對權益優勢就會擴大,未來攻擊的成功率也隨之提高,形成滾雪球效應。這種正反饋機制使得系統一旦開始被滲透,就會沿著權益失衡的軌道持續發展,直到達到某個穩定的不平衡狀態。

從防禦策略的角度來看,傳統攻擊已經發展出相對成熟的應對方法,主要集中在資料層面的統計檢測與過濾,Krum、Trimmed Mean、Median 等拜占庭強健聚合演算法能夠有效識別並排除異常的模型更新,這些方法的有效性已經在大量實驗中得到驗證。然而,PCCA 攻擊完全繞過了這些資料層防禦,因為它直接攻擊的是執行這些防禦演算法的驗證者本身,當驗證者被攻陷後,無論資料層的防禦設計得多麼精妙,都可以被選擇性地忽略或篡改。這揭示了一個層次化的依賴關係:資料層防禦的有效性完全依賴於共識層的安全性。要應對 PCCA,需要從根本上改變防禦思路,不能再依賴誠實多數假設,而是必須設計激勵相容的機制,使得理性攻擊者發現誠實行為才是其利益最大化的最優策略,這需要引入經濟懲罰、挑戰驗證等新的防禦維度,構建一個多層次的安全框架。

\section{安全目標}
\label{sec:security_goals}

基於前述對 PCCA 攻擊機制與影響的深入分析,本節將明確提出本研究所設計的防禦機制需要達成的安全目標。這些目標不僅要能夠有效防禦 PCCA 攻擊,更要在防禦過程中保持系統的去中心化特性與經濟激勵的合理性,避免引入新的安全風險或中心化依賴,以下將從五個層面逐一闘述這些安全目標的具體內涵與實現要求。

\subsection{防止委員會被惡意節點持續控制}

防禦機制的首要目標是破壞 PCCA 攻擊的自我強化循環,確保即使攻擊者在某一輪成功獲得委員會的超過三分之二席位,也無法將這種優勢轉化為長期的控制權。這個目標的實現需要從多個維度入手,首先系統必須具備檢測惡意委員會行為的能力,能夠識別出委員會是否在系統性地拒絕高品質提案或批准次優提案。其次,一旦檢測到可疑行為,必須有相應的懲罰機制能夠迅速介入,對參與作惡的委員會成員進行經濟制裁,例如透過罰沒 (Slashing) 機制沒收其部分或全部權益,這種懲罰的力度必須足夠大,使得攻擊者即使成功獲得短期經濟利益,也會因為被懲罰而遭受更大的長期損失。第三,懲罰機制的執行不能依賴中心化的仲裁者,而應該透過去中心化的挑戰與驗證流程來實現,任何節點都應該有權利對可疑的委員會決策提出質疑,並透過鏈上的驗證過程來證明其合理性。透過這種多層次的防禦設計,系統能夠確保攻擊者無法透過單次成功攻擊建立起持久的優勢地位。

\subsection{確保誠實節點的權益公平增長}

第二個核心目標是保護誠實節點的經濟利益,確保他們透過正常參與系統能夠持續獲得應得的獎勵,權益能夠穩定增長而不會被惡意委員會的排他性策略所剝奪。這個目標的達成需要重新審視獎勵分配機制,打破 PCCA 攻擊依賴的「提案被拒絕則所有相關節點零獎勵」的連動關係。一種可能的設計思路是引入備選獎勵通道,即使誠實節點的提案在某一輪被惡意委員會拒絕,但只要能夠證明其提案的品質確實優於被批准的提案,仍然可以透過挑戰機制獲得補償性獎勵。另一種思路是設計基於長期表現的獎勵平滑機制,使得單輪的獎勵分配不是全有或全無,而是基於節點的歷史貢獻與聲譽進行累積評估。此外,系統還需要確保即使在面臨攻擊的情況下,誠實節點的相對權益佔比不會持續下降,這可能需要引入反壟斷機制,例如對權益增長速度過快的節點進行額外審查。長期而言,只有當誠實行為能夠獲得穩定且可預期的經濟回報,理性節點才會選擇持續誠實參與,系統才能維持健康的參與者生態。

\subsection{維持模型收斂性與準確性}

儘管 PCCA 攻擊的主要目標是奪取網路控制權而非直接破壞模型,但防禦機制仍然需要確保在存在攻擊的情況下,聯邦學習的核心功能不受影響,模型能夠正常收斂並達到預期的準確率。這個目標的實現依賴於防禦機制能夠有效識別並拒絕次優或惡意的更新,具體而言,系統需要建立多層次的品質檢測機制,不僅在 Aggregator 層面執行拜占庭強健聚合,更要在 Verifier 層面引入獨立的品質驗證流程,例如透過在驗證集上測試聚合結果的性能表現,或者對比多個獨立聚合的一致性。當檢測到當輪的聚合結果明顯劣於歷史水平或存在異常模式時,系統應該有能力觸發特殊處理流程,例如要求重新聚合、延長驗證期或啟動社區投票。即使部分輪次受到攻擊影響,只要大多數輪次的更新品質能夠得到保證,整體訓練過程仍然能夠朝著正確方向推進。從長期收斂性的角度看,防禦機制應該確保最終模型的準確率與無攻擊場景相當或至少在可接受的誤差範圍內,證明系統具備抵禦攻擊的強健性。

\subsection{保持系統的去中心化特性}

在設計防禦機制時,一個容易陷入的誤區是為了提高安全性而引入中心化的信任假設或特權節點。本研究強調,防禦機制本身不應成為新的中心化風險來源,必須始終保持系統的去中心化本質。這意味著防禦機制不能依賴任何可信第三方或中心化仲裁者來判斷節點行為的善惡,也不能設置擁有特殊權限的超級節點來監督其他節點,所有的檢測、驗證與懲罰流程都應該透過去中心化的協議來實現,任何普通節點都應該有平等的權利參與挑戰與驗證過程。這種設計理念要求我們不能簡單地依賴誠實多數假設,而是要透過精巧的激勵機制設計,利用博弈論的原理使得理性節點自發選擇誠實行為。密碼學技術如零知識證明、可驗證計算等可以在這個過程中發揮重要作用,它們允許節點在不暴露私有資訊的前提下證明自己的計算正確性,為去中心化驗證提供了技術基礎。只有當防禦機制本身也是去中心化的,系統才能真正實現端到端的安全性,而不會在解決一個問題的同時創造新的安全隱患。

\subsection{激勵相容性}

最後但也是最根本的安全目標是實現激勵相容性 (Incentive Compatibility),這是應對理性攻擊者的核心策略。激勵相容性的含義是系統的機制設計應該使得理性節點的最優策略就是誠實行為,發動攻擊不僅不能帶來額外收益,反而會導致預期的經濟損失。從數學上表達,攻擊的預期收益 $E[\text{Payoff}]$ 必須為負,即 $E[\text{Payoff}] = P_{success} \cdot G_{attack} - P_{caught} \cdot L_{slash} < 0$,其中 $P_{success}$ 是攻擊成功的機率,$G_{attack}$ 是攻擊成功時獲得的經濟收益,$P_{caught}$ 是攻擊被檢測到的機率,$L_{slash}$ 是被懲罰時損失的權益數量。要確保這個不等式成立有幾種設計策略:第一種是提高檢測機率 $P_{caught}$,透過設計更敏感的異常檢測機制和更廣泛的挑戰參與機制使得惡意行為難以逃脫監督;第二種是大幅增加懲罰力度 $L_{slash}$,使其遠大於潛在的攻擊收益 $G_{attack}$,即使攻擊成功機率較高,但一旦被抓住就會損失慘重,理性節點不願意承擔這種風險;第三種是降低攻擊收益 $G_{attack}$,例如透過限制單輪獎勵的上限或將獎勵分散到多個輪次,使得單次成功攻擊的收益不足以覆蓋長期的作惡成本。只有當這種激勵結構被成功建立,系統才能從根本上消除理性攻擊者的作惡動機,實現自我維持的安全性。

\section{本章小結}
\label{sec:threat_summary}

本章系統性地構建了針對區塊鏈聯邦學習委員會架構的威脅模型,核心聚焦於一種新型的共識層攻擊:漸進式委員會佔領攻擊 (Progressive Committee Capture Attack, PCCA)。與傳統的資料層投毒攻擊著眼於破壞模型品質不同,PCCA 的目標在於透過經濟手段逐步奪取系統的治理權,最終實現對整個網路的持續性影響力。這種攻擊之所以危險,不僅在於其隱蔽性和自我強化特性,更在於它揭示了現有區塊鏈聯邦學習研究中普遍存在的一個系統性盲點:絕大多數研究在設計驗證機制時隱含地假設驗證者是誠實的或至少滿足誠實多數,但這個假設在去中心化環境下並沒有可靠的保證機制。

本章首先定義了理性攻擊者模型,明確了攻擊者以利益最大化而非單純破壞為目標的行為特徵,這種攻擊者模型更貼近現實世界中的威脅場景。在此基礎上,我們詳細剖析了 PCCA 的兩階段攻擊策略:在潛伏階段,攻擊者透過完美的誠實表現積累權益與信譽,耐心等待多個惡意節點同時被選入委員會的時機窗口;一旦獲得超過三分之二的席位優勢,攻擊立即進入佔領階段,根據對系統組件的控制程度採取戰略性餓死或全棧投毒策略。前者透過系統性地拒絕誠實提案來阻止誠實節點獲得獎勵,後者則在同時控制 Aggregator 和 Verifier 的情況下直接注入惡意更新。

透過權益增長動態分析,我們釐清了 PCCA 攻擊的實際影響特徵。值得注意的是,由於誠實節點仍能透過擔任 Update Provider 角色獲得部分獎勵,惡意節點的權益增長並非呈現指數式的無限擴張,而是會與誠實節點形成一種動態平衡,惡意節點的權益通常穩定在誠實節點的 1.1 至 1.2 倍左右。這種平衡的存在並不意味著攻擊不具威脅性,相反地,它揭示了攻擊者能夠建立並維持一種持久的「領先者優勢」,形成常態化的治理失衡狀態。基於這一威脅分析,本章提出了五個層次化的安全目標:防止委員會持續控制、確保誠實節點權益公平增長、維持模型收斂性與準確性、保持系統去中心化特性,以及實現激勵相容性。下一章將介紹本研究提出的防禦機制,展示如何透過挑戰增強型委員會架構,在不依賴誠實多數假設的前提下構建激勵相容的防禦體系,實現上述安全目標。

\end{ZhChapter}