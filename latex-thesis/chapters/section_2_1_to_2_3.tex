\chapter{背景知識與相關研究}
\label{chap:background-related}

本章旨在建立理解區塊鏈聯邦學習委員會安全性所需的理論基礎與技術背景。首先,本章將從聯邦學習的核心價值出發,闡明中央化架構面臨的信任困境,進而說明區塊鏈技術如何作為去中心化信任的基礎設施。接著,本章將介紹拜占庭容錯理論的基本原理,為理解委員會共識機制的安全性閾值提供數學基礎。在此基礎上,本章將探討區塊鏈聯邦學習如何從全節點共識演進至委員會架構,並分析委員會規模與安全性之間的權衡關係。隨後,本章將詳細介紹本研究採用的基準系統模型——BlockDFL 的委員會架構,包括其角色定義、運作流程與獎勵機制。最後,本章將回顧現有驗證方法的局限性,指出當前研究在面對策略性攻擊者時的盲區,從而定位本研究欲填補的關鍵缺口。

\section{聯邦學習與去中心化信任需求}
\label{sec:fl-trust}

\subsection{聯邦學習的核心機制}
\label{sec:fl-core}

聯邦學習是一種分散式機器學習典範,其核心創新在於實現「資料不動、模型動」的訓練機制 \cite{mcmahan2017communication}。在傳統的集中式機器學習中,所有訓練資料必須彙集至中央伺服器進行處理,這種做法在面對隱私敏感資料或資料傳輸成本高昂的場景時顯得力不從心。聯邦學習透過將訓練過程分散至資料所在的終端裝置,僅將模型更新而非原始資料上傳至伺服器進行聚合,從根本上改變了資料與運算的關係。這種架構使得醫療機構能夠在不共享病患紀錄的前提下協同訓練診斷模型,金融機構能夠在不揭露客戶交易資料的情況下建立風險評估系統,行動裝置製造商能夠利用數百萬用戶的使用習慣最佳化輸入法預測,而無需將敏感的打字內容上傳至雲端。

聯邦學習的標準訓練流程可概括為四個階段的迭代循環。在每一輪訓練中,中央伺服器首先將當前的全域模型參數分發給選定的客戶端;各客戶端隨後在本地私有資料上執行若干輪梯度下降,產生反映本地資料特性的模型更新;客戶端將這些更新上傳至伺服器;伺服器執行聚合演算法(最常見的是 FedAvg \cite{mcmahan2017communication})將各客戶端的更新整合為新的全域模型。此循環持續進行直至模型收斂或達到預設的訓練輪數。值得注意的是,聯邦學習面對的資料分布通常具有高度異質性:不同客戶端持有的資料量可能相差懸殊,資料的類別分布也往往呈現顯著差異,這種非獨立同分布(Non-IID)的特性為模型訓練與安全防護帶來了獨特的挑戰 \cite{kairouz2021advances}。

\subsection{中央化架構的信任困境}
\label{sec:centralized-trust}

儘管聯邦學習在資料隱私保護上取得了重要進展,其標準架構仍存在一個根本性的信任假設:所有參與者必須信任中央聚合伺服器會誠實地執行聚合運算並正確地分發結果。然而,在缺乏有效驗證機制的情況下,這項假設構成了系統的單點脆弱性。中央伺服器可能因遭受攻擊、內部人員惡意行為或系統故障而偏離預期行為,而客戶端對此幾乎無從察覺,更遑論採取補救措施。

中央化架構面臨的信任風險可歸納為三個層面。第一個層面是聚合正確性的不可驗證性。當伺服器宣稱某一全域模型是由特定客戶端更新聚合而成時,客戶端無法獨立驗證此宣稱的真實性。伺服器可能執行選擇性聚合——僅納入部分客戶端的更新而排除其他——或直接篡改聚合結果以植入後門。研究已證實,透過精心設計的模型修改,攻擊者可在不顯著影響主任務效能的情況下,使模型對特定輸入產生預設的錯誤輸出 \cite{bagdasaryan2020how}。第二個層面是單點故障風險。中央伺服器一旦因攻擊、硬體故障或網路問題而離線,整體訓練流程即刻中斷,且由於缺乏分散式的狀態同步機制,系統難以從中間狀態恢復。第三個層面是隱私保護的局限性。儘管聯邦學習避免了原始資料的直接傳輸,研究表明惡意的聚合伺服器仍可能透過分析客戶端提交的模型更新,推論出關於訓練資料的敏感資訊 \cite{zhu2019deep}。

這些信任風險在跨組織協作的場景中尤為突出。當多個相互獨立甚至存在競爭關係的機構希望聯合訓練模型時,由任何單一機構擔任中央聚合者都難以獲得其他參與者的充分信任。即便引入第三方作為中立的聚合服務提供者,仍無法從根本上消除對該第三方誠實性的依賴。這種信任困境限制了聯邦學習在高價值、高敏感場景中的應用潛力,也促使研究者開始探索去中心化的替代方案。

\subsection{區塊鏈作為去中心化信任基礎設施}
\label{sec:blockchain-trust}

區塊鏈技術的三項核心特性——不可篡改性、透明性與去中心化——恰好對應了中央化聯邦學習面臨的信任困境,使其成為建構去中心化聯邦學習系統的理想基礎設施。不可篡改性源於區塊鏈的鏈式雜湊結構:每個區塊包含前一區塊的雜湊值,任何對歷史資料的修改都將導致後續所有區塊的雜湊值連鎖變化,從而被網路中的其他節點立即偵測。這項特性確保了一旦聚合結果被記錄於區塊鏈,便無法在事後被悄然篡改。透明性則意味著所有被記錄的交易與狀態變更對全體參與者可見,客戶端可以驗證自己的更新是否被納入聚合,也可以審計歷史聚合過程是否遵循預定的規則。去中心化消除了對單一實體的信任依賴:區塊鏈網路由眾多獨立節點共同維護,即使部分節點失效或行為惡意,只要誠實節點佔據多數,系統仍能正確運作。

將區塊鏈整合至聯邦學習架構,可從多個層面強化系統的可信賴性。在聚合正確性方面,智能合約可編碼確定性的聚合規則,確保聚合過程按照預定邏輯執行,而非依賴聚合者的自我約束。聚合結果連同參與者資訊被記錄於區塊鏈,形成永久可查的審計軌跡。在系統可用性方面,區塊鏈的分散式架構天然具備容錯能力:即使部分節點離線,其他節點仍可維持系統運作,避免了中央伺服器故障導致的全面停擺。在激勵對齊方面,區塊鏈原生的代幣機制可用於設計精細的獎懲制度,對誠實貢獻者給予獎勵,對惡意行為者施加經濟懲罰,從而在博弈論意義上引導參與者趨向誠實行為。

然而,區塊鏈並非萬能的信任解決方案。區塊鏈共識機制本身需要假設惡意節點不超過特定比例——對於拜占庭容錯協議而言,這一閾值通常為三分之一。當攻擊者控制的節點超過此閾值時,區塊鏈的安全性保證將不再成立。此外,區塊鏈的共識過程涉及大量的節點間通訊,其延遲與頻寬成本可能與聯邦學習對快速迭代的需求產生張力。這些考量促使研究者發展出委員會架構等效率最佳化方案,但也隨之引入了新的安全性議題。下一節將首先介紹拜占庭容錯的理論基礎,為理解這些安全性議題提供必要的背景知識。

\section{拜占庭容錯的理論基礎}
\label{sec:bft-fundamentals}

\subsection{拜占庭將軍問題與容錯閾值}
\label{sec:byzantine-generals}

拜占庭將軍問題由 Lamport、Shostak 與 Pease 於 1982 年正式提出 \cite{lamport1982byzantine},是分散式系統容錯理論的基石。問題的設定源自一個軍事隱喻:拜占庭帝國的數支軍隊包圍敵城,各軍由一位將軍指揮,將軍們僅能透過信使相互通訊。然而,部分將軍可能是叛徒,他們會刻意傳遞錯誤訊息以阻撓忠誠將軍達成一致決策。問題的核心在於:如何設計一個協議,使得所有忠誠將軍能就「進攻」或「撤退」達成共識,即使存在叛徒試圖破壞協調?此問題的形式化定義包含兩個交互一致性條件:所有忠誠節點必須就相同的值達成共識(一致性),且若發起者是誠實的,則共識結果必須是發起者提出的值(正確性)。

拜占庭將軍問題存在一個根本性的數學限制:在僅使用口頭訊息的情況下,問題可解若且唯若誠實節點超過總數的三分之二。換言之,若系統中有 $n$ 個節點,最多只能容忍 $f$ 個拜占庭節點,其中 $n \geq 3f + 1$。此限制可透過最簡單的三節點、一叛徒場景直觀理解。考慮指揮官向兩位副官發送命令的情境:若指揮官是叛徒,他可能向副官 A 發送「進攻」,向副官 B 發送「撤退」;當兩位誠實副官相互交換收到的命令時,各自都會發現矛盾。然而,若副官 B 是叛徒而指揮官誠實,副官 B 可能向副官 A 謊稱「指揮官說撤退」。關鍵的洞察在於:從副官 A 的視角來看,這兩種情境完全無法區分——他都收到來自指揮官的「進攻」與來自 B 聲稱的「撤退」。任何確定性演算法在此情境下都必然失敗,這從根本上限制了拜占庭容錯系統的設計空間。

此三分之一閾值的數學根源在於 Quorum 交叉原理。為確保任何決策都獲得足夠的誠實節點背書,系統需要收集至少 $2f+1$ 個節點的確認。由於最多 $f$ 個節點可能是惡意的,$2f+1$ 個確認中必然包含至少 $f+1$ 個來自誠實節點。任意兩個大小為 $2f+1$ 的節點群體,其交集至少包含 $f+1$ 個節點,這確保了至少有一個誠實節點見證了兩次決策,從而防止系統對同一問題做出矛盾的決定。將節點總數代入約束條件 $n \geq 2f+1+f$,即得 $n \geq 3f+1$。

\subsection{實用拜占庭容錯協議的核心概念}
\label{sec:pbft-core}

拜占庭將軍問題的早期解法雖然在理論上可行,但其指數級的通訊複雜度使其僅具學術意義。1999 年,Castro 與 Liskov 提出實用拜占庭容錯協議(Practical Byzantine Fault Tolerance, PBFT)\cite{castro1999practical},首次將 BFT 共識的通訊複雜度降至多項式級別 $O(n^2)$,使其在實際系統中可行。PBFT 的設計目標是在部分同步網路模型下,以合理的效能代價換取對任意惡意行為的容錯能力。

PBFT 協議的運作依賴 $n = 3f+1$ 個副本節點,其中一個被指定為主節點(Primary),負責為客戶端請求分配序號並發起共識。協議透過三個階段達成共識:預準備(Pre-prepare)、準備(Prepare)與提交(Commit)。在預準備階段,主節點將請求連同分配的序號廣播給所有副本;在準備階段,收到預準備訊息的副本向其他所有副本廣播準備訊息,當某副本收集到 $2f$ 個匹配的準備訊息時,表明系統中有足夠多的節點認可此請求的序號分配;在提交階段,進入準備狀態的副本廣播提交訊息,當收集到 $2f+1$ 個提交訊息時,副本確信此請求已被系統接受,可以執行並回覆客戶端。三階段設計的核心目的是確保即使主節點是惡意的,也無法導致誠實節點對請求順序產生分歧。

PBFT 的通訊複雜度為 $O(n^2)$,這是因為在準備與提交階段,每個節點都需要向其他所有節點發送訊息。以 $n=7$(可容忍 2 個拜占庭節點)的配置為例,每輪共識約需交換 100 則訊息;當節點數增至 $n=22$(可容忍 7 個拜占庭節點)時,訊息數增至約 900 則。這種二次方增長限制了 PBFT 在大規模網路中的直接應用,實務部署通常限制在 10 至 20 個節點的規模。後續研究如 HotStuff \cite{yin2019hotstuff} 透過流水線化設計與閾值簽章技術,將複雜度進一步降至 $O(n)$,但在本研究關注的許可制聯盟鏈場景中,節點數量通常在 PBFT 可承受的範圍內。

\subsection{BFT 共識在區塊鏈聯邦學習中的角色}
\label{sec:bft-in-bcfl}

在區塊鏈聯邦學習系統中,拜占庭容錯共識扮演著確保聚合結果正確性的關鍵角色。與傳統區塊鏈應用(如加密貨幣交易)不同,聯邦學習的「交易」是模型更新,而「帳本狀態」是全域模型參數。當負責聚合的節點可能被攻陷或本身即為惡意參與者時,系統需要一個機制來驗證聚合結果的正確性,並在多個可能存在分歧的結果中達成共識。BFT 協議正是為此目的而設計:它確保只要惡意節點不超過總數的三分之一,系統就能就聚合結果達成一致,且該結果必然反映誠實多數的判斷。

然而,將 BFT 共識直接應用於大規模聯邦學習系統面臨顯著的效率挑戰。聯邦學習通常涉及數十至數百個參與者,若所有參與者都參與每一輪的 BFT 共識,$O(n^2)$ 的通訊複雜度將成為嚴重的效能瓶頸。更重要的是,聯邦學習需要頻繁迭代——典型的訓練過程可能包含數百至數千輪——每輪都執行完整的全網共識將導致訓練時間大幅延長。這種效率與安全性之間的張力,促使研究者發展出委員會架構:由一個小型的代表性子集執行共識,以較低的通訊成本達成近似的安全保證。下一節將詳細探討這種架構演進及其伴隨的安全性權衡。

\section{區塊鏈聯邦學習的委員會架構演進}
\label{sec:committee-evolution}

\subsection{從全節點到委員會的效率驅動}
\label{sec:full-to-committee}

區塊鏈聯邦學習的早期研究嘗試將傳統 BFT 共識直接應用於全體參與者,但很快便遭遇了可擴展性的瓶頸。以 BFLC \cite{li2021blockchain} 的實驗配置為例,當參與者數量達到 20 個時,採用完整 PBFT 共識的每輪延遲已超過 100 毫秒;若將參與者擴展至數百個規模,共識延遲將增長至秒級甚至更長,這對於需要快速迭代的聯邦學習訓練而言顯然無法接受。更根本的問題在於通訊頻寬的消耗:每輪共識中,每個節點都需要接收並處理來自其他所有節點的訊息,當節點數量增加時,網路負擔呈平方級增長,這在頻寬受限的邊緣運算環境中尤為致命。

委員會架構的核心理念是將共識責任委派給一個規模遠小於全網的代表性子集。令全網節點數為 $n$,委員會規模為 $c$,其中 $c \ll n$。委員會內部執行 BFT 共識的通訊成本為 $O(c^2)$,委員會決策結果廣播至全網的成本為 $O(n)$,總通訊成本為 $O(c^2 + n)$。當 $c$ 維持在較小的常數(如 7 至 20)時,此成本近似於線性 $O(n)$,相較於全節點 PBFT 的 $O(n^2)$ 實現了數量級的改善。FLCoin \cite{ren2024scalable} 的實驗資料印證了這一分析:在 500 個節點的網路中,採用規模為 100 的滑動視窗委員會,相較於全節點共識可減少約 90\% 的通訊開銷,共識延遲維持在 3 秒以內。

委員會架構的效率優勢使其迅速成為區塊鏈聯邦學習的主流設計典範。然而,這種效率的提升並非沒有代價:系統的安全性不再由全網的誠實多數保證,而是取決於委員會的組成是否可信。若攻擊者能夠控制委員會中超過三分之一的席位,便可操控共識結果,通過惡意的聚合提案或拒絕誠實的提案。這種從「全網安全」到「委員會安全」的轉變,將安全性分析的焦點從「全網惡意節點比例」轉移至「委員會選舉機制的抗操控能力」。

\subsection{委員會選舉機制的設計空間}
\label{sec:committee-selection}

委員會選舉機制決定了哪些節點將被選入委員會,其設計直接影響系統的安全性與公平性。現有方案大致可分為四種取向:隨機選擇、權益導向、聲譽導向與貢獻導向,各有其優勢與潛在風險。

隨機選擇機制透過密碼學隨機數決定委員會組成,其核心優勢在於不可預測性:攻擊者無法提前知曉哪些節點將被選中,因而難以針對性地部署攻擊。RapidChain \cite{zamani2018rapidchain} 採用分散式隨機數生成協議,確保選舉結果對所有參與者而言都是不可預測且可驗證的。然而,純粹的隨機選擇可能將惡意或低品質的節點選入委員會,且無法反映節點過去的行為表現。權益導向機制將選中機率與節點持有的權益(stake)掛鉤,持有越多權益的節點越可能被選入委員會。這種設計的理論基礎是經濟激勵對齊:高權益節點若行為惡意將面臨更大的經濟損失,因此傾向誠實。以太坊 2.0 的驗證者選舉即採用此機制。然而,權益導向可能導致「富者愈富」的中心化傾向,且無法防範願意承受經濟損失的攻擊者。

聲譽導向機制根據節點的歷史行為表現運算聲譽分數,高聲譽者優先被選入委員會。BESIFL \cite{chen2021robust} 追蹤各節點提交更新的品質,將聲譽作為委員會選舉的權重。此機制能有效過濾曾有惡意行為記錄的節點,但也面臨兩項挑戰:新加入者缺乏歷史記錄,可能陷入「冷啟動」困境;更重要的是,策略性攻擊者可透過長期的誠實行為累積聲譽,待時機成熟後再發動攻擊。貢獻導向機制以節點對聯邦學習的實質貢獻(如訓練資料量、模型品質)作為選舉依據。FLCoin \cite{ren2024scalable} 的滑動視窗機制即屬此類:節點透過提交有效的模型更新獲得「份額」,在視窗內持有份額的節點組成委員會。這種設計與聯邦學習的目標直接對齊,但貢獻指標可能被博弈——例如,攻擊者可先提交高品質更新以獲取委員會席位,再利用此席位通過惡意提案。

實際系統通常結合多種機制以平衡各方考量。BlockDFL \cite{qin2024blockdfl} 採用「權益加權的確定性隨機選擇」:選舉結果由前一區塊雜湊決定(確定性),但各節點被選中的機率與其權益成正比(權益加權)。這種混合設計試圖兼顧不可預測性與經濟激勵,但也繼承了權益導向機制的潛在風險——攻擊者可透過累積權益逐步提高其影響力。

\subsection{委員會規模與安全性的權衡分析}
\label{sec:committee-size-security}

委員會規模的選擇涉及效率與安全性之間的核心權衡。較小的委員會帶來更低的通訊成本與更快的共識速度,但也更容易被攻擊者滲透;較大的委員會提供更強的安全保證,但犧牲了效率優勢。理解這一權衡需要從機率論的角度分析委員會被攻破的風險。

當從 $n$ 個節點(其中 $m$ 個為惡意節點)中隨機選取 $c$ 個組成委員會時,委員會中恰有 $k$ 個惡意節點的機率遵循超幾何分佈:
\begin{equation}
P(X = k) = \frac{\binom{m}{k} \binom{n-m}{c-k}}{\binom{n}{c}}
\label{eq:hypergeometric}
\end{equation}
對於採用 BFT 共識的委員會,當惡意節點超過三分之一時委員會即被攻破。令 $\beta = 1/3$ 為此閾值,委員會被攻破的機率為:
\begin{equation}
P(\text{compromised}) = \sum_{k=\lceil \beta c \rceil}^{\min(m,c)} P(X=k)
\label{eq:compromised-prob}
\end{equation}

以具體數值說明此風險的量級。設全網有 $n=100$ 個節點,其中 30\% 為惡意節點($m=30$)。若委員會規模為 $c=7$(BlockDFL 的預設配置),運算可得惡意節點佔據超過三分之一席位(即至少 3 席)的機率約為 21\%。這意味著在每 5 輪選舉中,約有 1 輪會產生不安全的委員會組成。若將委員會規模提升至 $c=21$,此機率降至約 5\%;若進一步提升至 $c=50$,機率降至約 0.3\%。然而,委員會規模的增加直接推高了共識的通訊成本:$c=50$ 的委員會其內部通訊量約為 $c=7$ 的 50 倍。

上述分析揭示了委員會架構的一個根本性困境:在惡意節點比例固定的假設下,系統設計者必須在效率與安全性之間做出取捨。然而,這種「靜態快照」式的分析忽略了一個關鍵的動態因素:惡意節點的比例並非固定不變,而是可能隨時間演變。若委員會選舉機制與獎勵分配機制存在正反饋特性——獲得委員會席位的節點更容易在未來再次被選中——則攻擊者可能透過策略性的行為逐步提升其在系統中的影響力。這種「漸進式滲透」的威脅模式,正是本研究在後續章節將深入分析的核心議題。