\chapter{背景知識與相關研究}
\label{chap:background-related}

本章旨在建立理解本研究所需之技術基礎,並對現有研究進行系統性回顧。首先介紹聯邦學習的基本原理及其面臨的拜占庭威脅,接著探討傳統防禦機制的局限性與區塊鏈聯邦學習(BCFL)的興起。在相關研究部分,本章深入分析了現有 BCFL 方案在委員會選擇與驗證機制上的設計盲點,特別是針對「誠實多數假設」的依賴。最後,本章定義了本研究所採用的基準系統模型。

\section{聯邦式學習與拜占庭威脅 (Federated Learning and Byzantine Threats)}
\label{sec:fl_background}

聯邦學習(Federated Learning, FL)是由 McMahan 等人於 2017 年正式提出之分散式機器學習框架 \cite{mcmahan2017communication}。其核心目標在於多個參與方(Clients)協同訓練模型,而無需將原始資料集中於中央伺服器,從而解決資料隱私與孤島問題。

\subsection{聯邦式學習基礎 (Fundamentals of Federated Learning)}
在標準聯邦學習架構中,目標是最小化全域損失函數 $F(w)$:
\begin{equation}
    \min_{w} F(w) = \sum_{k=1}^{K} \frac{n_k}{n} F_k(w)
\end{equation}
其中 $K$ 為參與客戶端總數,$n_k$ 為第 $k$ 個客戶端之本地樣本數,$F_k(w)$ 為其本地損失函數。

經典的 \textit{FederatedAveraging} (FedAvg) 演算法透過週期性地收集客戶端模型更新 $w_{t+1}^k$,並在伺服器端進行加權聚合:
\begin{equation}
    w_{t+1} \leftarrow \sum_{k=1}^{K} \frac{n_k}{n} w_{t+1}^k
\end{equation}
此方法相較於同步隨機梯度下降(SGD)可顯著減少通訊開銷,但其安全性建立在中央聚合器完全誠實且客戶端皆非惡意的假設之上。

\subsection{拜占庭攻擊模型 (Byzantine Attack Models)}
在分散式環境中,系統必須面對拜占庭故障(Byzantine Fault)。根據 Blanchard 等人的定義,拜占庭節點可發送任意、潛在惡意的更新,並可能與其他惡意節點共謀。聯邦學習中的攻擊主要分為兩類:

\begin{enumerate}
    \item \textbf{資料投毒 (Data Poisoning)}:攻擊者汙染本地訓練資料(如標籤翻轉),導致模型學習錯誤的特徵。
    \item \textbf{模型投毒 (Model Poisoning)}:攻擊者直接操控上傳的梯度或模型參數。研究顯示,模型投毒比資料投毒更具威脅性。例如,Bagdasaryan 等人提出的模型替換攻擊(Model Replacement Attack) \cite{bagdasaryan2020how} 可在單一輪次內植入後門,並保持主任務的高準確率。
\end{enumerate}

\subsection{傳統拜占庭容錯聚合 (Traditional Byzantine-Robust Aggregation)}
為抵禦拜占庭攻擊,學界提出多種強健聚合演算法(Robust Aggregation Rules):

\begin{itemize}
    \item \textbf{Krum 及其變體} \cite{blanchard2017machine}:基於幾何距離選擇最接近多數節點的更新。Krum 選擇一個更新 $u^*$,使得其與最近 $n-f-2$ 個鄰居的歐式距離平方和最小。
    \item \textbf{裁剪均值 (Trimmed Mean)} \cite{yin2018byzantine}:在每個維度上移除最大與最小的 $\beta$ 比例數值後取平均,能有效抵禦統計極端值。
    \item \textbf{座標中位數 (Coordinate-wise Median)} \cite{yin2018byzantine}:取每個維度的中位數,具有高崩潰點(Breakdown Point)。
\end{itemize}

然而,這些防禦機制存在一個關鍵的局限性:\textbf{誠實聚合者假設}。如果不誠實的聚合器(Server)控制了聚合過程,它可以故意忽略防禦規則,甚至與惡意客戶端共謀。這構成了傳統聯邦學習的單點信任危機。

\section{區塊鏈聯邦式學習 (Blockchain-based Federated Learning, BCFL)}
\label{sec:bcfl_background}

區塊鏈聯邦式學習(BCFL)透過將分散式帳本技術(Distributed Ledger Technology, DLT)引入聯邦學習架構,從根本上重塑了多方協作訓練的信任模型。本節將探討 BCFL 的技術動機、架構演進脈絡、委員會共識機制的設計權衡,並深入分析現有方案在安全性上的結構性缺陷。

\subsection{技術動機:從中心化信任到去中心化共識}

\subsubsection{傳統聯邦學習的信任集中化困境}
儘管聯邦學習承諾「數據不出本地」,其標準架構仍高度依賴單一中央聚合器(Central Aggregator),這種中心化設計引入了三類關鍵的信任風險。首先是\textbf{聚合器的誠實性風險}:由於缺乏外部監督,中央伺服器可能執行選擇性聚合,故意排除特定客戶端的更新以操縱模型表現,甚至如同 Geiping 等人 \cite{geiping2020inverting} 指出,惡意伺服器可對梯度執行反演攻擊(Gradient Inversion Attack),從更新中重建原始訓練影像。其次是\textbf{單點故障(Single Point of Failure, SPOF)}:中央伺服器的可用性直接決定了整個系統的穩定性,任何網路攻擊或硬體故障皆會導致訓練全面中斷。最後是\textbf{拜占庭容錯能力的缺乏}:在惡意客戶端佔比超過 50\% 或中央伺服器本身遭入侵的情況下,傳統的穩健聚合演算法(如 Krum 或 Median)將失效,導致全域模型被投毒 \cite{fang2020local}。

\subsubsection{區塊鏈技術的解決方案}
區塊鏈技術的引入為上述問題提供了結構性的解方。\textbf{不可篡改性(Immutability)}確保了所有模型更新與聚合結果一旦上鏈便無法被回溯修改,為系統提供了可信任的審計軌跡,解決了結果篡改與抵賴問題。\textbf{智能合約的透明性(Transparency)}將聚合規則與客戶端選擇邏輯代碼化,使得所有參與者皆能驗證聚合過程的正確性,消除了黑箱操作的空間。\textbf{去中心化架構(Decentralization)}則通過點對點網路(P2P)取代了中央節點,利用共識機制(Consensus Mechanism)確保在部分節點失效或作惡的情況下,系統仍能維持運作並達成資料一致性。這種架構轉變將對單一實體的信任轉移至對密碼學協議與多數共識的信任,實現了更強健的安全性與容錯能力。

\subsection{BCFL 架構的演進:效率與安全的權衡}

BCFL 的發展歷程反映了學界在去中心化程度、通訊效率與安全性三者之間的權衡與探索。早期的研究,如 Kim 等人提出的 **BlockFL** \cite{kim2020blockchained},採用完全耦合架構(Fully Coupled Architecture),利用工作量證明(PoW)機制要求所有礦工驗證本地模型更新。這種設計雖然實現了極致的去中心化,但全節點驗證帶來了巨大的計算負擔,且 PoW 的共識延遲(Consensus Latency)嚴重拖累了模型訓練的迭代速度,使其難以應用於對時效性要求較高的邊緣運算場景。

為了克服擴展性瓶頸,**委員會共識架構(Committee-based Consensus)** 逐漸成為主流。Li 等人提出的 **BFLC** \cite{li2021blockchain} 與 Qin 等人的 **BlockDFL** \cite{qin2024blockdfl} 引入了代議制概念,從全體參與者中選出一個規模較小的驗證委員會(Verifier Committee)負責執行聚合與共識。此舉將共識的通訊複雜度從全網的 $O(n^2)$ 顯著降低至委員會內部的 $O(C^2)$ 或 $O(C)$。以 FLCoin \cite{ren2024scalable} 為例,其採用滑動視窗機制選取委員會,在維持安全性的前提下將通訊開銷降低了 90\%,並將共識延遲控制在數秒級別。然而,這種效率的提升也引入了新的攻擊面:系統的安全性從依賴全網多數誠實,轉變為依賴委員會成員的誠實性,這使得針對委員會的滲透攻擊成為可能。

\begin{table}[ht]
\centering
\caption{代表性 BCFL 系統比較}
\label{tab:bcfl_system_comparison}
\resizebox{\columnwidth}{!}{%
\begin{tabular}{|l|l|l|l|l|l|}
\hline
\textbf{系統} & \textbf{共識機制} & \textbf{通訊複雜度} & \textbf{容錯能力} & \textbf{主要創新} & \textbf{潛在局限} \\ \hline
BlockFL \cite{kim2020blockchained} & PoW & 依 PoW 難度 & 50\% 算力 & 首個 BCFL 框架 & 高能耗、高延遲 \\ \hline
BFLC \cite{li2021blockchain} & 委員會共識 & $O(C^2)$ & 33\% (3f+1) & K-fold 驗證 & 委員會負擔重 \\ \hline
Lu et al. \cite{lu2020blockchain} & PoTQ & IIoT 優化 & 33\% & 共識與訓練整合 & 需可信實體 \\ \hline
BlockDFL \cite{qin2024blockdfl} & PBFT 投票 & $O(A \times V)$ & \textbf{40\%} & 雙層評分 & 聚合器影響延遲 \\ \hline
FLCoin \cite{ren2024scalable} & 滑動視窗 & $O(s)$ 線性 & <25\% & 通訊大幅降低 & 視窗大小權衡 \\ \hline
\end{tabular}%
}
\end{table}

\subsection{關鍵技術組件與委員會安全分析}

\subsubsection{智能合約的功能職責}
在 BCFL 系統中,智能合約扮演著自動化管理者的角色,其功能通常劃分為四個核心模組:\textbf{註冊模組}負責維護參與者的身分與資格,並管理權益(Stake)的質押;\textbf{聚合模組}協調訓練輪次的同步,並管理模型更新的提交與聚合觸發;\textbf{驗證模組}是安全性的核心,負責執行預定義的校驗邏輯(如準確率測試或貢獻度評估)以過濾惡意更新;\textbf{獎勵模組}則依據 Shapley 值或其他貢獻度指標,自動分配代幣激勵,並對被偵測到的惡意行為執行罰沒(Slashing)。為了優化成本,現代架構多採用鏈下計算與鏈上驗證相結合的混合模式,僅將模型雜湊值與簡潔證明上鏈,而將高維度的參數儲存於 IPFS 等分散式存儲中。

\subsubsection{委員會選擇機制與風險分析}
委員會成員的選擇機制直接決定了系統的抗攻擊能力。現有文獻主要採用以下幾種策略:
\begin{itemize}
    \item \textbf{隨機選擇(Random Selection)}:如 BFLC,利用偽隨機函數選取驗證者,旨在防止特定節點被針對性攻擊。其缺點在於無法保證選出節點的品質,且若網路中存在大量惡意女巫節點(Sybil Nodes),委員會極易被滲透。
    \item \textbf{權益導向(Stake-based)}:類似權益證明(PoS),根據節點持有的權益代幣數量決定入選機率。這雖然提高了攻擊的經濟成本,但也容易導致「富者越富」的中心化傾向,且理性攻擊者可透過長期守序累積權益來發動後期攻擊。
    \item \textbf{聲譽導向(Reputation-based)}:依據歷史貢獻度進行選擇。然而,聲譽機制容易遭受共謀攻擊,惡意節點群可透過互相刷分來人為抬高聲譽。
\end{itemize}

從機率角度分析,若網路總節點數為 $n$,其中惡意節點數為 $m$,選取大小為 $C$ 的委員會。根據超幾何分佈,恰有 $k$ 個惡意節點入選的機率為 $P(X = k) = \frac{\binom{m}{k} \binom{n-m}{C-k}}{\binom{n}{C}}$。當惡意節點數超過 BFT 共識容忍閾值(如 $C/3$)時,委員會即被攻破。雖然增加 $C$ 可以降低被攻破的機率,但在 $n=100, m=30$ 的情境下,即便 $C=10$ 仍有約 3.88\% 的機率選出惡意委員會。對於追求高可靠性的系統而言,單純依賴隨機性顯然不足以提供充分的安全保障。

\subsection{現有研究的缺口}
儘管 BCFL 架構已日趨成熟,但在面對複雜攻擊場景時仍存在顯著的研究缺口。Nguyen 等人於 FedBlock \cite{nguyen2024fedblock} 中指出,現有的隨機選擇標準並非最佳,且缺乏有效的激勵機制來確保驗證者的持續誠實。更為關鍵的是,Taylor \& Francis 的回顧文獻 \cite{taylor2024blockchain} 明確警告,BFLC 與 FLCoin 等基於委員會的機制容易被惡意節點滲透。當攻擊者採取「漸進式委員會佔領」策略,即先表現誠實以獲取合法驗證權限,隨後在關鍵時刻發動攻擊時,現有的靜態防禦機制往往束手無策。這凸顯了對於一套具備動態偵測與事后問責能力的防禦機制的迫切需求。


\section{拜占庭容錯機制 (Byzantine Fault Tolerance Mechanisms)}
\label{sec:bft_background}

分散式系統在面對節點故障時,必須具備持續運作的能力。當故障不僅止於節點停機,而是涉及節點發送矛盾訊息或與其他惡意節點共謀時,系統便需要更強韌的拜占庭容錯機制。理解這些機制的基本原理與安全性閾值,是剖析 BCFL 系統安全性的必要前提。

\subsection{拜占庭將軍問題與容錯閾值}

\subsubsection{問題的起源與形式化定義}
拜占庭將軍問題由 Lamport 等人於 1982 年正式提出 \cite{lamport1982byzantine}。其核心在於:在存在叛徒(惡意節點)試圖破壞的情況下,忠誠的將軍(誠實節點)如何透過不可靠的通道達成一致決策。此問題定義了兩個交互一致性條件:(1) 所有忠誠副官必須執行相同的命令;(2) 若指揮官是忠誠的,則所有忠誠副官必須執行其命令。

\subsubsection{三分之一閾值的不可能性證明}
Lamport 證明了在僅使用口頭訊息的情況下,系統若要達成拜占庭容錯,總節點數 $n$ 與惡意節點數 $f$ 必須滿足 $n \geq 3f + 1$。這是因為當 $n=3f$ 時,接受者無法區分是傳送者惡意發送錯誤訊息,還是另一個轉發者在說謊。此理論極限確立了 PBFT 等共識協議的安全性邊界。

\subsubsection{故障模型區分}
表 \ref{tab:failure_models} 比較了分散式系統中最常見的兩類故障模型。在區塊鏈聯邦學習場景中,由於節點可能被攻擊者控制而表現出任意惡意行為,單純的崩潰容錯(Crash Fault Tolerance, CFT)不足以保證安全,必須採用拜占庭容錯(BFT)機制。

\begin{table}[ht]
\centering
\caption{崩潰故障與拜占庭故障之比較}
\label{tab:failure_models}
\begin{tabular}{|l|l|l|}
\hline
\textbf{特性} & \textbf{崩潰故障 (CFT)} & \textbf{拜占庭故障 (BFT)} \\ \hline
\textbf{行為特徵} & 節點停止運作,不再回應 & 節點可發送任意、矛盾訊息 \\ \hline
\textbf{容錯閾值} & $f < n/2$ & $f < n/3$ \\ \hline
\textbf{典型協議} & Paxos \cite{lamport2001paxos}, Raft \cite{ongaro2014search} & PBFT \cite{castro1999practical}, Tendermint \cite{tendermint} \\ \hline
\end{tabular}
\end{table}

\subsection{實用拜占庭容錯協議 (PBFT)}

\subsubsection{協議流程}
Castro 與 Liskov 於 1999 年提出的 PBFT 協議 \cite{castro1999practical} 首次將 BFT 共識的通訊複雜度從指數級降至多項式級別 $O(n^2)$。協議透過三個階段達成共識:
\begin{enumerate}
    \item \textbf{Pre-prepare}:主節點廣播請求與序號。
    \item \textbf{Prepare}:副本節點交換訊息以確認收到請求的一致性。
    \item \textbf{Commit}:副本節點確認全網已有足夠多的節點準備好執行請求。
\end{enumerate}
當節點收集到 $2f+1$ 個匹配的 Commit 訊息時,即確認達成共識。

\subsubsection{在 BCFL 中的角色}
在本研究架構中,PBFT 扮演著挑戰驗證的最終仲裁者角色。雖然其 $O(n^2)$ 的通訊複雜度限制了大規模擴展,但在聯盟鏈環境下(驗證者數量通常 $<20$),其通訊開銷是可接受的,且其成熟的安全性證明為系統提供了堅實的信任基礎。相較於 HotStuff \cite{yin2019hotstuff} 或 Algorand \cite{gilad2017algorand} 等新一代協議,PBFT 在小規模嚴格共識場景下仍具優勢。

\section{相關研究與盲點分析 (Related Work and Blind Spot Analysis)}

\label{sec:related_work}

本節基於近年文獻,深入分析現有 BCFL 方案在委員會選擇與安全性設計上的盲點。雖然現有方案解決了單點故障問題,但在面對具備長期策略的理性攻擊者時仍顯脆弱。

\subsection{委員會選擇機制的分類與局限}
現有的委員會選擇機制主要分為三類,各有其弱點:

\subsubsection{1. 隨機選擇 (Random Selection)}
如 BlockDFL \cite{qin2024blockdfl} 與 HoldOut SGD 採用雜湊環或 VRF(可驗證隨機函數)隨機選取驗證者。
\begin{itemize}
    \item \textbf{優勢}:公平性高,難以預測。
    \item \textbf{局限}:安全性完全依賴於「誠實多數」的機率分布。一旦惡意節點透過女巫攻擊(Sybil Attack)佔據網路多數,隨機選擇將失效。
\end{itemize}

\subsubsection{2. 基於權益的選擇 (Stake-based Selection)}
如 VBFL \cite{chen2021robust} 與 Biscotti \cite{shayan2021biscotti} 模仿权益证明(PoS),讓高权重的節點有更高機率進入委員會。
\begin{itemize}
    \item \textbf{優勢}:建立經濟門檻,增加攻擊成本。
    \item \textbf{局限}:存在「富者越富」效應。理性攻擊者可通過初期表現誠實累積權益,逐步提升被選機率,最終發動「漸進式委員會佔領攻擊」(Progressive Committee Capture)。
\end{itemize}

\subsubsection{3. 基於聲譽的選擇 (Reputation-based Selection)}
如 BFLC \cite{li2021blockchain} 與 CBRFL 根據歷史貢獻度選取高聲譽節點。
\begin{itemize}
    \item \textbf{優勢}:激勵高質量貢獻。
    \item \textbf{局限}:容易遭受共謀攻擊。惡意節點群可透過互相驗證與刷分(Reputation Inflation)來人為抬高聲譽,從而壟斷委員會席位。
\end{itemize}

\subsection{系統性盲點:驗證層的內部威脅}
綜合分析顯示,現有 BCFL 研究存在一個\textbf{系統性盲點}:絕大多數防禦機制(如 Krum, Trimmed Mean)僅針對\textbf{客戶端層}的攻擊,而假設\textbf{驗證層(或聚合層)}是誠實或至少大部分誠實的。

表 \ref{tab:defense_comparison} 總結了不同防禦層級的假設與局限:

\begin{table}[ht]
\centering
\caption{現有防禦機制的信任假設對比}
\label{tab:defense_comparison}
\begin{tabular}{|l|l|l|l|}
\hline
\textbf{防禦類型} & \textbf{代表方法} & \textbf{核心假設} & \textbf{主要局限} \\ \hline
傳統 Robust Aggregation & Krum, Median & 誠實聚合器 & 單點信任故障 \\ \hline
可驗證計算 (Verifiable FL) & zkFL \cite{wang2024zkfl}, ELSA & 誠實算力/證明 & 計算開銷極高 \\ \hline
區塊鏈委員會 (BCFL) & BlockDFL, BFLC & \textbf{誠實多數委員會} & 無法抵禦共謀/佔領 \\ \hline
\end{tabular}
\end{table}

\subsubsection{深入分析:漸進式委員會佔領攻擊 (Progressive Committee Capture Attack)}
現有委員會架構面臨的最深層威脅是「漸進式委員會佔領攻擊」。此攻擊利用權益或聲譽累積機制的正回饋特性,透過兩階段策略逐步控制系統:

\begin{enumerate}
    \item \textbf{潛伏階段 (Lurking Phase)}:攻擊者控制的節點在初期表現完全誠實,正常參與訓練並提交高品質的模型更新。透過持續的誠實行為,攻擊節點累積權益與聲譽,從而提高被選入委員會的機率。此階段中,攻擊者與誠實節點在行為上無法區分。
    \item \textbf{佔領階段 (Capture Phase)}:當攻擊節點首次在某一輪隨機過程中佔據委員會多數席位($>1/3$ 或 $>2/3$)時,他們啟動惡意行為。攻擊者利用多數優勢操控共識結果,核准投毒模型並將系統獎勵僅分配給自己控制的節點,同時拒絕誠實節點的更新。由於未來的委員會選舉通常基於持有的權益或聲譽,此操作使攻擊者的相對權益份額持續增加,進一步提高其在未來輪次佔據委員會多數的機率,形成自我強化的惡性循環。
\end{enumerate}

現有系統(如 BlockDFL \cite{qin2024blockdfl}, FLCoin \cite{ren2024scalable})缺乏有效的「事後問責」與「權力撤銷」機制來打破此循環。一旦攻擊者合法獲得了多數席位,他們產生的惡意區塊將被共識協議視為合法,系統無法自癒。本研究旨在填補此一缺口,提出能在驗證者共謀情況下仍能透過挑戰機制保證安全性的方案。

\section{系統模型與前置定義 (System Model and Preliminaries)}
\label{sec:system_model}

本節定義本研究所採用的基準系統模型。此模型基於 BlockDFL 委員會架構,作為後續威脅分析與防禦設計的基礎。

\subsection{網路模型}
本研究考慮一個去中心化的區塊鏈聯邦學習系統,由以下三種核心角色構成:
\begin{enumerate}
    \item \textbf{Update Providers (UP)}:原為客戶端 (Clients),集合記為 $\mathcal{U} = \{u_1, u_2, ..., u_N\}$。每個 Update Provider 持有本地私有資料集 $\mathcal{D}_i$,負責在本地進行模型訓練並提交更新。
    \item \textbf{Aggregators (AG)}:集合記為 $\mathcal{A} = \{a_1, a_2, ..., a_K\}$。負責收集 UP 的更新,執行初步聚合生成提案。Aggregator 的選擇基於權益。
    \item \textbf{Verifier Committee (VC)}:集合記為 $\mathcal{V} = \{v_1, v_2, ..., v_M\}$。Verifiers 組成委員會,負責驗證 Aggregator 的提案。委員會成員通過共識機制批准提案並上鏈。
\end{enumerate}

\subsection{聚合與共識流程}
在每個訓練輪次 $r$,系統執行以下流程:
\begin{enumerate}
    \item \textbf{本地訓練}:UP 訓練 model update $\Delta w_i$ 並發送給 AG。
    \item \textbf{初步聚合}:AG 生成聚合更新 $\Delta w_{agg}$ 並提交提案交易。
    \item \textbf{委員會驗證}:委員會 $\mathcal{V}_r$ 執行驗證邏輯(如 Krum 檢驗)。
    \item \textbf{共識決策}:委員會通過 BFT 共識對提案投票。
    \item \textbf{獎勵分配}:若提案通過,AG、UP 與投票贊成的 Verifiers 共同瓜分系統獎勵。
\end{enumerate}

\subsection{權益動態與攻擊面}
權益(Stake)在系統中扮演核心角色,既是選擇權重的依據,也是經濟獎勵的來源。這種「贏家通吃」的正反饋特性雖然激勵了誠實行為,但也創造了攻擊面:若攻擊者能策略性地累積權益,便能逐步掌控委員會。與傳統 PoS 不同,BCFL 中的攻擊者不僅能破壞共識,還能透過投毒模型永久損害全域模型的效能,且難以被傳統 BFT 機制偵測。
