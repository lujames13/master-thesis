\chapter{背景知識與相關研究}
\label{chap:background-related}

本章旨在建立理解本研究所需之技術基礎,並對現有研究進行系統性回顧。首先介紹聯邦學習與區塊鏈技術的結合動力,接著探討共識機制與經濟安全設計的基本原理。在相關研究部分,本章分析了現有區塊鏈聯邦學習方案在效率與安全性上的權衡,並指出現有方法在面對理性攻擊者時的局限性。最後,本章定義了本研究所採用的基準系統模型。

\section{聯邦式學習基礎 (Federated Learning Fundamentals)}
\label{sec:fl_fundamentals}

聯邦學習(Federated Learning, FL)是由McMahan等人於2017年正式提出之分散式機器學習框架 \cite{mcmahan2017communication}。其核心目標在於多個參與方(Clients)協同訓練模型,而無需將原始資料集中於中央伺服器,從而保護資料隱私。

\subsection{聯邦式學習的動機與定義}
在標準聯邦學習架構中,目標是最小化全域損失函數 $F(w)$:
\begin{equation}
    \min_{w} F(w) = \sum_{k=1}^{K} \frac{n_k}{n} F_k(w)
\end{equation}
其中 $K$ 為參與客戶端總數,$n_k$ 為第 $k$ 個客戶端之本地樣本數,$F_k(w)$ 為其本地損失函數。

\subsection{FedAvg 演算法與數學框架}
經典的 \textit{FederatedAveraging} (FedAvg) 演算法透過週期性地收集客戶端模型更新 $w_{t+1}^k$,並在伺服器端進行加權聚合:
\begin{equation}
    w_{t+1} \leftarrow \sum_{k=1}^{K} \frac{n_k}{n} w_{t+1}^k
\end{equation}
此方法雖然顯著降低了通訊開銷,但其安全性建立在中央聚合器完全誠實且客戶端皆非惡意的假設之上。在此框架下,任何單一聚合點的失效或惡意行為都將導致全局模型的崩潰。

\section{區塊鏈聯邦式學習 (Blockchain-based Federated Learning)}
\label{sec:bcfl_background}

為了消除對單一中央伺服器的依賴,研究者引入區塊鏈技術,提出區塊鏈式聯邦學習(BCFL)架構。在此架構中,去中心化帳本取代了傳統聚合器,提供不可篡改性與透明性。

\subsection{BCFL 的動機:解決信任問題}
BCFL 透過將模型聚合邏輯嵌入共識過程或智能合約,解決了聯邦學習中的「單點信任」問題。所有的模型更新、聚合歷史與獎勵分配均記錄於鏈上,確保了過程的可追溯性與可審計性。

\subsection{BCFL 基本架構與智能合約}
BCFL的發展經歷了從全節點共識到委員會機制的演進:
\begin{enumerate}
    \item \textbf{早期架構 (PoW-based)}:如 BlockFL \cite{kim2020blockchained} 使用工作量證明(PoW)達成共識,雖具備高度去中心化特性,但面臨高能耗與高延遲問題。
    \item \textbf{委員會共識 (Committee-based)}:如 BFLC \cite{li2021blockchain} 與 BlockDFL \cite{qin2024blockdfl}。為了提升效能,系統從全體參與者中選出一個子集(委員會)負責驗證與聚合。此種機制將通訊複雜度從 $O(n^2)$ 降低至 $O(C^2)$,其中 $C$ 為委員會大小。
\end{enumerate}
當前研究多採用智能合約來自動化執行聚合演算與獎勵發放,減少人為干預風險。

\section{拜占庭容錯機制 (Byzantine Fault Tolerance)}
\label{sec:bft_background}

在去中心化環境中,系統必須能夠抵禦拜占庭節點(發送任意或偽造訊息的節點)。

\subsection{PBFT 共識協議}
實用拜占庭容錯(Practical Byzantine Fault Tolerance, PBFT)是區塊鏈委員會常用的共識協議。它保證了在不超過 $1/3$ 節點失效的情況下,系統仍能達成一致性(Safety)與存活性(Liveness)。PBFT 的三階段投票流程(Pre-prepare, Prepare, Commit)確保了提案的終局性。

\subsection{委員會機制與效率優化}
為了進一步提升效能,現代 BCFL 方案普遍採用委員會架構。隨後的研究如 VBFL \cite{chen2021robust} 與 VFChain \cite{peng2022vfchain} 分別引入了基於權益的共識與可審計的聚合證明。這些方法將安全性與經濟質押(Staking)掛鉤,建立了加密經濟安全性(Crypto-economic Security)的基礎,確保攻擊成本高於潛在收益。

\section{相關研究與盲點分析 (Related Work and Blind Spot Analysis)}
\label{sec:related_work_merged}

本節基於 2023 年至 2025 年的區塊鏈聯邦學習(BCFL)文獻,對現有的防禦機制進行系統性分析。研究顯示,雖然大多數方案引入了拜占庭容錯聚合(Byzantine-robust Aggregation)來抵禦惡意客戶端(Malicious Clients),但在驗證層(Verification Layer)的安全性上存在顯著的盲點。

\subsection{現有文獻中的驗證者信任假設}
區塊鏈聯邦學習系統依賴礦工、驗證者或委員會成員執行 Krum、Trimmed Mean 等聚合演算法。然而,這些防禦機制僅在執行者誠實的前提下有效。我們的分析發現,約 93\% 的 BCFL 相關研究在不同程度上假設了驗證者的誠實性。

\subsubsection{顯式或隱式的誠實假設}
許多研究在威脅模型中未將驗證者列為潛在攻擊者。例如,BRFLATA \cite{li2025enhancing} 明確假設「服務器是可靠的」,僅針對客戶端與鏈路攻擊進行防禦。Szel\k{a}g 等人的調查 \cite{szelag2025adaptive} 雖然深入探討了適應性對手(Adaptive Adversaries),但仍隱含假設聚合器能正確執行防禦協議。

即便是強調去中心化的方案,如 BlockDFL \cite{qin2024blockdfl} 與 LiteChain \cite{chen2025litechain},也依賴於誠實多數假設。BlockDFL 指出系統安全性建立在「超過 2/3 的驗證者為誠實」的基礎上;LiteChain 則透過選舉機制選出委員會,並假設獲選成員值得信賴。這類假設在面對具備經濟動機的理性攻擊者(Rational Attackers)或共謀攻擊時顯得脆弱。

\subsubsection{依賴冗餘與輪替的信任機制}
部分研究試圖透過架構設計降低單點故障風險,但未根本解決惡意驗證問題。FLock \cite{flock2025} 利用多個聚合器進行狀態通道聚合,透過冗餘性提升容錯能力。BRFL \cite{song2024byzantine} 利用皮爾森相關係數(PPCC)動態選擇聚合節點,雖然實現了輪替,但仍假設被選中的高相關性節點是誠實的。

\subsection{針對惡意驗證者的防禦嘗試}
在近年的文獻中,僅有極少數研究直接面對驗證者可能完全惡意(Fully Malicious)的情況。

\begin{itemize}
    \item \textbf{KFC (Krum Federated Chain)} \cite{garcia2025krum}:這是目前唯一針對「所有驗證者皆可能惡意」場景提出完整防禦方案的研究。KFC 將 Krum 演算法與工作量證明(PoW)結合,設計了專屬的共識機制,確保即便驗證節點受損,系統仍能維持運作。
    \item \textbf{FedBlock} \cite{nguyen2024fedblock}:該研究明確指出了現有方法的缺陷,即「如果任何人都可以成為驗證者,僅假設誠實多數是不夠的」。然而,作者將「偵測並忽略惡意驗證者」列為未來工作,並未在文中提供具體解決方案。
    \item \textbf{Fantastyc} \cite{fantastyc2024}:該方案利用有效性證明(Validity Proofs)將驗證工作外包,能容忍高達 1/3 的拜占庭聚合者。儘管如此,它仍需維持誠實多數的底層假設。
\end{itemize}

\subsection{傳統聯邦學習與 BCFL 的信任模型對比}
值得注意的是,傳統(非區塊鏈)聯邦學習社群在處理惡意聚合器(Malicious Server)問題上,反而領先於 BCFL 社群。在傳統 FL 中,如 ELSA、zkFL \cite{chen2024zkml} 和 Mario 等方案,已廣泛利用零知識證明(ZKP)、可信執行環境(TEE)或多方計算(MPC)來強迫聚合器證明其計算正確性。這些研究將聚合器的惡意行為視為首要威脅(First-class Threat)。

相對地,BCFL 社群傾向認為「區塊鏈的去中心化」本身即能解決信任問題。然而,區塊鏈僅提供數據的不可篡改性(Immutability),並不能保證鏈下或鏈上計算邏輯(如 Krum 的執行)未被惡意節點操弄。

\subsection{系統性盲點與研究缺口}
綜合上述分析,現有 BCFL 研究存在一個系統性的盲點(Systematic Blind Spot):防禦機制高度集中於識別惡意客戶端的模型中毒攻擊(Model Poisoning),卻忽視了**驗證層本身的拜占庭風險**。

具體而言,當攻擊者透過賄賂、累積權益或女巫攻擊(Sybil Attack)掌控了委員會或驗證節點時,現有的 Krum 或 Trimmed Mean 防禦將被繞過或惡意執行。目前的文獻缺乏針對「驗證者共謀」與「基於權益的接管攻擊」的有效對策。本研究旨在填補此一缺口,提出一套在驗證者可能集體作惡的環境下,仍能透過經濟賽局機制保證系統安全性的方案。

\section{系統模型與前置定義 (System Model and Preliminaries)}
\label{sec:system_model}

本節定義本研究所採用的基準系統模型。此模型基於 BlockDFL 委員會架構,作為後續威脅分析與防禦設計的基礎。

\subsection{網路模型}

本研究考慮一個去中心化的區塊鏈聯邦學習系統,由以下三種核心角色構成:

\begin{enumerate}
    \item \textbf{Update Providers (UP)}:原為客戶端 (Clients),集合記為 $\mathcal{U} = \{u_1, u_2, ..., u_N\}$。每個 Update Provider 持有本地私有資料集 $\mathcal{D}_i$,負責在本地進行模型訓練並提交更新。
    
    \item \textbf{Aggregators (AG)}:集合記為 $\mathcal{A} = \{a_1, a_2, ..., a_K\}$。負責收集 UP 的更新,執行初步聚合生成提案。Aggregator 的選擇基於權益 (Stake-based)。
    
    \item \textbf{Verifier Committee (VC)}:集合記為 $\mathcal{V} = \{v_1, v_2, ..., v_M\}$。Verifiers 組成委員會,負責驗證 Aggregator 的提案。委員會成員通過共識機制批准提案並上鏈。
\end{enumerate}

\subsection{聚合與共識流程}

在每個訓練輪次 $r$,系統執行以下流程:
\begin{enumerate}
    \item \textbf{本地訓練}:UP 訓練 model update $\Delta w_i$ 並發送給 AG。
    \item \textbf{初步聚合}:AG 生成聚合更新 $\Delta w_{agg}$ 並提交提案交易。
    \item \textbf{委員會驗證}:委員會 $\mathcal{V}_r$ 執行驗證邏輯(如 Krum 檢驗)。
    \item \textbf{共識決策}:委員會通過 BFT 共識對提案投票。
    \item \textbf{獎勵分配}:若提案通過,AG、UP 與投票贊成的 Verifiers 共同瓜分系統獎勵。
\end{enumerate}

\subsection{獎勵機制與權益動態}

權益在系統中扮演雙重角色:
\begin{itemize}
    \item \textbf{選擇權重}:權益越高,被選入委員會或擔任 Aggregator 的機率越高。
    \item \textbf{經濟正反饋}:成功參與共識可獲得獎勵,進一步增加權益。
\end{itemize}
這種「贏家通吃」的正反饋特性雖然激勵了誠實行為,但也為後續描述的佔領攻擊埋下了伏筆。

\section{本章小結}
\label{sec:background_summary}

本章首先介紹了聯邦式學習的基本框架與 FedAvg 演算法,接著說明區塊鏈技術如何解決聯邦學習中的信任問題。在共識機制方面,本章詳細闡述了 PBFT 協議及其在委員會架構中的應用。

針對現有區塊鏈聯邦學習方案的分析顯示,基於驗證的方法(如 opML、zkML)受限於計算通用性,而基於委員會的方法(如 FLCoin、BlockDFL)雖提升了效率,但其安全性仍依賴「誠實多數假設」。當理性攻擊者透過權益累積逐步滲透委員會時,現有機制缺乏有效的偵測與防禦能力。

\ref{sec:system_model} 節定義的系統模型將作為後續分析的基礎。下一章將針對此架構,深入分析「漸進式委員會佔領攻擊」的威脅模型,揭示理性攻擊者如何利用權益機制的正反饋特性實現網路控制權的轉移。
