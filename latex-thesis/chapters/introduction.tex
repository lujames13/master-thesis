\chapter{緒論 (Introduction)}
\label{chapter:introduction}

\section{研究背景 (Background \& Context)}

隨著物聯網 (IoT) 設備的普及和隱私法規 (如 GDPR) 的日益嚴格,聯邦學習 (Federated Learning, FL) 作為一種隱私保護的分散式機器學習範式,獲得了廣泛關注。FL 允許客戶端在本地訓練模型並僅上傳模型更新,從而避免了原始數據的傳輸。然而,傳統的 FL 依賴於中心化的聚合伺服器 (Central Server),這不僅構成了單點故障 (Single Point of Failure),還面臨著服務器惡意篡改模型或遭受攻擊的風險。

為了解決這些問題,區塊鏈聯邦學習 (Blockchain-based Federated Learning, BCFL) 應運而生。BCFL 利用區塊鏈的去中心化、不可篡改和智能合約特性,取代了傳統的中心化聚合器。在 BCFL 架構中,模型更新的驗證和聚合通常由一組選定的「委員會 (Committee)」或「驗證者 (Verifiers)」負責。這種架構試圖通過共識機制 (Consensus Mechanism) 來確保模型更新的正確性,並通過加密貨幣激勵機制來鼓勵節點參與。

目前,主流的 BCFL 系統 (如 BlockDFL) 多採用基於權益 (Stake) 或聲譽 (Reputation) 的委員會選舉機制,並依賴拜占庭容錯 (BFT) 共識來達成決策。這些系統通常假設誠實節點佔據多數 (Honest Majority),並以此作為安全性的基石。

\section{問題陳述 (Problem Statement)}

儘管 BCFL 在去中心化方面取得了進展,但現有研究存在一個關鍵的「驗證盲點 (Verification Blind Spot)」。絕大多數 (約 93\%) 的現有文獻主要關注數據層面的投毒攻擊 (Data Poisoning),例如標籤翻轉或後門攻擊,並開發了如 Krum、Trimmed Mean 等魯棒聚合算法。然而,這些防禦機制隱含地假設執行聚合算法的「委員會」本身是誠實的。

本研究指出,這一假設在面對「理性攻擊者 (Rational Attacker)」時是極其脆弱的。我們定義了一種新的威脅模型——\textbf{委員會佔領 (Committee Capture)},特別是其具體實現形式\textbf{「漸進式委員會佔領攻擊 (Progressive Committee Capture Attack, PCCA)」}。在此攻擊中,理性攻擊者並非旨在破壞模型,而是追求利益最大化。他們通過以下兩個階段實施攻擊:

\begin{enumerate}
    \item \textbf{潛伏階段 (Infiltration Phase):} 攻擊者表現誠實,積累權益 (Stake) 以進入驗證者池。
    \item \textbf{佔領階段 (Capture Phase):} 一旦在委員會中獲得多數席位,攻擊者便實施「戰略性餓死 (Strategic Starvation)」,即拒絕打包誠實節點的更新,獨佔系統獎勵。
\end{enumerate}

這種攻擊導致誠實節點的權益停滯,而攻擊者的權益呈指數級增長,從而進一步鞏固其在未來委員會中的控制權。傳統的 BFT 共識機制面臨著「安全性與效率的兩難 (Security-Efficiency Dilemma)」:為了防禦共謀,必須擴大委員會規模 ($C$),但通訊複雜度 ($O(C^2)$) 會隨之呈二次方增長,導致系統效率急劇下降。

\section{研究目標 (Research Objectives)}

針對上述問題,本研究的主要目標是設計一種既能防禦理性共謀,又能保持高效率的 BCFL 架構。具體目標包括:

\begin{enumerate}
    \item \textbf{防禦委員會佔領:} 設計一種機制,使系統在面對理性驗證者共謀時仍能保持安全,打破對「誠實多數」的依賴。
    \item \textbf{解耦安全性與效率:} 打破傳統 BFT 系統中安全性與委員會規模的綁定關係,實現在使用小型委員會 (確保活性) 的同時,提供不亞於大型委員會的安全性。
    \item \textbf{實現激勵相容 (Incentive Compatibility):} 利用博弈論設計獎懲機制,使得「誠實行為」成為理性節點的納什均衡 (Nash Equilibrium) 策略。
\end{enumerate}

\section{研究貢獻 (Contributions)}

本研究的主要貢獻歸納如下:

\begin{enumerate}
    \item \textbf{識別新型威脅 (New Threat Identification):} 我們系統地分析了 BCFL 的驗證層漏洞,並定義了「漸進式委員會佔領攻擊 (PCCA)」。我們揭示了理性攻擊者如何利用權益機制進行中心化接管,這補充了現有文獻僅關注數據投毒的不足。

    \item \textbf{提出基於異步審計的激勵相容架構 (Incentive-Compatible Architecture with Asynchronous Audit):} 我們提出了一種結合「即時執行 (Immediate Execution)」與「異步審計 (Asynchronous Audit)」的新型防禦機制。
    \begin{itemize}
        \item 採用\textbf{即時執行}策略,移除傳統的驗證等待期,確保模型訓練的零延遲與高活性 (Liveness)。
        \item 引入\textbf{異步審計 (Asynchronous Audit)} 機制,允許挑戰者在事後對委員會的決策進行回溯性驗證。
        \item 設計\textbf{內部罰沒機制 (Internal Slashing)},確保 $Slashing \gg Gain$,從根本上遏制理性攻擊者的作惡動機。
    \end{itemize}

    \item \textbf{驗證與評估 (Evaluation):} 我們通過理論分析和模擬實驗驗證了所提機制的有效性。
    \begin{itemize}
        \item \textbf{安全性:} 在 30\% 節點共謀的極端情況下,本方案仍能維持模型收斂 (準確率 91.8\%),而對照組 (BlockDFL) 則崩潰 (65\%)。
        \item \textbf{效率:} 本方案成功將通訊複雜度從 $O(C_{large}^2)$ 降低至 $O(C_{small}^2)$ (正常情況),在保證高安全性的前提下實現了約 2~7 倍的效率提升。
    \end{itemize}
\end{enumerate}

\section{論文組織 (Thesis Organization)}

本論文共分為六章,組織結構如下:

\begin{itemize}
    \item \textbf{第一章 緒論 (Introduction):} 介紹研究背景、問題陳述、研究目標及貢獻。
    \item \textbf{第二章 背景知識 (Background):} 介紹聯邦學習、區塊鏈技術及博弈論基礎。
    \item \textbf{第三章 相關工作 (Related Work):} 回顧現有的 BCFL 方案及其局限性,特別是針對共識層安全的討論。
    \item \textbf{第四章 威脅模型 (Threat Model):} 詳細定義系統模型、攻擊者能力及 PCCA 攻擊策略。
    \item \textbf{第五章 架構設計 (Framework Design):} 闡述所提出的異步審計與即時執行架構,包括協議流程、審計機制及智能合約設計。
    \item \textbf{第六章 實驗評估 (Evaluation):} 展示模擬實驗結果,對比本方案與基準方案在模型效能、權益動態及系統效率上的表現。
    \item \textbf{第七章 結論與未來展望 (Conclusion and Future Work):} 總結全文並提出未來的研究方向。
\end{itemize}
