\begin{ZhChapter}

\chapter{緒論 (Introduction)}
\label{chapter:introduction}

隨著人工智慧與分散式運算技術的蓬勃發展,聯邦學習 (Federated Learning) \cite{mcmahan2017communication} 與區塊鏈技術的深度整合,催生了區塊鏈賦能的聯邦學習 (Blockchain-based Federated Learning, BCFL) 這一創新技術典範。此技術路徑之所以受到學界與產業界的高度重視,根本原因在於其成功回應了多方互不信任情境下協作機器學習的核心挑戰。在低軌衛星網路 (LEO) \cite{pokhrel2021blockchain, wu2024sharded, elmahallawy2025decentralized}、車聯網 (V2X) \cite{lu2020asynchronous, liu2021blockchain, pokhrel2020autonomous} 以及工業物聯網 (IIoT) \cite{lu2020blockchain, qu2020decentralized, eppbcfl2025} 等實際應用場景中,BCFL 展現了其獨特且難以替代的技術價值。以 LEO 衛星星座為代表的太空人工智慧應用場景尤具說明性意義,在此類場景中星地通訊窗口通常僅維持約五分鐘,且下行頻寬受限於每秒 8 Mbps 左右的水準 \cite{wu2024sharded},這種嚴苛的通訊條件使得傳統依賴地面站進行模型聚合的訓練方案在技術上幾乎無法實現。BCFL 透過在異質衛星營運商之間建立去中心化的信任基礎設施,成功將模型收斂所需時間縮減達三十小時之譜 \cite{elmahallawy2025decentralized},充分展現了此技術路徑在極端環境下的應用潛力。類似地,在工業 4.0 的發展背景下,BCFL 使得協作工廠能夠在完全不洩露各自商業機密的前提下共同進行預測性維護模型的訓練,實驗數據顯示此種架構可將通訊開銷較傳統集中式方案降低約百分之四十一 \cite{lu2020blockchain}。上述多元應用場景共同呈現出三項關鍵特徵:缺乏可信賴的中心化協調者、計算與通訊資源高度受限、以及訓練資料在統計分布上的顯著異質性,正是這些特徵的交織作用,使得 BCFL 逐步確立其作為通用去中心化學習架構首選方案的地位。

然而,當 BCFL 系統嘗試邁向大規模實際部署時,卻面臨著嚴峻的效能瓶頸,此問題在學術文獻中通常被稱為「可擴展性兩難困境」。究其根源,目前絕大多數 BCFL 系統採用實用拜占庭容錯協議 (Practical Byzantine Fault Tolerance, PBFT) \cite{castro1999practical} 或其各種變體作為底層共識機制,而此類協議固有的 $O(n^2)$ 訊息複雜度特性,導致系統效能隨著參與節點數量的增加而呈現急劇下降的趨勢。FLCoin \cite{qi2024scalable} 的實證研究提供了具體的量化證據:當參與節點數量達到一百個時,單一輪次的共識過程所產生的訊息交換量將突破兩萬條,共識延遲因而攀升至二十五秒以上,此延遲水準已經達到甚至超過模型訓練本身所需時間的量級,對於要求即時響應的應用場景而言構成了根本性的障礙。在車載網路 (VANET) 的極端測試條件下,一百輛車輛進行 BCFL 協作訓練將產生高達 360.57 MB 的資料傳輸量,單輪訓練的總通訊時間開銷達到 19.51 秒 \cite{vanetbcfl2024},這對於需要頻繁進行模型更新的車載智慧系統而言顯然是無法接受的效能表現。除了通訊效能問題之外,區塊鏈節點對於儲存空間的高需求同樣構成嚴峻挑戰,例如比特幣全節點需要約 200 GB 的儲存空間,而以太坊更是超過 465 GB,這與邊緣運算設備通常僅具備 KB 至 MB 等級記憶體容量的現實形成了尖銳的衝突 \cite{fedchain2024}。效能與資源的這種雙重束縛,使得要求所有節點參與驗證的傳統全節點架構在實際工業部署情境中越來越難以為繼,迫切需要創新的技術方案來突破這些根本性的限制。

正是為了化解上述可擴展性挑戰,學術界近年來將研究重心轉向「委員會機制 (Committee Mechanism)」的探索與發展,此機制的核心設計理念是將原本由全體節點共同承擔的驗證責任,集中委派給一個規模相對較小的驗證者委員會來執行。這種架構轉變的根本邏輯在於,透過大幅減少實際參與共識決策過程的節點數量,可以顯著降低系統的通訊複雜度與計算負擔。目前學界提出的主流委員會選拔機制包含數種不同的技術路徑:基於雜湊環結構的隨機抽樣方法 \cite{qin2024blockdfl}、依據持有代幣數量或權益進行加權選舉的方法 \cite{li2021blockchain, qi2024scalable}、以及利用可驗證隨機函數 (VRF) 實現的 Sortition 機制 \cite{shayan2021biscotti, weng2021deepchain, gilad2017algorand}。委員會機制的導入確實為系統效能帶來了立竿見影的改善效果,FLCoin \cite{qi2024scalable} 的研究報告顯示,透過採用滑動窗口選舉策略,系統成功將通訊開銷降低了百分之九十,並實現了 5.7 倍的訓練速度提升;BFLC \cite{li2021blockchain} 則藉由委員會驗證機制將共識延遲穩定控制在三秒以內。這些最佳化成果成功地將通訊複雜度從全網共識的 $O(n^2)$ 降至與委員會規模 $C$ 相關的 $O(C^2)$ 甚至 $O(C)$ 等級,極大提升了系統的吞吐量與響應速度。然而,這種為追求效率而將驗證權力集中於少數委員會成員的架構轉變,也在無形中引入了全新的安全攻擊面,對系統的長期安全性與穩定性構成了潛在威脅,而這些威脅至今尚未獲得學術界的充分探討。

深入剖析上述安全隱憂可以發現,現有委員會防禦機制普遍存在一個容易被研究者忽視的結構性弱點,即對「誠實多數假設」的過度依賴。儘管當前的區塊鏈聯邦學習系統已經廣泛導入 Krum、Trimmed Mean 或 Median 等具備拜占庭強健性的聚合演算法 \cite{blanchard2017machine, yin2018byzantine},但這些演算法能夠發揮預期防禦效果的根本前提是:執行聚合運算的委員會成員中,誠實節點必須穩定維持佔多數的地位。當前安全性研究的主要關注焦點集中在惡意客戶端上傳毒化梯度的資料層攻擊,卻在很大程度上忽略了一種更為隱蔽的威脅形式,即委員會成員之間可能形成的策略性共謀行為。一旦攻擊者成功控制委員會中超過三分之二的席位,便能夠串聯起來繞過所有現有的資料層強健聚合邏輯,甚至可以任意偽造聚合結果而無需擔心遭到系統的懲罰或追責。正如 FedBlock \cite{nguyen2024fedblock} 所指出,當任何參與者都可能成為驗證者時,系統不能僅仰賴誠實多數假設,而必須具備主動檢測並隔離惡意驗證者的能力,然而這項關鍵機制在現有的區塊鏈聯邦學習架構中普遍付之闕如,這揭示了當前安全防禦體系在共識決策層面的本質脆弱性。

上述分析揭示了當前研究領域中存在的一個關鍵「研究缺口」:現有的 BCFL 系統普遍缺乏能夠應對「漸進式委員會佔領攻擊 (Progressive Committee Capture Attack, PCCA)」的自我修復與防禦機制。在 PCCA 攻擊模式中,對手並非採取直接的暴力破壞策略,而是實施一種被稱為「策略性餓死 (Strategic Starvation)」的隱蔽手法,具體而言,攻擊者在成功掌控委員會之後,會優先處理與自身經濟利益相關的模型更新請求,同時系統性地拒絕為誠實參與者提供驗證服務,藉此操縱系統的獎勵分配機制與權益動態變化。值得特別關注的是,現有的各種解決方案大多僅聚焦於特定的防禦維度,例如基於同態加密 (Homomorphic Encryption, HE) \cite{zhang2020batchcrypt, miao2022privacy} 或權益證明 (Proof of Stake, PoS) \cite{chen2021robust} 的技術方案,雖然在隱私保護或准入控制等特定面向上確實有所突破,卻無法在委員會本身已經淪陷而不再可信的情況下,繼續保證模型聚合更新的正確性以及系統資源分配的公平性。由於缺乏事後的「可追溯審計能力」與「有效經濟威懾手段」,一旦誠實多數假設在某一輪次中被攻破,現有系統便缺乏任何機制來識別並懲罰惡意行為者,使得攻擊者得以在後續輪次中持續佔據優勢地位。因此,如何將系統的安全性保障與對共識節點集體誠實性的依賴進行有效解耦,已成為實現真正意義上去中心化人工智慧平台所必須跨越的關鍵技術門檻。

針對這一嚴峻的技術挑戰,本論文提出一套名為「挑戰者增強委員會架構 (Challenge-Augmented Committee Architecture, CACA)」的創新防禦體系,旨在為區塊鏈聯邦學習系統建構一道動態且具備自我修復能力的安全屏障。本研究的核心設計理念在於,透過引入異步審計機制與內部罰沒協議,賦予現有委員會架構檢測並汰除惡意驗證者的能力。這種設計使得系統能夠將「活性 (Liveness)」與「安全性 (Security)」這兩項往往相互牽制的系統屬性進行有效的解耦處理。在本架構下,即使委員會在特定輪次中並非完全可信甚至已經被惡意節點所捕獲,系統仍然能夠透過分散部署於全網的挑戰者網路,對委員會所做出的每一項決策進行事後審計與追責,進而檢舉並懲罰任何偽造或誤導性的聚合結果。這種創新機制不僅大幅提升了系統面對策略性攻擊時的韌性,更為去中心化環境下的模型協作訓練提供了一層額外且獨立的安全性保障,確保了系統在追求效率提升的同時,不會犧牲去中心化架構所應堅守的核心安全原則。

進一步而言,本研究在理論分析、協議設計與實驗驗證三個相互支撐的維度上均做出了具體而紮實的學術貢獻。在理論分析層面,本研究首次對漸進式委員會佔領攻擊 (PCCA) 模式進行了形式化的數學定義,並透過系統性的模擬實驗量化評估了該攻擊對系統長期激勵相容性所造成的深層破壞效應,藉此揭露了傳統防禦機制在面對具有演化特性的策略性攻擊時所存在的根本局限。基於上述理論發現,本研究進一步結合博弈論模型設計了一套精密的內部罰沒協議 (Internal Slashing Protocol),此協議透過精心設計的經濟懲罰機制,確保任何節點進行審計驗證所需付出的成本始終低於其從事惡意行為可能獲取的收益,從而促使誠實行為成為所有理性參與者在長期重複博弈下的唯一納什均衡策略 \cite{chiu2018incentive}。這些理論層面的創新貢獻在後續的大規模模擬實驗中獲得了充分的實證支持。實驗數據顯示,即使在百分之三十節點進行惡意共謀的極端對抗環境下,採用 CACA 架構的系統仍能將模型最終準確率穩定維持在百分之九十八點六以上的水準,同時在達成相同安全性保障指標的前提下將通訊開銷降低了百分之四十四點四,並將系統的最低不可用率從百分之二十大幅壓制至百分之五以下,顯著提升了系統在對抗性環境中的強健性與運作效率。

本論文後續章節的組織架構安排如下。在本緒論之後,第二章將系統性地奠定本研究所需的理論基礎,內容涵蓋聯邦學習的核心概念、區塊鏈共識機制的運作原理、以及拜占庭容錯技術的數學基礎,並針對現有去中心化聯邦學習領域的相關文獻進行批判性的回顧與評述。以第二章建立的知識基礎為起點,第三章將詳細定義本研究所採用的系統模型與威脅模型,並針對 PCCA 攻擊者的行為特徵、能力邊界與策略空間進行嚴謹的形式化描述。在完成理論模型的建構之後,第四章將提出 CACA 架構的核心設計方案,完整闡明具體的協議運作流程、關鍵演算法的設計細節、以及支撐整體架構的安全性證明邏輯。緊接著,第五章將透過一系列精心設計的針對性實驗與全面的數據比較分析,在多種不同的對抗場景設定下驗證本架構所具備的效能優勢與系統韌性。最終,第六章將總結本研究的主要發現與學術貢獻,並就本機制在未來分散式人工智慧生態系統中的應用潛力與可能的演進方向進行前瞻性的探討。

\end{ZhChapter}