\chapter{緒論 (Introduction)}
\label{chapter:introduction}

本章節介紹研究背景、動機、欲解決之關鍵問題以及本論文之主要貢獻。

\section{研究背景與動機}
隨著大數據與人工智慧的快速發展,隱私保護下的機器學習已成為重要趨勢。聯邦學習(Federated Learning, FL)提供了一種在不交換原始資料的情況下協同訓練模型的方法。然而,傳統 FL 仰賴中央伺服器,存在單點故障與信任依賴問題。

\section{問題陳述}
區塊鏈聯邦學習(BCFL)透過委員會機制提升了效率,但其安全性高度依賴「誠實大多數」假設。當惡意節點透過權益累積佔領委員會後,系統缺乏自癒能力。

\section{研究目標與貢獻}
本研究提出一種結合挑戰機制與罰沒處置的 Challenge-Augmented Committee 架構,旨在:
\begin{itemize}
    \item 降低大型委員會帶來的通訊開銷。
    \item 透過經濟懲罰機制確保系統安全性。
    \item 在委員會淪陷的情況下提供自動修復機制。
\end{itemize}
