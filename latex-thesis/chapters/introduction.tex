\chapter{緒論 (Introduction)}
\label{chapter:introduction}

隨著人工智慧與分散式運算技術的進步,聯邦學習 (Federated Learning) \cite{mcmahan2017communication} 與區塊鏈的結合,即區塊鏈賦能的聯邦學習 (Blockchain-based Federated Learning, BCFL),已成為解決多方互不信任情境下協作機器學習的核心技術路徑。在諸如低軌衛星網路 (LEO) \cite{pokhrel2021blockchain, wu2024sharded, elmahallawy2025decentralized}、車聯網 (V2X) \cite{lu2020asynchronous, liu2021blockchain, pokhrel2020autonomous} 以及工業物聯網 (IIoT) \cite{lu2020blockchain, qu2020decentralized, eppbcfl2025} 等實際應用場景中,BCFL 展現了其不可替代的重要性。特別是以 LEO 衛星星座為代表的太空 AI 應用場景,星地通訊窗口通常僅約 5 分鐘,且下行頻寬受限於 8Mbps 左右 \cite{wu2024sharded},這使得依賴地面站聚合的傳統模型訓練方案難以實施。BCFL 通過在異質衛星營運商間建立去中心化信任層,成功將收斂時間減少達 30 小時 \cite{elmahallawy2025decentralized}。同樣地,在工業 4.0 的背景下,BCFL 允許協作工廠在不洩露商業機密的前提下進行預測性維護,實驗資料顯示其通訊開銷可較集中式架構減少約 41\% \cite{lu2020blockchain}。這些場景共同呈現出無可信中心、資源受限與資料高度異質的特徵,促使 BCFL 成為通用去中心化學習架構的首選方案。

然而,BCFL 在邁向大規模部署時面臨著嚴峻的效率瓶頸,這在業界被稱為「可擴展性兩難」。目前絕大多數 BCFL 系統採用 PBFT (Practical Byzantine Fault Tolerance) \cite{castro1999practical} 或其變體作為共識機制,其 $O(n^2)$ 的訊息複雜度在節點數增加時會導致效能急劇下降。根據 FLCoin \cite{qi2024scalable} 的實證研究,當參與節點數達到 100 個時,單輪共識產生的訊息量將超過 20,000 條,導致共識延遲攀升至 25 秒以上,此延遲水平已達到模型訓練時間的量級,嚴重阻礙了即時應用。在極端的車載網路 (VANET) 實測中,100 輛車進行 BCFL 協作會產生 360.57 MB 的巨大資料量,單輪訓練的總通訊開銷高達 19.51 秒 \cite{vanetbcfl2024},這對於頻繁互動的車載環境而言是不可接受的。此外,區塊鏈節點對儲存的高需求,例如比特幣需 200GB,乙太坊超過 465GB,與邊緣設備 KB 至 MB 級的有限記憶體形成強烈衝突 \cite{fedchain2024}。這種效能與資源的雙重束縛,使得全節點驗證的傳統架構在實際工業部署中顯得難以維繫,亟需創新的解決方案來突破這些限制。

為了解決上述可擴展性挑戰,學界近年來轉向研究「委員會機制 (Committee Mechanism)」,其核心思想是將驗證責任從全體節點縮減至一組小型驗證者委員會。這種方法旨在透過減少參與共識的節點數量來顯著降低通訊複雜度與計算負擔。目前主流的選拔機制包含基於雜湊環的隨機抽樣 \cite{qin2024blockdfl}、基於幣齡或權益的權重選舉 \cite{li2021blockchain, qi2024scalable} 以及基於預言機 (VRF) 的 Sortition 機制 \cite{shayan2021biscotti, weng2021deepchain, gilad2017algorand}。委員會機制的引入立竿見影地改善了系統效能:FLCoin \cite{qi2024scalable} 通過滑動窗口選舉將通訊開銷降低了 90\%,並實現了 5.7 倍的訓練加速;BFLC \cite{li2021blockchain} 則利用委員會驗證成功將共識延遲穩定在 3 秒以內。這些最佳化雖成功將通訊複雜度降至與委員會規模 $C$ 相關的 $O(C^2)$ 或 $O(C)$,極大地提升了系統的吞吐量與響應速度。然而,這種為了效率而進行的算力與權力的集中,也同時引入了新的、尚未被充分探討的安全攻擊面,對系統的長期穩定性構成潛在威脅。

然而,現有委員會防禦機制存在一個被普遍忽視的結構性弱點:對「誠實多數假設」的過度依賴。儘管 Krum、Trimmed Mean 或 Median 等拜占庭強健聚合演算法 \cite{blanchard2017machine, yin2018byzantine} 已被廣泛採用於區塊鏈聯邦學習系統中,這些演算法的有效運作仰賴一個關鍵前提,即執行聚合運算的委員會成員中,誠實節點必須佔據多數。現有的威脅模型大多聚焦於惡意客戶端上傳毒化梯度的情境,卻未充分考量委員會成員之間可能形成的共謀行為。當攻擊者控制的節點在委員會中取得超過三分之二的席位時,他們便能協同繞過所有資料層的強健聚合演算法,甚至偽造聚合結果而不受任何懲罰。FedBlock \cite{nguyen2024fedblock} 的研究明確指出,現有機制在面對具備長期策略的委員會共謀攻擊時存在顯著漏洞,這凸顯了當前安全防禦體系在共識層的脆弱性。

上述現象揭示了一個關鍵的「研究缺口 (Research Gap)」:現有 BCFL 缺乏應對「漸進式委員會佔領攻擊 (Progressive Committee Capture Attack, PCCA)」的自癒機制。在 PCCA 中,對手並非採取暴力破壞,而是實施「策略性餓死 (Strategic Starvation)」,亦即在掌控委員會後,優先打包與自身利益相關的更新,並拒絕為誠實參與者提供驗證服務,從而操縱獎勵分配與權益動態。值得注意的是,現有的解決方案多聚焦於特定維度,例如基於同態加密 (Homomorphic Encryption, HE) \cite{zhang2020batchcrypt, miao2022privacy} 或權益證明 (Proof of Stake, PoS) \cite{chen2021robust} 的方案,雖然在隱私保護與准入控制上有所突破,卻無法在委員會本身已不再可信的情況下,保證模型更新的正確性與資源分配的公平性。由於缺乏事後的「可追溯審計」與「有效威懾」,一旦誠實多數假設在某一輪次被攻破,系統權力將產生雪崩式的中心化。因此,如何解耦安全性與共識節點集體信用,成為實現真正去中心化 AI 平台的最後一哩路。

針對這一嚴峻挑戰,本論文提出一套名為「挑戰者增強委員會架構 (Challenge-Augmented Committee Architecture, CACA)」的創新防禦體系,旨在為區塊鏈聯邦學習系統建立一個動態且可自我修復的安全屏障。本研究的核心理念在於打破傳統「先驗證、後提交」的靜態防禦範式,轉向「即時執行、異步審計、罰沒威懾」的主動安全模式。這種設計上的轉向,使得系統能夠將「活性」(Liveness) 與「安全性」(Security) 進行實質性的解耦。即使在委員會不完全可信甚至被惡意捕獲的情況下,系統仍能透過分佈式的挑戰者網路對委員會的決策進行事後追責,進而檢舉被偽造或誤導的聚合結果。這種機制不僅提升了系統的韌性,更為去中心化環境下的模型協作提供了一層額外的安全性保障,確保了在效率提升的同時,不犧牲核心的去中心化安全原則。

進一步而言,本研究在理論分析、協議開發與實驗驗證三個維度上均做出了具體貢獻。在理論層面,我們首次形式化定義了漸進式委員會佔領攻擊 (PCCA) 模式,並透過模擬量化了該攻擊對系統長期激勵相容性的深度破壞力,以此揭露傳統防禦機制在面對演化型攻擊時的局限性。基於上述發現,本研究進而結合博弈論模型設計了一套精密的內部罰沒協議 (Internal Slashing Protocol),透過懲罰機制確保節點的審計成本始終低於其潛在的作惡收益,促使誠實行為成為理性參與者在長期互動下的納什均衡 \cite{chiu2018incentive}。最終,這些理論創見在模擬實驗中得到了充分驗證。數據顯示,在 30\% 惡意共謀的極端環境下,CACA 仍能將模型準確率穩定維持在 98.6\% 以上,同時在相同的安全性指標下將通訊開銷降低了 44.4\%,並將系統最低不可用率從 20\% 壓制至 5\% 以下,顯著提升了系統的魯棒性與效率。

本論文餘下部分的組織結構編排如下。在緒論之後,第二章將奠定研究基礎,深入探討聯邦學習、區塊鏈共識機制與拜占庭容錯技術等核心背景知識,並對現有的去中心化聯邦學習文獻進行批判性評述。以此為起點,第三章將定義詳盡的系統模型與威脅模型,針對 PCCA 攻擊者的關鍵特徵進行形式化描述。在完成理論模型構建後,第四章將提出 CACA 的核心設計方案,涵蓋具體協議流程、演算法設計及其背後的安全性證明邏輯。緊接著,第五章將透過一系列針對性實驗與數據對比,在多種對抗場景下驗證本架構的效能優勢與韌性。最終,第六章將總結全研究的主要發現,並探討本機制在未來分散式人工智慧生態中的應用潛力與演進方向。