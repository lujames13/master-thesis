\chapter{緒論 (Introduction)}
\label{chapter:introduction}

隨著人工智慧與分佈式計算技術的進步,區塊鏈賦能的聯邦學習 (Blockchain-based Federated Learning, BCFL) 已成為解決多方互不信任情境下協作機器學習的核心技術路徑。在諸如低軌衛星網絡 (LEO) \cite{pokhrel2021blockchain, wu2024sharded, elmahallawy2025decentralized}、車聯網 (V2X) \cite{lu2020asynchronous, liu2021blockchain, pokhrel2020autonomous} 以及工業物聯網 (IIoT) \cite{lu2020blockchain, qu2020decentralized, eppbcfl2025} 等實際應用場景中,BCFL 展現了其不可替代的重要性。特別是以 LEO 衛星星座為代表的太空 AI 應用場景,星地通訊窗口通常僅約 5 分鐘,且下行帶寬受限於 8Mbps 左右 \cite{wu2024sharded},使得依賴地面站聚合的傳統模型訓練方案難以實施。BCFL 通過在異質衛星營運商間建立去中心化信任層,成功將收斂時間減少達 30 小時 \cite{elmahallawy2025decentralized}。同樣地,在工業 4.0 的背景下,BCFL 允許協作工廠在不洩露商業機密的前提下進行預測性維護,實驗數據顯示其通訊開銷可較集中式架構減少約 41\% \cite{lu2020blockchain}。這些場景共同呈現出「無可信中心」、「資源受限」與「數據高度異質」的特徵,促使 BCFL 成為通用去中心化學習架構的首選方案。

然而,BCFL 在邁向大規模部署時面臨著嚴峻的效率瓶頸,這在業界被稱為「可擴展性兩難」。目前絕大多數 BCFL 系統採用 PBFT (Practical Byzantine Fault Tolerance) 或其變體作為共識機制,其 $O(n^2)$ 的訊息複雜度在節點數增加時會導致效能急劇下降。根據 FLCoin \cite{qi2024scalable} 的實證研究,當參與節點數達到 100 個時,單輪共識產生的訊息量將超過 20,000 條,導致共識延遲攀升至 25 秒以上,此延遲水平已達到模型訓練時間的量級。在極端的車載網路 (VANET) 實測中,100 輛車進行 BCFL 協作會產生 360.57 MB 的巨大數據量,單輪訓練的總通訊開銷高達 19.51 秒 \cite{vanetbcfl2024}。此外,區塊鏈節點對存儲的高需求 (如比特幣需 200GB,乙太坊超過 465GB) 與邊緣設備 KB 至 MB 級的有限內存形成強烈衝突 \cite{fedchain2024}。這種效能與資源的雙重束縛,使得全節點驗證的傳統架構在實際工業部署中顯得難以維繫。

為了解決上述可擴展性挑戰,學界近年來轉向研究「委員會機制 (Committee Mechanism)」,其核心思想是將驗證責任從全體節點縮減至一組小型驗證者委員會。目前主流的選拔機制包含基於哈希環的隨機抽樣 \cite{qin2024blockdfl}、基於幣齡或權益的權重選舉 \cite{li2021blockchain, qi2024scalable} 以及基於預言機 (VRF) 的 Sortition 機制 \cite{shayan2021biscotti, weng2021deepchain}。委員會機制的引入立竿見影地改善了系統效能:FLCoin \cite{qi2024scalable} 通過滑動窗口選舉將通訊開銷降低了 90\%,並實現了 5.7 倍的訓練加速;BFLC \cite{li2021blockchain} 則利用委員會驗證成功將共識延遲穩定在 3 秒以內。這些優化雖成功將通訊複雜度降至與委員會規模 $C$ 相關的 $O(C^2)$ 或 $O(C)$。然而,這種為了效率而進行的「算力與權力集中」也同時引入了新的、尚未被充分探討的安全攻擊面。

最令學界擔憂的危機在於現有委員會防禦機制對「誠實多數假設 (Honest Majority Assumption)」的過度依賴。根據 2024 年針對拜占庭強健聯邦學習的全面調查 \cite{li2025enhancing, xing2023zero},目前超過 93.3\% 的 BCFL 研究雖部署了 Krum、Trimmed Mean 或 Median 等防禦算法,但皆隱含地假設執行這些算法的實體 (即委員會成員) 是絕對誠實的。現有的威脅模型大多只考慮惡意客戶端上傳毒化梯度,卻忽略了「理性驗證者 (Rational Verifiers)」的危害。最新研究指出,理性對手可以先透過合法行為積累聲譽,一旦在委員中取得超過 33\% (針對 BFT 系統) 或 50\% (針對一般投票系統) 的主導權,即可輕易繞過所有魯棒聚合算法,甚至偽造聚合結果而不受懲罰。BlockDFL \cite{qin2024blockdfl} 與 FedBlock \cite{nguyen2024fedblock} 等前沿工作亦坦言,現有機制無法抵禦具備長期策略的委員會共謀攻擊。

上述現象揭示了一個關鍵的「研究缺口 (Research Gap)」:現有 BCFL 缺乏應對「漸進式委員會佔領 (Progressive Committee Capture, PCC)」的自癒機制。在 PCCA 攻擊中,對手並非採取暴力破壞,而是實施「策略性餓死 (Strategic Starvation)」——即在掌控委員會後,優先打包與自身利益相關的更新,並拒絕為誠實參與者提供驗證服務,從而操縱獎勵分配與權益動態。由於缺乏事後的「可追溯審計」與「有效威懾」,一旦誠實多數假設在某一輪次被攻破,系統權力將產生雪崩式的中心化。現有的基於同態加密或權益證明的方案雖然能保護隱私,卻無法在委員會本身已不再可信的情況下,保證模型更新的正確性與資源分配的公平性。如何解耦安全性與共識節點集體信用,成為實現真正去中心化 AI 平台的最後一哩路。

針對這一挑戰,本文提出了一種「挑戰者增強委員會架構 (Challenge-Augmented Committee Architecture, CACA)」,旨在為 BCFL 引入一種全新的安全性保險機制。本研究提出的核心思想是「即時執行、異步審計、罰沒威懾」,這與傳統的「先驗證、後提交」模式有本質區別。我們的主要創新點在於將系統的「活性 (Liveness)」與「安全性 (Security)」進行解耦:即使在委員會不完全可信、甚至被捕獲的情況下,系統仍能通過去中心化的挑戰者網絡來檢舉委員會的錯誤決策。具體貢獻概括如下:(1) 我們首次定義並模擬量化了漸進式委員會佔領攻擊對 BCFL 長期激勵相容性的破壞力;(2) 我們提出了一套基於博弈論設計的「內部罰沒 (Internal Slashing)」協議,確保審計成本低於作惡罰金,從而使得誠實行為成為理性節點的納什均衡;(3) 實驗結果顯示,在 30\% 惡意共謀的極端環境下,本框架仍能維持 91.8\% 的模型準確率,且在 500 節點規模下提供了相較於 BlockDFL 顯著的效率與回原能力。

本論文的組織結構編排如下:第一章為緒論,闡明研究動機、目標與貢獻。第二章介紹聯邦學習、區塊鏈底層架構及拜占庭容錯技術等背景知識。第三章對現有的去中心化聯邦學習文獻進行分類與批判性評述。第四章定義本研究的系統模型與 PCC 攻擊者的行為特徵。第五章詳細描述 CACA 的具體設計流程、智能合約實現及安全協議。第六章呈現模擬實驗的參數設定與效能對比結果,驗證所提架構的有效性。第七章對全論文進行總結,並探討本研究在大型語言模型 (LLM) 與 Web3 領域的未來延伸方向。
