\begin{ZhChapter}

\chapter{結論與未來展望}
\label{chap:conclusion}

\section{研究總結}

區塊鏈聯邦學習的委員會架構在追求執行效率的同時,隱含著對「誠實多數假設」的過度依賴。本研究針對此安全性缺口進行系統性分析,從威脅識別、形式化建模到防禦機制設計,建構了完整的理論與實踐框架。我們識別並定義了「漸進式委員會佔領攻擊」,揭示理性攻擊者如何透過策略性權益累積逐步滲透委員會治理結構,從根本上顛覆去中心化系統的安全假設。

為彌補此缺口,本論文提出審計驅動型委員會 BlockDFL (AC-BlockDFL),其核心設計哲學在於將安全性保障與委員會規模解耦。透過異步審計機制與內部罰沒協議,本架構實現從「門檻安全性」向「經濟安全性」的典範轉移:與其執著於將被攻破機率壓至趨近於零,不如確保即使委員會被攻破,攻擊者也無法獲取正向收益。這種視角轉換使系統在面對理性對手時,仍能維持運作活性與模型聚合正確性,同時享有小規模委員會的效率優勢。

\section{研究貢獻}

本研究的首要貢獻在於對漸進式委員會佔領攻擊進行形式化威脅建模與實證驗證。我們定義了此攻擊的兩階段演化模型,刻畫攻擊者如何在潛伏階段偽裝誠實以累積權益,並在獲得控制權後切換至佔領階段執行惡意策略。長期模擬實驗證實了傳統委員會架構確實存在權益固化與治理失效風險,為防禦機制的必要性提供論據基礎。

第二項貢獻體現在經濟懲罰機制對攻擊者誘因結構的重塑。實驗結果顯示,罰沒機制成功打破惡意節點的權益累積正反饋循環,實現「漸進式淨化」效果。罰沒機制不僅懲罰個別惡意行為,更將攻擊失敗後果轉化為永久性治理排除,內部化作惡的外部性成本,迫使理性節點趨向誠實策略。

第三項貢獻在於證明「事前預防」轉向「事後追責」的架構創新能有效打破安全性與通訊開銷的強耦合。透過引入經濟安全性作為獨立保障路徑,本架構在維持等效安全性的前提下顯著降低通訊成本,對資源受限的邊緣運算場景具有重要實務價值。

\section{未來展望}

本研究所提出的架構在應對理性攻擊者時展現優越防禦能力,然而仍存在若干值得探索的方向。首先,本研究的防禦策略建立在聯邦學習固有自癒能力之上,採用「僅懲罰不回滾」原則。當攻擊者目標從理性獲利轉為純粹破壞時,自癒機制的有效性邊界將成為關鍵問題,未來可探討如何設計高效的模型回溯復原機制。

其次,本研究驗證了罰沒機制在特定威脅環境下的淨化效果,但在惡意節點絕對數量更大的場景中,淨化所需輪次將延長。未來可探討如何透過動態調整委員會規模或設計更積極的挑戰觸發策略來最佳化淨化效率。

最後,實際部署環境往往具有高度異質性與動態性,包括低軌衛星網路的通訊窗口限制、工業物聯網的資源差異等。未來可探討如何建構自適應委員會機制,根據網路威脅監控資料與節點資源狀態動態調整配置,同時確保不會產生安全缺口或成為新的攻擊向量。

\end{ZhChapter}