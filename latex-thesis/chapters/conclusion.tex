\chapter{結論與未來展望 (Conclusion and Future Work)}
\label{chap:conclusion}

\section{研究總結 (Summary of Research)}
本研究針對區塊鏈聯邦學習 (Blockchain-based Federated Learning, BCFL) 在委員會架構下過度依賴「誠實多數假設」的安全漏洞進行了系統性分析。我們識別出一種針對權益機制缺陷的「漸進式委員會佔領攻擊 (Progressive Committee Capture Attack, PCCA)」,揭示了理性攻擊者如何透過累積治理資源,規避傳統的資料層防禦。為了彌補這一安全性缺口,本論文提出「挑戰增強型委員會架構 (Challenge-Augmented Committee Architecture, CACA)」,其核心設計哲學在於安全性與治理規模的解耦。透過引入異步審計與內部罰沒協議,我們將系統的安全防禦從「門檻安全性 (Threshold Security)」轉向「經濟安全性 (Economic Security)」,確保系統在面對具備策略性的理性對手時,仍能維持高度的活性與模型聚合的正確性。

\section{研究發現與貢獻 (Research Findings and Contributions)}
本研究的主要發現與貢獻總結如下:

\begin{itemize}
    \item 定義並驗證 PCCA 的威脅演化:本研究首次定義了漸進式委員會佔領攻擊的兩階段模型(潛伏與佔領),並量化了權益機制正反饋如何加速網路控制權的轉移。實驗證實,傳統架構(如 BlockDFL)在長期運行中存在顯著的財富固化與治理失效風險。
    \item 強化系統在極端環境下的服務能力:透過 CACA 的挑戰機制,系統在遭受 30\% 惡意共謀的壓力下,能有效將成功受擊頻率壓制在極低水平。數據顯示,本架構不僅能將最低不可用率從 20\% 降至 5\% 以下,更能在 Non-IID 資料分佈下維持與 IID 環境相近的收斂穩定性。
    \item 重塑理性攻擊者的誘因結構 (Incentive Realignment):長期賽局實驗顯示,罰沒機制能有效打破惡意節點的「權益累積循環」。數據指出,攻擊失敗導致的治理權益驟降(至誠實節點的 22.6\%),實質上內部化了作惡的外部性成本,使得攻擊的預期收益遠低於潛在損失。這種經濟上的不對稱性,迫使理性節點趨向誠實策略,從而實現了無須依賴中心化仲裁的去中心化治理平衡。

    \item 打破安全性與通訊開銷的強耦合:本研究證明了「事前預防」轉向「事後追責」的效率優勢。在維持相同安全性邊界的前提下,CACA 允許系統在常態下僅維持輕量級的小型委員會運作(如 $c=5$),成功減少了約 44.4\% 的通訊冗餘,為資源受限的邊緣運算場景提供具擴展性的防禦方案。
\end{itemize}

\section{未來展望 (Future Work)}
本研究提出的挑戰增強型委員會架構 (CACA) 在應對理性攻擊者時展現了優越的經濟防禦力。基於現有成果,未來研究可朝以下兩個方向進一步延伸:

\subsection{聯邦學習自癒界限與災難性恢復機制}
本研究目前仰賴聯邦學習本身的自癒能力來抵銷惡意梯度,並對攻擊者實施「僅懲罰不回滾」的策略以維持系統活性。然而,未來研究可進一步探討在更極端的攻擊行為(如旨在徹底毀滅模型的非理性拜占庭攻擊)下,自癒能力的失效界限。當「全棧投毒」場景注入的更新足以導致模型發生不可逆的發散時,如何設計一套高效的「模型回溯復原機制」將成為核心課題。此機制的挑戰在於,如何在偵測到災難性損害後,精準且低開銷地將模型狀態回溯至受攻擊前的檢查點,同時避免因頻繁回溯導致誠實節點的算力嚴重浪費。

\subsection{針對多樣化應用情境之自適應委員會設計}
本研究證實了小規模委員會配合挑戰機制能在常態下提供極高的效率。但在實際應用中,如低軌衛星網路 (LEO) 的通訊窗口限制、或是工業物聯網 (IoT) 中邊緣設備的異質資源約束,其面臨的威脅水平與環境壓力各不相同。未來研究可探討如何建構一套「自適應委員會」機制,根據當前網路的威脅監控數據與應用場景特徵,動態調整委員會的規模或選拔權重門檻。此方向的主要挑戰在於,如何在動態變化的環境中,始終維持足夠的經濟安全性 (Economic Security) 閾值,並確保效率優化不會因過度縮減委員會而產生不可預見的安全缺口。