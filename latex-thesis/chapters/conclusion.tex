\begin{ZhChapter}

\chapter{結論與未來展望 (Conclusion and Future Work)}
\label{chap:conclusion}

\section{研究總結 (Summary of Research)}

區塊鏈聯邦學習作為解決多方互不信任情境下協作機器學習的創新技術典範,其委員會架構在追求執行效率的同時,卻隱含著對「誠實多數假設」的過度依賴。本研究針對此安全性缺口進行了系統性的分析與回應,從威脅識別、形式化建模到防禦機制設計,建構了一套完整的理論與實踐框架。研究過程中,我們首先識別並定義了「漸進式委員會佔領攻擊」這一針對權益機制設計缺陷的隱蔽性威脅,揭示了理性攻擊者如何透過策略性的權益累積來逐步滲透委員會治理結構,最終規避傳統資料層防禦機制的監控。這種攻擊的危險性並非體現在對單一輪次模型品質的破壞,而是從根本上顛覆了去中心化系統的安全假設,能夠將表面上維持分散治理形態的聯邦學習系統,實質上重新集權化至攻擊者手中。

為了彌補上述安全性缺口,本論文提出了挑戰增強型委員會架構,其核心設計哲學在於將系統的安全性保障與委員會規模進行解耦。透過引入異步審計機制與內部罰沒協議,本架構實現了從傳統「門檻安全性」向「經濟安全性」的典範轉移。在門檻安全性的框架下,系統被迫透過擴大委員會規模來降低被攻破的機率,這種做法不僅帶來高昂的通訊成本,更無法從根本上消除攻擊誘因。經濟安全性則採取截然不同的策略:與其執著於將被攻破的機率壓制至趨近於零,不如確保即使委員會被攻破,攻擊者也無法從中獲取正向收益。這種從「預防攻擊發生」到「消除攻擊誘因」的視角轉換,使得系統在面對具備策略性思維的理性對手時,仍能維持高度的運作活性與模型聚合的正確性,同時享有小規模委員會所帶來的效率優勢。

\section{研究發現與貢獻 (Research Findings and Contributions)}

本研究的首要貢獻在於對漸進式委員會佔領攻擊進行了形式化的威脅建模與實證驗證。我們首次定義了此攻擊的兩階段演化模型,清晰刻畫了攻擊者如何在潛伏階段透過完美的誠實偽裝來累積權益與信譽,並在獲得委員會控制權後切換至佔領階段執行戰略性餓死或全棧投毒策略。2000 輪的長期模擬實驗量化呈現了權益機制正反饋特性如何加速網路控制權的轉移:在缺乏防禦機制的 BlockDFL 架構中,惡意節點的權益比值穩定收斂至誠實節點的 1.3 倍,這種持續性的治理優勢使得攻擊者在整個實驗期間共成功發動了 107 次未受懲罰的委員會佔領攻擊。這些實證數據有力地證實了傳統委員會架構在長期運行中確實存在顯著的權益固化與治理失效風險,為本研究提出防禦機制的必要性提供了堅實的論據基礎。

本研究的第二項核心貢獻體現在經濟懲罰機制對攻擊者誘因結構的根本性重塑。長期賽局實驗的結果顯示,罰沒機制成功打破了惡意節點的權益累積正反饋循環,實現了「漸進式淨化」的防禦效果。在 2000 輪的實驗觀察期內,挑戰增強型委員會架構透過五次階梯式的罰沒制裁,將惡意節點的權益比值從初始的 1.0 逐步壓制至最終的 0.37,意味著攻擊者的平均權益僅為誠實節點的三分之一強。更值得關注的是,隨著惡意節點權益基數的縮減,其組織後續攻擊的難度持續提高,第 1332 輪的最後一次罰沒之後長達 668 輪的觀察期內再無任何成功攻擊的記錄。這種動態變化的深層意涵在於,罰沒機制不僅懲罰了個別的惡意行為,更將攻擊失敗的後果轉化為永久性的治理排除,實質上內部化了作惡的外部性成本,使得攻擊的預期收益遠低於潛在損失,從而迫使理性節點趨向誠實策略。

本研究的第三項貢獻在於證明了「事前預防」轉向「事後追責」的架構創新能夠有效打破安全性與通訊開銷之間的強耦合關係。傳統的門檻安全性思維將委員會規模視為安全性的唯一保障手段,追求更高的安全性必然要求更大的委員會,而更大的委員會必然帶來更高的通訊成本。挑戰增強型委員會架構透過引入經濟安全性作為獨立的第二條保障路徑,成功弱化了這種強耦合。實驗結果顯示,在維持等效安全性保證的前提下,本架構允許系統採用規模為 7 的委員會,相較於達成相同安全閾值所需的規模 9 委員會,通訊成本降低約 39.5\%。這種效率提升對於資源受限的邊緣運算場景具有重要的實務價值,使得區塊鏈聯邦學習技術能夠更廣泛地應用於低軌衛星網路、車聯網、工業物聯網等對通訊頻寬與延遲敏感的部署環境。

\section{未來展望 (Future Work)}

本研究所提出的挑戰增強型委員會架構在應對理性攻擊者時展現了優越的經濟防禦能力,然而在更廣泛的應用情境與更極端的威脅條件下,仍存在若干值得深入探索的研究方向。

\subsection{聯邦學習自癒界限與災難性恢復機制}

本研究的防禦策略建立在聯邦學習固有自癒能力的基礎之上,採用「僅懲罰不回滾」的處置原則來維持系統的運作活性。這種設計選擇源於對機器學習容錯特性的認識:偶發性的模型參數偏差能夠被後續輪次中來自誠實節點的正確更新逐步修正,引導模型重新收斂至正確的優化方向。然而,此假設在面對更極端的攻擊行為時可能面臨挑戰。當攻擊者的目標從理性的經濟獲利轉變為純粹的系統破壞,例如不計成本地執行旨在徹底癱瘓模型的非理性拜占庭攻擊時,自癒機制的有效性邊界便成為關鍵問題。未來研究可深入探討在何種攻擊強度與頻率的組合下,聯邦學習的自我修復能力將達到失效臨界點,以及當全棧投毒場景注入的惡意更新足以導致模型發生不可逆發散時,如何設計一套高效的模型回溯復原機制。此機制的核心挑戰在於平衡安全性與效率:一方面需要能夠在偵測到災難性損害後精準地將模型狀態回溯至受攻擊前的檢查點,另一方面又必須避免因過於頻繁的回溯操作而導致誠實節點的運算資源嚴重浪費。

\subsection{大規模惡意節點場景下的漸進式淨化效率}

本研究的實驗驗證了罰沒機制在初始惡意節點佔比 30\% 的威脅環境下能夠透過五次階梯式制裁完成系統的漸進式淨化。然而,由於每次罰沒僅能懲罰當輪實際參與共謀攻擊的惡意節點,其數量受限於委員會規模,因此在惡意節點絕對數量更大的場景中,淨化過程所需的輪次與時間將相應延長。舉例而言,若系統規模擴展至 200 個節點且其中 60 個為惡意節點,在委員會規模維持為 7 的配置下,完整清除所有惡意參與者可能需要經歷更多次的罰沒事件。未來研究可探討如何最佳化漸進式淨化的效率,例如透過動態調整委員會規模來加速惡意節點的暴露與清除,或設計更積極的挑戰觸發策略來提高罰沒事件的發生頻率。此方向的核心權衡在於,加速淨化的措施可能帶來額外的系統開銷或增加誤判風險,如何在淨化效率與系統穩定性之間取得適當平衡將是關鍵的設計考量。

\subsection{針對多樣化應用情境之自適應委員會設計}

本研究證實了小規模委員會配合挑戰機制能在常態下提供極高的運作效率,然而實際的部署環境往往具有高度的異質性與動態性。在低軌衛星網路的應用場景中,衛星與地面站之間的通訊窗口受到軌道週期的嚴格限制,可能僅有數分鐘的連線時間來完成模型同步與共識達成;在工業物聯網的邊緣運算環境中,不同設備的運算能力、網路頻寬、電力供應等資源條件可能存在數量級的差異;在跨組織的聯盟學習場景中,參與方的加入與退出可能導致網路威脅水平的動態變化。面對這些多樣化的應用情境,固定的委員會配置難以同時滿足所有場景的最佳化需求。未來研究可探討如何建構一套自適應的委員會機制,能夠根據當前網路的威脅監控資料、節點資源狀態、以及應用場景的特定約束條件,動態調整委員會的規模或成員選拔的權重門檻。此方向的主要挑戰在於,如何在動態變化的環境中始終維持足夠的經濟安全性閾值,確保效率最佳化的追求不會因過度縮減委員會而產生不可預見的安全缺口,同時避免自適應機制本身成為新的攻擊向量。

\end{ZhChapter}